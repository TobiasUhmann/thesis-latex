As described in the introduction in Chapter~\ref{ch:1_introduction}, the Power model is an ensemble model, consisting of a rule-based and a text-based component, whose goal is the comprehensible prediction of new facts. Given a query entity with a few known facts and descriptive texts, Power implements this requirement by providing prioritized lists of rules and sentences that were crucial for the predicted facts. Considering the application scenario in which a user uses the model for support of manual knowledge graph completion, it is also important that correct predictions are ranked as high as possible among an entity's predicted facts.

Technically, the Power ensemble consists of three components as shown in Figure~\ref{fig:4_approach/power_architecture}. Besides the rule-based component, named \emph{Ruler}, and the text-based component, named \emph{Texter}, it also contains the so-called \emph{Aggregator} that combines the Ruler's and Texter's predictions. If both Ruler and Texter predict a fact, the final prediction yielded by the Aggregator comes with both, the Ruler's prioritzed list of decisive rules and the Texter's ranked list of sentences, and is considered more reliable than a fact predicted by only one of the components.

\begin{figure}[t]
    \centering
    \includegraphics[width=\textwidth]{4_approach/power_architecture}
    \caption{Power model consisting of the Ruler, which processes a query entity's known facts, the Texter, which processes the entity's texts, and the Aggregator, that combines all predictions. All prediction lists are sorted by confidence and come with sorted lists of explaining rules and texts.}
    \label{fig:4_approach/power_architecture}
\end{figure}

Depending on the dataset at hand, either the Ruler or Texter might perform better. In general, the Texter should perform better for graphs that contain many similar facts, such as $(x, part~of, Asia)$ with $x \in \{China, India, Japan, \dots\}$, while the Ruler also works well on diverse datasets but cannot handle open-world entities about which no facts are known. Together, they complement each other: The Ruler handles very rare facts while the Texter can bootstrap the prediction process on open-world entities until the Ruler can join when some facts are available for rule application. In Chapter~\ref{sec:5_experiments/5_aggregator}, the Ruler's and Texter's evaluation results are compared to each other as well as to the results achieved by the Aggregator.

\section{Texter}
\label{sec:4_approach/1_texter}
With a Power dataset in place and the evaluatin metrics defined, the next step is training the Power model's components. In case of the Texter, however, an intermediate step is necessary. The Texter is trained in respect to a certain set of relation-tail tuples that make up its classes and requires a fixed number of sentences per entity. Given a fixed number of classes $c$, the $c$ most common relation-tail tuples in the training facts are determined and become the Texter's classes. A Texter dataset is created from the Power dataset that specifies which of the classes hold true for each entity. Furthermore, the Texter dataset contains the specified number of randomly samples sentences for each entity. Table~\ref{tab:5_experiments/4_texter/texter_dataset} shows an excerpt from a Texter dataset.

\begin{table}[h]
    \centering
    \begin{tabular}{ r l r r r r r r r r l }
    \toprule
    
    \multicolumn{1}{c}{\textbf{Ent}} &
    \multicolumn{1}{l}{\textbf{Entity Label}} &
    \multicolumn{8}{c}{\textbf{Classes}} &
    \multicolumn{1}{l}{\textbf{Sentences}} \\
    
    \midrule

    25 & Gabriel Yared    & 1 & 1 & 0 & 0 & 0 & 0 & 0 & 0 & Lebanese composer.        \\
    26 & Bridesmaids      & 0 & 0 & 0 & 0 & 0 & 1 & 1 & 1 & 2011 film by paul feig.   \\
    27 & Portugal         & 0 & 0 & 0 & 0 & 0 & 0 & 0 & 0 & Republic in \dots         \\
    28 & Serpico          & 0 & 0 & 0 & 0 & 0 & 1 & 1 & 1 & 1973 american crime \dots \\
    29 & Italian          & 0 & 0 & 0 & 0 & 0 & 0 & 0 & 0 & Romance language.         \\
    30 & Catherine Keener & 0 & 0 & 1 & 1 & 1 & 0 & 0 & 0 & American actress.         \\
    
    \bottomrule
\end{tabular}

    \caption{Excerpt from a Texter dataset from the FB Power split and the fb-owe-1-clean text set for a Texter with eight classes}
    \label{tab:5_experiments/4_texter/texter_dataset}
\end{table}

In general, the Texter dataset's classes array is very sparse, especially for diverse knowledge graphs without frequently occurring relation-tail tuples. Therefore, it is important to weight the classes when calculating the loss during training to punish false negatives more heavily than false positives. Otherwise, the Texter would learn that always predicting 0 keeps the loss as low as possible. Table~\ref{tab:5_experiments/4_texter/classes} gives an overview of the imbalance between the top 100 classes that were selected for this evaluation. As can be seen, the FB split contains not only more facts than the CDE split, but also fewer diverse facts, which should facilitate the Texter's training.

\begin{table}[h]
    \centering
    \begin{tabular}{| l | r | r | l |}
    \hline
    
    \multicolumn{1}{|c|}{\textbf{Split}} &
    \multicolumn{1}{|c|}{\textbf{Rank}} &
    \multicolumn{1}{|c|}{\textbf{Freq}} &
    \multicolumn{1}{|c|}{\textbf{Class = (Relation, Tail)}} \\

    \hline \hline

    \multirow{3}{*}{CDE}
    & 1   & 19.78 & (country of citizenship, United States of America)   \\
    & 2   & 19.10 & (occupation, writer)                                 \\
    & 3   & 17.06 & (languages spoken, written, or signed, English)      \\

    \hline

    \multirow{3}{*}{CDE}
    & 98  & 1.08  & (genre, hip hop music)                               \\
    & 99  & 1.07  & (member of, International Finance Corporation)       \\
    & 100 & 1.07  & (record label, Columbia Records)                     \\

    \hline \hline

    \multirow{3}{*}{FB}
    & 1   & 24.34 & (\dots/webpage/category, /m/08mbj5d)                 \\
    & 2   & 19.67 & (/people/person/gender, male organism)               \\
    & 3   & 17.38 & (\dots/marriage/type\_of\_union, marriage)           \\

    \hline

    \multirow{3}{*}{FB}
    & 98  & 1.35  & (\dots/award\_nomination/award, Screen Actors \dots) \\
    & 99  & 1.32  & (/film/film/genre, fantasy)                          \\
    & 100 & 1.31  & (\dots/film\_release\_region, Israel)                \\

    \hline
\end{tabular}

    \caption{Most and least common classes on the CDE and FB splits. Frequencies are given in percent.}
    \label{tab:5_experiments/4_texter/classes}
\end{table}

Evaluation is generally performed against the Texter dataset's test subset, following the form shown in Table~\ref{tab:5_experiments/4_texter/texter_dataset}. That means that the Texter's predictions are evaluated against the classes it was trained on and that the results are comparable to the validation results during training. The metrics chosen are macro precision, recall and F1 over the classes. This is rather demanding, as rare classes are equally weighted as the common classes, that represent more facts, but it follows the objective of predicting all classes as good as possible. For the final comparison with Ruler and Aggregator, however, the Texter is also evaluated against the Power split's test facts, which includes facts the Texter is unable to predict, using the common metrics discussed before in Section~\ref{sec:5_experiments/3_metrics}.

The following Subsection~\ref{subsec:5_experiments/4_texter/1_zero_rule} applies the presented metrics to several baselines to give an idea of the problem's complexity. Given the reference values from the baselines, Subsections~\ref{subsec:5_experiments/4_texter/2_static} and~\ref{subsec:5_experiments/4_texter/3_context} present two ways of implementing the Texter's embedding block using static and contextual word embeddings, whereby the latter is used in the final Power implementation described in Chapter~\ref{ch:4_approach}. To distinguish the two variants, they are also referred to as static and contextual Texter, respectively. Both sections include several experiments that led to the final versions of the models. In addition, each experiment compares the simple and the attentive versions of the Texter to see which variations affect the attention mechanism of the latter.

\subsection{Zero Rule Baselines}
\label{subsec:5_experiments/4_texter/1_zero_rule}
Zero-rule baselines are naive models that give an idea of how well a problem could be tackled by a naive approach, such as simply guessing random solutions. Any model performing worse than a zero-rule baseline cannot be considered useful. For example, in a coin toss, one would already achieve 50\% accuracy, precision and recall by randomly guessing between heads and tails. A coin-flip predicting model that does not exceed that performance would not be considered useful.

The purpose of this subsection is to determine the absolute baseline for the classification problem at hand that the Texter has to exceed. Therefore, the best zero-rule for the binary multi-label classification the Texter dataset poses has to be found. Randomly guessing with a 50:50 chance which of an entity's classes hold true is one naive strategy, but there are better ones when it comes to unbalanced classes such as the Texter dataset's ones. For example, for a class with an 80:20 chance to be true, randomly guessing true in 80 of 100 cases leads to better accurracy than choosing equally between true and false.

In general, performance for a random guessing strategy can be calculated. For example, accuracy in an unbalanced classification scenario with two possible outcomes 0 and 1 can be calculated as $P(\hat{y} = y) = P(\hat{y} = 0) \cdot P(y = 0) + P(\hat{y} = 1) \cdot P(y = 1)$ where $y$ is the ground truth outcome and $\hat{y}$ is the predicted outcome. While $P(y = 0)$ and $P(y = 1)$ depend on the regarded class' outcome distribution, the values of $P(\hat{y} = 0)$ and $P(\hat{y} = 1)$, and thus the probability to predict the ground truth value, depend on the guessing strategy. In this evaluation, four potential guessing strategies were considered with respect to each class:

\begin{itemize}
    \item \emph{Uniformly} guessing between 0 and 1, i.e. $P(\hat{y} = 0)$ = $P(\hat{y} = 1)$ = 0.5

    \item \emph{Stratified} guessing, which means that each outcome is guessed according to its frequency, i.e. $P(\hat{y} = 0)$ = $P(y = 0)$ and $P(\hat{y} = 1)$ = $P(y = 1)$

    \item Always guessing the \emph{most frequent} outcome, i.e.  $P(\hat{y} = 0)$ = 1 and $P(\hat{y} = 1)$ = 0 if 0 is the most common outcome

    \item \emph{Constantly guessing 1}, i.e. $P(\hat{y} = 0)$ = 0 and $P(\hat{y} = 1)$ = 1
\end{itemize}

Similarly, precision, recall and F1 could also be calculated for each class of the Texter dataset. Averaging each class' metrics over all classes would finally yield each zero-rule strategy's performance for the whole Texter dataset. For this evaluation, the zero-rule strategies were implemented and measured instead. The values in \autoref{tab:5_experiments/3_texter/1_zero_rule/results} were gathered by running each of the four zero-rule strategies ten times and averaging the values. For each zero-rule strategy, performance is shown for the most common class, the least common class and the average over all classes. As shown before, the Texter datasets' classes arrays are very sparse, so it was expected that constantly predicting the most common output false for all classes would lead to the highest accurracy. However, since F1 is the more relevant metric in regard to predicting true positives, constantly predicting true was expected to be the best zero-rule strategy.

\begin{table}[t]
    \makebox[\textwidth][c]{
        \begin{tabular}{| l | r | r | r | r | r | r | r | r | r |}
    \hline
    
    \multicolumn{1}{|c|}{\textbf{Split}} &
    \multicolumn{1}{|c|}{\textbf{mAP}} &
    \multicolumn{4}{|c|}{\textbf{Macro over ents}} &
    \multicolumn{4}{|c|}{\textbf{Micro over facts}} \\
    
    \multicolumn{1}{|c|}{} &
    \multicolumn{1}{|c|}{} &
    \multicolumn{1}{|c|}{\textbf{Prec}} &
    \multicolumn{1}{|c|}{\textbf{Rec}} &
    \multicolumn{1}{|c|}{\textbf{F1}} &
    \multicolumn{1}{|c|}{\textbf{Supp}} &
    \multicolumn{1}{|c|}{\textbf{Prec}} &
    \multicolumn{1}{|c|}{\textbf{Rec}} &
    \multicolumn{1}{|c|}{\textbf{F1}} &
    \multicolumn{1}{|c|}{\textbf{Supp}} \\
    
    \hline \hline
    
    CDE-0 & 0.84 &
    100.00 & 0.84 & 0.84 & \num{13.32} &
    100.00 & 0.00 & 0.00 & \num{25255} \\
    
    CDE-25 & 22.06 &
    66.90 & 24.27 & 32.02 & \num{13.32} &
    62.67 & 22.93 & 33.57 & \num{25255} \\
    
    CDE-50 & 29.26 &
    61.41 & 33.14 & 40.29 & \num{13.32} &
    58.47 & 31.52 & 40.96 & \num{25255} \\
    
    CDE-75 & 33.23 &
    57.88 & 38.10 & 43.59 & \num{13.32} &
    55.43 & 3638 & 43.93 & \num{25255} \\
    
    CDE-100 & 35.77 &
    55.62 & 41.48 & 45.28 & \num{13.32} &
    53.49 & 39.72 & 45.59 & \num{25255} \\
    
    \hline
    
    FB-0 & 3.19 &
    100.00 & 03.19 & 3.19 & \num{18.76} &
    100.00 & 0.00 & 0.00 & \num{15312} \\
    
    FB-25 & 27.41 &
    73.46 & 30.56 & 37.06 & \num{18.76} &
    69.28 & 30.15 & 42.02 & \num{15312} \\
    
    FB-50 & 33.39 &
    68.36 & 37.61 & 42.90 & \num{18.76} &
    64.90 & 36.76 & 46.94 & \num{15312} \\
    
    FB-75 & 36.22 &
    64.93 & 41.37 & 45.14 & \num{18.76} &
    62.76 & 40.82 & 49.47 & \num{15312} \\
    
    FB-100 & 38.43 &
    63.08 & 44.34 & 46.97 & \num{18.76} &
    60.98 & 43.51 & 50.79 & \num{15312} \\
    
    \hline
\end{tabular}

    }
    \caption{Evaluation of various zero-rule baselines on the CDE and FB splits. The best average values per split are marked bold (column-wise). Thus, uniform sampling sets the baseline on the CDE split (20.19\% F1), while constantly predicting true leads to the best result on the FB split (9.43\% F1).}
    \label{tab:5_experiments/3_texter/1_zero_rule/results}
\end{table}

As it turned out, constantly predicting true for all classes is indeed the best random guessing strategoy for the Texter dataset derived from the FB split, leading to an F1 score of 9.43\% which serves as the absolute baseline for any Texter predicting facts for the FB split. Surprisingly, the best zero-rule strategy for the Texter dataset derived from the CDE split is randomly guessing between true and false, leading to over 20.19\% F1 score, which is almost twice as much as the value achieved when constantly guessing true. Those two percentages have to be kept in mind when regarding any performances achieved by actually learning models.


\subsection{Static Word Embeddings}
\label{subsec:5_experiments/4_texter/2_static}
First, the embedding block in both, the simple and the attentive Texter, is investigated when using static, pre-trained word embeddings, i.e. a word is embedded independent from where it occurs in the sentence. The advantages include ease of implementation using a simple look-up table, few hyperparameters and quick training. However, the sentence embedding cannot properly capture the meaning of markings and maskings via special tokens such as those in sentences from the "cde-irt-5-marked" or "cde-irt-5-masked" text sets exemplified in Table~\ref{tab:5_experiments/1_data_sources/2_text_sets/text_sets_table}. Therefore, all experiments in this section are conducted on the clean text sets containing no special tokens. During training, the settings that proved to be best during the experiments were used. In particular, these are the use of an NLP-based tokenizer, pre-trained fastText~\ref{} word embeddings that are finetuned during training, mean pooling to form sentence embeddings, the sigmoid function in the attention block in the case of the attentive texter, class weights in the computation of the loss, and the Adam optimizer to apply gradients during backpropagation. Table~\ref{tab:5_experiments/4_texter/2_static/results} shows the final results over all clean text sets for both the simple and the attentive versions of the Texter. As for all following tables presenting Texter evaluation results, for compactness, only the name of the text set is given for the respective Texter dataset, because it implies the matching Power split.

\begin{table}[t]
    \centering
    \begin{tabular}{| l | r | r | r | r | r | r | r | r | r |}
    \hline
    
    \multicolumn{1}{|c|}{\textbf{Split}} &
    \multicolumn{1}{|c|}{\textbf{mAP}} &
    \multicolumn{4}{|c|}{\textbf{Macro over ents}} &
    \multicolumn{4}{|c|}{\textbf{Micro over facts}} \\
    
    \multicolumn{1}{|c|}{} &
    \multicolumn{1}{|c|}{} &
    \multicolumn{1}{|c|}{\textbf{Prec}} &
    \multicolumn{1}{|c|}{\textbf{Rec}} &
    \multicolumn{1}{|c|}{\textbf{F1}} &
    \multicolumn{1}{|c|}{\textbf{Supp}} &
    \multicolumn{1}{|c|}{\textbf{Prec}} &
    \multicolumn{1}{|c|}{\textbf{Rec}} &
    \multicolumn{1}{|c|}{\textbf{F1}} &
    \multicolumn{1}{|c|}{\textbf{Supp}} \\
    
    \hline \hline
    
    CDE-0 & 0.84 &
    100.00 & 0.84 & 0.84 & \num{13.32} &
    100.00 & 0.00 & 0.00 & \num{25255} \\
    
    CDE-25 & 22.06 &
    66.90 & 24.27 & 32.02 & \num{13.32} &
    62.67 & 22.93 & 33.57 & \num{25255} \\
    
    CDE-50 & 29.26 &
    61.41 & 33.14 & 40.29 & \num{13.32} &
    58.47 & 31.52 & 40.96 & \num{25255} \\
    
    CDE-75 & 33.23 &
    57.88 & 38.10 & 43.59 & \num{13.32} &
    55.43 & 3638 & 43.93 & \num{25255} \\
    
    CDE-100 & 35.77 &
    55.62 & 41.48 & 45.28 & \num{13.32} &
    53.49 & 39.72 & 45.59 & \num{25255} \\
    
    \hline
    
    FB-0 & 3.19 &
    100.00 & 03.19 & 3.19 & \num{18.76} &
    100.00 & 0.00 & 0.00 & \num{15312} \\
    
    FB-25 & 27.41 &
    73.46 & 30.56 & 37.06 & \num{18.76} &
    69.28 & 30.15 & 42.02 & \num{15312} \\
    
    FB-50 & 33.39 &
    68.36 & 37.61 & 42.90 & \num{18.76} &
    64.90 & 36.76 & 46.94 & \num{15312} \\
    
    FB-75 & 36.22 &
    64.93 & 41.37 & 45.14 & \num{18.76} &
    62.76 & 40.82 & 49.47 & \num{15312} \\
    
    FB-100 & 38.43 &
    63.08 & 44.34 & 46.97 & \num{18.76} &
    60.98 & 43.51 & 50.79 & \num{15312} \\
    
    \hline
\end{tabular}

    \caption{Final evluation results for the simple and the attentive Texter using static word embeddings - for each text set, the better model is marked in terms of precision, recall and F1}
    \label{tab:5_experiments/4_texter/2_static/results}
\end{table}

The numbers show that both the simple and the attentive Texter, outperform the zero rule baselines by far. However, the attentive model does not outperform the simple version as hoped. On most datasets, the attentive model achieves better precision, but the simple Texter has higher recall. Overall, the simple modell performs better, especially on datasets with multiple sentences per entity which should have been in favor of the attentive model. Interestingly, both Texters perform best given the short OWE sentence due to high recall scores, which implies that a single, good entity description is more useful than many vague contexts for knowledge graph completion. Similar to the OWE sentences, the CDE sentences from the entities' Wikipedia introductions perform best on the CDE. However, the results are not far ahead of those for the text set with 30 IRT sentences, so that the observation could also be formulated in reverse: 15 randomly sampled entity contexts are just as good as a selected high-quality description.

In addition to the final evaluation results on the test data, Figure~\ref{fig:5_experiments/4_texter/2_static/plot_valid_curves} shows the development of the F1 score for different text sets during training on the FB graph. After 200 episodes of training, the simple and attentive versions of the Texter converge against similar values. The only notable difference between the two is that the simple model reaches its optimal training time earlier before it starts overfitting on most datasets. Thereby, the IRT text sets with few sentences are more prone to overfitting than the ones with 15 or 30 sentences per entity. The short OWE texts, on the other hand, seem to be immune to overfitting. Although training could be stopped after 50 episodes without major defficiencies in this case, all following experiments are performed with 200 episodes to ensure that all models reach their full potential even if changes lead to slower training.

\begin{figure}[t]
    \centering
    \subfloat[Simple Model]{
    \input{5_experiments/3_texter/2_static/plot_valid_curves/a_simple/a_simple}
    \label{fig:5_experiments/3_texter/2_static/plot_valid_curves/a_simple}
}
\hskip 5pt
\subfloat[Attentive Model]{
    \input{5_experiments/3_texter/2_static/plot_valid_curves/b_attentive/b_attentive}
    \label{fig:5_experiments/3_texter/2_static/plot_valid_curves/b_attentive}
}

    \caption{Validation F1 curves during training the Texter on the FB split with 1 OWE sentence (red), 30 IRT sentences (orange), 15 IRT sentences (yellow), 5 IRT sentences (black) or 1 IRT sentence (green) per entity}
    \label{fig:5_experiments/4_texter/2_static/plot_valid_curves}
\end{figure}

What is not visible in Figure~\ref{fig:5_experiments/4_texter/2_static/plot_valid_curves} is the information how the average F1 score is composed of the class-wise F1 scores, which is addressed by Figure~\ref{fig:5_experiments/4_texter/2_static/plot_class_curves}. It compares the F1 scores between the most common classes, the least common classes, and the average F1 score over all classes. The graphs reveal that predictions for common classes are much more reliable. The three most common classes on the CDE split, with frequencies of almonst 20\% each, reach performances between 50\% and 80\%, whereas the three least common classes, with frequencies of around 1\% each, reach strongly divergent values, from 0\% to 60\%. Despite the tendency that frequent classes perform better, however, a class' performance cannot be calculated directly from its frequence. The least common classes have very similar frequencies but perform very differently. For the second least common class prediction does not work at all, while the third least common class performs even better than the third most common class. There are also significant performance differences between the three most frequent classes although the frequencies are similar. Another, more obvious, finding is that frequent classes are learned within fewer episodes than less common classes. Interestingly, the attentive Texter seems to learn faster than the simple Texter. Finally, the graph shows that both Texters converge to similar values for all classes, as might have been assumed from the similar macro F1 values.

\begin{figure}[t]
    \centering
    \subfloat[Simple Model]{
    \input{a_appendix/static_classes_1/cde_irt_5_simple/cde_irt_5_simple}
    \label{fig:5_experiments/4_texter/2_static/plot_class_curves/a_simple}
}
\hskip 5pt
\subfloat[Attentive Model]{
    \input{a_appendix/static_classes_1/cde_irt_5_attentive/cde_irt_5_attentive}
    \label{fig:5_experiments/4_texter/2_static/plot_class_curves/b_attentive}
}
    \caption{Class-wise validation F1 curves during training the Texter on the CDE split with 5 IRT sentences per entity - comparison between the most common classes (red, orange, yellow), the least common classes (green, blue, purple) and the average value over all classes (black)}
    \label{fig:5_experiments/4_texter/2_static/plot_class_curves}
\end{figure}

Looking at the results for the other datasets shown in Tables~\ref{fig:a_appendix/static_classes_1}~--~\ref{fig:a_appendix/static_classes_3} in Appendix~\ref{ch:a_appendix}, similar patterns can be observed on other text sets with only a few exceptions. However, although there are no major differences between the simple and the attentive Texter, it is noticeable that the results for rare classes do not improve steadily when increasing the number of sentences in case of the IRT text sets. Multiple training runs on the same data showed that this was not due to different initializations of the model. Instead, separate experiments showed that it is the random selection of IRT sentences that affects the performance of individual rare classes. In the respective experiment, new fb-irt-1-clean text sets were built by repeatedly selecting random sentences from the larger fb-irt-30-clean dataset. The overall F1 score over all classes, however, stayed the same for all runs.

With the final results for static word embeddings at hand, the following subsections show what impact various model changes and hyperparameters have. The affected model components are examined in the order in which they are invoked during training, beginning with the tokenizer and ending with the optimizer. Finally, the attentive Texter's attention mechanism is investigated to explain why it does not improve upon the simple Texter as hoped.

\subsubsection{Changing the tokenizer}
\label{subsubsec:5_experiments/4_texter/2_static/1_tokenizer}
The first variable component in both the simple and the attentive Texter is the tokenizer. A naive tokenizer might split the input sentences on any whitespace such as spaces, tabs and line breaks. The problem with that simple approach is the large number of resulting tokens blowing up the vocabulary. For example, the sentence "Hello, world!" would be split into the two tokens "Hello," and "world!", which would be different from "Hello" and "world". As a consequence, none of a word's "impure" occurrences contribute to learning the words embedding, but rather offer the model a way to overfit. One possible solution would be deleting punctuation characters, numbers, dates and other "polluting" characters and unique tokens. However, this could also lead to the destruction of correct tokens, such as "U.K.". Instead, this work uses a tokenizer from the NLP library SpaCy \cite{SpaCy} which splits the sentence "Welcome to the U.K.!" into the tokens "Welcome", "to", "the", "U.K." and "!".

%\begin{table}[h]
%    \centering
%    \input{tables/experiments/static/grid_search.tex}
%    \caption{Impact of different tokenizers on vocabulary size and evaluation results}
%    \label{table:experiments/static/tokenizer}
%\end{table}

Table~\ref{table:experiments/static/tokenizer} shows the tokenizer choice's impact on the vocabulary size and the evaluation results for two text sets on FB15k-237. For the large text set "fb-irt-30", the vocabulary's size halves when using the SpaCy tokenizer rather than splitting on whitespace and the F1 score increases slightly. On "fb-owe-1", the vocabulary's size increases by only 30\%, but the performance gap is over 5\%, probably because every "lost" token means a bigger loss on the small text set. Table~\ref{table:static_tokenizers_all} in Appendix~\ref{ch:a_appendix} contains the full evaluation over all text sets where similar results can be observed on the "cde-irt-30" and "cde-cde-1" text sets.


\subsubsection{Initializing word embeddings randomly}
\label{subsubsec:5_experiments/4_texter/2_static/2_emb_size}
Upon tokenization, the sentences' tokens are embedded -- in the context of this chapter using static word embeddings. For this, randomly initialized, or as shown in the next subsection, pre-trained embeddings can be used. When using randomly initialized embeddings, the questions of what embedding size and which random distribution to chose arises. As random distribution the standard normal distribution $N(0, 1)$ is set. The optimal embedding size is determined via grid search. It is related to, among other things, the amount of available training data and is determined via grid search. Small embeddings can hold little specific information and generalize well, whereas large embeddings can represent subtleties of individual tokens but are also more susceptible to overfitting. The grid search covers a wide range of embedding sizes around the expected optimal embedding size of about 200, a usual size in current models, including unrealistically small sizes to find out to what degree it is worth increasing the size. Overall, expectations for the experiment were limited because of the moderately large amount of training. When using the maximum embedding size of 1000, performance was expected to decrease due to overfitting. \autoref{tab:5_experiments/3_texter/2_static/2_emb_size/grid_search} shows the experiment's results.

\begin{table}[t]
    \centering
    \input{5_experiments/3_texter/2_static/2_emb_size/grid_search}
    \caption{Static Texter with randomly initialized word embeddings of varying size. Numbers show F1 scores. Best result per row marked bold. The simple Texter profits from very large embeddings, while the attentive Texter's performance decreases from medium sizes on.}
    \label{tab:5_experiments/3_texter/2_static/2_emb_size/grid_search}
\end{table}

At first glance, one can see that training does actually work and that the simple Texter performs much better with randomly initialized embeddings. Against expectations, the attentive Texter reaches its top performance at an embedding size of only around 30 to 100, while the simple model benefits from very large embeddings beyond 300. Apart from one outlier, the performances of both models evolve similarly for small embedding sizes until the attentive Texter stagnates early at medium sizes. For the attentive model, the poor performance seems reasonable since semantically similar words, onto which the same class embedding should match, have completely different initial values, which should complicate the class embeddings' training. Another aspect that catches the eye when looking at \autoref{tab:5_experiments/3_texter/2_static/2_emb_size/grid_search} is that even one-dimensional embeddings deliver surprisingly good results. In the case of the CDE split, the appearance is deceptive, as the F1 values are actually below the best zero-rule baseline, which reaches about 20\%, but for the FB split, the results are indeed better than the best zero-rule baseline with only 9\%. Overall, it can be said that randomly initialized embeddings can work, but, despite the good results for the simple Texter, they are not considered in further experiments, as pre-trained embeddings yield better results for both models as shown in the next subsection.


\subsubsection{Using pre-trained word embeddings}
\label{subsubsec:5_experiments/4_texter/2_static/3_pre_trained}
Studies show that pre-trained word embeddings can produce better results after a shorter training period - especially when little training data is available~\cite{}. Especially for the attentive model, there was the assumption that pre-trained embeddings could help the model to focus on training the class embeddings. Until the unexpectedly good experiment with randomly initialized embeddings, pre-trained embeddings even seemed indispensable.

When pre-trained embeddings are provided, Texter's embedding block initializes as many of the vocabulary's words with them as possible. Tokens for which no pre-trained embeddings are available are still initialized randomly. Therefore, when using pre-trained embeddings, it is important to use a tokenizer that produces tokens similar to the ones produced during pre-training - another reason why SpaCy's tokenizer led to better results than the Whitespace tokenizer.

Just as the models used in pre-training use different tokenizers, the resulting pre-trained embeddings also differ. In the experiment at hand, a total of 13 variants of three different types of pre-trained embedding sets were tried. The embedding sets vary in embedding size, vocabulary size and the text corpus pre-training was conducted on. Table~\ref{tab:5_experiments/4_texter/2_static/3_pre_trained/vector_sets} lists all embedding sets tried during the experiment.

\begin{table}[h]
    \centering
    \input{5_experiments/4_texter/2_static/3_pre_trained/vector_sets}
    \caption{Pre-trained word embedding sets considered for evaluation}
    \label{tab:5_experiments/4_texter/2_static/3_pre_trained/vector_sets}
\end{table}

The three types of embedding differ in how word embeddings are obtained:

\begin{itemize}
    \item \textbf{\emph{GloVe}}~\cite{Pennington2014GloveGV}, coined from "global vectors", is an unsupervised learning algorithm dedicated to obtaining the popular, equally named embeddings. GloVe learns embeddings for whole words so that co-occurring, semantically similar words are close in embedding space. Thereby, remarkable relations between related word embeddings emerge, such as the equation $king - queen = man - woman$. Since English has many different words, GloVe has a very large vocabulary.

    \item \textbf{\emph{fastText}}~\cite{Bojanowski2017EnrichingWV,Mikolov2018AdvancesIP}  is based on the idea that words can be viewed as sums of n-grams they exist of. Leveraging the internal structure of words has the advantage that rare or even unknown words can be handled. For example, different declensions of verbs can be related without resorting to stemming, that is, without reducing words to their root. The approach of composing words from n-grams also offers the potential advantage of a small vocabulary.

    \item \textbf{\emph{Charagram}}, referred to as charngram in the library used for implementing Power's Texter, forms word embeddings, as the name suggests, from charachter n-grams as well. Charagram was introduced at the same time as fastText and serves as an alternative to fastText in the experiment.
\end{itemize}

In the experiment all 13 embedding sets were evaluated. For brevity, Table~\ref{tab:5_experiments/4_texter/2_static/3_pre_trained/vector_sets} is limited to the evaluation results for the Charagram, fastText and GloVe embeddings obtained from training on Wikipedia and Gigaword. Thus, the selected embedding sets cover all three embedding types and give an impression of the influence the embedding size has by comparing the four otherwise equal GloVe embedding sets. Table~\ref{} in Appendix~\ref{ch:a_appendix} contains the second part of the table listing the results for the remaining six GloVe embedding sets.

The expected outcome of the experiment was a significant performance increase for both, simple and attentive Texter, just as other works did. Between the embedding sets, medium deviations were assumed, depending on how similar the data during pre-training resembled those of the IRT text sets. In case of the text sets with CDE and IRT sentences taken from Wikipedia, the GloVe embeddings were therefore the favorite candidate. Regarding the embedding type, there were no clear expectations, as both a large GloVe vocabulary and compound charachter n-grams should be able to cover the vocabulary of the natural language texts. However, it was assumed that a larger text corpus used for pre-training would lead to a noteable performance improvement, for example for fasttext.en.300d compared to fasttext.simple.300d or glove.840B.300d compared to glove.42B.300d. Finally, it was expected that an increase in embedding size would have a similar strong effect as for randomly initialized embeddings.

\begin{table}[h]
    \input{5_experiments/4_texter/2_static/3_pre_trained/grid_search}
    \makebox[\textwidth][c]{
    }
    \caption{Pre-trained embs}
    \label{tab:5_experiments/4_texter/2_static/3_pre_trained/grid_search}
\end{table}

As Tables~\ref{tab:5_experiments/4_texter/2_static/3_pre_trained/vector_sets} and~\ref{table:appendix/static_vectors_2} show, many of the assumptions were not met to the extent expected. Overall, it does not matter very much which embedding set is chosen. All three embedding types offer similar top performances with their respective best embedding sets, a multiplication of pre-training data is not necessarily better, as a look into Table~\ref{table:appendix/static_vectors_2} reveals, and even increasing the embedding size causes only a marginal improvement. Overall, pre-trained embeddings yield better results than randomly initialized ones, so that the last statement can also be formulated the other way around, namely that pre-trained embeddings already work well with small embedding sizes. Besides the small difference between the embedding sets, it is noticeable that the attentive Texter benefits particularly strongly from pre-trained embeddings, although the simple model still performs better on most text sets. On closer inspection, there is a tendency for the attentive model to work better with charachter n-grams, while the simple model might work slightly better with GloVe embeddings. However, the latter performance gain is so small that all further experiments were performed with character n-grams for the sake of clarity. Among the n-gram based Charagram and fastText embeddings, the slightly better fastText embeddings were chosen, and among these again the smaller fasttext.simple.300d embeddings set whose performance is barely distinguishable from fasttext.en.300d.


\subsubsection{Freezing pre-trained embeddings}
\label{subsubsec:5_experiments/4_texter/2_static/4_update_vectors}
\begin{table}[t!]
    \makebox[\textwidth][c]{
        \input{5_experiments/3_texter/2_static/4_update_vectors/grid_search}
    }
    \caption{Static Texters when (not) freezing the pre-trained embeddings. Numbers show F1 scores. Best result per row marked bold. Freezing pre-trained word embeddings leads to worse results in every case.}
    \label{tab:5_experiments/3_texter/2_static/4_update_vectors/grid_search}
\end{table}

When using pre-trained word embeddings, the embeddings adapt to the new training data during fine-tuning and may lose their special properties~\cite{He2019AnalyzingTF}. To prevent this, the new training data used for fine-tuning is sometimes mixed in with the training data used during pre-training. Another option is to freeze the pre-trained word embeddings during training so that other parameters must align themselves more strongly while the word embeddings remain constant. The latter approach was considered in an attempt to enhance the attentive Texter's class embeddings. The outcome of the experiment was uncertain. On the one hand, the model is deprived of a large part of its parameters, leaving only the class embeddings and linear layers to be learned, on the other hand, it could focus the training on the class embeddings and prevent overfitting. \autoref{tab:5_experiments/3_texter/2_static/4_update_vectors/grid_search} shows the results. In addition to the fasttext.simple.300d embeddings, charngram.300d, and glove.6B.300d were also examined, as it was conceivable that embedding freezing would have different effects depending on the embedding type.

As the numbers inevitably show, freezing the pre-trained embeddings results in worse performance for every dataset and every type of embedding which is why no further experiments make use of it. Nevertheless, it is interesting to observe how different the negative effects are: CharNGram shows immense performance losses while GloVe embeddings handle the restriction well on text sets with few IRT sentences per entity. Nonetheless, even if the negative effects cannot be compensated, the assumption that freezing the word embeddings has a positive effect on the attention mechanism seems to be valid, since the attentive Texter's performance drops less on text sets with multiple IRT sentences compared to the simple model.


\subsubsection{Pooling}
\label{subsubsec:5_experiments/4_texter/2_static/5_pooling}
from embedded tokens multiple ways to chose from
for classification usually take class embedding, as done by bert authors (?), captures whole sentence meaning
alternative idea is to average word embeddings instead
while on it, also tried average cls + word embs and average all embs including out of sentence, i.e. padding
table \ref{tab:5_experiments/4_texter/3_context/2_pooling/grid_search} shows results

\begin{table}[h]
    \centering
    \input{5_experiments/4_texter/3_context/2_pooling/grid_search}
    \caption{Pooling}
    \label{tab:5_experiments/4_texter/3_context/2_pooling/grid_search}
\end{table}

best is to average all embs followed by word embs followed by cls default
but difference not big, absolute ?\%


\subsubsection{Activation Function}
\label{subsubsec:5_experiments/4_texter/2_static/6_activation}
In case of the attentive Texter, the sentence embeddings generated by the embedding block are subsequently passed on to the attention block where they are scalar multiplied with the class embeddings to obtain the attention values that specify how well a class embedding matches each of the entity's sentences. For normalization purposes a classes' attentions are further pushed through a non-linear activation function before the resulting values serve as weight factors in calculating the class-specific entity embeddings returned from the attention block.

Initially, the softmax function was considered as the activation function, so that the attention mechanism is forced to compare all of an entities sentencies to each other. On the other hand, it is problematic that the attention weights sum to 1 even if none of the sentences, or all of them, fit the class. In the first case, the class embedding converges towards unrelated sentence embeddings during backpropagation and in the second case several insightful sentences are not fully utilized, because each one's gradient is lower than possible. Therefore, the sigmoid function was considered as an alternative, allowing each sentence embedding to be weighted by its attention value in $[0, 1]$ individually. While on it, the gererally popular ReLu function was also tested. Out of interest in the principle utility of a non-linear activation function, it was also measured what happens when no activation function is applied at all. Both the relu function and the omission of an activation function allow outliers with a large scalar product to have great influence on the learning process, which was expected to have a negative impact. In the comparison between Softmax and Sigmoid, the outcome was uncertain. \autoref{tab:5_experiments/3_texter/2_static/6_activation/grid_search} shows the results of varying the activation function for the attentive Texter only, as the simple model does not have an attention block.

\begin{table}[t]
    \centering
    \input{5_experiments/3_texter/2_static/6_activation/grid_search}
    \caption{Static Texters with various activation functions in the attention block. Numbers show F1 scores. Best result per row marked bold. The sigmoid function works best. Using no activation function works surprisingly well.}
    \label{tab:5_experiments/3_texter/2_static/6_activation/grid_search}
\end{table}

In line with expectations, the Sigmoid and Softmax functions produce the best results. For all text sets that provide multiple sentences per entity, however, Sigmoid is ahead by a few percentage points. On single-sentence text sets Sigmoid and Softmax perform similiarly. Interestingly, not using an activation function performs only slightly worse than Softmax on the CDE split and even similarly well on the FB split. On the text set, even the best results are achieved without an activation function. The otherwise promising ReLu function was inferior to the other functions in this experiment. For all further experiments, the sigmoid function was used as the standard.


\subsubsection{Appplying class weights}
\label{subsubsec:5_experiments/4_texter/2_static/7_weight_factor}
After the entity embeddings have been formed from the sentence embeddings in the attention block, they are passed through the classification block which outputs the final class logits of the forward() function. In case of the simple model, the attention block is omitted and the sentence embeddings are used directly for classification. In either case, the forward() function's class logits are taken as input by the loss function during training to calculate the model's loss in regard to the ground truth class labels. To make the logits comparable with the usual 0 and 1 ground truth labels, they are normalized to range $[0, 1]$ during the loss calculation - usually by applying the sigmoid function.

For the multilabel problem at hand, the binary cross-entropy (BCE) loss function is used. A convenient property of the BCE function is that it produces very large loss values for very wrong predictions which contributes to an initially fast learning process. To counteract the unbalanced classes in the Power dataset, the weighted binary cross entropy loss function (wBCE) shown in Equation~\ref{eq:5_experiments/4_texter/2_static/7_weight_factor/wbce} is used. It calculates the loss for the multi-label output logits $x$ and the respective ground truth labels $y$, both of which are c-dimensional vectors with $c$ being the number of output classes. Without the weights, the model would learn that it gets off best by always making negative predictions for very rare classes. This would be good for high accuracy, but is bad for the metrics used during evaluation, which do not measure true negatives. A class' weight $w_c$ is calculated as the reciprocal of the class' frequence it occurs in the training data with. A class that is true for every fifth entity, for example, would be assigned a class weight of five.

\begin{align}
    wBCE(x, y) = - \frac{1}{C} \sum_{c = 1}^C w_c \cdot log(\sigma(x)) + (1 - y) \cdot log(1 - \sigma(x))
    \label{eq:5_experiments/4_texter/2_static/7_weight_factor/wbce}
\end{align}

In an experiment, whose results fill Table~\ref{tab:5_experiments/4_texter/2_static/7_weight_factor/grid_search}, it should be verified that the use of class weights has a positive effect. Furthermore, it should be ensured that the choice to calculate the weights as the reciprocal of the class frequencies is optimal. Therefore, also smaller and larger weights were included in the comparison by halving and doubling the classes' reciprocals, respectively.

\begin{table}[t]
    \centering
    \input{5_experiments/4_texter/2_static/7_weight_factor/grid_search}
    \caption{Applying different class weights}
    \label{tab:5_experiments/4_texter/2_static/7_weight_factor/grid_search}
\end{table}

The empirical values are in line with the theoretical expectations. Although omitting class weights produces useful values - especially on the FB split where the 100 most frequent classes have higher frequencies than on the CDE split - applying class weights significantly improve performance. Thereby, it seems more important to apply weights at all than to fine-tune the weights. In most cases, it does not matter much whether the weights are halved or doubled. On average, the plain reciprocals seem to be the best choice, which is why they are set as the default.


\subsubsection{Choosing an optimizer}
\label{subsubsec:5_experiments/4_texter/2_static/8_optimizer}
% TODO moved to approach

\begin{table}[t]
    \centering
    \input{5_experiments/3_texter/2_static/8_optimizer/grid_search}
    \caption{Static Texter when applying different learning rates during training. Numbers show F1 scores. Best result per row marked bold. The simple Texter can be trained with high learning rates while the attentive Texter is more sensible, especially on datasets with many sentences per entity.}
    \label{tab:5_experiments/3_texter/2_static/8_optimizer/grid_search}
\end{table}

In this work, SGD with momentum~\cite{Qian1999OnTM} and Adam~\cite{Kingma2015AdamAM} are tried. SGD with Momentum serves as a representative of the classical, non-adaptive gradient descent methods for which are generally better suited to find the minimum of a loss function~\cite{Wilson2017TheMV}. The adaptive optimizer Adam, on the other hand, is particularly popular~\cite{AdamPopular} and is generally considered fast and good. In the experiment, SGD's momentum constant was set to 0.9.

Directly related to the optimizer is the learning rate. Both optimizers were tested with different learning rates from a range that should allow training within a reasonable time. It was expected that lower learning rates always lead to a better result given sufficient training time, but that the improvements become smaller and smaller as the learning rate decreases, so that a sufficiently good learning rate can be declared for each optimizer. SGD with momentum was expected perform slightly better after a longer training time. Depending on the ratio of additional training time to gained performance, a decision should be made between the optimizers.

\begin{figure}[t]
    \centering
    \input{5_experiments/3_texter/2_static/8_optimizer/sgd_vs_adam/sgd_vs_adam}
    \caption{Static Texter optimized via SGD and Adam with high to low learning rates (red = 1.0, orange, yellow, black, green, blue, purple = 0.001) on the fb-owe-1-clean text set. The Adam optimizer converges much faster, but does not work with high learning rates when training the attentive Texter.}
    \label{fig:5_experiments/3_texter/2_static/8_optimizer/sgd_vs_adam/sgd_vs_adam}
\end{figure}

Unfortunately, even at the highest learning rate, SGD converged too slowly to estimate definitive results. Therefore, \autoref{tab:5_experiments/3_texter/2_static/8_optimizer/grid_search} only shows the results for Adam. The four plots in \autoref{fig:5_experiments/3_texter/2_static/8_optimizer/sgd_vs_adam/sgd_vs_adam} illustrate the training of the simple and the attentive Texter using SGD and Adam with different learning rates. The upper plots show SGD's slowly converging curves while Adam's bottom plots paint a contrary picture. Even with the highest learning rate, SGD is slower than Adam with the lowest learning rate and would most likely need at least 200 more episodes to converge. An optimizer-independent observation is that the complex model appears to be more sensitive to learning rates that are too high. For SGD the highest learning rate leads to a slower training progress and for Adam the training does not work at all at the two highest learning rates. This information is also reflected in \autoref{tab:5_experiments/3_texter/2_static/8_optimizer/grid_search}, which also reveals another fact that is not directly evident from the plot: A lower learning rate does not automatically result in better performance. However, the nearest explanation is that training is simply not finished at that point. \autoref{tab:5_experiments/3_texter/2_static/8_optimizer/grid_search} suggests that the training of the attentive Texter is somewhat slower, but at a learning rate of 0.003, the training of both models appears to converge after the specified 200 epochs, which is why this learning rate is used in the other experiments.


\subsubsection{Inspecting the attention mechanism}
\label{subsubsec:5_experiments/4_texter/2_static/9_attention}
In the course of all the described experiments, the performance of both the simple and the complex Texter were steadily improved. What was missing was the hoped-for increase in performance of the attentive Texter compared to the simple version. Instead, the simple Texter actually performs better on many text sets. In order to get to the bottom of the cause, the attention mechanism should be investigated to find out whether the basic assumption that the class embeddings react to sentences in which class-typical keywords appear can be verified. In particular, it was examined (1) whether the trained class embeddings match keywords that are typical for the respective class, (2) whether the attention block favors sentences that would yield clear classification results on their own and (3) how the attention value related to a sentence results from the sentence's individual words.

Regarding the first question, the assumption was that the class embeddings converge against keywords that are particularly meaningful for a class during training. To confirm this conjecture, the class embeddings $class_c$, of a trained texter were scalar multiplied with all word embeddings $word_w$ of the vocabulary, the resulting scalar products $\langle class_c, word_w \rangle$ were sorted, and the values with the largest absolute values $|\langle class_c, word_w \rangle|$ were examined for plausibility. Thereby, the indices $1 <= c <= |C|$ and $1 <= w <= |V|$ specify one of the classes from the class set $C$ and a word from the vocabulary $V$, respectively. It was expected that among these top candidates mainly words would be found that clearly speak for or against the corresponding class. Table shows the top words the attentive Texter reacts to after being trained on the CDE split with cde-irt-5-clean texts.

\begin{table}[h]
    \centering
    \input{5_experiments/4_texter/2_static/9_attention/top_words}
    \caption{Top words}
    \label{tab:5_experiments/4_texter/2_static/9_attention/top_words}
\end{table}

Contrary to the assumption, the top words do not show a clear relation to the respective classes. With a few exceptions such as "Maryland" and "BBC" which appear further down in the suggestions for (from, USA) or (speaks, English), the words appear rather random. Given the rather infrequent occurrence of top words, overfitting seemed the most obvious explanation. However, the top words change with repeated training runs, so it seemed more likely that the word embeddings simply happened to be close to the learned class embeddings. As long as this only affects a small proportion of top words and the class embeddings are mostly close to relevant words, this should have little impact on the generalizability of the model. Whether this relation exists to more frequent keywords was investigated in the third experiment.

Before that, the second question was clarified whether the attention mechanism weights the sentences most heavily that are also the most meaningful on their own. For this purpose, the attentive Texter was extended by a method that bypasses the attention mechanism during inference. The trained model is still passed multiple sentences per entity when called, but instead of these being calculated entity embeddings in the attention block, the sentence embeddings are pushed directly through the classification block. That function, denoted as $\psi$ in the following, finally returns the non-aggregated class logits for each individual sentence. However, since the linear layers of the complex model were actually trained on the entity embeddings, this experiment only works with a texter in whose attention block the softmax function is used. This is because the Sigmoid version, unlike the Softmax version, does not have the property that the individual set embeddings have the same order of magnitude as the entity embeddings. Therefore, the results of the attentive model trained with Softmax on the cde-irt-5-clean texts are shown below. However, since the performance of the Softmax version is almost as high as that of the Sigmoid version, the results should be representative. Table~\ref{tab:5_experiments/4_texter/2_static/9_attention/kari_soft} presents exemplary values for the entity Kari Hotakainen, a Finnish writer. In addition to the five rather short sentences about Kari as well as the model predictions and the ground truth concerning the four most frequent classes in the CDE split, for each class-sentence combination it is stated to which proportion the sentence embedding normally enters into the entity embedding and which result the function $\psi$ yields when the attention mechanism is bypassed. Considering only the selected classes, the positive predictions for logits above the zero threshold in the example result in a precision of 33\%, a recall of 100\%, and thus an overall F1 score of 33\% and thereby close to the cde-irt-5-clean text set's average value of 37.6\%. However, it must be mentioned that the observable effectiveness of the attention mechanism is rather above average. The example was chosen mainly because of the short sentences.

\begin{table}[h]
    \centering
    \input{5_experiments/4_texter/2_static/9_attention/kari_soft}
    \caption{Kari}
    \label{tab:5_experiments/4_texter/2_static/9_attention/kari_soft}
\end{table}

For the exmample entity, the attentive Texter does indeed prefer those sentences that would give a clear result on their own in case of three of the four classes. In the fourth case, the model heavily incorporates a sentence that has a rather uncertain outcome. Despite that the respective prediction would have been correct on its own, the model should not have focused on it. Looking at other sentences that have been heavily weighted, further misjudgments become apparent, such as the cases where the texter relied on the fifth and second sentences to predict the classes (speaks, English) and (is, actor), respectively, which both lead to unreliable decisions on their own. Overall, however, it is noticeable that meaningful sentences are more strongly included in the entity embeddings.

However, the effective focus on sentences with a clear prediction alone is of no use if the unambiguous decision made on the basis of the individual sentences is wrong. For example, in Table~\ref{tab:5_experiments/4_texter/2_static/9_attention/kari_soft}, the first three sentences incorrectly indicate that Kari Hotakainen speaks English. Also, he is predicted to be an actor which is not the case which is probably due to the (is, actor) class's general high posterior probability as the dataset contains many actors. Otherwise, the sentences lead to largely correct classifications and was less certain about its two false positives than about its two correct predictions.

What is still missing, however, is a clear prioritization of the sentences. For example, as a human being, one would give the last sentence a much higher weighting when it comes to the negation of the class (from, USA) due to the fragment "by finnish author kari hotakainen". But, using static word embeddings, the model cannot recognize at this point that the word "finnish" is much more relevant than in the preceding sentence in which the same word does not refer to the entity itself. When the experiment is repeated, it is also noticeable that the class logits for the entire entity fluctuate only slightly, but change noticeably between individual sentences, which also changes the prioritization of the sentences.

This uncertainty about the order of the sentences also exists in the final model that uses the sigmoid function, although to a lesser extent. Table~\ref{tab:5_experiments/4_texter/2_static/9_attention/kari} shows, analogous to Table~\ref{tab:5_experiments/4_texter/2_static/9_attention/kari_soft}, the attention values after the sigmoid function has been applied instead of the softmax, which is why the probabilities no longer sum up to 100\% columnwise. Furthermore, Table~\ref{tab:5_experiments/4_texter/2_static/9_attention/kari_soft} misses the results of the $\psi$ function which cannot be calculated here as explained before. The shown figures imply that the sigmoid version achieves a higher overall precision due to the correct prediction of (is, actor) and that the inclusion of uncertain sentences is generally lower, but even here the order between the relevant sentences varies among training runs.

\begin{table}[h]
    \centering
    \input{5_experiments/4_texter/2_static/9_attention/kari}
    \caption{Kari}
    \label{tab:5_experiments/4_texter/2_static/9_attention/kari}
\end{table}

Another interesting insight is provided by the third experiment on the attention mechanism, which investigates how attention values derive from word embeddings. Prerequisite for the experiment is the usage of mean pooling, i.e. the calculation of the sentence embeddings as the mean of the respective word embeddings. In this case the scalar product of a class embedding $class_c$ and a sentence embedding $sent_s$ can be broken down to the mean of the scalar products between the class embedding and the sentence's word embeddings $word_w$ where $1 <= w <= N$ is the words position within the sentence:

\[
    \langle class_c, sent_s \rangle
    = \langle class_c, \frac{1}{N} \sum_{w=1}^N word_w \rangle
    = \frac{1}{N} \sum_{w=1}^N \langle class_c, word_w \rangle
\]

Leveraging this relationship, Figure~\ref{fig:5_experiments/4_texter/2_static/9_attention/kari_softmax} illustrates which words influence the overall scalar products between the class and sentence embeddings of the example entity Kari Hotakainen the most. Words that have a similar or opposite embedding to the class embedding are highlighted in yellow and purple, respectively. The expectation was that class-specific keywords would play a large role. During the experiment, it was noticed by coincidence that the result for Softmax- and Sigmoid-based attention mechanisms differ significantly. Therefore, each sentence is shown twice per class - once for Softmax and once for Sigmoid.

\begin{figure}
    \centering
    \includegraphics[width=\textwidth]{5_experiments/4_texter/2_static/9_attention/kari_softmax}
    \caption{Sent}
    \label{fig:5_experiments/4_texter/2_static/9_attention/kari_softmax}
\end{figure}

Although the use of the softmax or the sigmoid function in the attention block leads to very similar F1 scores the training of the word embeddings seems to differ significantly in both cases. When using sigmoid, the learned class embeddings attend on class-relevant keywords as expected whereas this is not the case with softmax. For example, the sigmoid version learns that the words "novel," "author," and "biography" are closely related to class (is, author) and that Finns tend to be likely to speak English. Conversely, "Finnish" and the name "Juha" militate against an origin from the US, and, apparently, authors are rarely actors at the same time. On the other hand, the softmax version attends to the names "Kimi" and "Kari" whose causal relationsships to (from, USA) and (is, actor) is less obvious.

In summary, the assumptions regarding the attention mechanism are only partially supported empirically and no reliable prioritization between an entity's sentences is established, which is probably the reason for the missing performance improvement of the attentive Texter compared to the attention-less version. One possible reason has already been mentioned: With static word embeddings it is not possible for the model to detect whether a potentially class-relevant keyword refers to the entity or not. This becomes even more obvious with a long sentence like the following of which there are many in the IRT text sentences:

\begin{displayquote}
    The Chilean writer Ricardo Cuadros said that McOndo irreverence for Latin American literary tradition, its thematic–stylistic concentration upon the pop culture of the United States, and the literatures’ apolitical tone, are dismissive of the literary ideas, writing style, and narrative techniques of the generation of Latin American writers (García Márquez, Vargas Llosa, Carpentier, Fuentes, et al.) who lived under, opposed, and (occasionally) were repressed by dictators.
\end{displayquote}

The sentece is associated with a Latin American writer who was classified as an American due to the keywords "United", "States" and the twice occurring "American" whose relation to "Latin" cannot be captured by static word embeddings. When trying to identify the entity in question, a second problem reveals itself in long sentences: Even as a human, it is not recognizable that, with all the named entities, the text is about the Peruvian writer Mario Vargas Llosa and not the Chilean Ricardo Cuadros. To address these problems, the next step is to use transformers that can recognize relationships between words and include special markers in the text sets that have not been used so far.



\subsection{Contextual Word Embeddings}
\label{subsec:5_experiments/4_texter/3_context}
Modern NLP models continue where classical models with static word embedding reach their limits when it comes to long sentences in which the relationship between the words is even more important for the words', and thus, the overall sentence's meaning. For this purpose, the particularly successful transformers use internal attention mechanisms that learn for each word which other words in the sentence are particularly relevant and should therefore be included in the word's embedding. This results in a context-dependent embedding for each occurrence of the word.

Besides embedding usual words and some standard tokens that represent unknown words and paddings, the here used DistilBERT also supports the special [CLS] and [SEP] tokens introduced by the BERT model. The [CLS] token's purpose is to capture the meaning of the sentence as a whole during training. In addition, the Texter supports the definition of custom tokens, such as the ones used for markings or maskings in the IRT text sets. This additional information should lead to a significant performance increase, even when the entity mention is masked, as it preserves the information what the surrounding sentence is about.

For the Texter's downstream architecture, the change from a lookup table for embeddings to the use of DistilBERT does not mean a big change as it only affects the embedding block -- the classification block and, in the case of the attentive version, the attention block remain untouched. In particular, design choices from the experiments with static word embeddings, such as the usage of the sigmoid function in the attention block, are kept. During training, however, the optimizer has to be adjusted to accommodate the deep transformers within the embedding block.

\begin{table}
    \makebox[\textwidth][c]{
        \begin{tabular}{| l | r | r | r | r | r | r | r | r | r |}
    \hline
    
    \multicolumn{1}{|c|}{\textbf{Split}} &
    \multicolumn{1}{|c|}{\textbf{mAP}} &
    \multicolumn{4}{|c|}{\textbf{Macro over ents}} &
    \multicolumn{4}{|c|}{\textbf{Micro over facts}} \\
    
    \multicolumn{1}{|c|}{} &
    \multicolumn{1}{|c|}{} &
    \multicolumn{1}{|c|}{\textbf{Prec}} &
    \multicolumn{1}{|c|}{\textbf{Rec}} &
    \multicolumn{1}{|c|}{\textbf{F1}} &
    \multicolumn{1}{|c|}{\textbf{Supp}} &
    \multicolumn{1}{|c|}{\textbf{Prec}} &
    \multicolumn{1}{|c|}{\textbf{Rec}} &
    \multicolumn{1}{|c|}{\textbf{F1}} &
    \multicolumn{1}{|c|}{\textbf{Supp}} \\
    
    \hline \hline
    
    CDE-0 & 0.84 &
    100.00 & 0.84 & 0.84 & \num{13.32} &
    100.00 & 0.00 & 0.00 & \num{25255} \\
    
    CDE-25 & 22.06 &
    66.90 & 24.27 & 32.02 & \num{13.32} &
    62.67 & 22.93 & 33.57 & \num{25255} \\
    
    CDE-50 & 29.26 &
    61.41 & 33.14 & 40.29 & \num{13.32} &
    58.47 & 31.52 & 40.96 & \num{25255} \\
    
    CDE-75 & 33.23 &
    57.88 & 38.10 & 43.59 & \num{13.32} &
    55.43 & 3638 & 43.93 & \num{25255} \\
    
    CDE-100 & 35.77 &
    55.62 & 41.48 & 45.28 & \num{13.32} &
    53.49 & 39.72 & 45.59 & \num{25255} \\
    
    \hline
    
    FB-0 & 3.19 &
    100.00 & 03.19 & 3.19 & \num{18.76} &
    100.00 & 0.00 & 0.00 & \num{15312} \\
    
    FB-25 & 27.41 &
    73.46 & 30.56 & 37.06 & \num{18.76} &
    69.28 & 30.15 & 42.02 & \num{15312} \\
    
    FB-50 & 33.39 &
    68.36 & 37.61 & 42.90 & \num{18.76} &
    64.90 & 36.76 & 46.94 & \num{15312} \\
    
    FB-75 & 36.22 &
    64.93 & 41.37 & 45.14 & \num{18.76} &
    62.76 & 40.82 & 49.47 & \num{15312} \\
    
    FB-100 & 38.43 &
    63.08 & 44.34 & 46.97 & \num{18.76} &
    60.98 & 43.51 & 50.79 & \num{15312} \\
    
    \hline
\end{tabular}

    }
    \caption{Final evaluation of the contextual Texter on all text sets. Results of the static Texter are given for comparison. The contextual Texter is evaluated against the Texter dataset's test subset (F1) and against all facts from the respective split (F1 all, mAP all). The contextual Texter outperforms the simple Texter in general, especially when leveraging markings in the text. The attentive Texter profits more from contextual word embeddings. Still, the simple Texter performs better in terms of mAP.}
    \label{tab:5_experiments/3_texter/3_context/results}
\end{table}

After the switch to DistilBERT, training of the Texter takes longer due to the large number of parameters in the transformer, but can be terminated after 50 epochs. \autoref{tab:5_experiments/3_texter/3_context/results} shows the evaluation results for the final Texter after that time. In addition to the previously regarded macro F1 score over all classes, \autoref{tab:5_experiments/3_texter/3_context/results} also provides F1 and mAP over the entities from the Power split -- including those the Texter cannot predict. In addition to the contextual Texters' evaluation results, the static Texters' final results from \autoref{subsec:5_experiments/3_texter/2_static} are given for comparison where available. Furthermore, \autoref{ch:a_appendix} contains the detailed \autoref{tab:a_appendix/context_final_prec_rec} that provides the precision and recall values.

The final evaluation results reveal some interesting facts: First, the contextual Texter performs better on clean text sets than the static Texter most of the time, but not always. While the contextual Texter performs better on almost all text sets containing IRT sentences, it does not for the CDE and OWE sentences. Second, the CDE and OWE text sets show best, that the attentive Texter benefits more from contextual word embeddings than the simple model. Third, it is obvious that markings bring great improvements, as was expected, while the masked text sets are in between the clean and marked ones in terms of F1. In terms of mAP, however, the masked text sets lead to better predictions than their marked counterparts in multiple cases. Fourth, the evaluating against all facts from the Power split leads to significantly worse results on the FB split than it does on the CDE split, which is probably due to the reason that the FB classes cover a smaller portion of the larger FB split.

Overall, it can be stated, that the Texter performs better when using contextual word embeddings, especially when marked texts are available. Therefore, the contextual Texter is the default in the Power model. However, depending on the given text set as well as hardware support, static word embeddings might be a noteworthy alternative. When comparing the simple and attentive versions of the contextual Texter, the simple version yields slightly better results. Still, the attentive Texter is kept due to its ability to explain its text-based decision to a certain point.

In the following, subsections~\ref{subsubsec:5_experiments/3_texter/3_context/1_sent_len}~--~\ref{subsubsec:5_experiments/3_texter/3_context/3_optimizer} will look at some experiments on the updated embedding block and the adjusted optimizer used to train the deep contextual Texter.

\subsubsection{Varying sentence length}
\label{subsubsec:5_experiments/3_texter/3_context/1_sent_len}
bert takes batch of fixed length sents, longer sents are cut, shorter sents are padded
still, last tok is always [SEP], despite cut. also first is [CLS]
mask specifies which tokens are words and which are padding
example: ["short sent", "long sent"] become ...
depending on max sent len, more or less tokens are lost
tried several sent lengths as shown in table \ref{tab:5_experiments/4_texter/3_context/1_sent_len/grid_search}

\begin{table}[h]
    \centering
    \input{5_experiments/4_texter/3_context/1_sent_len/grid_search}
    \caption{Sent len}
    \label{tab:5_experiments/4_texter/3_context/1_sent_len/grid_search}
\end{table}

result is that performance increases strongly first until certain point where does not really matter anymore
decided to go with 64


\subsubsection{Pooling}
\label{subsubsec:5_experiments/3_texter/3_context/2_pooling}
from embedded tokens multiple ways to chose from
for classification usually take class embedding, as done by bert authors (?), captures whole sentence meaning
alternative idea is to average word embeddings instead
while on it, also tried average cls + word embs and average all embs including out of sentence, i.e. padding
table \ref{tab:5_experiments/4_texter/3_context/2_pooling/grid_search} shows results

\begin{table}[h]
    \centering
    \input{5_experiments/4_texter/3_context/2_pooling/grid_search}
    \caption{Pooling}
    \label{tab:5_experiments/4_texter/3_context/2_pooling/grid_search}
\end{table}

best is to average all embs followed by word embs followed by cls default
but difference not big, absolute ?\%


\subsubsection{Optimizer}
\label{subsubsec:5_experiments/3_texter/3_context/3_optimizer}
% TODO moved to approach

\begin{table}[t]
    \centering
    \input{5_experiments/3_texter/2_static/8_optimizer/grid_search}
    \caption{Static Texter when applying different learning rates during training. Numbers show F1 scores. Best result per row marked bold. The simple Texter can be trained with high learning rates while the attentive Texter is more sensible, especially on datasets with many sentences per entity.}
    \label{tab:5_experiments/3_texter/2_static/8_optimizer/grid_search}
\end{table}

In this work, SGD with momentum~\cite{Qian1999OnTM} and Adam~\cite{Kingma2015AdamAM} are tried. SGD with Momentum serves as a representative of the classical, non-adaptive gradient descent methods for which are generally better suited to find the minimum of a loss function~\cite{Wilson2017TheMV}. The adaptive optimizer Adam, on the other hand, is particularly popular~\cite{AdamPopular} and is generally considered fast and good. In the experiment, SGD's momentum constant was set to 0.9.

Directly related to the optimizer is the learning rate. Both optimizers were tested with different learning rates from a range that should allow training within a reasonable time. It was expected that lower learning rates always lead to a better result given sufficient training time, but that the improvements become smaller and smaller as the learning rate decreases, so that a sufficiently good learning rate can be declared for each optimizer. SGD with momentum was expected perform slightly better after a longer training time. Depending on the ratio of additional training time to gained performance, a decision should be made between the optimizers.

\begin{figure}[t]
    \centering
    \input{5_experiments/3_texter/2_static/8_optimizer/sgd_vs_adam/sgd_vs_adam}
    \caption{Static Texter optimized via SGD and Adam with high to low learning rates (red = 1.0, orange, yellow, black, green, blue, purple = 0.001) on the fb-owe-1-clean text set. The Adam optimizer converges much faster, but does not work with high learning rates when training the attentive Texter.}
    \label{fig:5_experiments/3_texter/2_static/8_optimizer/sgd_vs_adam/sgd_vs_adam}
\end{figure}

Unfortunately, even at the highest learning rate, SGD converged too slowly to estimate definitive results. Therefore, \autoref{tab:5_experiments/3_texter/2_static/8_optimizer/grid_search} only shows the results for Adam. The four plots in \autoref{fig:5_experiments/3_texter/2_static/8_optimizer/sgd_vs_adam/sgd_vs_adam} illustrate the training of the simple and the attentive Texter using SGD and Adam with different learning rates. The upper plots show SGD's slowly converging curves while Adam's bottom plots paint a contrary picture. Even with the highest learning rate, SGD is slower than Adam with the lowest learning rate and would most likely need at least 200 more episodes to converge. An optimizer-independent observation is that the complex model appears to be more sensitive to learning rates that are too high. For SGD the highest learning rate leads to a slower training progress and for Adam the training does not work at all at the two highest learning rates. This information is also reflected in \autoref{tab:5_experiments/3_texter/2_static/8_optimizer/grid_search}, which also reveals another fact that is not directly evident from the plot: A lower learning rate does not automatically result in better performance. However, the nearest explanation is that training is simply not finished at that point. \autoref{tab:5_experiments/3_texter/2_static/8_optimizer/grid_search} suggests that the training of the attentive Texter is somewhat slower, but at a learning rate of 0.003, the training of both models appears to converge after the specified 200 epochs, which is why this learning rate is used in the other experiments.





\section{Ruler}
\label{sec:4_approach/2_ruler}
While the Texter processes the text information attached to the knowledge graph's entities, the Ruler exploits patterns in the graph structure to predict missing facts. Given an entity $x$ with a set of known facts $K$ containing facts of the form $(x, rel_k, tail_k)$ with $1 <= k <= |K|$, $rel_k \in R$, and $tail_k \in E$ being any relation or entity, respectively, the Ruler leverages entity-related rules of the form $(x, rel_m, tail_m) <= (x, rel_k, tail_k)$ with $1 <= m <= |M|$ to predict a set of missing facts $M$. Therefore, the rules required for the inference process have to be mined from the knowledge graph, beforehand. That rule mining process can be viewed as the equivalent to the Texter's training process and some paragraphs in this thesis will refer to rule mining as ``training'' a Ruler. Compatible to the Texter, the Ruler is limited to the prediction of facts $(x, rel, tail)$ that contain the query entity as their head, as well. However, when the Ruler is applied to all entities $e \in E$, all facts of the form $(e, rel, x)$ are predicted at some point as far as the mined rules support it.

For rule mining, AnyBURL, the bottom-up rule mining algorithm, by Christian Meilicke et al.~\cite{Meilicke2019AnytimeBR} is used. It is an anytime algorithm, meaning that it can be interrupted anytime and still yield valid results. Bottom-up rule mining refers to the fact that the algorithm starts with concrete paths in the graph and tries to generalize those paths to rules instead of coming up with rules initially and searching for evidence later, which would be a top-down approach. Out of all possible Horn rules that might describe patterns in the graph, AnyBURL is restricted to rules that can be generalized from so-called \emph{ground path rules}. A ground path rule does not contain variables, but only constants, and must not contain any cycles in its body. \autoref{eq:4_approach/2_ruler/ground_path_rule} describes the general form of a ground path rule of length $n$, meaning that it consists out of the head fact and $n$ body facts.

\begin{align}
(c_0, h, c_1)
    \Leftarrow (c_1, b_1, c_2), \dots, (c_n, b_n, c_{n+1}) &&
    c_k \neq c_l \forall k, l \in \{1, \dots, n+1\}, k \ne l
    \label{eq:4_approach/2_ruler/ground_path_rule}
\end{align}

Notably, despite the rule body being free of cycles, the ground path rule as a whole can still be cyclic if $c_0 = c_{n+1}$. Ground path rules are derived directly from randomly sampled paths in the graph and are subsequently generalized to rules that replace some of the constants with variables. If further supporting paths can be found for a general rule, it is kept. In their paper on AnyBURL, Meilicke et al. show that any rule that can be generalized from a ground path rules can be generalized to one of the three rule types formulated in Equations~\ref{eq:4_approach/2_ruler/c}~--~\ref{eq:4_approach/2_ruler/ac2}. Thereby, $C$-type rules can only be generalized from cyclic ground path rules, $AC2$ rules can only be generalized from acyclic ground path rules and $AC1$ can be generalized from both, cyclic and acyclic ground path rules. The following paragraphs outline the core algorithm used to mine such rules and derive some example rules from the small graph introduced in \autoref{ch:1_introduction}. \autoref{fig:4_approach/2_ruler/rule_graph} shows an annotated subset of the graph that illustrates the rules.

\begin{align}
    C   && (Y, h, X)   &\Leftarrow (X, b_1, A_2), \dots, (A_n, b_n, Y)
    \label{eq:4_approach/2_ruler/c} \\
    AC1 && (c_0, h, X) &\Leftarrow (X, b_1, A_2), \dots, (A_n, b_n, c_{n+1})
    \label{eq:4_approach/2_ruler/ac1} \\
    AC2 && (c_0, h, X) &\Leftarrow (X, b_1, A_2), \dots, (A_n, b_n, A_{n+1})
    \label{eq:4_approach/2_ruler/ac2}
\end{align}

\begin{figure}[t]
    \centering
    \includegraphics{4_approach/2_ruler/rule_graph}
    \caption{Subset of the previously introduced example graph with highlighted facts that form an acyclic (red + green) and a cyclic (red + blue) path.  ``Amsterdam'' and ``Netherlands'' have been abbreviated to ``AMS'' and ``NL''.}
    \label{fig:4_approach/2_ruler/rule_graph}
\end{figure}

Essentially, AnyBURL repeatedly samples random paths from the graph, generalizes them to all possible rule types, looks for further paths that match the gained rules and keeps those rules it finds further evidence for. For example, in search of rules of length two, i.e. rules that have a body consisting of two facts, AnyBURL might randomly sample the two paths in~\ref{eq:4_approach/2_ruler/path_1} and~\ref{eq:4_approach/2_ruler/path_2} from the graph. Note, that parentheses denote entities, brackets denote relations and that the path does not need to follow directed edges in the graph. Furthermore, a close look at the paths reveals that the second path is cyclic as it starts and ends at the entity ``Dutch''.

\begin{align}
(Dutch)
    \leftarrow [speaks] - (Ed) - [married~to] \rightarrow (Lisa) - [born~in] \rightarrow (AMS)
    \label{eq:4_approach/2_ruler/path_1} \\
    (Dutch) \leftarrow [speaks] - (Ed) - [lives~in] \rightarrow (NL) - [has~lang] \rightarrow (Dutch)
    \label{eq:4_approach/2_ruler/path_2}
\end{align}

From those paths, AnyBURL would then derive the constant-only ground path rules~\ref{eq:4_approach/2_ruler/acyclic_ground_path} and~\ref{eq:4_approach/2_ruler/cyclic_ground_path} by taking the path's first part as the rule's head and the remaining parts to form the rule's body.

\begin{align}
(Ed, speaks, Dutch)
    &\Leftarrow (Ed, married~to, Lisa), (Lisa, born~in, AMS)
    \label{eq:4_approach/2_ruler/acyclic_ground_path} \\
    (Ed, speaks, Dutch) &\Leftarrow (Ed, lives~in, NL), (NL, has~lang, Dutch)
    \label{eq:4_approach/2_ruler/cyclic_ground_path}
\end{align}

The acyclic ground path rule in~\ref{eq:4_approach/2_ruler/acyclic_ground_path} can be generalized to the $AC1$ rule~\ref{eq:4_approach/2_ruler/acyclic_ac1} and the $AC2$ rule~\ref{eq:4_approach/2_ruler/acyclic_ac2} while the cyclic ground path rule~\ref{eq:4_approach/2_ruler/cyclic_ground_path} can be generalized to the $C$ rule~\ref{eq:4_approach/2_ruler/cyclic_c} and the $AC1$ rule~\ref{eq:4_approach/2_ruler/cyclic_ac1}.

\begin{align}
    AC1 && (X, speaks, Dutch) &\Leftarrow (X, married~to, A_2), (A_2, born~in, AMS)
    \label{eq:4_approach/2_ruler/acyclic_ac1} \\
    AC2 && (X, speaks, Dutch) &\Leftarrow (X, married~to, A_2), (A_2, born~in, A_3)
    \label{eq:4_approach/2_ruler/acyclic_ac2} \\
        C   && (X, speaks, Y) &\Leftarrow (X, lives~in, A_2), (A_2, has~lang, Y)
    \label{eq:4_approach/2_ruler/cyclic_c} \\
    AC1 && (X, speaks, Dutch) &\Leftarrow (X, lives~in, A_2), (A_2, has~lang, Dutch)
    \label{eq:4_approach/2_ruler/cyclic_ac1}
\end{align}

Next, every rule candidate is scored by looking for further paths that match the rule's body and checking whether the graph also contains the fact predicted by the rule, i.e. whether the rule is correct in that case. Thereby, the number of paths that match the rule body is called the rule's \emph{support} while the ratio of times the rule is correct over its total support is called \emph{confidence}. Taking the cyclic rule~\ref{eq:4_approach/2_ruler/cyclic_c} as an example, AnyBURL would search for further evidence and find the path $(Lisa) - [lives~in] \rightarrow (NL) - [has~lang] \rightarrow (Dutch)$ that matches the rule body, increasing the rule's support to two, so far. However, the example graph does not contain the rule's predicted fact $(Lisa, speaks, Dutch)$, so the rule's support drops from 1 to $\frac{1}{2}$. Since rules that only apply to a single case or only once in every thousandth case are not very useful, AnyBURL drops rules with a support of 1 or confidence below 0.0001 by default~\cite{AnyBURL}. It is noteworthy that, although some rules are more general than others, such as~\ref{eq:4_approach/2_ruler/acyclic_ac2} compared to~\ref{eq:4_approach/2_ruler/acyclic_ac1}, the more specific ones are still kept as they might end up with higher confidence for their special case during the ongoing mining process.

The process described by the above example is repeated until only a few new rules of the same length $n$ can be found. AnyBURL then continues its search for rules of length $n + 1$ until it terminates after a fixed number of time steps. \autoref{code:anyburl} shows the slightly adjusted pseudocode from the AnyBURL paper. The sampling and scoring process discussed above is implemented as the body of the inner while loop. The outer for loop implements the repeated check for the saturation of rules of the current length and the eventual proceeding to rules of increased length.

\begin{listing}[t]
    \begin{lstlisting}
        AnyBURL(G, sat, Q, i, ts):
            n = 2
            R = $\emptyset$
            for i times:
                $R_s = \emptyset$
                start = current_time()
                while current_time() < start + ts:
                    p = sample_path(G, n)
                    $R_p$ = generate_rules(p)
                    for $r$ in $R_p$:
                        score($r$)
                        if $Q$($r)$:
                            $R_s$ = $R_s \cup {r}$

                $R_s^{'}$ = $R_s \cap R$
                if $|R_s^{'}| / |R_s| > sat$:
                    n = n + 1
                $R$ = $R \cup R_s$

            return $R$
    \end{lstlisting}
    \caption{The AnyBURL rule mining algorithm takes a graph $G$, a saturation level $sat$, a quality criterion $Q$, and a number of iterations $i$, each of a timespan $ts$, as input and produces a ruleset $R$.}
    \label{code:anyburl}
\end{listing}

A walk through the pseudocode reads as follows: Given the Graph $G$, the saturation threshold $s$, the quality criterion $Q$, a number of iterations $i$ and the timespan $ts$ each iteration endures, AnyBURL starts with an empty ruleset $R$, that will be extended after each iteration and returned in the end. The initial length of the randomly sampled paths is $n=2$, allowing to find the shortest possible rules of length 1. During the first iteration of duration $ts$, AnyBURL fills the ruleset $R_s$, which keeps the rules found in the current iteration, by repeatedly sampling paths, generating rules from the paths, scoring the resulting rules, and keeping those with sufficient support and confidence. At the end of the iteration, when the timespan $ts$ has passed, $R_s^{'}$ is calculated as the set of rules mined during the iteration that were already known. If the share of already known rules mined during the current iteration exceeds the saturation threshold, the algorithm starts searching for rules of increased length. Otherwise, it continues with the current length. In both cases, the iteration's rules are added to the overall ruleset $R$. If the specified number of total iterations is reached, AnyBURL terminates and returns the mined rules $R$. In practice, AnyBURL saves the mined rules in a text file at the end and at configurable points during mining.

With the stored rules from AnyBURL in place, the Ruler is prepared for inference. Conceptually, given an entity and its known facts, the Ruler loads the rules, filters out further rules that do not meet the Ruler's quality demands, and applies the remaining, useful rules to all known facts. All rules that can be applied successfully are kept together with their confidence. From all the facts predicted by the applied rules, already known facts from the existing graph are filtered out. The remaining facts are sorted by confidence and returned to the user -- together with the rules that predicted them as an explanation for the user. If multiple rules predict the same fact, the fact is assigned the highest confidence of those rules and is returned together with all of them. The Ruler's extra quality criterion mentioned above further restricts the considered rules to those with confidence greater 50\%, because AnyBURL's minimum confidence threshold of 0.0001 allows many rules that predict false positives. For open-world entities, this algorithm implies an empty result set as no rule can cover an entity that is not connected to any other entity and all the facts predicted for the train entities will be filtered out. In those cases, the Power model has to rely solely on the Texter.



\section{Aggregator}
\label{sec:4_approach/3_aggregator}
The aggregator has the task of merging the predicted facts from Ruler and Texter. As envisioned in \autoref{sec:4_approach/3_aggregator} and illustrated in \autoref{fig:4_approach/3_aggregator/lucy}, it was hoped that merging the facts leads to higher average precision because facts predicted by both components are likely to be correct and should be ranked higher. In addition, the Aggregator should be able to estimate how reliable the predictions of Ruler and Texter are in relation to each other, which is implemented in the form of the weight parameter $\alpha$ as described in \autoref{eq:4_approach/3_aggregator/conf_aggregator}.

\autoref{tab:5_experiments/5_aggregator/results} shows the final evaluation results for the Aggregator, and thus the final evaluation results for the Power model, for a number of graph-text combinations. As fact splits, the splits with 50\% known test facts were chosen, as for the final Ruler evaluation in \autoref{sec:5_experiments/4_ruler}. The respective results for the CDE-50 and FB-50 splits from \autoref{tab:5_experiments/4_ruler/results} were taken over into \autoref{tab:5_experiments/5_aggregator/results} for easier comparability. Similarly, the chosen text sets are the ones from the final Texter evaluation in \autoref{subsec:5_experiments/3_texter/3_context}. Again, \autoref{tab:5_experiments/5_aggregator/results} duplicates the respective results from \autoref{tab:5_experiments/3_texter/3_context/results} for ease of comparison. The last two columns then contain the new Aggregator measurements for the combination of the corresponding Ruler and Texter.

\begin{table}[t]
    \makebox[\textwidth][c]{
        \begin{tabular}{| l | r | r | r | r | r | r | r | r | r |}
    \hline
    
    \multicolumn{1}{|c|}{\textbf{Split}} &
    \multicolumn{1}{|c|}{\textbf{mAP}} &
    \multicolumn{4}{|c|}{\textbf{Macro over ents}} &
    \multicolumn{4}{|c|}{\textbf{Micro over facts}} \\
    
    \multicolumn{1}{|c|}{} &
    \multicolumn{1}{|c|}{} &
    \multicolumn{1}{|c|}{\textbf{Prec}} &
    \multicolumn{1}{|c|}{\textbf{Rec}} &
    \multicolumn{1}{|c|}{\textbf{F1}} &
    \multicolumn{1}{|c|}{\textbf{Supp}} &
    \multicolumn{1}{|c|}{\textbf{Prec}} &
    \multicolumn{1}{|c|}{\textbf{Rec}} &
    \multicolumn{1}{|c|}{\textbf{F1}} &
    \multicolumn{1}{|c|}{\textbf{Supp}} \\
    
    \hline \hline
    
    CDE-0 & 0.84 &
    100.00 & 0.84 & 0.84 & \num{13.32} &
    100.00 & 0.00 & 0.00 & \num{25255} \\
    
    CDE-25 & 22.06 &
    66.90 & 24.27 & 32.02 & \num{13.32} &
    62.67 & 22.93 & 33.57 & \num{25255} \\
    
    CDE-50 & 29.26 &
    61.41 & 33.14 & 40.29 & \num{13.32} &
    58.47 & 31.52 & 40.96 & \num{25255} \\
    
    CDE-75 & 33.23 &
    57.88 & 38.10 & 43.59 & \num{13.32} &
    55.43 & 3638 & 43.93 & \num{25255} \\
    
    CDE-100 & 35.77 &
    55.62 & 41.48 & 45.28 & \num{13.32} &
    53.49 & 39.72 & 45.59 & \num{25255} \\
    
    \hline
    
    FB-0 & 3.19 &
    100.00 & 03.19 & 3.19 & \num{18.76} &
    100.00 & 0.00 & 0.00 & \num{15312} \\
    
    FB-25 & 27.41 &
    73.46 & 30.56 & 37.06 & \num{18.76} &
    69.28 & 30.15 & 42.02 & \num{15312} \\
    
    FB-50 & 33.39 &
    68.36 & 37.61 & 42.90 & \num{18.76} &
    64.90 & 36.76 & 46.94 & \num{15312} \\
    
    FB-75 & 36.22 &
    64.93 & 41.37 & 45.14 & \num{18.76} &
    62.76 & 40.82 & 49.47 & \num{15312} \\
    
    FB-100 & 38.43 &
    63.08 & 44.34 & 46.97 & \num{18.76} &
    60.98 & 43.51 & 50.79 & \num{15312} \\
    
    \hline
\end{tabular}

    }
    \caption{Final Aggregator results, i.e. final results for the Power model. The results of the Ruler and Texter, whose predictions the Aggregator combines, are also shown for comparison. Although the Aggregator does not outperform its respective Ruler and Texter in terms of F1 score, it does for mAP.}
    \label{tab:5_experiments/5_aggregator/results}
\end{table}

As the mAP values show, the Aggregator performs several percentage points better than the Ruler and Texter on their own, with the improvement on the CDE split being more obvious. However, the relatively small increase on the FB split suggests that the true positives of Ruler and Texter almost coincide there. For the CDE split, on the other hand, manually peeking into the predictions reveals that the improved mAP mainly results from complementary true positives -- and not so much from improved ranks of joint predictions. Looking at the values of simple and attentive Texter, it is also noticeable that the lead of the simple Texter over the attentive Texter shrinks when adding the Ruler. Likewise, the lead of the text sets with many sentences and with high-quality sentences shrinks. Finally, the different aptitudes for Ruler and Overall, the Aggregator results are even similar between the two splits, while previously, models performed significantly better on the FB split.

Two experiments that will be mentioned only briefly here, because of their unspectacular results, concerning the calculation of the Aggregator's confidence as per \autoref{eq:4_approach/3_aggregator/conf_aggregator}: First, in the beginning, experiments were conducted on the computation of the combined confidence $conf_{Aggregator}$ in cases where facts are predicted by Ruler and Texter. As combining methods, calculating the maximum and the mean of $conf_{Ruler}$ and $conf_{Texter}$ were evaluated, but it soon became apparent that summing them up much better accommodates the fact that a fact predicted by Ruler and Texter deserves very high confidence. Second, experiments showed that taking into account the weight parameter $\alpha$ between Ruler and Texter yields only marginal performance improvements in the tenths of a percent range because the confidence values of Ruler and Texter seem to be very comparable after all and thus always yield $\alpha$ values close to 0.5. In detail, Ruler and Texer were both a bit too optimistic about their predictions in the experiments -- but they were equally overconfident.

