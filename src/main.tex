\documentclass[11pt,a4paper]{report}

\include{preamble}

\usepackage{amsmath}
\usepackage{csquotes}
\usepackage{hyperref}
\usepackage{makecell}
\usepackage{multirow}
\usepackage{pgfplots}
\usepackage{siunitx}
\usepackage{tabularx}
\usepackage{xcolor}

\pgfplotscreateplotcyclelist{tb}{
    thick,tb_color_1 \\
    thick,tb_color_2 \\
    thick,tb_color_3 \\
    thick,tb_color_4 \\
    thick,tb_color_5 \\
    thick,tb_color_6 \\
    thick,tb_color_7 \\
}

\definecolor{tb_color_1}{HTML}{b71c1c}
\definecolor{tb_color_2}{HTML}{ff6f00}
\definecolor{tb_color_3}{HTML}{ffeb3b}
\definecolor{tb_color_4}{HTML}{212121}
\definecolor{tb_color_5}{HTML}{4caf50}
\definecolor{tb_color_6}{HTML}{2196f3}
\definecolor{tb_color_7}{HTML}{9c27b0}

\newcommand{\welchethesis}{Master}
\newcommand{\thesisofwas}{of Science}
\newcommand{\titel}{An Open-World Extension to Rule-Based Knowledge Graph Completion}
\newcommand{\autor}{Tobias Uhmann}
\newcommand{\datum}{17. Mai 2021}
\newcommand{\ort}{Wiesbaden}
\newcommand{\referent}{Prof.\ Dr.\ Adrian Ulges}
\newcommand{\korreferent}{Dr.\ Jörn Hees}

\begin{document}

    \include{vorspann}

    \begin{abstract}

        Knowledge graphs store facts about objects in a graph structure. The graph's nodes represent entities while its edges represent relationships between the objects. Typical application scenarios include voice assistants and recommender systems. However, large knowledge graphs are usually incomplete and need to be updated continuously. AI approaches can support this task by suggesting probable facts that can be inferred from the existing graph.

        Most recent knowledge graph completion models approached the problem successfully by embedding the graph in a low-dimensional vector space. Further extensions to those embedding-based approaches make use of textual information to enable predictions about open-world entities for whom no facts are known in the existing graph. Alternative to embedding-based approaches, symbolic, rule-based systems capture patterns in the graph structure in form of logical rules. Recent works have shown that rule-based models can perform as well as embedding-based approaches and have the advantage that the model's rule-based decisions are comprehensible by humans.

        In this work, the Power model, an ensemble model consisting of a rule-based and a text-based KGC model is introduced. The rule-based component employs rules mined by the rule miner AnyBURL to make predictions based on the graph's structure, while the text-based component leverages additional text information about the entities to improve performance and enable prediction for open-world entities, which cannot be handled by the rule-based system. Furthermore, Power explains its predictions by giving the rules and text prioritization that led to its decision. The text prioritization is achieved by using an attention mechanism that focuses on those texts that are most relevant. The texts are encoded using contextual word embeddings produced by a variant of the state-of-the-art BERT transformer, which is shown to improve performance compared to the usage of conventional, static word embeddings. Finally, it is shown that the ensemble model produces better results than the rule- and text-based models on their own.

    \end{abstract}

    \setcounter{tocdepth}{4}
    \tableofcontents
    \newpage


    \chapter{Introduction}
    \label{ch:1_introduction}
    \emph{Knowledge graphs} (KGs) power search engines, question-answering systems and social networks. They are especially useful to manage dynamic, diverse data and have drawn significant attention from industry and academia in recent years~\cite{}. Today, many big tech companies, such as Amazaon~\cite{AmazonKG}, Facebook~\cite{}, Google~\cite{} and Microsoft~\cite{MicrosoftKG} run their own enterprise knowledge graphs. But, although the term "Knowledge Graph" became popular only in 2012 when Google announced its equally named system~\cite{}, the concept behind it is much older and even the term itself has been used back in the 70s~\cite{}.

Since then many definitions for knowledge graphs have been given, often contradicting each other, but in general they all describe knowledge graphs as directed graphs whose nodes and edges represent objects and their relationships to each other. In this work, a knowledge graph $G$ is formally defined as a set of known facts, which are a subset of all facts that could theoretically be constructed between the set of entities $E$ and the set of relations $R$:

\[
    G \subseteq E \times R \times E
\]

Due to their mathematical structure, facts are also referred to as \emph{triples}. A fact $(head, rel, tail)$ consists of a \emph{head} entity, a \emph{tail} entity and a relation between the two. Relations are usually not symetrical and therefore represented by directed edges. Figure~\ref{fig:1_introduction/knowledge_graph} shows an example knowledge graph as it could be used in a social network. It stores knowledge about concrete objects as well as general concepts, such as the fact $(Ed, has~gender, male)$. Symmetrical relations, such as "married to" are represented by two directed edges.

\begin{figure}[t]
    \centering
    \includegraphics[]{1_introduction/knowledge_graph}
    \caption{Example Knowledge Graph. Some facts missing}
    \label{fig:1_introduction/knowledge_graph}
\end{figure}

Besides displaying known facts using solid lines, Figure~\ref{figure:example_knowledge_graph} also shows dotted lines which indicate facts that hold true in reality but which are missing from the graph, like the fact $(Lucy, lives in, Netherlands)$. In fact, real-world knowledge graphs are often rather incomplete for several reasons: Firstly, it is difficult to manually create and update a large graph in the first place, and secondly, facts are sometimes deliberately left out if they can be derived from other facts. An example for the latter would be the transitive relation "lies in": Given a graph that stores $(Paris, lies in, France)$ and $(France, lies in Europe)$, it can be derived that Paris lies in Europe. This circumstance, that given facts can be relied on, but missing facts are not automatically false, but rather unknown, is known as \emph{open-world scenario}~\cite{}. If, instead, missing facts were actually false, the \emph{closed-world assumption} would hold instead.

When looking close at Figure~\ref{fig:1_introduction/knowledge_graph}, patterns can be found in the graph structure. For example, two out of three people born in Amsterdam also speak Dutch. One could therefore correctly assume that this is also true for the third entity Lucy. Furthermore, the fact that person A is married to person B also seems to imply that person B is married to person A. Two fundamentally different approaches to capturing these patterns are (1) searching for logical rules that generalize individual cases or (2) projecting the graph into a vector space, also known as \emph{embedding}, in which the patterns can easily be registered by machines.

Most state-of-the-art models follow the embedding approach, as it proved effective and efficient for a wide range of tasks including image \cite{}, natural language \cite{} and graph processing \cite{}. Embedding-based approaches are also called latent, because the exact structure of their embedding space is incomprehensible, or latent, to humans. In contrast, classical rule-based approaches are denoted as symbolic and bring the advantage that their rules are meaningful to humans - other from a 300-dimensional vector in an embedding space. In their paper on the AnyBURL rule miner \cite{}, Melicke et al. that rule-based models are capable to perform equally well to embedding-based approaches and demand further research on rule-based approaches.

Following that suggestion, this work builds a rule-based knowledge graph completion model on top of AnyBURL and combines it with a text-based model, forming the \emph{Power} ensemble model, which stands for "\textbf{P}robabilistic \textbf{o}pen-\textbf{w}orld \textbf{e}xtension to \textbf{r}ule-based KGC". Besides improving the rule-based component's fact predictions by leveraging additional text information, the text-based component enables predictions for open-world entities for which no facts are given. One of the top aims for both rule-based and text-based prediction was explaining the models decision. Therefore, each predicted fact is given with a lists of the rules and sentences that led to it.

As an example, considering the entity Lucy in Figure~\ref{fig:1_introduction/knowledge_graph}, who is known to be born in Amsterdam, the Power model would suggest that she lives in the Netherlands and give the rule $(X, lives in, Netherlands) <= (X, born in, Amsterdam)$ with a confidence of 67\% as an explanation. If multiple rules predict the same fact, all rules with a confidence greater 50\% are given in sorted order. In parallel to the rule-based prediction, a text about Lucy, like "Lucy is from the Netherlands", might lead to a 90\% prediction of $(Lucy, lives~in, Netherlands)$ by the text-based component. When multiple texts are given, they are ordered by relevance, as well. For open-world entities like John, the rule-based component cannot apply any rules, so analyzing John's texts is the only way to possibly predict his knowledge of Dutch or the fact that he is male. Again, multile texts, like "John is Dutch" and "He is an actor", are sorted by relevance to explain to the user that the fact $(John, speaks, Dutch)$ was inferred due the first sentence while the prediction of $(John, has~gender, male)$ is primarily based on the second one.

power source code available under MIT licence on GitHub
also, repos for experiments (power-experiment-*)



    \chapter{Basics}
    \label{ch:2_basics}
    \section{Knowledge Graphs}
\label{sec:3_basics/1_knowledge_graphs}
various definitions, here: <KG definition>

fact = head, rel, tail

related terms = knowledge base, triple store, ontology, graph database

closed/open world assumption
entities, relations
"closed world entity", "open world entity"



\section{Rules}
\label{sec:3_basics/2_rules}
the facts in a KG are not random, can detect patterns
patterns can be described with first order logic, e.g. if a-r-b and a-r-c then a-r-d, or e.g. if a-r-b and a-r-c then a-r-d
in practice restrict to horn rules, humanly understandable and computational efficient (?)
example horn rule:

<example horn rule>

consists of rule body and head
body consists of one or more atoms.
atoms resemble facts, but head and tail can be variable

rule mining = find rules

\subsection{Rule Types}
\label{subsec:2_basics/2_rules/1_rule_types}
different types of rules can be found by different rule miners
e.g. cyclic/acyclic []

cyclic means ..

<example cyclic rule>

example for acylic:

<example acyclic rule>


\subsection{Rule Types}
\label{subsec:2_basics/2_rules/2_rule_metrics}
\input{2_basics/2_rules/2_rule_metrics/rule_metrics}



\section{Neural Networks}
\label{sec:3_basics/3_neural_networks}
ml in general detects patterns in data
beside classic approaches like SVMs one import model type are neural nets
deep nets possible because much data nowadays
trained by forwarding input through net, calcing loss function and adjusting weights during backpropagation to minimize loss
during backpropagation sgd or similar
validate during training, keep out test data
hyperparameters, hyperparameter optimization
loss function: mse, cross entropy, binary cross entropy
feature/embedding space, manual feature engineering vs. automatic feature learning
eval/train mode (calc gradients or not)
softmax, sigmoid





NLP = natural language processing
machine learning, primer about training, datasets at <link> []

embedding based: embed tokens in (relatively) low-dim vector space,
that has properties like man/woman -> king/queen
word2vec, glove

pre-training
many works showed that pre-training on large general-purpose corpus with
subsequent finetuning yields better results~\cite{}

transformers
attention based
transformer = encoder + decoder
depending on problem one or both better suited, generally good: bert = encoder
bert revolutionalized nlp



\section{Metrics}
\label{sec:3_basics/4_metrics}
When designing and implementing machine learning models, scientists act on experience when it comes to architectural decisions and hyperparameter choices. In the early stages of a model, a trained eye on processing examples and observing the loss curve enable rapid progress. However, as the model matures, it becomes essential to quantify its performance with respect to comprehensive validation and test sets. Besides the selection of appropriate validation and test data, it is important to choose a meaningful metric that fits the problem. For example, when all of a models predictions are equally relevant, one would aim for an overall high precision, whereas a use case that involves a human processing the results manually, such as a web search, for example, one would choose a ranking metric that rewards good results at the top of a list. The purpose of this section is to explain those metrics relevant for this work. Section~\ref{fig:2_basics/4_metrics/1_confusion_matrix} defines basic terms used by the following sections. Sections~\ref{subsec:2_basics/4_metrics/2_accuracy} and~\ref{subsec:2_basics/4_metrics/3_prf} then present the general purpose metrics accuracy, precision, recall and F1 score while Sections~\ref{subsec:2_basics/4_metrics/4_mrr} and~\ref{subsec:2_basics/4_metrics/5_map} discuss the ranking metrics MRR and mAP\@.

\subsection{Confusion Matrix}
\label{subsec:2_basics/4_metrics/1_confusion_matrix}
All considered metrics are defined with the help of terms from a so-called confusion matrix as shown in Figure~\ref{fig:3_basics/4_metrics/1_confusion_matrix}. The matrix emerges from the general scenario in which predictions are made for a set of objects that can be either true or false. A prediction is true if its statement is consistent with reality about the object, also called ground truth, and false if the prediction contradicts reality. An example scenario would be an image recognition that has to determine whether a photo shows a cat or not. Then, four mutually exclusive types of predictions can be distinguished:

\begin{itemize}
    \item \textbf{\emph{True positives}} (TP) are predictions stating that a condition holds true when this is indeed the case. In the image recognition example, this would correspond to the case where the model correctly classifies a cat image as a cat.

    \item \textbf{\emph{False positives}} (FP) are negative predictions about objects where the condition is actually true, e.g. declaring an animal as cat although it is not. This type of error is also referred to as a \emph{Type 1 error}.

    \item \textbf{\emph{False negatives}} (FN) are another kind of erroneous predictions, also referred to as \emph{Type 2 errors}. They represent the case that an element with a true condition was classified as false - was overlooked, so to speak. An example would be a cat image not recognized as a cat.

    \item \textbf{\emph{True negatives}} (TN) are similar to true positives in that they are correct predictions, as well. They consist of rejective predictions about objects where the condition does indeed not apply, for example by recognizing that there is no cat in a cat-less photo.
\end{itemize}

Although it is generally desirable to obtain as many correct predictions as possible, true positives and negates are often differently important and errors of type one and two differently severe. In medicine, for example, not recognizing a disease could be much worse than accidentally diagnosing a healthy person as ill. Conversely, not recognizing guilt in a lawsuit might be less serious than convicting someone who is innocent. In the context of knowledge graph completion under the open-world assumption, the focus lies on true positives since the KGC model cannot make any qualified statements about non-applying facts without negative samples in the graph. Omitting a fact from the prediction only means that the model has too little evidence for that fact, not that it can falsify it.

\begin{figure}[t]
    \centering
    \includegraphics[width=\textwidth]{3_basics/4_metrics/1_confusion_matrix/matrix}
    \caption{Confusion Matrix}
    \label{fig:3_basics/4_metrics/1_confusion_matrix}
\end{figure}

"positive", "negative", "true", "false" predictions, "positive", "negative" elements

cat example
ir scenario


\subsection{Accuracy}
\label{subsec:2_basics/4_metrics/2_accuracy}
One of the most common general-purpose metrics is \emph{accuracy}, which measures a models overall capability to make correct positive and negative predictions. In case of binary classification, accuracy is the rate of correct predictions over all predictions:

\begin{align}
    Accuracy = \frac{TP + TN}{TP + TN + FP + FN}
    \label{eq:2_basics/2_metrics/1_accuracy/accuracy}
\end{align}

Colloquially speaking, accuracy answers the question of how good predictions are in general. It takes values in $[0, 1]$, whereby higher is better. However, although accuracy is a useful and intuitive metric in general, it can be misleading when it comes to inbalanced classes, because a model can simply reach high accuracy by always predicting the predominant class. For example, if nine out of ten ground truth values are false, a model could reach 90\% accuracy by always predicting false. To counteract this, balanced accuracy~\cite{Mower2005PREPMtPR} can be used instead. However, as mention earlier, as negative predictions do not play a big role for KGC models in an open-world scenario, accuracy will only play a minor role in this work, anyway.


\subsection{Precision, Recall and F1}
\label{subsec:2_basics/4_metrics/3_prf}
\emph{Precision}, \emph{recall} and \emph{F1} score focus on the quality of positive predictions. Precision gives an impression of how reliable positive predictions are, recall tells how many of the actual positive elements are declared as such, and F1 is a measure that combines both precision and recall in one value. All three metrics take values between 0 and 1 with higher being better.

Precision is the ratio of true positive predictions to all positive predictions as noted in~\ref{eq:2_basics/4_metrics/3_prf/precision}. In the above cat example, correctly identifying three out of five cats but also classifying one dog as a cat results in a precision of $3 / (3 + 1) = 0.75$. The two missed out cats do not play a role. Precision is also called \emph{positive predictive value} (PPV).

\begin{align}
    Precision = \frac{TP}{TP + FP}
    \label{eq:2_basics/4_metrics/3_prf/precision}
\end{align}

Recall, on the other hand, compares the number of true positives the the number of all ground truth positives as shown in~\ref{eq:2_basics/4_metrics/3_prf/recall}. Looking again at the cat example, identifying three out of five cats in total leads to a recall of $3 / (3 + 2) = 0.6$. The dog that was mistaken as a cat does not count in. Recall is also referred to as \emph{true positive rate} (TPR).

\begin{align}
    Recall = \frac{TP}{TP + FN}
    \label{eq:2_basics/4_metrics/3_prf/recall}
\end{align}

Precision and recall are directly dependent on each other. A cautious model that only predicts positives when it is absolutely sure achieves high precision but low recall. Conversely, it is easy to achieve optimal recall by making positive predictions for all elements, though precision will suffer in that case. The F score serves as a measure that reaches a high value when a reasonable balance between precision and recall is found. Equation~\ref{eq:2_basics/4_metrics/3_prf/f_beta} shows the formula for the general $F_\beta$ score whose parameter $\beta$ determines whether the focus should rather be shifted to precision or recall. Setting $\beta = 1$ yields the $F_1$ score in~\ref{eq:2_basics/4_metrics/3_prf/f_1} as the harmonic mean in which precision and recall are equally weighted.

\begin{align}
    F_\beta &= (1 + \beta^2) \cdot \frac{Precision \cdot Recall}{\beta^2 \cdot Precision + Recall}
    \label{eq:2_basics/4_metrics/3_prf/f_beta} \\
    F_1 &= 2 \cdot \frac{Precision \cdot Recall}{Precision + Recall}
    \label{eq:2_basics/4_metrics/3_prf/f_1}
\end{align}

When including more than one class in the evaluation, for example when predicting for each photo whether it is a cat, a dog or a horse, the question arises how to combine the results of the respective classes. Three of several possibilities are as follows:

\begin{itemize}
    \item Not combining the class results at all preserves the class-wise information but convoluted with a large number of classes.

    \item Merging all classes' predictions and calculating precision, recall and F score over all predictions is called \emph{micro} averaging. Underrepresented classes, which typically show poorer performance due to lack of training data, are less reflected in micro precision, recall and F score.

    \item Calculating each class' scores individually and averaging the class-wise metrics yields \emph{macro} precision, recall and F score. In this case, underrepresented classes have the same impact as classes with many ground truth positives. Macro values are therefore often worse than the corresponding micro values.
\end{itemize}

For very rare classes, there may be zero positives within a subset of the entire data set. If a good model does not predict false positives in this case, precision is undefined. In the context of a macro averaging, precision could be considered 0 because the model does not make a correct prediction, it could be defined as 1 because the model does not make an incorrect prediction, or the class could be excluded from averaging. Similarly, a withholding model might always make negative predictions so that recall is undefined. Therefore, the same considerations must be made for Recall and F Score.


\subsection{Mean Reciprocal Rank}
\label{subsec:2_basics/4_metrics/4_mrr}
The above presented accuracy, precision, recall and F1 metrics are useful in a classification task where no priorization among the predictions is required. However, in an \emph{information retrieval (IR)} scenario, such as a web search, for example, a model might yield a sorted list of predictions where the ranking actually plays a major role. Usually, IR scenarios do not differentiate between positive and negative predictions, but rather between more or less relevant predictions that are returned by decreasing relevance.

In those cases it is more important to rank relevant items as high as possible among the overall results than it is assign the correct probability, or class if there is probability threshold, to each item. When it is most important to receive a correct top-most prediction for each query, the \emph{mean reciprocal rank (MRR)} is the metric of choice. Given the results of $n$ queries it is calculated as per Equation~\ref{subsec:2_basics/2_metrics/3_mrr}, where $rank_i$ is the rank of the top-most relevant item among the predictions of the $i$th query results. Each reciprocal rank, and thus the mean over all reciprocal ranks, lies in $(0, 1]$, with higher being better. If a query result does not contain any relevant item, the reciprocal rank is undefined. Depending on the use case, the object might be skipped or assigned a specific value. A typical application scenario for MRR would be the evaluation of a voice assistant that has to respond with the single most relevant answer it gets from a model.

\begin{align}
    MRR = \frac{1}{n} \sum_{i=1}^{n} \frac{1}{rank_i}
    \label{eq:2_basics/2_metrics/3_mrr/mrr}
\end{align}


\subsection{Mean Average Precision}
\label{subsec:2_basics/4_metrics/5_map}
precision, recall, f1 do not consider order of predictions
power yields probabilities for facts, makes sense to rank probable facts higher
like google search

\[
    AP@n = \frac{1}{GTP} \sum_{k=1}^{n} P@n \cdot rel@n
\]

then, mean over all entities = mean of all average precisions = mAP

\[
    mAP = \frac{1}{N} \sum_{i=1}^{N} AP_i
\]





    \chapter{Related Work}
    \label{ch:3_related_work}
    \section{Text-Based}
\label{sec:3_related_work/1_text_based}
Most state-of-the-art models follow the second, embedding approach, which proved successful to many tasks outside graph applications, like image~\cite{} or natural language processing~\cite{}, for example. On knowledge graphs, the idea is to map each entity onto a certain spot in the relatively low-dimensional embedding space, whereby low means a few hundred dimensions, which is low compared to manually designed feature vectors. Relations, on the other hand are represented by two entities' relative positions to each other. By training an embedding model, the entity embeddings are moved in the embedding space such that their relative locations match the relations in the original graph. The learned embedding space then enables quick inference, because calculating the embeddings' floating point numbers is very cheap for modern processors.


state-of-the-art \cite{Wang2017KnowledgeGE}
embedding-based represents graph in low-dimensional vector space
entities in space, relationships encoded as relative position
embeddings learned to that similar embeddings close, different embeddings far apart


Villmov et al OWE \cite{Shah2019AnOE}
extension to embedding-based model
embed graph into vector space
embed texts into vector space
align graphs
apply alignment function to new ents in text space




\section{Rule-Based}
\label{sec:3_related_work/2_rule_based}
classical approach
find rules, apply to graph, infer facts
advantage: humanly understandable
problem: processing time

AnyBURL



\section{Combined}
\label{sec:3_related_work/3_combined}
\input{3_related_work/3_combined/combined}



    \chapter{Approach}
    \label{ch:4_approach}
    In general, the goal of knowledge graph completion is the prediction of missing knowledge, i.e. the prediction of missing facts. In practice, one is often interested in the facts of a particular entity. A useful model might thus accept an entity together with known facts and text about it and predict new facts from that information. For example, given the entity "Angela Merkel", the fact that it is a person who lives in Germany, and associated sentences about politics, a useful model might predict facts like $(Angela~Merkel, speaks, German)$ or $(Angela~Merkel, has~occupation, politician)$. To obtain all knowledge, one could then iteratively apply the model to all entities.

Aside from getting bare, recommended facts it would be desirable to get a rationale for the model's decision as well as a ranking of the predicted facts, ranging from highly ranked facts the model is sure about to lowly-ranked facts that might or might not be helpful.

The POWER model implements those concepts by invoking a rule-based and a text-based component whose predictions are then combined. The so-called \emph{Ruler} component leverages patterns in the graph structure of the existing knowledge graph while the \emph{Texter} component performs natural language processing on the entities' sentences to predict new facts. The \emph{Aggregator} component then combines the Ruler's and the Texter's predictions, returning the final output of the model. All predictions include the confidence for the predicted fact as well as the rules or sentences the prediction is based on.

\begin{figure}[t]
    \centering
    \includegraphics[width=\textwidth]{4_approach/power_architecture}
    \caption{POWER model architecture; Given an entity's known facts and textual description, the Ruler processes the known facts, the Texter processes the textual information and the Aggregator combines both components predictions to the final result}
    \label{fig:4_approach/power_architecture}
\end{figure}

The Texter performs particularly well on common facts but cannot predict rare facts while the ruler can potentially cover all kinds of facts seen in the train set but is unable to predict anything for open-world entities on its own. Through combining the Ruler's and Texter's predictions one obtains predictions with higher average precision than each component achieves on its own. The exact results vary, depending on the choice of hyperparameters during training, like the Texter's learning rate. Furthermore, each of the components can be adjusted architecturally as demonstrated by the experiments in chapter~\ref{ch:5_experiments}.


\section{Texter}
\label{sec:4_approach/1_texter}
With a Power dataset in place and the evaluatin metrics defined, the next step is training the Power model's components. In case of the Texter, however, an intermediate step is necessary. The Texter is trained in respect to a certain set of relation-tail tuples that make up its classes and requires a fixed number of sentences per entity. Given a fixed number of classes $c$, the $c$ most common relation-tail tuples in the training facts are determined and become the Texter's classes. A Texter dataset is created from the Power dataset that specifies which of the classes hold true for each entity. Furthermore, the Texter dataset contains the specified number of randomly samples sentences for each entity. Table~\ref{tab:5_experiments/4_texter/texter_dataset} shows an excerpt from a Texter dataset.

\begin{table}[h]
    \centering
    \begin{tabular}{ r l r r r r r r r r l }
    \toprule
    
    \multicolumn{1}{c}{\textbf{Ent}} &
    \multicolumn{1}{l}{\textbf{Entity Label}} &
    \multicolumn{8}{c}{\textbf{Classes}} &
    \multicolumn{1}{l}{\textbf{Sentences}} \\
    
    \midrule

    25 & Gabriel Yared    & 1 & 1 & 0 & 0 & 0 & 0 & 0 & 0 & Lebanese composer.        \\
    26 & Bridesmaids      & 0 & 0 & 0 & 0 & 0 & 1 & 1 & 1 & 2011 film by paul feig.   \\
    27 & Portugal         & 0 & 0 & 0 & 0 & 0 & 0 & 0 & 0 & Republic in \dots         \\
    28 & Serpico          & 0 & 0 & 0 & 0 & 0 & 1 & 1 & 1 & 1973 american crime \dots \\
    29 & Italian          & 0 & 0 & 0 & 0 & 0 & 0 & 0 & 0 & Romance language.         \\
    30 & Catherine Keener & 0 & 0 & 1 & 1 & 1 & 0 & 0 & 0 & American actress.         \\
    
    \bottomrule
\end{tabular}

    \caption{Excerpt from a Texter dataset from the FB Power split and the fb-owe-1-clean text set for a Texter with eight classes}
    \label{tab:5_experiments/4_texter/texter_dataset}
\end{table}

In general, the Texter dataset's classes array is very sparse, especially for diverse knowledge graphs without frequently occurring relation-tail tuples. Therefore, it is important to weight the classes when calculating the loss during training to punish false negatives more heavily than false positives. Otherwise, the Texter would learn that always predicting 0 keeps the loss as low as possible. Table~\ref{tab:5_experiments/4_texter/classes} gives an overview of the imbalance between the top 100 classes that were selected for this evaluation. As can be seen, the FB split contains not only more facts than the CDE split, but also fewer diverse facts, which should facilitate the Texter's training.

\begin{table}[h]
    \centering
    \begin{tabular}{| l | r | r | l |}
    \hline
    
    \multicolumn{1}{|c|}{\textbf{Split}} &
    \multicolumn{1}{|c|}{\textbf{Rank}} &
    \multicolumn{1}{|c|}{\textbf{Freq}} &
    \multicolumn{1}{|c|}{\textbf{Class = (Relation, Tail)}} \\

    \hline \hline

    \multirow{3}{*}{CDE}
    & 1   & 19.78 & (country of citizenship, United States of America)   \\
    & 2   & 19.10 & (occupation, writer)                                 \\
    & 3   & 17.06 & (languages spoken, written, or signed, English)      \\

    \hline

    \multirow{3}{*}{CDE}
    & 98  & 1.08  & (genre, hip hop music)                               \\
    & 99  & 1.07  & (member of, International Finance Corporation)       \\
    & 100 & 1.07  & (record label, Columbia Records)                     \\

    \hline \hline

    \multirow{3}{*}{FB}
    & 1   & 24.34 & (\dots/webpage/category, /m/08mbj5d)                 \\
    & 2   & 19.67 & (/people/person/gender, male organism)               \\
    & 3   & 17.38 & (\dots/marriage/type\_of\_union, marriage)           \\

    \hline

    \multirow{3}{*}{FB}
    & 98  & 1.35  & (\dots/award\_nomination/award, Screen Actors \dots) \\
    & 99  & 1.32  & (/film/film/genre, fantasy)                          \\
    & 100 & 1.31  & (\dots/film\_release\_region, Israel)                \\

    \hline
\end{tabular}

    \caption{Most and least common classes on the CDE and FB splits. Frequencies are given in percent.}
    \label{tab:5_experiments/4_texter/classes}
\end{table}

Evaluation is generally performed against the Texter dataset's test subset, following the form shown in Table~\ref{tab:5_experiments/4_texter/texter_dataset}. That means that the Texter's predictions are evaluated against the classes it was trained on and that the results are comparable to the validation results during training. The metrics chosen are macro precision, recall and F1 over the classes. This is rather demanding, as rare classes are equally weighted as the common classes, that represent more facts, but it follows the objective of predicting all classes as good as possible. For the final comparison with Ruler and Aggregator, however, the Texter is also evaluated against the Power split's test facts, which includes facts the Texter is unable to predict, using the common metrics discussed before in Section~\ref{sec:5_experiments/3_metrics}.

The following Subsection~\ref{subsec:5_experiments/4_texter/1_zero_rule} applies the presented metrics to several baselines to give an idea of the problem's complexity. Given the reference values from the baselines, Subsections~\ref{subsec:5_experiments/4_texter/2_static} and~\ref{subsec:5_experiments/4_texter/3_context} present two ways of implementing the Texter's embedding block using static and contextual word embeddings, whereby the latter is used in the final Power implementation described in Chapter~\ref{ch:4_approach}. To distinguish the two variants, they are also referred to as static and contextual Texter, respectively. Both sections include several experiments that led to the final versions of the models. In addition, each experiment compares the simple and the attentive versions of the Texter to see which variations affect the attention mechanism of the latter.

\subsection{Zero Rule Baselines}
\label{subsec:5_experiments/4_texter/1_zero_rule}
Zero-rule baselines are naive models that give an idea of how well a problem could be tackled by a naive approach, such as simply guessing random solutions. Any model performing worse than a zero-rule baseline cannot be considered useful. For example, in a coin toss, one would already achieve 50\% accuracy, precision and recall by randomly guessing between heads and tails. A coin-flip predicting model that does not exceed that performance would not be considered useful.

The purpose of this subsection is to determine the absolute baseline for the classification problem at hand that the Texter has to exceed. Therefore, the best zero-rule for the binary multi-label classification the Texter dataset poses has to be found. Randomly guessing with a 50:50 chance which of an entity's classes hold true is one naive strategy, but there are better ones when it comes to unbalanced classes such as the Texter dataset's ones. For example, for a class with an 80:20 chance to be true, randomly guessing true in 80 of 100 cases leads to better accurracy than choosing equally between true and false.

In general, performance for a random guessing strategy can be calculated. For example, accuracy in an unbalanced classification scenario with two possible outcomes 0 and 1 can be calculated as $P(\hat{y} = y) = P(\hat{y} = 0) \cdot P(y = 0) + P(\hat{y} = 1) \cdot P(y = 1)$ where $y$ is the ground truth outcome and $\hat{y}$ is the predicted outcome. While $P(y = 0)$ and $P(y = 1)$ depend on the regarded class' outcome distribution, the values of $P(\hat{y} = 0)$ and $P(\hat{y} = 1)$, and thus the probability to predict the ground truth value, depend on the guessing strategy. In this evaluation, four potential guessing strategies were considered with respect to each class:

\begin{itemize}
    \item \emph{Uniformly} guessing between 0 and 1, i.e. $P(\hat{y} = 0)$ = $P(\hat{y} = 1)$ = 0.5

    \item \emph{Stratified} guessing, which means that each outcome is guessed according to its frequency, i.e. $P(\hat{y} = 0)$ = $P(y = 0)$ and $P(\hat{y} = 1)$ = $P(y = 1)$

    \item Always guessing the \emph{most frequent} outcome, i.e.  $P(\hat{y} = 0)$ = 1 and $P(\hat{y} = 1)$ = 0 if 0 is the most common outcome

    \item \emph{Constantly guessing 1}, i.e. $P(\hat{y} = 0)$ = 0 and $P(\hat{y} = 1)$ = 1
\end{itemize}

Similarly, precision, recall and F1 could also be calculated for each class of the Texter dataset. Averaging each class' metrics over all classes would finally yield each zero-rule strategy's performance for the whole Texter dataset. For this evaluation, the zero-rule strategies were implemented and measured instead. The values in \autoref{tab:5_experiments/3_texter/1_zero_rule/results} were gathered by running each of the four zero-rule strategies ten times and averaging the values. For each zero-rule strategy, performance is shown for the most common class, the least common class and the average over all classes. As shown before, the Texter datasets' classes arrays are very sparse, so it was expected that constantly predicting the most common output false for all classes would lead to the highest accurracy. However, since F1 is the more relevant metric in regard to predicting true positives, constantly predicting true was expected to be the best zero-rule strategy.

\begin{table}[t]
    \makebox[\textwidth][c]{
        \input{5_experiments/3_texter/1_zero_rule/results}
    }
    \caption{Evaluation of various zero-rule baselines on the CDE and FB splits. The best average values per split are marked bold (column-wise). Thus, uniform sampling sets the baseline on the CDE split (20.19\% F1), while constantly predicting true leads to the best result on the FB split (9.43\% F1).}
    \label{tab:5_experiments/3_texter/1_zero_rule/results}
\end{table}

As it turned out, constantly predicting true for all classes is indeed the best random guessing strategoy for the Texter dataset derived from the FB split, leading to an F1 score of 9.43\% which serves as the absolute baseline for any Texter predicting facts for the FB split. Surprisingly, the best zero-rule strategy for the Texter dataset derived from the CDE split is randomly guessing between true and false, leading to over 20.19\% F1 score, which is almost twice as much as the value achieved when constantly guessing true. Those two percentages have to be kept in mind when regarding any performances achieved by actually learning models.


\subsection{Static Word Embeddings}
\label{subsec:5_experiments/4_texter/2_static}
First, the embedding block in both, the simple and the attentive Texter, is investigated when using static, pre-trained word embeddings, i.e. a word is embedded independent from where it occurs in the sentence. The advantages include ease of implementation using a simple look-up table, few hyperparameters and quick training. However, the sentence embedding cannot properly capture the meaning of markings and maskings via special tokens such as those in sentences from the "cde-irt-5-marked" or "cde-irt-5-masked" text sets exemplified in Table~\ref{tab:5_experiments/1_data_sources/2_text_sets/text_sets_table}. Therefore, all experiments in this section are conducted on the clean text sets containing no special tokens. During training, the settings that proved to be best during the experiments were used. In particular, these are the use of an NLP-based tokenizer, pre-trained fastText~\ref{} word embeddings that are finetuned during training, mean pooling to form sentence embeddings, the sigmoid function in the attention block in the case of the attentive texter, class weights in the computation of the loss, and the Adam optimizer to apply gradients during backpropagation. Table~\ref{tab:5_experiments/4_texter/2_static/results} shows the final results over all clean text sets for both the simple and the attentive versions of the Texter. As for all following tables presenting Texter evaluation results, for compactness, only the name of the text set is given for the respective Texter dataset, because it implies the matching Power split.

\begin{table}[t]
    \centering
    \input{5_experiments/4_texter/2_static/results}
    \caption{Final evluation results for the simple and the attentive Texter using static word embeddings - for each text set, the better model is marked in terms of precision, recall and F1}
    \label{tab:5_experiments/4_texter/2_static/results}
\end{table}

The numbers show that both the simple and the attentive Texter, outperform the zero rule baselines by far. However, the attentive model does not outperform the simple version as hoped. On most datasets, the attentive model achieves better precision, but the simple Texter has higher recall. Overall, the simple modell performs better, especially on datasets with multiple sentences per entity which should have been in favor of the attentive model. Interestingly, both Texters perform best given the short OWE sentence due to high recall scores, which implies that a single, good entity description is more useful than many vague contexts for knowledge graph completion. Similar to the OWE sentences, the CDE sentences from the entities' Wikipedia introductions perform best on the CDE. However, the results are not far ahead of those for the text set with 30 IRT sentences, so that the observation could also be formulated in reverse: 15 randomly sampled entity contexts are just as good as a selected high-quality description.

In addition to the final evaluation results on the test data, Figure~\ref{fig:5_experiments/4_texter/2_static/plot_valid_curves} shows the development of the F1 score for different text sets during training on the FB graph. After 200 episodes of training, the simple and attentive versions of the Texter converge against similar values. The only notable difference between the two is that the simple model reaches its optimal training time earlier before it starts overfitting on most datasets. Thereby, the IRT text sets with few sentences are more prone to overfitting than the ones with 15 or 30 sentences per entity. The short OWE texts, on the other hand, seem to be immune to overfitting. Although training could be stopped after 50 episodes without major defficiencies in this case, all following experiments are performed with 200 episodes to ensure that all models reach their full potential even if changes lead to slower training.

\begin{figure}[t]
    \centering
    \input{5_experiments/4_texter/2_static/plot_valid_curves/plot_valid_curves}
    \caption{Validation F1 curves during training the Texter on the FB split with 1 OWE sentence (red), 30 IRT sentences (orange), 15 IRT sentences (yellow), 5 IRT sentences (black) or 1 IRT sentence (green) per entity}
    \label{fig:5_experiments/4_texter/2_static/plot_valid_curves}
\end{figure}

What is not visible in Figure~\ref{fig:5_experiments/4_texter/2_static/plot_valid_curves} is the information how the average F1 score is composed of the class-wise F1 scores, which is addressed by Figure~\ref{fig:5_experiments/4_texter/2_static/plot_class_curves}. It compares the F1 scores between the most common classes, the least common classes, and the average F1 score over all classes. The graphs reveal that predictions for common classes are much more reliable. The three most common classes on the CDE split, with frequencies of almonst 20\% each, reach performances between 50\% and 80\%, whereas the three least common classes, with frequencies of around 1\% each, reach strongly divergent values, from 0\% to 60\%. Despite the tendency that frequent classes perform better, however, a class' performance cannot be calculated directly from its frequence. The least common classes have very similar frequencies but perform very differently. For the second least common class prediction does not work at all, while the third least common class performs even better than the third most common class. There are also significant performance differences between the three most frequent classes although the frequencies are similar. Another, more obvious, finding is that frequent classes are learned within fewer episodes than less common classes. Interestingly, the attentive Texter seems to learn faster than the simple Texter. Finally, the graph shows that both Texters converge to similar values for all classes, as might have been assumed from the similar macro F1 values.

\begin{figure}[t]
    \centering
    \input{5_experiments/4_texter/2_static/plot_class_curves/plot_class_curves}
    \caption{Class-wise validation F1 curves during training the Texter on the CDE split with 5 IRT sentences per entity - comparison between the most common classes (red, orange, yellow), the least common classes (green, blue, purple) and the average value over all classes (black)}
    \label{fig:5_experiments/4_texter/2_static/plot_class_curves}
\end{figure}

Looking at the results for the other datasets shown in Tables~\ref{fig:a_appendix/static_classes_1}~--~\ref{fig:a_appendix/static_classes_3} in Appendix~\ref{ch:a_appendix}, similar patterns can be observed on other text sets with only a few exceptions. However, although there are no major differences between the simple and the attentive Texter, it is noticeable that the results for rare classes do not improve steadily when increasing the number of sentences in case of the IRT text sets. Multiple training runs on the same data showed that this was not due to different initializations of the model. Instead, separate experiments showed that it is the random selection of IRT sentences that affects the performance of individual rare classes. In the respective experiment, new fb-irt-1-clean text sets were built by repeatedly selecting random sentences from the larger fb-irt-30-clean dataset. The overall F1 score over all classes, however, stayed the same for all runs.

With the final results for static word embeddings at hand, the following subsections show what impact various model changes and hyperparameters have. The affected model components are examined in the order in which they are invoked during training, beginning with the tokenizer and ending with the optimizer. Finally, the attentive Texter's attention mechanism is investigated to explain why it does not improve upon the simple Texter as hoped.

\subsubsection{Changing the tokenizer}
\label{subsubsec:5_experiments/4_texter/2_static/1_tokenizer}
\input{5_experiments/4_texter/2_static/1_tokenizer/tokenizer}

\subsubsection{Initializing word embeddings randomly}
\label{subsubsec:5_experiments/4_texter/2_static/2_emb_size}
\input{5_experiments/4_texter/2_static/2_emb_size/emb_size}

\subsubsection{Using pre-trained word embeddings}
\label{subsubsec:5_experiments/4_texter/2_static/3_pre_trained}
\input{5_experiments/4_texter/2_static/3_pre_trained/pre_trained}

\subsubsection{Freezing pre-trained embeddings}
\label{subsubsec:5_experiments/4_texter/2_static/4_update_vectors}
\input{5_experiments/4_texter/2_static/4_update_vectors/update_vectors}

\subsubsection{Pooling}
\label{subsubsec:5_experiments/4_texter/2_static/5_pooling}
\input{5_experiments/4_texter/2_static/5_pooling/pooling}

\subsubsection{Activation Function}
\label{subsubsec:5_experiments/4_texter/2_static/6_activation}
\input{5_experiments/4_texter/2_static/6_activation/activation}

\subsubsection{Appplying class weights}
\label{subsubsec:5_experiments/4_texter/2_static/7_weight_factor}
\input{5_experiments/4_texter/2_static/7_weight_factor/weight_factor}

\subsubsection{Choosing an optimizer}
\label{subsubsec:5_experiments/4_texter/2_static/8_optimizer}
\input{5_experiments/4_texter/2_static/8_optimizer/optimizer}

\subsubsection{Inspecting the attention mechanism}
\label{subsubsec:5_experiments/4_texter/2_static/9_attention}
\input{5_experiments/4_texter/2_static/9_attention/attention}


\subsection{Contextual Word Embeddings}
\label{subsec:5_experiments/4_texter/3_context}
Modern NLP models continue where classical models with static word embedding reach their limits when it comes to long sentences in which the relationship between the words is even more important for the words', and thus, the overall sentence's meaning. For this purpose, the particularly successful transformers use internal attention mechanisms that learn for each word which other words in the sentence are particularly relevant and should therefore be included in the word's embedding. This results in a context-dependent embedding for each occurrence of the word.

Besides embedding usual words and some standard tokens that represent unknown words and paddings, the here used DistilBERT also supports the special [CLS] and [SEP] tokens introduced by the BERT model. The [CLS] token's purpose is to capture the meaning of the sentence as a whole during training. In addition, the Texter supports the definition of custom tokens, such as the ones used for markings or maskings in the IRT text sets. This additional information should lead to a significant performance increase, even when the entity mention is masked, as it preserves the information what the surrounding sentence is about.

For the Texter's downstream architecture, the change from a lookup table for embeddings to the use of DistilBERT does not mean a big change as it only affects the embedding block -- the classification block and, in the case of the attentive version, the attention block remain untouched. In particular, design choices from the experiments with static word embeddings, such as the usage of the sigmoid function in the attention block, are kept. During training, however, the optimizer has to be adjusted to accommodate the deep transformers within the embedding block.

\begin{table}
    \makebox[\textwidth][c]{
        \input{5_experiments/3_texter/3_context/results}
    }
    \caption{Final evaluation of the contextual Texter on all text sets. Results of the static Texter are given for comparison. The contextual Texter is evaluated against the Texter dataset's test subset (F1) and against all facts from the respective split (F1 all, mAP all). The contextual Texter outperforms the simple Texter in general, especially when leveraging markings in the text. The attentive Texter profits more from contextual word embeddings. Still, the simple Texter performs better in terms of mAP.}
    \label{tab:5_experiments/3_texter/3_context/results}
\end{table}

After the switch to DistilBERT, training of the Texter takes longer due to the large number of parameters in the transformer, but can be terminated after 50 epochs. \autoref{tab:5_experiments/3_texter/3_context/results} shows the evaluation results for the final Texter after that time. In addition to the previously regarded macro F1 score over all classes, \autoref{tab:5_experiments/3_texter/3_context/results} also provides F1 and mAP over the entities from the Power split -- including those the Texter cannot predict. In addition to the contextual Texters' evaluation results, the static Texters' final results from \autoref{subsec:5_experiments/3_texter/2_static} are given for comparison where available. Furthermore, \autoref{ch:a_appendix} contains the detailed \autoref{tab:a_appendix/context_final_prec_rec} that provides the precision and recall values.

The final evaluation results reveal some interesting facts: First, the contextual Texter performs better on clean text sets than the static Texter most of the time, but not always. While the contextual Texter performs better on almost all text sets containing IRT sentences, it does not for the CDE and OWE sentences. Second, the CDE and OWE text sets show best, that the attentive Texter benefits more from contextual word embeddings than the simple model. Third, it is obvious that markings bring great improvements, as was expected, while the masked text sets are in between the clean and marked ones in terms of F1. In terms of mAP, however, the masked text sets lead to better predictions than their marked counterparts in multiple cases. Fourth, the evaluating against all facts from the Power split leads to significantly worse results on the FB split than it does on the CDE split, which is probably due to the reason that the FB classes cover a smaller portion of the larger FB split.

Overall, it can be stated, that the Texter performs better when using contextual word embeddings, especially when marked texts are available. Therefore, the contextual Texter is the default in the Power model. However, depending on the given text set as well as hardware support, static word embeddings might be a noteworthy alternative. When comparing the simple and attentive versions of the contextual Texter, the simple version yields slightly better results. Still, the attentive Texter is kept due to its ability to explain its text-based decision to a certain point.

In the following, subsections~\ref{subsubsec:5_experiments/3_texter/3_context/1_sent_len}~--~\ref{subsubsec:5_experiments/3_texter/3_context/3_optimizer} will look at some experiments on the updated embedding block and the adjusted optimizer used to train the deep contextual Texter.

\subsubsection{Varying sentence length}
\label{subsubsec:5_experiments/3_texter/3_context/1_sent_len}
\input{5_experiments/3_texter/3_context/1_sent_len/sent_len}

\subsubsection{Pooling}
\label{subsubsec:5_experiments/3_texter/3_context/2_pooling}
\input{5_experiments/3_texter/3_context/2_pooling/pooling}

\subsubsection{Optimizer}
\label{subsubsec:5_experiments/3_texter/3_context/3_optimizer}
\input{5_experiments/3_texter/3_context/3_optimizer/optimizer}




\section{Ruler}
\label{sec:4_approach/2_ruler}
While the Texter processes the text information attached to the knowledge graph's entities, the Ruler exploits patterns in the graph structure to predict missing facts. Given an entity $x$ with a set of known facts $K$ containing facts of the form $(x, rel_k, tail_k)$ with $1 <= k <= |K|$, $rel_k \in R$, and $tail_k \in E$ being any relation or entity, respectively, the Ruler leverages entity-related rules of the form $(x, rel_m, tail_m) <= (x, rel_k, tail_k)$ with $1 <= m <= |M|$ to predict a set of missing facts $M$. Therefore, the rules required for the inference process have to be mined from the knowledge graph, beforehand. That rule mining process can be viewed as the equivalent to the Texter's training process and some paragraphs in this thesis will refer to rule mining as ``training'' a Ruler. Compatible to the Texter, the Ruler is limited to the prediction of facts $(x, rel, tail)$ that contain the query entity as their head, as well. However, when the Ruler is applied to all entities $e \in E$, all facts of the form $(e, rel, x)$ are predicted at some point as far as the mined rules support it.

For rule mining, AnyBURL, the bottom-up rule mining algorithm, by Christian Meilicke et al.~\cite{Meilicke2019AnytimeBR} is used. It is an anytime algorithm, meaning that it can be interrupted anytime and still yield valid results. Bottom-up rule mining refers to the fact that the algorithm starts with concrete paths in the graph and tries to generalize those paths to rules instead of coming up with rules initially and searching for evidence later, which would be a top-down approach. Out of all possible Horn rules that might describe patterns in the graph, AnyBURL is restricted to rules that can be generalized from so-called \emph{ground path rules}. A ground path rule does not contain variables, but only constants, and must not contain any cycles in its body. \autoref{eq:4_approach/2_ruler/ground_path_rule} describes the general form of a ground path rule of length $n$, meaning that it consists out of the head fact and $n$ body facts.

\begin{align}
(c_0, h, c_1)
    \Leftarrow (c_1, b_1, c_2), \dots, (c_n, b_n, c_{n+1}) &&
    c_k \neq c_l \forall k, l \in \{1, \dots, n+1\}, k \ne l
    \label{eq:4_approach/2_ruler/ground_path_rule}
\end{align}

Notably, despite the rule body being free of cycles, the ground path rule as a whole can still be cyclic if $c_0 = c_{n+1}$. Ground path rules are derived directly from randomly sampled paths in the graph and are subsequently generalized to rules that replace some of the constants with variables. If further supporting paths can be found for a general rule, it is kept. In their paper on AnyBURL, Meilicke et al. show that any rule that can be generalized from a ground path rules can be generalized to one of the three rule types formulated in Equations~\ref{eq:4_approach/2_ruler/c}~--~\ref{eq:4_approach/2_ruler/ac2}. Thereby, $C$-type rules can only be generalized from cyclic ground path rules, $AC2$ rules can only be generalized from acyclic ground path rules and $AC1$ can be generalized from both, cyclic and acyclic ground path rules. The following paragraphs outline the core algorithm used to mine such rules and derive some example rules from the small graph introduced in \autoref{ch:1_introduction}. \autoref{fig:4_approach/2_ruler/rule_graph} shows an annotated subset of the graph that illustrates the rules.

\begin{align}
    C   && (Y, h, X)   &\Leftarrow (X, b_1, A_2), \dots, (A_n, b_n, Y)
    \label{eq:4_approach/2_ruler/c} \\
    AC1 && (c_0, h, X) &\Leftarrow (X, b_1, A_2), \dots, (A_n, b_n, c_{n+1})
    \label{eq:4_approach/2_ruler/ac1} \\
    AC2 && (c_0, h, X) &\Leftarrow (X, b_1, A_2), \dots, (A_n, b_n, A_{n+1})
    \label{eq:4_approach/2_ruler/ac2}
\end{align}

\begin{figure}[t]
    \centering
    \includegraphics{4_approach/2_ruler/rule_graph}
    \caption{Subset of the previously introduced example graph with highlighted facts that form an acyclic (red + green) and a cyclic (red + blue) path.  ``Amsterdam'' and ``Netherlands'' have been abbreviated to ``AMS'' and ``NL''.}
    \label{fig:4_approach/2_ruler/rule_graph}
\end{figure}

Essentially, AnyBURL repeatedly samples random paths from the graph, generalizes them to all possible rule types, looks for further paths that match the gained rules and keeps those rules it finds further evidence for. For example, in search of rules of length two, i.e. rules that have a body consisting of two facts, AnyBURL might randomly sample the two paths in~\ref{eq:4_approach/2_ruler/path_1} and~\ref{eq:4_approach/2_ruler/path_2} from the graph. Note, that parentheses denote entities, brackets denote relations and that the path does not need to follow directed edges in the graph. Furthermore, a close look at the paths reveals that the second path is cyclic as it starts and ends at the entity ``Dutch''.

\begin{align}
(Dutch)
    \leftarrow [speaks] - (Ed) - [married~to] \rightarrow (Lisa) - [born~in] \rightarrow (AMS)
    \label{eq:4_approach/2_ruler/path_1} \\
    (Dutch) \leftarrow [speaks] - (Ed) - [lives~in] \rightarrow (NL) - [has~lang] \rightarrow (Dutch)
    \label{eq:4_approach/2_ruler/path_2}
\end{align}

From those paths, AnyBURL would then derive the constant-only ground path rules~\ref{eq:4_approach/2_ruler/acyclic_ground_path} and~\ref{eq:4_approach/2_ruler/cyclic_ground_path} by taking the path's first part as the rule's head and the remaining parts to form the rule's body.

\begin{align}
(Ed, speaks, Dutch)
    &\Leftarrow (Ed, married~to, Lisa), (Lisa, born~in, AMS)
    \label{eq:4_approach/2_ruler/acyclic_ground_path} \\
    (Ed, speaks, Dutch) &\Leftarrow (Ed, lives~in, NL), (NL, has~lang, Dutch)
    \label{eq:4_approach/2_ruler/cyclic_ground_path}
\end{align}

The acyclic ground path rule in~\ref{eq:4_approach/2_ruler/acyclic_ground_path} can be generalized to the $AC1$ rule~\ref{eq:4_approach/2_ruler/acyclic_ac1} and the $AC2$ rule~\ref{eq:4_approach/2_ruler/acyclic_ac2} while the cyclic ground path rule~\ref{eq:4_approach/2_ruler/cyclic_ground_path} can be generalized to the $C$ rule~\ref{eq:4_approach/2_ruler/cyclic_c} and the $AC1$ rule~\ref{eq:4_approach/2_ruler/cyclic_ac1}.

\begin{align}
    AC1 && (X, speaks, Dutch) &\Leftarrow (X, married~to, A_2), (A_2, born~in, AMS)
    \label{eq:4_approach/2_ruler/acyclic_ac1} \\
    AC2 && (X, speaks, Dutch) &\Leftarrow (X, married~to, A_2), (A_2, born~in, A_3)
    \label{eq:4_approach/2_ruler/acyclic_ac2} \\
        C   && (X, speaks, Y) &\Leftarrow (X, lives~in, A_2), (A_2, has~lang, Y)
    \label{eq:4_approach/2_ruler/cyclic_c} \\
    AC1 && (X, speaks, Dutch) &\Leftarrow (X, lives~in, A_2), (A_2, has~lang, Dutch)
    \label{eq:4_approach/2_ruler/cyclic_ac1}
\end{align}

Next, every rule candidate is scored by looking for further paths that match the rule's body and checking whether the graph also contains the fact predicted by the rule, i.e. whether the rule is correct in that case. Thereby, the number of paths that match the rule body is called the rule's \emph{support} while the ratio of times the rule is correct over its total support is called \emph{confidence}. Taking the cyclic rule~\ref{eq:4_approach/2_ruler/cyclic_c} as an example, AnyBURL would search for further evidence and find the path $(Lisa) - [lives~in] \rightarrow (NL) - [has~lang] \rightarrow (Dutch)$ that matches the rule body, increasing the rule's support to two, so far. However, the example graph does not contain the rule's predicted fact $(Lisa, speaks, Dutch)$, so the rule's support drops from 1 to $\frac{1}{2}$. Since rules that only apply to a single case or only once in every thousandth case are not very useful, AnyBURL drops rules with a support of 1 or confidence below 0.0001 by default~\cite{AnyBURL}. It is noteworthy that, although some rules are more general than others, such as~\ref{eq:4_approach/2_ruler/acyclic_ac2} compared to~\ref{eq:4_approach/2_ruler/acyclic_ac1}, the more specific ones are still kept as they might end up with higher confidence for their special case during the ongoing mining process.

The process described by the above example is repeated until only a few new rules of the same length $n$ can be found. AnyBURL then continues its search for rules of length $n + 1$ until it terminates after a fixed number of time steps. \autoref{code:anyburl} shows the slightly adjusted pseudocode from the AnyBURL paper. The sampling and scoring process discussed above is implemented as the body of the inner while loop. The outer for loop implements the repeated check for the saturation of rules of the current length and the eventual proceeding to rules of increased length.

\begin{listing}[t]
    \begin{lstlisting}
        AnyBURL(G, sat, Q, i, ts):
            n = 2
            R = $\emptyset$
            for i times:
                $R_s = \emptyset$
                start = current_time()
                while current_time() < start + ts:
                    p = sample_path(G, n)
                    $R_p$ = generate_rules(p)
                    for $r$ in $R_p$:
                        score($r$)
                        if $Q$($r)$:
                            $R_s$ = $R_s \cup {r}$

                $R_s^{'}$ = $R_s \cap R$
                if $|R_s^{'}| / |R_s| > sat$:
                    n = n + 1
                $R$ = $R \cup R_s$

            return $R$
    \end{lstlisting}
    \caption{The AnyBURL rule mining algorithm takes a graph $G$, a saturation level $sat$, a quality criterion $Q$, and a number of iterations $i$, each of a timespan $ts$, as input and produces a ruleset $R$.}
    \label{code:anyburl}
\end{listing}

A walk through the pseudocode reads as follows: Given the Graph $G$, the saturation threshold $s$, the quality criterion $Q$, a number of iterations $i$ and the timespan $ts$ each iteration endures, AnyBURL starts with an empty ruleset $R$, that will be extended after each iteration and returned in the end. The initial length of the randomly sampled paths is $n=2$, allowing to find the shortest possible rules of length 1. During the first iteration of duration $ts$, AnyBURL fills the ruleset $R_s$, which keeps the rules found in the current iteration, by repeatedly sampling paths, generating rules from the paths, scoring the resulting rules, and keeping those with sufficient support and confidence. At the end of the iteration, when the timespan $ts$ has passed, $R_s^{'}$ is calculated as the set of rules mined during the iteration that were already known. If the share of already known rules mined during the current iteration exceeds the saturation threshold, the algorithm starts searching for rules of increased length. Otherwise, it continues with the current length. In both cases, the iteration's rules are added to the overall ruleset $R$. If the specified number of total iterations is reached, AnyBURL terminates and returns the mined rules $R$. In practice, AnyBURL saves the mined rules in a text file at the end and at configurable points during mining.

With the stored rules from AnyBURL in place, the Ruler is prepared for inference. Conceptually, given an entity and its known facts, the Ruler loads the rules, filters out further rules that do not meet the Ruler's quality demands, and applies the remaining, useful rules to all known facts. All rules that can be applied successfully are kept together with their confidence. From all the facts predicted by the applied rules, already known facts from the existing graph are filtered out. The remaining facts are sorted by confidence and returned to the user -- together with the rules that predicted them as an explanation for the user. If multiple rules predict the same fact, the fact is assigned the highest confidence of those rules and is returned together with all of them. The Ruler's extra quality criterion mentioned above further restricts the considered rules to those with confidence greater 50\%, because AnyBURL's minimum confidence threshold of 0.0001 allows many rules that predict false positives. For open-world entities, this algorithm implies an empty result set as no rule can cover an entity that is not connected to any other entity and all the facts predicted for the train entities will be filtered out. In those cases, the Power model has to rely solely on the Texter.



\section{Aggregator}
\label{sec:4_approach/3_aggregator}
The aggregator has the task of merging the predicted facts from Ruler and Texter. As envisioned in \autoref{sec:4_approach/3_aggregator} and illustrated in \autoref{fig:4_approach/3_aggregator/lucy}, it was hoped that merging the facts leads to higher average precision because facts predicted by both components are likely to be correct and should be ranked higher. In addition, the Aggregator should be able to estimate how reliable the predictions of Ruler and Texter are in relation to each other, which is implemented in the form of the weight parameter $\alpha$ as described in \autoref{eq:4_approach/3_aggregator/conf_aggregator}.

\autoref{tab:5_experiments/5_aggregator/results} shows the final evaluation results for the Aggregator, and thus the final evaluation results for the Power model, for a number of graph-text combinations. As fact splits, the splits with 50\% known test facts were chosen, as for the final Ruler evaluation in \autoref{sec:5_experiments/4_ruler}. The respective results for the CDE-50 and FB-50 splits from \autoref{tab:5_experiments/4_ruler/results} were taken over into \autoref{tab:5_experiments/5_aggregator/results} for easier comparability. Similarly, the chosen text sets are the ones from the final Texter evaluation in \autoref{subsec:5_experiments/3_texter/3_context}. Again, \autoref{tab:5_experiments/5_aggregator/results} duplicates the respective results from \autoref{tab:5_experiments/3_texter/3_context/results} for ease of comparison. The last two columns then contain the new Aggregator measurements for the combination of the corresponding Ruler and Texter.

\begin{table}[t]
    \makebox[\textwidth][c]{
        \begin{tabular}{| l | r | r | r | r | r | r | r | r | r |}
    \hline
    
    \multicolumn{1}{|c|}{\textbf{Split}} &
    \multicolumn{1}{|c|}{\textbf{mAP}} &
    \multicolumn{4}{|c|}{\textbf{Macro over ents}} &
    \multicolumn{4}{|c|}{\textbf{Micro over facts}} \\
    
    \multicolumn{1}{|c|}{} &
    \multicolumn{1}{|c|}{} &
    \multicolumn{1}{|c|}{\textbf{Prec}} &
    \multicolumn{1}{|c|}{\textbf{Rec}} &
    \multicolumn{1}{|c|}{\textbf{F1}} &
    \multicolumn{1}{|c|}{\textbf{Supp}} &
    \multicolumn{1}{|c|}{\textbf{Prec}} &
    \multicolumn{1}{|c|}{\textbf{Rec}} &
    \multicolumn{1}{|c|}{\textbf{F1}} &
    \multicolumn{1}{|c|}{\textbf{Supp}} \\
    
    \hline \hline
    
    CDE-0 & 0.84 &
    100.00 & 0.84 & 0.84 & \num{13.32} &
    100.00 & 0.00 & 0.00 & \num{25255} \\
    
    CDE-25 & 22.06 &
    66.90 & 24.27 & 32.02 & \num{13.32} &
    62.67 & 22.93 & 33.57 & \num{25255} \\
    
    CDE-50 & 29.26 &
    61.41 & 33.14 & 40.29 & \num{13.32} &
    58.47 & 31.52 & 40.96 & \num{25255} \\
    
    CDE-75 & 33.23 &
    57.88 & 38.10 & 43.59 & \num{13.32} &
    55.43 & 3638 & 43.93 & \num{25255} \\
    
    CDE-100 & 35.77 &
    55.62 & 41.48 & 45.28 & \num{13.32} &
    53.49 & 39.72 & 45.59 & \num{25255} \\
    
    \hline
    
    FB-0 & 3.19 &
    100.00 & 03.19 & 3.19 & \num{18.76} &
    100.00 & 0.00 & 0.00 & \num{15312} \\
    
    FB-25 & 27.41 &
    73.46 & 30.56 & 37.06 & \num{18.76} &
    69.28 & 30.15 & 42.02 & \num{15312} \\
    
    FB-50 & 33.39 &
    68.36 & 37.61 & 42.90 & \num{18.76} &
    64.90 & 36.76 & 46.94 & \num{15312} \\
    
    FB-75 & 36.22 &
    64.93 & 41.37 & 45.14 & \num{18.76} &
    62.76 & 40.82 & 49.47 & \num{15312} \\
    
    FB-100 & 38.43 &
    63.08 & 44.34 & 46.97 & \num{18.76} &
    60.98 & 43.51 & 50.79 & \num{15312} \\
    
    \hline
\end{tabular}

    }
    \caption{Final Aggregator results, i.e. final results for the Power model. The results of the Ruler and Texter, whose predictions the Aggregator combines, are also shown for comparison. Although the Aggregator does not outperform its respective Ruler and Texter in terms of F1 score, it does for mAP.}
    \label{tab:5_experiments/5_aggregator/results}
\end{table}

As the mAP values show, the Aggregator performs several percentage points better than the Ruler and Texter on their own, with the improvement on the CDE split being more obvious. However, the relatively small increase on the FB split suggests that the true positives of Ruler and Texter almost coincide there. For the CDE split, on the other hand, manually peeking into the predictions reveals that the improved mAP mainly results from complementary true positives -- and not so much from improved ranks of joint predictions. Looking at the values of simple and attentive Texter, it is also noticeable that the lead of the simple Texter over the attentive Texter shrinks when adding the Ruler. Likewise, the lead of the text sets with many sentences and with high-quality sentences shrinks. Finally, the different aptitudes for Ruler and Overall, the Aggregator results are even similar between the two splits, while previously, models performed significantly better on the FB split.

Two experiments that will be mentioned only briefly here, because of their unspectacular results, concerning the calculation of the Aggregator's confidence as per \autoref{eq:4_approach/3_aggregator/conf_aggregator}: First, in the beginning, experiments were conducted on the computation of the combined confidence $conf_{Aggregator}$ in cases where facts are predicted by Ruler and Texter. As combining methods, calculating the maximum and the mean of $conf_{Ruler}$ and $conf_{Texter}$ were evaluated, but it soon became apparent that summing them up much better accommodates the fact that a fact predicted by Ruler and Texter deserves very high confidence. Second, experiments showed that taking into account the weight parameter $\alpha$ between Ruler and Texter yields only marginal performance improvements in the tenths of a percent range because the confidence values of Ruler and Texter seem to be very comparable after all and thus always yield $\alpha$ values close to 0.5. In detail, Ruler and Texer were both a bit too optimistic about their predictions in the experiments -- but they were equally overconfident.


For example, the class (is, married) should especially to sethe assumption was that sentences containing words such as "married", "spouse" or "single" would help in rating the class (type\_of\_union, marriage)



    \chapter{Experiments}
    \label{ch:5_experiments}
    Chapter~\ref{ch:4_approach} has presented the final Power model that resulted from a series of experiments conducted on the model's Texter, Ruler and Aggregator components. This chapter presents some of those experiments and presents the evaluation results of both, the experiments' and the final model's results. Thereby the evaluation was performed on data originating from the knowledge graphs and text sets presented in Section~\ref{sec:5_experiments/1_base_datasets} that were used as a basis for the \emph{Power datasets} presented in~\ref{sec:5_experiments/2_power_datasets}. The Power datasets in turn were used to train and evaluate Texter, Ruler and Aggregator as shown in sections~\ref{sec:5_experiments/4_texter}~--~\ref{sec:5_experiments/6_aggregator}.


\section{Base Datasets}
\label{sec:5_experiments/1_base_datasets}
Fundamentally, a knowledge graph and textual entity descriptions are required that the Power model can be trained and used upon. The graphs chosen for this work are the FB15K-237~\cite{Toutanova2015ObservedVL} subset of the popular Freebase~\cite{Bollacker2008FreebaseAC} graph and the CoDEx~\cite{Safavi2020CoDExAC} graph, a knowledge graph constructed with the aim of providing an improved benchmark for knowledge graph completion. For both graphs, the IRT datasets created by Hammann~\cite{}, which split the graph facts into training, validation and test sets and provide entity contexts, are used. The following sub sections give an overview of the graphs' dimensions and list some example texts.

\subsection{Knowledge Graphs}
\label{subsec:5_experiments/1_base_datasets/1_knowledge_graphs}
various definitions, here: <KG definition>

fact = head, rel, tail

related terms = knowledge base, triple store, ontology, graph database

closed/open world assumption
entities, relations
"closed world entity", "open world entity"


\subsection{Text Sets}
\label{subsec:5_experiments/1_base_datasets/2_text_sets}
In addition to the open-world fact splits of FB15k-237 and CoDExM, Hamann provides multiple text sets for each split's entities. Thereby, the text sets' contents vary in quality and quantity, ranging from text sets that offer single, very specific entity descriptions to text sets that provide multiple low-quality contexts not necessarily describing the entity directly. Some of the text sets contain plain text, some mark the entity's mentions within the text via special tokens, and some mask the entity mention. Together with the choice between CoDEx-M and FB15k-237, this allows for graph-text combinations suiting different real-world scenarios.

Table~\ref{tab:5_experiments/1_base_datasets/2_text_sets/text_sets_table} lists some example sentences from selected text sets and shortly describes the naming schema behind the text sets' names that will be used throughout this chapter. For example, the text set named "cde-irt-5-marked" indictates that it contains up to five marked sentences for each of the CoDEx-M entities. Thereby, "irt" marks the origin of the sentences: The \emph{IRT sentences} introduced in the Hamann's IRT paper~\cite{} are randomly sampled entity contexts from the English Wikipedia that mention the entity in a more or less meaningful way anywhere in the sentence. In contrast, the \emph{OWE sentences} are very compact entity descriptions, often consisting of only a few words, created by Villmov et al. for their work on open-world KGC~\cite{Shah2019AnOE}. A middle between the vagueness of the IRT sentences and the compactness of the OWE sentences are the \emph{CDE sentences} provided by the authors of the CoDEx paper~\cite{}, which are the entities' first sentence from their respective Wikipedia page. Table~\ref{tab:5_experiments/1_base_datasets/2_text_sets/text_sets_table} in Appendix~\ref{ch:a_appendix} shows the full list of all text sets used during evaluation.

\begin{table}
    \centering
    \input{5_experiments/1_base_datasets/2_text_sets/text_sets_table}
    \caption{Example sentences from some of the text sets - in some text sets the entity mention is marked or masked via special tokens}
    \label{tab:5_experiments/1_base_datasets/2_text_sets/text_sets_table}
\end{table}




\section{Power Datasets}
\label{sec:5_experiments/2_power_datasets}
Theoretically an IRT fact split and matching text set are sufficient for training the Power model. However, when evaluat However, when evaluating the trained model against the IRT split's open-world entities, Power would perform only mediocre as it could only use its Texter component for inference, since the open-world entities do not provide any facts the Ruler can work with. Therefore, considering the intended few-shot scenario, the IRT splits are transformed to \emph{Power splits}, that involve some known facts for the test entities, by dividing the IRT split's open-world facts into so-called \emph{known facts}, that are allowed to be used for rule application during inference, and \emph{unknown facts}, that stay unknown. Meanwhile, the close-world entities' training and validation facts are merged as there is no need for the latter. Figure~\ref{fig:5_experiments/2_power_datasets/splits} illustrates the repartitioning.

\begin{figure}[t]
    \centering
    \includegraphics[width=\textwidth]{5_experiments/2_power_datasets/splits}
    \caption{Repartitioning an IRT fact split into a Power split by merging the train subsets and splitting validation and test subsets into facts are known and unknown during inference, respectively}
    \label{fig:5_experiments/2_power_datasets/splits}
\end{figure}

The percentage of known and unknown facts can be varied when creating the Power split to study the Ruler's effectiveness on entities with more or less given knowledge. Table~\ref{tab:5_experiments/2_power_datasets/power_splits} lists the splits used for the evaluation. In practice, few-shot splits are particularly interesting. Since there are only about 20 facts per entity on average, the fact splits with 5\% and 15\% known facts can be considered few-shot scenarios. The splits with 0\% known facts correspond to an zero-shot open-world scenario. For readability, the Power splits based on the CoDEx-M and FB15k-237 splits are abbreviate to "CDE" and "FB".

\begin{table}[h]
    \centering
    Theoretically an IRT fact split and matching text set are sufficient for training the Power model. However, when evaluating the trained model against the IRT split's open-world entities, Power would perform only mediocre as it could only use its Texter component for inference, since the open-world entities do not provide any facts the Ruler can work with. Therefore, the IRT splits are transformed to \emph{Power splits}, which divide the validation and test facts into so-called \emph{known facts} and \emph{unknown facts}. During inference, only known facts may be used to apply rules, while both known and unknown facts form the ground truth during evaluation - neither known nor unknown validation and test facts are seen during rule mining. IRT separate closed-world training and validation facts are merged as there is no need for the latter. \autoref{fig:5_experiments/2_power_splits/power_split} illustrates the repartitioning.

\begin{figure}[t]
    \centering
    \includegraphics[width=\textwidth]{5_experiments/2_power_splits/power_split}
    \caption{Repartitioning an IRT fact split into a Power split by merging the training fact subsets and splitting open-world validation and test subsets into facts are known and unknown during inference, respectively.}
    \label{fig:5_experiments/2_power_splits/power_split}
\end{figure}

The percentage of known and unknown facts can be varied when creating the Power split to study the Ruler's effectiveness on entities with more or less given knowledge. \autoref{tab:5_experiments/2_power_splits/power_splits_table} lists the splits used for the evaluation. For readability, the Power splits based on the CoDEx-M and FB15k-237 splits are generally referred to and ``CDE'' and ``FB'', respectively. An integer suffix specifies the chosen percentage of known validation and test facts. For example, the CDE-50 Power split denotes the Power split created from the IRT CoDEx-M split with 50\% of the validation and test facts being available for rule application during inference. In practice, the CDE-15 and FB-5 splits are particullarly interesting, because they represent few-shot scenarios with only one or two known test facts per entity. The CDE-0 and FB-0 splits correspond to zero-shot open-world scenarios.

\begin{table}[t]
    \centering
    \input{5_experiments/2_power_splits/power_splits_table}
    \caption{Power splits with varying ratios of known validation and test fact. For example, ``CDE-50'' denotes the CoDEx-M-based Power split with half of the test facts being available for rule application during inference while the FB-0 Power split does not reveal any of the FB15k-237 facts during inference.}
    \label{tab:5_experiments/2_power_splits/power_splits_table}
\end{table}

Along with the creation of the Power split, the question of appropriate metrics for the evaluation arises. While evaluation scenarios for classical fact splits typically consider micro precision, recall and F1 across all facts, this work regards macro precision, recall, F1 and mAP averaged over the test entities, as this comes closer to the goal of producing top predictions for as many entities as possible -- while the micro metrics could achieve good results if the model focused on overall good predictions in favor of entities with few ground truth facts. It should be noted, however, that not further described experiments with the metrics yielded similar results to the macro ones. Mean average precison has been preferred over mean reciprocal rank as a ranking metric, because it does not focus on the single best prediction for an entity, which comes closer to the intended use case of suggesting multiple facts for KGC, as well. For entities for which no predictions are made, precision is mathematically undefined. In this case, the precision for that entity is set to 1 since no incorrect predictions were made. Similarly, recall and mAP are undefined when there are no ground truth facts for an entity. In these rare cases, recall and mAP are set to 1 if no predictions were made for the entity and 0 otherwise.

    \caption{Power splits with varying ratios of known test entities - for example, "CDE-50" denotes the CoDEx-M-based Power split with half of the test entities being known during inference while the FB-0 Power split does not reveal any of the FB15k-237 facts during inference}
    \label{tab:5_experiments/2_power_datasets/power_splits}
\end{table}



\section{Evaluation Metrics}
\label{sec:5_experiments/3_metrics}
When designing and implementing machine learning models, scientists act on experience when it comes to architectural decisions and hyperparameter choices. In the early stages of a model, a trained eye on processing examples and observing the loss curve enable rapid progress. However, as the model matures, it becomes essential to quantify its performance with respect to comprehensive validation and test sets. Besides the selection of appropriate validation and test data, it is important to choose a meaningful metric that fits the problem. For example, when all of a models predictions are equally relevant, one would aim for an overall high precision, whereas a use case that involves a human processing the results manually, such as a web search, for example, one would choose a ranking metric that rewards good results at the top of a list. The purpose of this section is to explain those metrics relevant for this work. Section~\ref{fig:2_basics/4_metrics/1_confusion_matrix} defines basic terms used by the following sections. Sections~\ref{subsec:2_basics/4_metrics/2_accuracy} and~\ref{subsec:2_basics/4_metrics/3_prf} then present the general purpose metrics accuracy, precision, recall and F1 score while Sections~\ref{subsec:2_basics/4_metrics/4_mrr} and~\ref{subsec:2_basics/4_metrics/5_map} discuss the ranking metrics MRR and mAP\@.

\subsection{Confusion Matrix}
\label{subsec:2_basics/4_metrics/1_confusion_matrix}
All considered metrics are defined with the help of terms from a so-called confusion matrix as shown in Figure~\ref{fig:3_basics/4_metrics/1_confusion_matrix}. The matrix emerges from the general scenario in which predictions are made for a set of objects that can be either true or false. A prediction is true if its statement is consistent with reality about the object, also called ground truth, and false if the prediction contradicts reality. An example scenario would be an image recognition that has to determine whether a photo shows a cat or not. Then, four mutually exclusive types of predictions can be distinguished:

\begin{itemize}
    \item \textbf{\emph{True positives}} (TP) are predictions stating that a condition holds true when this is indeed the case. In the image recognition example, this would correspond to the case where the model correctly classifies a cat image as a cat.

    \item \textbf{\emph{False positives}} (FP) are negative predictions about objects where the condition is actually true, e.g. declaring an animal as cat although it is not. This type of error is also referred to as a \emph{Type 1 error}.

    \item \textbf{\emph{False negatives}} (FN) are another kind of erroneous predictions, also referred to as \emph{Type 2 errors}. They represent the case that an element with a true condition was classified as false - was overlooked, so to speak. An example would be a cat image not recognized as a cat.

    \item \textbf{\emph{True negatives}} (TN) are similar to true positives in that they are correct predictions, as well. They consist of rejective predictions about objects where the condition does indeed not apply, for example by recognizing that there is no cat in a cat-less photo.
\end{itemize}

Although it is generally desirable to obtain as many correct predictions as possible, true positives and negates are often differently important and errors of type one and two differently severe. In medicine, for example, not recognizing a disease could be much worse than accidentally diagnosing a healthy person as ill. Conversely, not recognizing guilt in a lawsuit might be less serious than convicting someone who is innocent. In the context of knowledge graph completion under the open-world assumption, the focus lies on true positives since the KGC model cannot make any qualified statements about non-applying facts without negative samples in the graph. Omitting a fact from the prediction only means that the model has too little evidence for that fact, not that it can falsify it.

\begin{figure}[t]
    \centering
    \includegraphics[width=\textwidth]{3_basics/4_metrics/1_confusion_matrix/matrix}
    \caption{Confusion Matrix}
    \label{fig:3_basics/4_metrics/1_confusion_matrix}
\end{figure}

"positive", "negative", "true", "false" predictions, "positive", "negative" elements

cat example
ir scenario


\subsection{Accuracy}
\label{subsec:2_basics/4_metrics/2_accuracy}
One of the most common general-purpose metrics is \emph{accuracy}, which measures a models overall capability to make correct positive and negative predictions. In case of binary classification, accuracy is the rate of correct predictions over all predictions:

\begin{align}
    Accuracy = \frac{TP + TN}{TP + TN + FP + FN}
    \label{eq:2_basics/2_metrics/1_accuracy/accuracy}
\end{align}

Colloquially speaking, accuracy answers the question of how good predictions are in general. It takes values in $[0, 1]$, whereby higher is better. However, although accuracy is a useful and intuitive metric in general, it can be misleading when it comes to inbalanced classes, because a model can simply reach high accuracy by always predicting the predominant class. For example, if nine out of ten ground truth values are false, a model could reach 90\% accuracy by always predicting false. To counteract this, balanced accuracy~\cite{Mower2005PREPMtPR} can be used instead. However, as mention earlier, as negative predictions do not play a big role for KGC models in an open-world scenario, accuracy will only play a minor role in this work, anyway.


\subsection{Precision, Recall and F1}
\label{subsec:2_basics/4_metrics/3_prf}
\emph{Precision}, \emph{recall} and \emph{F1} score focus on the quality of positive predictions. Precision gives an impression of how reliable positive predictions are, recall tells how many of the actual positive elements are declared as such, and F1 is a measure that combines both precision and recall in one value. All three metrics take values between 0 and 1 with higher being better.

Precision is the ratio of true positive predictions to all positive predictions as noted in~\ref{eq:2_basics/4_metrics/3_prf/precision}. In the above cat example, correctly identifying three out of five cats but also classifying one dog as a cat results in a precision of $3 / (3 + 1) = 0.75$. The two missed out cats do not play a role. Precision is also called \emph{positive predictive value} (PPV).

\begin{align}
    Precision = \frac{TP}{TP + FP}
    \label{eq:2_basics/4_metrics/3_prf/precision}
\end{align}

Recall, on the other hand, compares the number of true positives the the number of all ground truth positives as shown in~\ref{eq:2_basics/4_metrics/3_prf/recall}. Looking again at the cat example, identifying three out of five cats in total leads to a recall of $3 / (3 + 2) = 0.6$. The dog that was mistaken as a cat does not count in. Recall is also referred to as \emph{true positive rate} (TPR).

\begin{align}
    Recall = \frac{TP}{TP + FN}
    \label{eq:2_basics/4_metrics/3_prf/recall}
\end{align}

Precision and recall are directly dependent on each other. A cautious model that only predicts positives when it is absolutely sure achieves high precision but low recall. Conversely, it is easy to achieve optimal recall by making positive predictions for all elements, though precision will suffer in that case. The F score serves as a measure that reaches a high value when a reasonable balance between precision and recall is found. Equation~\ref{eq:2_basics/4_metrics/3_prf/f_beta} shows the formula for the general $F_\beta$ score whose parameter $\beta$ determines whether the focus should rather be shifted to precision or recall. Setting $\beta = 1$ yields the $F_1$ score in~\ref{eq:2_basics/4_metrics/3_prf/f_1} as the harmonic mean in which precision and recall are equally weighted.

\begin{align}
    F_\beta &= (1 + \beta^2) \cdot \frac{Precision \cdot Recall}{\beta^2 \cdot Precision + Recall}
    \label{eq:2_basics/4_metrics/3_prf/f_beta} \\
    F_1 &= 2 \cdot \frac{Precision \cdot Recall}{Precision + Recall}
    \label{eq:2_basics/4_metrics/3_prf/f_1}
\end{align}

When including more than one class in the evaluation, for example when predicting for each photo whether it is a cat, a dog or a horse, the question arises how to combine the results of the respective classes. Three of several possibilities are as follows:

\begin{itemize}
    \item Not combining the class results at all preserves the class-wise information but convoluted with a large number of classes.

    \item Merging all classes' predictions and calculating precision, recall and F score over all predictions is called \emph{micro} averaging. Underrepresented classes, which typically show poorer performance due to lack of training data, are less reflected in micro precision, recall and F score.

    \item Calculating each class' scores individually and averaging the class-wise metrics yields \emph{macro} precision, recall and F score. In this case, underrepresented classes have the same impact as classes with many ground truth positives. Macro values are therefore often worse than the corresponding micro values.
\end{itemize}

For very rare classes, there may be zero positives within a subset of the entire data set. If a good model does not predict false positives in this case, precision is undefined. In the context of a macro averaging, precision could be considered 0 because the model does not make a correct prediction, it could be defined as 1 because the model does not make an incorrect prediction, or the class could be excluded from averaging. Similarly, a withholding model might always make negative predictions so that recall is undefined. Therefore, the same considerations must be made for Recall and F Score.


\subsection{Mean Reciprocal Rank}
\label{subsec:2_basics/4_metrics/4_mrr}
The above presented accuracy, precision, recall and F1 metrics are useful in a classification task where no priorization among the predictions is required. However, in an \emph{information retrieval (IR)} scenario, such as a web search, for example, a model might yield a sorted list of predictions where the ranking actually plays a major role. Usually, IR scenarios do not differentiate between positive and negative predictions, but rather between more or less relevant predictions that are returned by decreasing relevance.

In those cases it is more important to rank relevant items as high as possible among the overall results than it is assign the correct probability, or class if there is probability threshold, to each item. When it is most important to receive a correct top-most prediction for each query, the \emph{mean reciprocal rank (MRR)} is the metric of choice. Given the results of $n$ queries it is calculated as per Equation~\ref{subsec:2_basics/2_metrics/3_mrr}, where $rank_i$ is the rank of the top-most relevant item among the predictions of the $i$th query results. Each reciprocal rank, and thus the mean over all reciprocal ranks, lies in $(0, 1]$, with higher being better. If a query result does not contain any relevant item, the reciprocal rank is undefined. Depending on the use case, the object might be skipped or assigned a specific value. A typical application scenario for MRR would be the evaluation of a voice assistant that has to respond with the single most relevant answer it gets from a model.

\begin{align}
    MRR = \frac{1}{n} \sum_{i=1}^{n} \frac{1}{rank_i}
    \label{eq:2_basics/2_metrics/3_mrr/mrr}
\end{align}


\subsection{Mean Average Precision}
\label{subsec:2_basics/4_metrics/5_map}
precision, recall, f1 do not consider order of predictions
power yields probabilities for facts, makes sense to rank probable facts higher
like google search

\[
    AP@n = \frac{1}{GTP} \sum_{k=1}^{n} P@n \cdot rel@n
\]

then, mean over all entities = mean of all average precisions = mAP

\[
    mAP = \frac{1}{N} \sum_{i=1}^{N} AP_i
\]




\section{Texter}
\label{sec:5_experiments/4_texter}
With a Power dataset in place and the evaluatin metrics defined, the next step is training the Power model's components. In case of the Texter, however, an intermediate step is necessary. The Texter is trained in respect to a certain set of relation-tail tuples that make up its classes and requires a fixed number of sentences per entity. Given a fixed number of classes $c$, the $c$ most common relation-tail tuples in the training facts are determined and become the Texter's classes. A Texter dataset is created from the Power dataset that specifies which of the classes hold true for each entity. Furthermore, the Texter dataset contains the specified number of randomly samples sentences for each entity. Table~\ref{tab:5_experiments/4_texter/texter_dataset} shows an excerpt from a Texter dataset.

\begin{table}[h]
    \centering
    \begin{tabular}{ r l r r r r r r r r l }
    \toprule
    
    \multicolumn{1}{c}{\textbf{Ent}} &
    \multicolumn{1}{l}{\textbf{Entity Label}} &
    \multicolumn{8}{c}{\textbf{Classes}} &
    \multicolumn{1}{l}{\textbf{Sentences}} \\
    
    \midrule

    25 & Gabriel Yared    & 1 & 1 & 0 & 0 & 0 & 0 & 0 & 0 & Lebanese composer.        \\
    26 & Bridesmaids      & 0 & 0 & 0 & 0 & 0 & 1 & 1 & 1 & 2011 film by paul feig.   \\
    27 & Portugal         & 0 & 0 & 0 & 0 & 0 & 0 & 0 & 0 & Republic in \dots         \\
    28 & Serpico          & 0 & 0 & 0 & 0 & 0 & 1 & 1 & 1 & 1973 american crime \dots \\
    29 & Italian          & 0 & 0 & 0 & 0 & 0 & 0 & 0 & 0 & Romance language.         \\
    30 & Catherine Keener & 0 & 0 & 1 & 1 & 1 & 0 & 0 & 0 & American actress.         \\
    
    \bottomrule
\end{tabular}

    \caption{Excerpt from a Texter dataset from the FB Power split and the fb-owe-1-clean text set for a Texter with eight classes}
    \label{tab:5_experiments/4_texter/texter_dataset}
\end{table}

In general, the Texter dataset's classes array is very sparse, especially for diverse knowledge graphs without frequently occurring relation-tail tuples. Therefore, it is important to weight the classes when calculating the loss during training to punish false negatives more heavily than false positives. Otherwise, the Texter would learn that always predicting 0 keeps the loss as low as possible. Table~\ref{tab:5_experiments/4_texter/classes} gives an overview of the imbalance between the top 100 classes that were selected for this evaluation. As can be seen, the FB split contains not only more facts than the CDE split, but also fewer diverse facts, which should facilitate the Texter's training.

\begin{table}[h]
    \centering
    \begin{tabular}{| l | r | r | l |}
    \hline
    
    \multicolumn{1}{|c|}{\textbf{Split}} &
    \multicolumn{1}{|c|}{\textbf{Rank}} &
    \multicolumn{1}{|c|}{\textbf{Freq}} &
    \multicolumn{1}{|c|}{\textbf{Class = (Relation, Tail)}} \\

    \hline \hline

    \multirow{3}{*}{CDE}
    & 1   & 19.78 & (country of citizenship, United States of America)   \\
    & 2   & 19.10 & (occupation, writer)                                 \\
    & 3   & 17.06 & (languages spoken, written, or signed, English)      \\

    \hline

    \multirow{3}{*}{CDE}
    & 98  & 1.08  & (genre, hip hop music)                               \\
    & 99  & 1.07  & (member of, International Finance Corporation)       \\
    & 100 & 1.07  & (record label, Columbia Records)                     \\

    \hline \hline

    \multirow{3}{*}{FB}
    & 1   & 24.34 & (\dots/webpage/category, /m/08mbj5d)                 \\
    & 2   & 19.67 & (/people/person/gender, male organism)               \\
    & 3   & 17.38 & (\dots/marriage/type\_of\_union, marriage)           \\

    \hline

    \multirow{3}{*}{FB}
    & 98  & 1.35  & (\dots/award\_nomination/award, Screen Actors \dots) \\
    & 99  & 1.32  & (/film/film/genre, fantasy)                          \\
    & 100 & 1.31  & (\dots/film\_release\_region, Israel)                \\

    \hline
\end{tabular}

    \caption{Most and least common classes on the CDE and FB splits. Frequencies are given in percent.}
    \label{tab:5_experiments/4_texter/classes}
\end{table}

Evaluation is generally performed against the Texter dataset's test subset, following the form shown in Table~\ref{tab:5_experiments/4_texter/texter_dataset}. That means that the Texter's predictions are evaluated against the classes it was trained on and that the results are comparable to the validation results during training. The metrics chosen are macro precision, recall and F1 over the classes. This is rather demanding, as rare classes are equally weighted as the common classes, that represent more facts, but it follows the objective of predicting all classes as good as possible. For the final comparison with Ruler and Aggregator, however, the Texter is also evaluated against the Power split's test facts, which includes facts the Texter is unable to predict, using the common metrics discussed before in Section~\ref{sec:5_experiments/3_metrics}.

The following Subsection~\ref{subsec:5_experiments/4_texter/1_zero_rule} applies the presented metrics to several baselines to give an idea of the problem's complexity. Given the reference values from the baselines, Subsections~\ref{subsec:5_experiments/4_texter/2_static} and~\ref{subsec:5_experiments/4_texter/3_context} present two ways of implementing the Texter's embedding block using static and contextual word embeddings, whereby the latter is used in the final Power implementation described in Chapter~\ref{ch:4_approach}. To distinguish the two variants, they are also referred to as static and contextual Texter, respectively. Both sections include several experiments that led to the final versions of the models. In addition, each experiment compares the simple and the attentive versions of the Texter to see which variations affect the attention mechanism of the latter.

\subsection{Zero Rule Baselines}
\label{subsec:5_experiments/4_texter/1_zero_rule}
Zero-rule baselines are naive models that give an idea of how well a problem could be tackled by a naive approach, such as simply guessing random solutions. Any model performing worse than a zero-rule baseline cannot be considered useful. For example, in a coin toss, one would already achieve 50\% accuracy, precision and recall by randomly guessing between heads and tails. A coin-flip predicting model that does not exceed that performance would not be considered useful.

The purpose of this subsection is to determine the absolute baseline for the classification problem at hand that the Texter has to exceed. Therefore, the best zero-rule for the binary multi-label classification the Texter dataset poses has to be found. Randomly guessing with a 50:50 chance which of an entity's classes hold true is one naive strategy, but there are better ones when it comes to unbalanced classes such as the Texter dataset's ones. For example, for a class with an 80:20 chance to be true, randomly guessing true in 80 of 100 cases leads to better accurracy than choosing equally between true and false.

In general, performance for a random guessing strategy can be calculated. For example, accuracy in an unbalanced classification scenario with two possible outcomes 0 and 1 can be calculated as $P(\hat{y} = y) = P(\hat{y} = 0) \cdot P(y = 0) + P(\hat{y} = 1) \cdot P(y = 1)$ where $y$ is the ground truth outcome and $\hat{y}$ is the predicted outcome. While $P(y = 0)$ and $P(y = 1)$ depend on the regarded class' outcome distribution, the values of $P(\hat{y} = 0)$ and $P(\hat{y} = 1)$, and thus the probability to predict the ground truth value, depend on the guessing strategy. In this evaluation, four potential guessing strategies were considered with respect to each class:

\begin{itemize}
    \item \emph{Uniformly} guessing between 0 and 1, i.e. $P(\hat{y} = 0)$ = $P(\hat{y} = 1)$ = 0.5

    \item \emph{Stratified} guessing, which means that each outcome is guessed according to its frequency, i.e. $P(\hat{y} = 0)$ = $P(y = 0)$ and $P(\hat{y} = 1)$ = $P(y = 1)$

    \item Always guessing the \emph{most frequent} outcome, i.e.  $P(\hat{y} = 0)$ = 1 and $P(\hat{y} = 1)$ = 0 if 0 is the most common outcome

    \item \emph{Constantly guessing 1}, i.e. $P(\hat{y} = 0)$ = 0 and $P(\hat{y} = 1)$ = 1
\end{itemize}

Similarly, precision, recall and F1 could also be calculated for each class of the Texter dataset. Averaging each class' metrics over all classes would finally yield each zero-rule strategy's performance for the whole Texter dataset. For this evaluation, the zero-rule strategies were implemented and measured instead. The values in \autoref{tab:5_experiments/3_texter/1_zero_rule/results} were gathered by running each of the four zero-rule strategies ten times and averaging the values. For each zero-rule strategy, performance is shown for the most common class, the least common class and the average over all classes. As shown before, the Texter datasets' classes arrays are very sparse, so it was expected that constantly predicting the most common output false for all classes would lead to the highest accurracy. However, since F1 is the more relevant metric in regard to predicting true positives, constantly predicting true was expected to be the best zero-rule strategy.

\begin{table}[t]
    \makebox[\textwidth][c]{
        \input{5_experiments/3_texter/1_zero_rule/results}
    }
    \caption{Evaluation of various zero-rule baselines on the CDE and FB splits. The best average values per split are marked bold (column-wise). Thus, uniform sampling sets the baseline on the CDE split (20.19\% F1), while constantly predicting true leads to the best result on the FB split (9.43\% F1).}
    \label{tab:5_experiments/3_texter/1_zero_rule/results}
\end{table}

As it turned out, constantly predicting true for all classes is indeed the best random guessing strategoy for the Texter dataset derived from the FB split, leading to an F1 score of 9.43\% which serves as the absolute baseline for any Texter predicting facts for the FB split. Surprisingly, the best zero-rule strategy for the Texter dataset derived from the CDE split is randomly guessing between true and false, leading to over 20.19\% F1 score, which is almost twice as much as the value achieved when constantly guessing true. Those two percentages have to be kept in mind when regarding any performances achieved by actually learning models.


\subsection{Static Word Embeddings}
\label{subsec:5_experiments/4_texter/2_static}
First, the embedding block in both, the simple and the attentive Texter, is investigated when using static, pre-trained word embeddings, i.e. a word is embedded independent from where it occurs in the sentence. The advantages include ease of implementation using a simple look-up table, few hyperparameters and quick training. However, the sentence embedding cannot properly capture the meaning of markings and maskings via special tokens such as those in sentences from the "cde-irt-5-marked" or "cde-irt-5-masked" text sets exemplified in Table~\ref{tab:5_experiments/1_data_sources/2_text_sets/text_sets_table}. Therefore, all experiments in this section are conducted on the clean text sets containing no special tokens. During training, the settings that proved to be best during the experiments were used. In particular, these are the use of an NLP-based tokenizer, pre-trained fastText~\ref{} word embeddings that are finetuned during training, mean pooling to form sentence embeddings, the sigmoid function in the attention block in the case of the attentive texter, class weights in the computation of the loss, and the Adam optimizer to apply gradients during backpropagation. Table~\ref{tab:5_experiments/4_texter/2_static/results} shows the final results over all clean text sets for both the simple and the attentive versions of the Texter. As for all following tables presenting Texter evaluation results, for compactness, only the name of the text set is given for the respective Texter dataset, because it implies the matching Power split.

\begin{table}[t]
    \centering
    \input{5_experiments/4_texter/2_static/results}
    \caption{Final evluation results for the simple and the attentive Texter using static word embeddings - for each text set, the better model is marked in terms of precision, recall and F1}
    \label{tab:5_experiments/4_texter/2_static/results}
\end{table}

The numbers show that both the simple and the attentive Texter, outperform the zero rule baselines by far. However, the attentive model does not outperform the simple version as hoped. On most datasets, the attentive model achieves better precision, but the simple Texter has higher recall. Overall, the simple modell performs better, especially on datasets with multiple sentences per entity which should have been in favor of the attentive model. Interestingly, both Texters perform best given the short OWE sentence due to high recall scores, which implies that a single, good entity description is more useful than many vague contexts for knowledge graph completion. Similar to the OWE sentences, the CDE sentences from the entities' Wikipedia introductions perform best on the CDE. However, the results are not far ahead of those for the text set with 30 IRT sentences, so that the observation could also be formulated in reverse: 15 randomly sampled entity contexts are just as good as a selected high-quality description.

In addition to the final evaluation results on the test data, Figure~\ref{fig:5_experiments/4_texter/2_static/plot_valid_curves} shows the development of the F1 score for different text sets during training on the FB graph. After 200 episodes of training, the simple and attentive versions of the Texter converge against similar values. The only notable difference between the two is that the simple model reaches its optimal training time earlier before it starts overfitting on most datasets. Thereby, the IRT text sets with few sentences are more prone to overfitting than the ones with 15 or 30 sentences per entity. The short OWE texts, on the other hand, seem to be immune to overfitting. Although training could be stopped after 50 episodes without major defficiencies in this case, all following experiments are performed with 200 episodes to ensure that all models reach their full potential even if changes lead to slower training.

\begin{figure}[t]
    \centering
    \input{5_experiments/4_texter/2_static/plot_valid_curves/plot_valid_curves}
    \caption{Validation F1 curves during training the Texter on the FB split with 1 OWE sentence (red), 30 IRT sentences (orange), 15 IRT sentences (yellow), 5 IRT sentences (black) or 1 IRT sentence (green) per entity}
    \label{fig:5_experiments/4_texter/2_static/plot_valid_curves}
\end{figure}

What is not visible in Figure~\ref{fig:5_experiments/4_texter/2_static/plot_valid_curves} is the information how the average F1 score is composed of the class-wise F1 scores, which is addressed by Figure~\ref{fig:5_experiments/4_texter/2_static/plot_class_curves}. It compares the F1 scores between the most common classes, the least common classes, and the average F1 score over all classes. The graphs reveal that predictions for common classes are much more reliable. The three most common classes on the CDE split, with frequencies of almonst 20\% each, reach performances between 50\% and 80\%, whereas the three least common classes, with frequencies of around 1\% each, reach strongly divergent values, from 0\% to 60\%. Despite the tendency that frequent classes perform better, however, a class' performance cannot be calculated directly from its frequence. The least common classes have very similar frequencies but perform very differently. For the second least common class prediction does not work at all, while the third least common class performs even better than the third most common class. There are also significant performance differences between the three most frequent classes although the frequencies are similar. Another, more obvious, finding is that frequent classes are learned within fewer episodes than less common classes. Interestingly, the attentive Texter seems to learn faster than the simple Texter. Finally, the graph shows that both Texters converge to similar values for all classes, as might have been assumed from the similar macro F1 values.

\begin{figure}[t]
    \centering
    \input{5_experiments/4_texter/2_static/plot_class_curves/plot_class_curves}
    \caption{Class-wise validation F1 curves during training the Texter on the CDE split with 5 IRT sentences per entity - comparison between the most common classes (red, orange, yellow), the least common classes (green, blue, purple) and the average value over all classes (black)}
    \label{fig:5_experiments/4_texter/2_static/plot_class_curves}
\end{figure}

Looking at the results for the other datasets shown in Tables~\ref{fig:a_appendix/static_classes_1}~--~\ref{fig:a_appendix/static_classes_3} in Appendix~\ref{ch:a_appendix}, similar patterns can be observed on other text sets with only a few exceptions. However, although there are no major differences between the simple and the attentive Texter, it is noticeable that the results for rare classes do not improve steadily when increasing the number of sentences in case of the IRT text sets. Multiple training runs on the same data showed that this was not due to different initializations of the model. Instead, separate experiments showed that it is the random selection of IRT sentences that affects the performance of individual rare classes. In the respective experiment, new fb-irt-1-clean text sets were built by repeatedly selecting random sentences from the larger fb-irt-30-clean dataset. The overall F1 score over all classes, however, stayed the same for all runs.

With the final results for static word embeddings at hand, the following subsections show what impact various model changes and hyperparameters have. The affected model components are examined in the order in which they are invoked during training, beginning with the tokenizer and ending with the optimizer. Finally, the attentive Texter's attention mechanism is investigated to explain why it does not improve upon the simple Texter as hoped.

\subsubsection{Changing the tokenizer}
\label{subsubsec:5_experiments/4_texter/2_static/1_tokenizer}
\input{5_experiments/4_texter/2_static/1_tokenizer/tokenizer}

\subsubsection{Initializing word embeddings randomly}
\label{subsubsec:5_experiments/4_texter/2_static/2_emb_size}
\input{5_experiments/4_texter/2_static/2_emb_size/emb_size}

\subsubsection{Using pre-trained word embeddings}
\label{subsubsec:5_experiments/4_texter/2_static/3_pre_trained}
\input{5_experiments/4_texter/2_static/3_pre_trained/pre_trained}

\subsubsection{Freezing pre-trained embeddings}
\label{subsubsec:5_experiments/4_texter/2_static/4_update_vectors}
\input{5_experiments/4_texter/2_static/4_update_vectors/update_vectors}

\subsubsection{Pooling}
\label{subsubsec:5_experiments/4_texter/2_static/5_pooling}
\input{5_experiments/4_texter/2_static/5_pooling/pooling}

\subsubsection{Activation Function}
\label{subsubsec:5_experiments/4_texter/2_static/6_activation}
\input{5_experiments/4_texter/2_static/6_activation/activation}

\subsubsection{Appplying class weights}
\label{subsubsec:5_experiments/4_texter/2_static/7_weight_factor}
\input{5_experiments/4_texter/2_static/7_weight_factor/weight_factor}

\subsubsection{Choosing an optimizer}
\label{subsubsec:5_experiments/4_texter/2_static/8_optimizer}
\input{5_experiments/4_texter/2_static/8_optimizer/optimizer}

\subsubsection{Inspecting the attention mechanism}
\label{subsubsec:5_experiments/4_texter/2_static/9_attention}
\input{5_experiments/4_texter/2_static/9_attention/attention}


\subsection{Contextual Word Embeddings}
\label{subsec:5_experiments/4_texter/3_context}
Modern NLP models continue where classical models with static word embedding reach their limits when it comes to long sentences in which the relationship between the words is even more important for the words', and thus, the overall sentence's meaning. For this purpose, the particularly successful transformers use internal attention mechanisms that learn for each word which other words in the sentence are particularly relevant and should therefore be included in the word's embedding. This results in a context-dependent embedding for each occurrence of the word.

Besides embedding usual words and some standard tokens that represent unknown words and paddings, the here used DistilBERT also supports the special [CLS] and [SEP] tokens introduced by the BERT model. The [CLS] token's purpose is to capture the meaning of the sentence as a whole during training. In addition, the Texter supports the definition of custom tokens, such as the ones used for markings or maskings in the IRT text sets. This additional information should lead to a significant performance increase, even when the entity mention is masked, as it preserves the information what the surrounding sentence is about.

For the Texter's downstream architecture, the change from a lookup table for embeddings to the use of DistilBERT does not mean a big change as it only affects the embedding block -- the classification block and, in the case of the attentive version, the attention block remain untouched. In particular, design choices from the experiments with static word embeddings, such as the usage of the sigmoid function in the attention block, are kept. During training, however, the optimizer has to be adjusted to accommodate the deep transformers within the embedding block.

\begin{table}
    \makebox[\textwidth][c]{
        \input{5_experiments/3_texter/3_context/results}
    }
    \caption{Final evaluation of the contextual Texter on all text sets. Results of the static Texter are given for comparison. The contextual Texter is evaluated against the Texter dataset's test subset (F1) and against all facts from the respective split (F1 all, mAP all). The contextual Texter outperforms the simple Texter in general, especially when leveraging markings in the text. The attentive Texter profits more from contextual word embeddings. Still, the simple Texter performs better in terms of mAP.}
    \label{tab:5_experiments/3_texter/3_context/results}
\end{table}

After the switch to DistilBERT, training of the Texter takes longer due to the large number of parameters in the transformer, but can be terminated after 50 epochs. \autoref{tab:5_experiments/3_texter/3_context/results} shows the evaluation results for the final Texter after that time. In addition to the previously regarded macro F1 score over all classes, \autoref{tab:5_experiments/3_texter/3_context/results} also provides F1 and mAP over the entities from the Power split -- including those the Texter cannot predict. In addition to the contextual Texters' evaluation results, the static Texters' final results from \autoref{subsec:5_experiments/3_texter/2_static} are given for comparison where available. Furthermore, \autoref{ch:a_appendix} contains the detailed \autoref{tab:a_appendix/context_final_prec_rec} that provides the precision and recall values.

The final evaluation results reveal some interesting facts: First, the contextual Texter performs better on clean text sets than the static Texter most of the time, but not always. While the contextual Texter performs better on almost all text sets containing IRT sentences, it does not for the CDE and OWE sentences. Second, the CDE and OWE text sets show best, that the attentive Texter benefits more from contextual word embeddings than the simple model. Third, it is obvious that markings bring great improvements, as was expected, while the masked text sets are in between the clean and marked ones in terms of F1. In terms of mAP, however, the masked text sets lead to better predictions than their marked counterparts in multiple cases. Fourth, the evaluating against all facts from the Power split leads to significantly worse results on the FB split than it does on the CDE split, which is probably due to the reason that the FB classes cover a smaller portion of the larger FB split.

Overall, it can be stated, that the Texter performs better when using contextual word embeddings, especially when marked texts are available. Therefore, the contextual Texter is the default in the Power model. However, depending on the given text set as well as hardware support, static word embeddings might be a noteworthy alternative. When comparing the simple and attentive versions of the contextual Texter, the simple version yields slightly better results. Still, the attentive Texter is kept due to its ability to explain its text-based decision to a certain point.

In the following, subsections~\ref{subsubsec:5_experiments/3_texter/3_context/1_sent_len}~--~\ref{subsubsec:5_experiments/3_texter/3_context/3_optimizer} will look at some experiments on the updated embedding block and the adjusted optimizer used to train the deep contextual Texter.

\subsubsection{Varying sentence length}
\label{subsubsec:5_experiments/3_texter/3_context/1_sent_len}
\input{5_experiments/3_texter/3_context/1_sent_len/sent_len}

\subsubsection{Pooling}
\label{subsubsec:5_experiments/3_texter/3_context/2_pooling}
\input{5_experiments/3_texter/3_context/2_pooling/pooling}

\subsubsection{Optimizer}
\label{subsubsec:5_experiments/3_texter/3_context/3_optimizer}
\input{5_experiments/3_texter/3_context/3_optimizer/optimizer}




\section{Ruler}
\label{sec:5_experiments/5_ruler}
While the Texter processes the text information attached to the knowledge graph's entities, the Ruler exploits patterns in the graph structure to predict missing facts. Given an entity $x$ with a set of known facts $K$ containing facts of the form $(x, rel_k, tail_k)$ with $1 <= k <= |K|$, $rel_k \in R$, and $tail_k \in E$ being any relation or entity, respectively, the Ruler leverages entity-related rules of the form $(x, rel_m, tail_m) <= (x, rel_k, tail_k)$ with $1 <= m <= |M|$ to predict a set of missing facts $M$. Therefore, the rules required for the inference process have to be mined from the knowledge graph, beforehand. That rule mining process can be viewed as the equivalent to the Texter's training process and some paragraphs in this thesis will refer to rule mining as ``training'' a Ruler. Compatible to the Texter, the Ruler is limited to the prediction of facts $(x, rel, tail)$ that contain the query entity as their head, as well. However, when the Ruler is applied to all entities $e \in E$, all facts of the form $(e, rel, x)$ are predicted at some point as far as the mined rules support it.

For rule mining, AnyBURL, the bottom-up rule mining algorithm, by Christian Meilicke et al.~\cite{Meilicke2019AnytimeBR} is used. It is an anytime algorithm, meaning that it can be interrupted anytime and still yield valid results. Bottom-up rule mining refers to the fact that the algorithm starts with concrete paths in the graph and tries to generalize those paths to rules instead of coming up with rules initially and searching for evidence later, which would be a top-down approach. Out of all possible Horn rules that might describe patterns in the graph, AnyBURL is restricted to rules that can be generalized from so-called \emph{ground path rules}. A ground path rule does not contain variables, but only constants, and must not contain any cycles in its body. \autoref{eq:4_approach/2_ruler/ground_path_rule} describes the general form of a ground path rule of length $n$, meaning that it consists out of the head fact and $n$ body facts.

\begin{align}
(c_0, h, c_1)
    \Leftarrow (c_1, b_1, c_2), \dots, (c_n, b_n, c_{n+1}) &&
    c_k \neq c_l \forall k, l \in \{1, \dots, n+1\}, k \ne l
    \label{eq:4_approach/2_ruler/ground_path_rule}
\end{align}

Notably, despite the rule body being free of cycles, the ground path rule as a whole can still be cyclic if $c_0 = c_{n+1}$. Ground path rules are derived directly from randomly sampled paths in the graph and are subsequently generalized to rules that replace some of the constants with variables. If further supporting paths can be found for a general rule, it is kept. In their paper on AnyBURL, Meilicke et al. show that any rule that can be generalized from a ground path rules can be generalized to one of the three rule types formulated in Equations~\ref{eq:4_approach/2_ruler/c}~--~\ref{eq:4_approach/2_ruler/ac2}. Thereby, $C$-type rules can only be generalized from cyclic ground path rules, $AC2$ rules can only be generalized from acyclic ground path rules and $AC1$ can be generalized from both, cyclic and acyclic ground path rules. The following paragraphs outline the core algorithm used to mine such rules and derive some example rules from the small graph introduced in \autoref{ch:1_introduction}. \autoref{fig:4_approach/2_ruler/rule_graph} shows an annotated subset of the graph that illustrates the rules.

\begin{align}
    C   && (Y, h, X)   &\Leftarrow (X, b_1, A_2), \dots, (A_n, b_n, Y)
    \label{eq:4_approach/2_ruler/c} \\
    AC1 && (c_0, h, X) &\Leftarrow (X, b_1, A_2), \dots, (A_n, b_n, c_{n+1})
    \label{eq:4_approach/2_ruler/ac1} \\
    AC2 && (c_0, h, X) &\Leftarrow (X, b_1, A_2), \dots, (A_n, b_n, A_{n+1})
    \label{eq:4_approach/2_ruler/ac2}
\end{align}

\begin{figure}[t]
    \centering
    \includegraphics{4_approach/2_ruler/rule_graph}
    \caption{Subset of the previously introduced example graph with highlighted facts that form an acyclic (red + green) and a cyclic (red + blue) path.  ``Amsterdam'' and ``Netherlands'' have been abbreviated to ``AMS'' and ``NL''.}
    \label{fig:4_approach/2_ruler/rule_graph}
\end{figure}

Essentially, AnyBURL repeatedly samples random paths from the graph, generalizes them to all possible rule types, looks for further paths that match the gained rules and keeps those rules it finds further evidence for. For example, in search of rules of length two, i.e. rules that have a body consisting of two facts, AnyBURL might randomly sample the two paths in~\ref{eq:4_approach/2_ruler/path_1} and~\ref{eq:4_approach/2_ruler/path_2} from the graph. Note, that parentheses denote entities, brackets denote relations and that the path does not need to follow directed edges in the graph. Furthermore, a close look at the paths reveals that the second path is cyclic as it starts and ends at the entity ``Dutch''.

\begin{align}
(Dutch)
    \leftarrow [speaks] - (Ed) - [married~to] \rightarrow (Lisa) - [born~in] \rightarrow (AMS)
    \label{eq:4_approach/2_ruler/path_1} \\
    (Dutch) \leftarrow [speaks] - (Ed) - [lives~in] \rightarrow (NL) - [has~lang] \rightarrow (Dutch)
    \label{eq:4_approach/2_ruler/path_2}
\end{align}

From those paths, AnyBURL would then derive the constant-only ground path rules~\ref{eq:4_approach/2_ruler/acyclic_ground_path} and~\ref{eq:4_approach/2_ruler/cyclic_ground_path} by taking the path's first part as the rule's head and the remaining parts to form the rule's body.

\begin{align}
(Ed, speaks, Dutch)
    &\Leftarrow (Ed, married~to, Lisa), (Lisa, born~in, AMS)
    \label{eq:4_approach/2_ruler/acyclic_ground_path} \\
    (Ed, speaks, Dutch) &\Leftarrow (Ed, lives~in, NL), (NL, has~lang, Dutch)
    \label{eq:4_approach/2_ruler/cyclic_ground_path}
\end{align}

The acyclic ground path rule in~\ref{eq:4_approach/2_ruler/acyclic_ground_path} can be generalized to the $AC1$ rule~\ref{eq:4_approach/2_ruler/acyclic_ac1} and the $AC2$ rule~\ref{eq:4_approach/2_ruler/acyclic_ac2} while the cyclic ground path rule~\ref{eq:4_approach/2_ruler/cyclic_ground_path} can be generalized to the $C$ rule~\ref{eq:4_approach/2_ruler/cyclic_c} and the $AC1$ rule~\ref{eq:4_approach/2_ruler/cyclic_ac1}.

\begin{align}
    AC1 && (X, speaks, Dutch) &\Leftarrow (X, married~to, A_2), (A_2, born~in, AMS)
    \label{eq:4_approach/2_ruler/acyclic_ac1} \\
    AC2 && (X, speaks, Dutch) &\Leftarrow (X, married~to, A_2), (A_2, born~in, A_3)
    \label{eq:4_approach/2_ruler/acyclic_ac2} \\
        C   && (X, speaks, Y) &\Leftarrow (X, lives~in, A_2), (A_2, has~lang, Y)
    \label{eq:4_approach/2_ruler/cyclic_c} \\
    AC1 && (X, speaks, Dutch) &\Leftarrow (X, lives~in, A_2), (A_2, has~lang, Dutch)
    \label{eq:4_approach/2_ruler/cyclic_ac1}
\end{align}

Next, every rule candidate is scored by looking for further paths that match the rule's body and checking whether the graph also contains the fact predicted by the rule, i.e. whether the rule is correct in that case. Thereby, the number of paths that match the rule body is called the rule's \emph{support} while the ratio of times the rule is correct over its total support is called \emph{confidence}. Taking the cyclic rule~\ref{eq:4_approach/2_ruler/cyclic_c} as an example, AnyBURL would search for further evidence and find the path $(Lisa) - [lives~in] \rightarrow (NL) - [has~lang] \rightarrow (Dutch)$ that matches the rule body, increasing the rule's support to two, so far. However, the example graph does not contain the rule's predicted fact $(Lisa, speaks, Dutch)$, so the rule's support drops from 1 to $\frac{1}{2}$. Since rules that only apply to a single case or only once in every thousandth case are not very useful, AnyBURL drops rules with a support of 1 or confidence below 0.0001 by default~\cite{AnyBURL}. It is noteworthy that, although some rules are more general than others, such as~\ref{eq:4_approach/2_ruler/acyclic_ac2} compared to~\ref{eq:4_approach/2_ruler/acyclic_ac1}, the more specific ones are still kept as they might end up with higher confidence for their special case during the ongoing mining process.

The process described by the above example is repeated until only a few new rules of the same length $n$ can be found. AnyBURL then continues its search for rules of length $n + 1$ until it terminates after a fixed number of time steps. \autoref{code:anyburl} shows the slightly adjusted pseudocode from the AnyBURL paper. The sampling and scoring process discussed above is implemented as the body of the inner while loop. The outer for loop implements the repeated check for the saturation of rules of the current length and the eventual proceeding to rules of increased length.

\begin{listing}[t]
    \begin{lstlisting}
        AnyBURL(G, sat, Q, i, ts):
            n = 2
            R = $\emptyset$
            for i times:
                $R_s = \emptyset$
                start = current_time()
                while current_time() < start + ts:
                    p = sample_path(G, n)
                    $R_p$ = generate_rules(p)
                    for $r$ in $R_p$:
                        score($r$)
                        if $Q$($r)$:
                            $R_s$ = $R_s \cup {r}$

                $R_s^{'}$ = $R_s \cap R$
                if $|R_s^{'}| / |R_s| > sat$:
                    n = n + 1
                $R$ = $R \cup R_s$

            return $R$
    \end{lstlisting}
    \caption{The AnyBURL rule mining algorithm takes a graph $G$, a saturation level $sat$, a quality criterion $Q$, and a number of iterations $i$, each of a timespan $ts$, as input and produces a ruleset $R$.}
    \label{code:anyburl}
\end{listing}

A walk through the pseudocode reads as follows: Given the Graph $G$, the saturation threshold $s$, the quality criterion $Q$, a number of iterations $i$ and the timespan $ts$ each iteration endures, AnyBURL starts with an empty ruleset $R$, that will be extended after each iteration and returned in the end. The initial length of the randomly sampled paths is $n=2$, allowing to find the shortest possible rules of length 1. During the first iteration of duration $ts$, AnyBURL fills the ruleset $R_s$, which keeps the rules found in the current iteration, by repeatedly sampling paths, generating rules from the paths, scoring the resulting rules, and keeping those with sufficient support and confidence. At the end of the iteration, when the timespan $ts$ has passed, $R_s^{'}$ is calculated as the set of rules mined during the iteration that were already known. If the share of already known rules mined during the current iteration exceeds the saturation threshold, the algorithm starts searching for rules of increased length. Otherwise, it continues with the current length. In both cases, the iteration's rules are added to the overall ruleset $R$. If the specified number of total iterations is reached, AnyBURL terminates and returns the mined rules $R$. In practice, AnyBURL saves the mined rules in a text file at the end and at configurable points during mining.

With the stored rules from AnyBURL in place, the Ruler is prepared for inference. Conceptually, given an entity and its known facts, the Ruler loads the rules, filters out further rules that do not meet the Ruler's quality demands, and applies the remaining, useful rules to all known facts. All rules that can be applied successfully are kept together with their confidence. From all the facts predicted by the applied rules, already known facts from the existing graph are filtered out. The remaining facts are sorted by confidence and returned to the user -- together with the rules that predicted them as an explanation for the user. If multiple rules predict the same fact, the fact is assigned the highest confidence of those rules and is returned together with all of them. The Ruler's extra quality criterion mentioned above further restricts the considered rules to those with confidence greater 50\%, because AnyBURL's minimum confidence threshold of 0.0001 allows many rules that predict false positives. For open-world entities, this algorithm implies an empty result set as no rule can cover an entity that is not connected to any other entity and all the facts predicted for the train entities will be filtered out. In those cases, the Power model has to rely solely on the Texter.



\section{Aggregator}
\label{sec:5_experiments/6_aggregator}
The aggregator has the task of merging the predicted facts from Ruler and Texter. As envisioned in \autoref{sec:4_approach/3_aggregator} and illustrated in \autoref{fig:4_approach/3_aggregator/lucy}, it was hoped that merging the facts leads to higher average precision because facts predicted by both components are likely to be correct and should be ranked higher. In addition, the Aggregator should be able to estimate how reliable the predictions of Ruler and Texter are in relation to each other, which is implemented in the form of the weight parameter $\alpha$ as described in \autoref{eq:4_approach/3_aggregator/conf_aggregator}.

\autoref{tab:5_experiments/5_aggregator/results} shows the final evaluation results for the Aggregator, and thus the final evaluation results for the Power model, for a number of graph-text combinations. As fact splits, the splits with 50\% known test facts were chosen, as for the final Ruler evaluation in \autoref{sec:5_experiments/4_ruler}. The respective results for the CDE-50 and FB-50 splits from \autoref{tab:5_experiments/4_ruler/results} were taken over into \autoref{tab:5_experiments/5_aggregator/results} for easier comparability. Similarly, the chosen text sets are the ones from the final Texter evaluation in \autoref{subsec:5_experiments/3_texter/3_context}. Again, \autoref{tab:5_experiments/5_aggregator/results} duplicates the respective results from \autoref{tab:5_experiments/3_texter/3_context/results} for ease of comparison. The last two columns then contain the new Aggregator measurements for the combination of the corresponding Ruler and Texter.

\begin{table}[t]
    \makebox[\textwidth][c]{
        \begin{tabular}{| l | r | r | r | r | r | r | r | r | r |}
    \hline
    
    \multicolumn{1}{|c|}{\textbf{Split}} &
    \multicolumn{1}{|c|}{\textbf{mAP}} &
    \multicolumn{4}{|c|}{\textbf{Macro over ents}} &
    \multicolumn{4}{|c|}{\textbf{Micro over facts}} \\
    
    \multicolumn{1}{|c|}{} &
    \multicolumn{1}{|c|}{} &
    \multicolumn{1}{|c|}{\textbf{Prec}} &
    \multicolumn{1}{|c|}{\textbf{Rec}} &
    \multicolumn{1}{|c|}{\textbf{F1}} &
    \multicolumn{1}{|c|}{\textbf{Supp}} &
    \multicolumn{1}{|c|}{\textbf{Prec}} &
    \multicolumn{1}{|c|}{\textbf{Rec}} &
    \multicolumn{1}{|c|}{\textbf{F1}} &
    \multicolumn{1}{|c|}{\textbf{Supp}} \\
    
    \hline \hline
    
    CDE-0 & 0.84 &
    100.00 & 0.84 & 0.84 & \num{13.32} &
    100.00 & 0.00 & 0.00 & \num{25255} \\
    
    CDE-25 & 22.06 &
    66.90 & 24.27 & 32.02 & \num{13.32} &
    62.67 & 22.93 & 33.57 & \num{25255} \\
    
    CDE-50 & 29.26 &
    61.41 & 33.14 & 40.29 & \num{13.32} &
    58.47 & 31.52 & 40.96 & \num{25255} \\
    
    CDE-75 & 33.23 &
    57.88 & 38.10 & 43.59 & \num{13.32} &
    55.43 & 3638 & 43.93 & \num{25255} \\
    
    CDE-100 & 35.77 &
    55.62 & 41.48 & 45.28 & \num{13.32} &
    53.49 & 39.72 & 45.59 & \num{25255} \\
    
    \hline
    
    FB-0 & 3.19 &
    100.00 & 03.19 & 3.19 & \num{18.76} &
    100.00 & 0.00 & 0.00 & \num{15312} \\
    
    FB-25 & 27.41 &
    73.46 & 30.56 & 37.06 & \num{18.76} &
    69.28 & 30.15 & 42.02 & \num{15312} \\
    
    FB-50 & 33.39 &
    68.36 & 37.61 & 42.90 & \num{18.76} &
    64.90 & 36.76 & 46.94 & \num{15312} \\
    
    FB-75 & 36.22 &
    64.93 & 41.37 & 45.14 & \num{18.76} &
    62.76 & 40.82 & 49.47 & \num{15312} \\
    
    FB-100 & 38.43 &
    63.08 & 44.34 & 46.97 & \num{18.76} &
    60.98 & 43.51 & 50.79 & \num{15312} \\
    
    \hline
\end{tabular}

    }
    \caption{Final Aggregator results, i.e. final results for the Power model. The results of the Ruler and Texter, whose predictions the Aggregator combines, are also shown for comparison. Although the Aggregator does not outperform its respective Ruler and Texter in terms of F1 score, it does for mAP.}
    \label{tab:5_experiments/5_aggregator/results}
\end{table}

As the mAP values show, the Aggregator performs several percentage points better than the Ruler and Texter on their own, with the improvement on the CDE split being more obvious. However, the relatively small increase on the FB split suggests that the true positives of Ruler and Texter almost coincide there. For the CDE split, on the other hand, manually peeking into the predictions reveals that the improved mAP mainly results from complementary true positives -- and not so much from improved ranks of joint predictions. Looking at the values of simple and attentive Texter, it is also noticeable that the lead of the simple Texter over the attentive Texter shrinks when adding the Ruler. Likewise, the lead of the text sets with many sentences and with high-quality sentences shrinks. Finally, the different aptitudes for Ruler and Overall, the Aggregator results are even similar between the two splits, while previously, models performed significantly better on the FB split.

Two experiments that will be mentioned only briefly here, because of their unspectacular results, concerning the calculation of the Aggregator's confidence as per \autoref{eq:4_approach/3_aggregator/conf_aggregator}: First, in the beginning, experiments were conducted on the computation of the combined confidence $conf_{Aggregator}$ in cases where facts are predicted by Ruler and Texter. As combining methods, calculating the maximum and the mean of $conf_{Ruler}$ and $conf_{Texter}$ were evaluated, but it soon became apparent that summing them up much better accommodates the fact that a fact predicted by Ruler and Texter deserves very high confidence. Second, experiments showed that taking into account the weight parameter $\alpha$ between Ruler and Texter yields only marginal performance improvements in the tenths of a percent range because the confidence values of Ruler and Texter seem to be very comparable after all and thus always yield $\alpha$ values close to 0.5. In detail, Ruler and Texer were both a bit too optimistic about their predictions in the experiments -- but they were equally overconfident.




    \chapter{Conclusion}
    \label{ch:6_conclusion}
    Knowledge graphs are used in more and more domains and research on AI assisted knowledge graph completion enjoys increasing interest. Due to the ongoing exponential information growth~\cite{}, especially of freely accessible texts on the web~\cite{}, it is becoming easier to obtain additional entity information that can be used to improve the predictions of KGC models.

In this work, the Power ensemble model was developed which uses complementary rule-based and text-based components to determine new facts for a knowledge graph for whose entities textual descriptions are available. The rule-based component exploits patterns in the graph structure found by the rule miner AnyBURL while the text-based component uses the state-of-the-art transformer DistilBERT to effectively evaluate texts. Thanks to the rule-based component and an attention mechanism within the text-based component, the Power model also has the advantage over existing embedding-based models that it can provide human-understandable information about which rules and which texts were decisive for the predictions.

In a series of experiments, the model was evaluated on several combinations of graphs and text sets. Furthermore, different variations of the components were compared and the optimal conditions for training were determined. For the text-based component, this yielded the surprising finding that very good results can already be obtained without the use of computationally intensive transformers. Moreover, the attention mechanism was qualitatively tested for its functionality - and although the attention mechanism does not improve the performance, it nevertheless causes the comprehensible prioritization of the sentences. Overall, it was found that, depending on the graph and text inventory, sometimes the rule-based component and sometimes the text-based component provided better predictions, but the Power model as a whole always provided better results than either component on its own.

Building on the results of this thesis, future studies could address the following aspects in order to further improve the quality of the predicted facts:

\begin{itemize}
    \item The predicted facts could be used as a basis for the inference of further facts by repeatedly applying the known rules to the new facts. The predicted facts would be extended step by step until at some point they would correspond to the hull of all facts derivable from the given facts. Thereby, the probabilities of the underlying premises would have to be taken into account when calculating the probabilities of derived facts. With such an extension, transitive relations, such as ``is part of'' could be fully exploited and the rule-based component could participate in processing open-world entity which lack initially known facts.

    \item The model could be extended to include head prediction, i. e. given an entity $x$, predict facts of the form $(head, rel, x)$ without having to wait for the processing of the entity $head$.

    \item The rule-based component is currently limited to processing the particularly common, reliable AnyBURL rules of type $AC_1$ of length 1, and could be extended to rules of types $AC_2$ and $C$, as well as rules of arbitrary length.

    \item In terms of the ranking scenario, evaluation of rules with a confidence lower than 0.5 could be considered to achieve higher average precision. At the same time, new metrics like precision@k could be considered to check if the increased mAP is also reflected in the top predictions.

    \item It could be investigated whether splitting the classifier used in the text-based component into several smaller classifiers gives an advantage in regard to unbalanced classes.
\end{itemize}


    \newpage

    \bibliographystyle{plain}

    \begin{btSect}{main}
        \section*{References}
        \btPrintCited
    \end{btSect}

    \begin{btSect}{online}
        \section*{Online Sources}
        \btPrintCited
    \end{btSect}


    \appendix


    \chapter{Tables and Plots}
    \label{ch:a_appendix}
    \begin{table}
    \centering
    \begin{tabular}{ l l }
    \toprule

    \multicolumn{1}{l}{\textbf{Text Set}} &
    \multicolumn{1}{l}{\textbf{Description}} \\

    \midrule

    cde-cde-1-clean   & One CDE sentence per CDE entity                \\

    \addlinespace

    cde-irt-1-clean   & One IRT sentence per CDE entity                \\
    cde-irt-1-marked  & One marked IRT sentence per CDE entity         \\
    cde-irt-1-masked  & One masked IRT sentence per CDE entity         \\

    \addlinespace

    cde-irt-5-clean   & Up to five IRT sentences per CDE entity        \\
    cde-irt-5-marked  & Up to five marked IRT sentences per CDE entity \\
    cde-irt-5-masked  & Up to five masked IRT sentences per CDE entity \\

    \addlinespace

    cde-irt-15-clean  & Up to 15 IRT sentences per CDE entity          \\
    cde-irt-15-marked & Up to 15 marked IRT sentences per CDE entity   \\
    cde-irt-15-masked & Up to 15 masked IRT sentences per CDE entity   \\

    \addlinespace

    cde-irt-30-clean  & Up to 30 IRT sentences per CDE entity          \\
    cde-irt-30-marked & Up to 30 marked IRT sentences per CDE entity   \\
    cde-irt-30-masked & Up to 30 masked IRT sentences per CDE entity   \\

    \midrule

    fb-owe-1-clean    & One OWE sentence per FB entity                 \\

    \addlinespace

    fb-irt-1-clean    & One IRT sentence per FB entity                 \\
    fb-irt-1-marked   & One marked IRT sentence per FB entity          \\
    fb-irt-1-masked   & One masked IRT sentence per FB entity          \\

    \addlinespace

    fb-irt-5-clean    & Up to 5 IRT sentences per FB entity            \\
    fb-irt-5-marked   & Up to 5 marked IRT sentences per FB entity     \\
    fb-irt-5-masked   & Up to 5 masked IRT sentences per FB entity     \\

    \addlinespace

    fb-irt-15-clean   & Up to 15 IRT sentences per FB entity           \\
    fb-irt-15-marked  & Up to 15 marked IRT sentences per FB entity    \\
    fb-irt-15-masked  & Up to 15 masked IRT sentences per FB entity    \\

    \addlinespace

    fb-irt-30-clean   & Up to 30 IRT sentences per FB entity           \\
    fb-irt-30-marked  & Up to 30 marked IRT sentences per FB entity    \\
    fb-irt-30-masked  & Up to 30 masked IRT sentences per FB entity    \\

    \bottomrule
\end{tabular}

    \caption{List of the IRT~\cite{} text sets for the CoDEx-M (CDE) \cite{Safavi2020CoDExAC} and FB15k-237 (FB) \cite{Toutanova2015ObservedVL} entities used during evaluation}
    \label{tab:a_appendix/text_sets_all}
\end{table}

\begin{table}
    \centering
    \begin{tabular}{ l c r r r c r r r }
    \toprule

    \multicolumn{1}{l}{\textbf{Text Set}} & \phantom &
    \multicolumn{3}{c}{\textbf{Simple Texter}} & \phantom &
    \multicolumn{3}{c}{\textbf{Attending Texter}} \\

    \cmidrule{3-5}
    \cmidrule{7-9}

    \multicolumn{1}{c}{} & \phantom{abc} &
    \multicolumn{1}{c}{\textbf{Prec}} &
    \multicolumn{1}{c}{\textbf{Rec}} &
    \multicolumn{1}{c}{\textbf{F1}} & \phantom{abc} &
    \multicolumn{1}{c}{\textbf{Prec}} &
    \multicolumn{1}{c}{\textbf{Rec}} &
    \multicolumn{1}{c}{\textbf{F1}} \\

    \midrule

    cde-cde-1-clean   && 33.76 & 71.28 & 44.10 && 42.53 & 60.62 & 49.26 \\

    \addlinespace

    cde-irt-1-clean   && 24.66 & 56.31 & 32.83 && 28.07 & 41.03 & 32.80 \\
    cde-irt-1-marked  && 28.31 & 56.48 & 36.60 && 30.00 & 43.56 & 35.02 \\
    cde-irt-1-masked  && 25.30 & 52.19 & 32.98 && 27.64 & 35.08 & 30.49 \\

    \addlinespace

    cde-irt-5-clean   && 35.51 & 54.70 & 42.37 && 34.61 & 48.01 & 39.81 \\
    cde-irt-5-marked  && 37.63 & 52.25 & 43.26 && 37.12 & 49.56 & 41.88 \\
    cde-irt-5-masked  && 38.00 & 53.70 & 43.76 && 34.42 & 48.01 & 39.50 \\

    \addlinespace

    cde-irt-15-clean  && 40.13 & 54.34 & 45.22 && 39.88 & 47.16 & 42.80 \\
    cde-irt-15-marked && 37.89 & 55.32 & 44.50 && 39.14 & 51.96 & 44.04 \\
    cde-irt-15-masked && 43.26 & 54.55 & 47.37 && 38.90 & 50.39 & 43.39 \\

    \addlinespace

    cde-irt-30-clean  && 40.09 & 52.41 & 44.47 && 36.74 & 57.78 & 43.83 \\
    cde-irt-30-marked && 37.94 & 52.37 & 43.39 && 40.78 & 58.07 & 46.82 \\
    cde-irt-30-masked && 45.66 & 55.15 & 49.24 && 37.85 & 58.56 & 45.05 \\

    \midrule

    fb-owe-1-clean    && 41.43 & 89.33 & 53.62 && 46.49 & 87.44 & 57.87 \\

    \addlinespace

    fb-irt-1-clean    && 33.18 & 61.56 & 40.19 && 33.76 & 52.08 & 39.56 \\
    fb-irt-1-marked   && 38.84 & 69.48 & 47.89 && 39.40 & 61.50 & 47.07 \\
    fb-irt-1-masked   && 33.28 & 69.95 & 43.07 && 35.36 & 58.49 & 43.11 \\

    \addlinespace

    fb-irt-5-clean    && 48.08 & 56.23 & 46.18 && 43.82 & 59.30 & 49.53 \\
    fb-irt-5-marked   && 46.91 & 64.77 & 53.16 && 45.84 & 71.21 & 54.28 \\
    fb-irt-5-masked   && 44.68 & 65.95 & 50.08 && 45.54 & 55.88 & 48.74 \\

    \addlinespace

    fb-irt-15-clean   && 49.07 & 58.46 & 49.69 && 49.72 & 69.46 & 56.93 \\
    fb-irt-15-marked  && 46.68 & 69.70 & 54.85 && 48.44 & 79.19 & 58.68 \\
    fb-irt-15-masked  && 44.39 & 79.43 & 53.74 && 47.79 & 71.52 & 56.34 \\

    \addlinespace

    fb-irt-30-clean   && 46.00 & 58.87 & 50.69 && 46.08 & 60.59 & 51.44 \\
    fb-irt-30-marked  && 45.50 & 61.38 & 51.10 && 52.54 & 64.47 & 56.78 \\
    fb-irt-30-masked  && 41.20 & 60.15 & 47.56 && 47.52 & 72.42 & 56.25 \\

    \bottomrule
\end{tabular}

    \caption{Context Final Precision Recall}
    \label{tab:a_appendix/context_final_prec_rec}
\end{table}

some text 1

\begin{figure}[h]
    \centering
    \subfloat[cde-cde-1-simple]{
    \input{a_appendix/static_classes_1/cde_cde_1_simple/cde_cde_1_simple}
    \label{fig:a_appendix/static_classes_1/cde_cde_1_simple}
}
\hskip 5pt
\subfloat[cde-cde-1-attentive]{
    \input{a_appendix/static_classes_1/cde_cde_1_attentive/cde_cde_1_attentive}
    \label{fig:a_appendix/static_classes_1/cde_cde_1_attentive}
}

\subfloat[cde-irt-1-simple]{
    \input{a_appendix/static_classes_1/cde_irt_1_simple/cde_irt_1_simple}
    \label{fig:a_appendix/static_classes_1/cde_irt_1_simple}
}
\hskip 5pt
\subfloat[cde-irt-1-attentive]{
    \input{a_appendix/static_classes_1/cde_irt_1_attentive/cde_irt_1_attentive}
    \label{fig:a_appendix/static_classes_1/cde_irt_1_attentive}
}

\subfloat[cde-irt-5-simple]{
    \input{a_appendix/static_classes_1/cde_irt_5_simple/cde_irt_5_simple}
    \label{fig:a_appendix/static_classes_1/cde_irt_5_simple}
}
\hskip 5pt
\subfloat[cde-irt-5-attentive]{
    \input{a_appendix/static_classes_1/cde_irt_5_attentive/cde_irt_5_attentive}
    \label{fig:a_appendix/static_classes_1/cde_irt_5_attentive}
}

\subfloat[cde-irt-15-simple]{
    \input{a_appendix/static_classes_1/cde_irt_15_simple/cde_irt_15_simple}
    \label{fig:a_appendix/static_classes_1/cde_irt_15_simple}
}
\hskip 5pt
\subfloat[cde-irt-15-attentive]{
    \input{a_appendix/static_classes_1/cde_irt_15_attentive/cde_irt_15_attentive}
    \label{fig:a_appendix/static_classes_1/cde_irt_15_attentive}
}

    \caption{Development of the F1 score on the validation data during training}
    \label{fig:a_appendix/static_classes_1}
\end{figure}

some text 2

\begin{figure}[h]
    \centering
    \subfloat[cde-irt-30-simple]{
    \input{a_appendix/static_classes_2/cde_irt_30_simple/cde_irt_30_simple}
    \label{fig:a_appendix/static_classes_2/cde_irt_30_simple}
}
\hskip 5pt
\subfloat[cde-irt-30-attentive]{
    \input{a_appendix/static_classes_2/cde_irt_30_attentive/cde_irt_30_attentive}
    \label{fig:a_appendix/static_classes_2/cde_irt_30_attentive}
}

\subfloat[fb-irt-1-simple]{
    \begin{tikzpicture}
    \begin{axis}[
        axis lines = middle,
        cycle list name = tb,
        grid = both,
        legend pos = outer north east,
        scale = 0.6,
        xlabel = epoch,
        ylabel = F1,
    ]
        \addplot table [x = Step, y = Value, col sep = comma] {a_appendix/static_classes_2/fb_irt_1_simple/class_1.csv};
        \addplot table [x = Step, y = Value, col sep = comma] {a_appendix/static_classes_2/fb_irt_1_simple/class_2.csv};
        \addplot table [x = Step, y = Value, col sep = comma] {a_appendix/static_classes_2/fb_irt_1_simple/class_3.csv};
        \addplot table [x = Step, y = Value, col sep = comma] {a_appendix/static_classes_2/fb_irt_1_simple/class_avg.csv};
        \addplot table [x = Step, y = Value, col sep = comma] {a_appendix/static_classes_2/fb_irt_1_simple/class_98.csv};
        \addplot table [x = Step, y = Value, col sep = comma] {a_appendix/static_classes_2/fb_irt_1_simple/class_99.csv};
        \addplot table [x = Step, y = Value, col sep = comma] {a_appendix/static_classes_2/fb_irt_1_simple/class_100.csv};
    \end{axis}
\end{tikzpicture}

    \label{fig:a_appendix/static_classes_2/fb_irt_1_simple}
}
\hskip 5pt
\subfloat[fb-irt-1-attentive]{
    \input{a_appendix/static_classes_2/fb_irt_1_attentive/fb_irt_1_attentive}
    \label{fig:a_appendix/static_classes_2/fb_irt_1_attentive}
}

\subfloat[fb-irt-5-simple]{
    \input{a_appendix/static_classes_2/fb_irt_5_simple/fb_irt_5_simple}
    \label{fig:a_appendix/static_classes_2/fb_irt_5_simple}
}
\hskip 5pt
\subfloat[fb-irt-5-attentive]{
    \begin{tikzpicture}
    \begin{axis}[
        axis lines = middle,
        cycle list name = tb,
        grid = both,
        legend pos = outer north east,
        scale = 0.6,
        xlabel = epoch,
        ylabel = F1,
    ]
        \addplot table [x = Step, y = Value, col sep = comma] {a_appendix/static_classes_2/fb_irt_5_attentive/class_1.csv};
        \addplot table [x = Step, y = Value, col sep = comma] {a_appendix/static_classes_2/fb_irt_5_attentive/class_2.csv};
        \addplot table [x = Step, y = Value, col sep = comma] {a_appendix/static_classes_2/fb_irt_5_attentive/class_3.csv};
        \addplot table [x = Step, y = Value, col sep = comma] {a_appendix/static_classes_2/fb_irt_5_attentive/class_avg.csv};
        \addplot table [x = Step, y = Value, col sep = comma] {a_appendix/static_classes_2/fb_irt_5_attentive/class_98.csv};
        \addplot table [x = Step, y = Value, col sep = comma] {a_appendix/static_classes_2/fb_irt_5_attentive/class_99.csv};
        \addplot table [x = Step, y = Value, col sep = comma] {a_appendix/static_classes_2/fb_irt_5_attentive/class_100.csv};
    \end{axis}
\end{tikzpicture}

    \label{fig:a_appendix/static_classes_2/fb_irt_5_attentive}
}

\subfloat[fb-irt-15-simple]{
    \input{a_appendix/static_classes_2/fb_irt_15_simple/fb_irt_15_simple}
    \label{fig:a_appendix/static_classes_2/fb_irt_15_simple}
}
\hskip 5pt
\subfloat[fb-irt-15-attentive]{
    \input{a_appendix/static_classes_2/fb_irt_15_attentive/fb_irt_15_attentive}
    \label{fig:a_appendix/static_classes_2/fb_irt_15_attentive}
}

    \caption{Development of the F1 score on the validation data during training}
    \label{fig:a_appendix/static_classes_2}
\end{figure}

some text 3

\begin{figure}[h]
    \centering
    \subfloat[fb-irt-30-simple]{
    \input{a_appendix/static_classes_3/fb_irt_30_simple/fb_irt_30_simple}
    \label{fig:a_appendix/static_classes_3/fb_irt_30_simple}
}
\hskip 5pt
\subfloat[fb-irt-30-attentive]{
    \input{a_appendix/static_classes_3/fb_irt_30_attentive/fb_irt_30_attentive}
    \label{fig:a_appendix/static_classes_3/fb_irt_30_attentive}
}

\subfloat[fb-owe-1-simple]{
    \input{a_appendix/static_classes_3/fb_owe_1_simple/fb_owe_1_simple}
    \label{fig:a_appendix/static_classes_3/fb_owe_1_simple}
}
\hskip 5pt
\subfloat[fb-owe-1-attentive]{
    \input{a_appendix/static_classes_3/fb_owe_1_attentive/fb_owe_1_attentive}
    \label{fig:a_appendix/static_classes_3/fb_owe_1_attentive}
}

    \caption{Development of the F1 score on the validation data during training}
    \label{fig:a_appendix/static_classes_3}
\end{figure}

some text 4


\end{document}
