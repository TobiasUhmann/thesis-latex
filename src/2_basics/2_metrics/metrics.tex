When designing and implementing machine learning models, scientists act on experience when it comes to architectural decisions and hyperparameter choices. In the early stages of a model, a trained eye on processing examples and observing the loss curve enable rapid progress. However, as the model matures, it becomes essential to quantify its performance with respect to comprehensive validation and test sets. Besides the selection of appropriate validation and test data, it is important to choose a meaningful metric that fits the problem. For example, when all of a models predictions are equally relevant, one would aim for an overall high precision, whereas a use case that involves a human processing the results manually, such as a web search, for example, one would choose a ranking metric that rewards good results at the top of a list.

All considered metrics are defined by terms from the so-called confusion matrix shown in Figure~\ref{fig:2_basics/2_metrics/1_confusion_matrix}. The matrix applies to the general scenario in which predictions are made about a certain condition for a set of objects. Thereby, an object's actual condition can be positive or negative, also referred to as its \emph{ground truth}, as can be the object's predicted condition. A prediction is \emph{true}, or correct, if its predicted condition is consistent with the object's actual condition and \emph{false} otherwise. Furthermore, objects whose prediction is positive are called \emph{positives}, while objects whose predictions are negative are called \emph{negatives}. Depending on their actual and their predicted condition, each object falls into one of four distinct categories:

\begin{figure}[t]
    \centering
    \includegraphics{2_basics/2_metrics/confusion_matrix}
    \caption{Confusion matrix dividing objects into four distinctive groups depending on their actual condition and the condition predicted by a model.}
    \label{fig:2_basics/2_metrics/1_confusion_matrix}
\end{figure}

\begin{itemize}
    \item \textbf{\emph{True positives}} (TP) are objects for which the regarded condition is positive and whose predicted conditions is also positive.

    \item \textbf{\emph{False positives}} (FP) are objects for which the regarded condition is positive but whose predicted condition is negative. This type of error is referred to as a \emph{Type I error}.

    \item \textbf{\emph{False negatives}} (FN) are objects for which the condition is actually negative but whose predicted condition is positive. This second kind of errornous prediction is also referred to as \emph{Type II error}.

    \item \textbf{\emph{True negatives}} (TN) are objects for which the condition is actually negative and whose predicted condition is also negative.
\end{itemize}

Although it is generally desirable to obtain as many correct predictions as possible, true positives and true negatives are often differently important and errors of type one and two differently severe. In medicine, for example, not recognizing a disease could be much worse than accidentally diagnosing a healthy person as ill. Conversely, not recognizing guilt in a lawsuit might be less serious than convicting someone who is innocent. In the context of knowledge graph completion under the open-world assumption, the focus lies on true positives since the KGC model cannot make any qualified statements about false facts without negative samples in the graph. Omitting a fact from the prediction only means that the model has too little evidence for that fact, not that it can falsify it.

%The purpose of this section is to explain those metrics relevant for this work. Section~\ref{fig:2_basics/2_metrics/1_confusion_matrix} defines basic terms used by the following sections. Sections~\ref{subsec:2_basics/2_metrics/1_accuracy} and~\ref{subsec:2_basics/2_metrics/2_prf} then present the general purpose metrics accuracy, precision, recall and F1 score while Sections~\ref{subsec:2_basics/2_metrics/3_mrr} and~\ref{subsec:2_basics/2_metrics/4_map} discuss the ranking metrics MRR and mAP\@.

\subsection{Accuracy}
\label{subsec:2_basics/2_metrics/1_accuracy}
One of the most common general-purpose metrics is \emph{accuracy}, which measures a models overall capability to make correct positive and negative predictions. In case of binary classification, accuracy is the rate of correct predictions over all predictions:

\begin{align}
    Accuracy = \frac{TP + TN}{TP + TN + FP + FN}
    \label{eq:2_basics/2_metrics/1_accuracy/accuracy}
\end{align}

Colloquially speaking, accuracy answers the question of how good predictions are in general. It takes values in $[0, 1]$, whereby higher is better. However, although accuracy is a useful and intuitive metric in general, it can be misleading when it comes to inbalanced classes, because a model can simply reach high accuracy by always predicting the predominant class. For example, if nine out of ten ground truth values are false, a model could reach 90\% accuracy by always predicting false. To counteract this, balanced accuracy~\cite{Mower2005PREPMtPR} can be used instead. However, as mention earlier, as negative predictions do not play a big role for KGC models in an open-world scenario, accuracy will only play a minor role in this work, anyway.


\subsection{Precision, Recall and F1}
\label{subsec:2_basics/2_metrics/2_prf}
\emph{Precision}, \emph{recall} and \emph{F1} score focus on the quality of positive predictions. Precision gives an impression of how reliable positive predictions are, recall tells how many of the actual positive elements are declared as such, and F1 is a measure that combines both precision and recall in one value. All three metrics take values between 0 and 1 with higher being better.

Precision is the ratio of true positive predictions to all positive predictions as noted in~\ref{eq:2_basics/4_metrics/3_prf/precision}. In the above cat example, correctly identifying three out of five cats but also classifying one dog as a cat results in a precision of $3 / (3 + 1) = 0.75$. The two missed out cats do not play a role. Precision is also called \emph{positive predictive value} (PPV).

\begin{align}
    Precision = \frac{TP}{TP + FP}
    \label{eq:2_basics/4_metrics/3_prf/precision}
\end{align}

Recall, on the other hand, compares the number of true positives the the number of all ground truth positives as shown in~\ref{eq:2_basics/4_metrics/3_prf/recall}. Looking again at the cat example, identifying three out of five cats in total leads to a recall of $3 / (3 + 2) = 0.6$. The dog that was mistaken as a cat does not count in. Recall is also referred to as \emph{true positive rate} (TPR).

\begin{align}
    Recall = \frac{TP}{TP + FN}
    \label{eq:2_basics/4_metrics/3_prf/recall}
\end{align}

Precision and recall are directly dependent on each other. A cautious model that only predicts positives when it is absolutely sure achieves high precision but low recall. Conversely, it is easy to achieve optimal recall by making positive predictions for all elements, though precision will suffer in that case. The F score serves as a measure that reaches a high value when a reasonable balance between precision and recall is found. Equation~\ref{eq:2_basics/4_metrics/3_prf/f_beta} shows the formula for the general $F_\beta$ score whose parameter $\beta$ determines whether the focus should rather be shifted to precision or recall. Setting $\beta = 1$ yields the $F_1$ score in~\ref{eq:2_basics/4_metrics/3_prf/f_1} as the harmonic mean in which precision and recall are equally weighted.

\begin{align}
    F_\beta &= (1 + \beta^2) \cdot \frac{Precision \cdot Recall}{\beta^2 \cdot Precision + Recall}
    \label{eq:2_basics/4_metrics/3_prf/f_beta} \\
    F_1 &= 2 \cdot \frac{Precision \cdot Recall}{Precision + Recall}
    \label{eq:2_basics/4_metrics/3_prf/f_1}
\end{align}

When including more than one class in the evaluation, for example when predicting for each photo whether it is a cat, a dog or a horse, the question arises how to combine the results of the respective classes. Three of several possibilities are as follows:

\begin{itemize}
    \item Not combining the class results at all preserves the class-wise information but convoluted with a large number of classes.

    \item Merging all classes' predictions and calculating precision, recall and F score over all predictions is called \emph{micro} averaging. Underrepresented classes, which typically show poorer performance due to lack of training data, are less reflected in micro precision, recall and F score.

    \item Calculating each class' scores individually and averaging the class-wise metrics yields \emph{macro} precision, recall and F score. In this case, underrepresented classes have the same impact as classes with many ground truth positives. Macro values are therefore often worse than the corresponding micro values.
\end{itemize}

For very rare classes, there may be zero positives within a subset of the entire data set. If a good model does not predict false positives in this case, precision is undefined. In the context of a macro averaging, precision could be considered 0 because the model does not make a correct prediction, it could be defined as 1 because the model does not make an incorrect prediction, or the class could be excluded from averaging. Similarly, a withholding model might always make negative predictions so that recall is undefined. Therefore, the same considerations must be made for Recall and F Score.


\subsection{Mean Reciprocal Rank}
\label{subsec:2_basics/2_metrics/3_mrr}
The above presented accuracy, precision, recall and F1 metrics are useful in a classification task where no priorization among the predictions is required. However, in an \emph{information retrieval (IR)} scenario, such as a web search, for example, a model might yield a sorted list of predictions where the ranking actually plays a major role. Usually, IR scenarios do not differentiate between positive and negative predictions, but rather between more or less relevant predictions that are returned by decreasing relevance.

In those cases it is more important to rank relevant items as high as possible among the overall results than it is assign the correct probability, or class if there is probability threshold, to each item. When it is most important to receive a correct top-most prediction for each query, the \emph{mean reciprocal rank (MRR)} is the metric of choice. Given the results of $n$ queries it is calculated as per Equation~\ref{subsec:2_basics/2_metrics/3_mrr}, where $rank_i$ is the rank of the top-most relevant item among the predictions of the $i$th query results. Each reciprocal rank, and thus the mean over all reciprocal ranks, lies in $(0, 1]$, with higher being better. If a query result does not contain any relevant item, the reciprocal rank is undefined. Depending on the use case, the object might be skipped or assigned a specific value. A typical application scenario for MRR would be the evaluation of a voice assistant that has to respond with the single most relevant answer it gets from a model.

\begin{align}
    MRR = \frac{1}{n} \sum_{i=1}^{n} \frac{1}{rank_i}
    \label{eq:2_basics/2_metrics/3_mrr/mrr}
\end{align}


\subsection{Mean Average Precision}
\label{subsec:2_basics/2_metrics/4_map}
precision, recall, f1 do not consider order of predictions
power yields probabilities for facts, makes sense to rank probable facts higher
like google search

\[
    AP@n = \frac{1}{GTP} \sum_{k=1}^{n} P@n \cdot rel@n
\]

then, mean over all entities = mean of all average precisions = mAP

\[
    mAP = \frac{1}{N} \sum_{i=1}^{N} AP_i
\]

