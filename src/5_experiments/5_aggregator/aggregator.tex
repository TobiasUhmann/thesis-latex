The aggregator has the task of merging the predicted facts from Ruler and Texter. As envisioned in \autoref{sec:4_approach/3_aggregator} and illustrated in \autoref{fig:4_approach/3_aggregator/lucy}, it was hoped that merging the facts leads to higher average precision because facts predicted by both components are likely to be correct and should be ranked higher. In addition, the Aggregator should be able to estimate how reliable the predictions of Ruler and Texter are in relation to each other, which is implemented in the form of the weight parameter $\alpha$ as described in \autoref{eq:4_approach/3_aggregator/conf_aggregator}.

\autoref{tab:5_experiments/5_aggregator/results} shows the final evaluation results for the Aggregator, and thus the final evaluation results for the Power model, for a number of graph-text combinations. As fact splits, the splits with 50\% known test facts were chosen, as for the final Ruler evaluation in \autoref{sec:5_experiments/4_ruler}. The respective results for the CDE-50 and FB-50 splits from \autoref{tab:5_experiments/4_ruler/results} were taken over into \autoref{tab:5_experiments/5_aggregator/results} for easier comparability. Similarly, the chosen text sets are the ones from the final Texter evaluation in \autoref{subsec:5_experiments/3_texter/3_context}. Again, \autoref{tab:5_experiments/5_aggregator/results} duplicates the respective results from \autoref{tab:5_experiments/3_texter/3_context/results} for ease of comparison. The last two columns then contain the new Aggregator measurements for the combination of the corresponding Ruler and Texter.

\begin{table}[t]
    \makebox[\textwidth][c]{
        \begin{tabular}{| l | r | r | r | r | r | r | r | r | r |}
    \hline
    
    \multicolumn{1}{|c|}{\textbf{Split}} &
    \multicolumn{1}{|c|}{\textbf{mAP}} &
    \multicolumn{4}{|c|}{\textbf{Macro over ents}} &
    \multicolumn{4}{|c|}{\textbf{Micro over facts}} \\
    
    \multicolumn{1}{|c|}{} &
    \multicolumn{1}{|c|}{} &
    \multicolumn{1}{|c|}{\textbf{Prec}} &
    \multicolumn{1}{|c|}{\textbf{Rec}} &
    \multicolumn{1}{|c|}{\textbf{F1}} &
    \multicolumn{1}{|c|}{\textbf{Supp}} &
    \multicolumn{1}{|c|}{\textbf{Prec}} &
    \multicolumn{1}{|c|}{\textbf{Rec}} &
    \multicolumn{1}{|c|}{\textbf{F1}} &
    \multicolumn{1}{|c|}{\textbf{Supp}} \\
    
    \hline \hline
    
    CDE-0 & 0.84 &
    100.00 & 0.84 & 0.84 & \num{13.32} &
    100.00 & 0.00 & 0.00 & \num{25255} \\
    
    CDE-25 & 22.06 &
    66.90 & 24.27 & 32.02 & \num{13.32} &
    62.67 & 22.93 & 33.57 & \num{25255} \\
    
    CDE-50 & 29.26 &
    61.41 & 33.14 & 40.29 & \num{13.32} &
    58.47 & 31.52 & 40.96 & \num{25255} \\
    
    CDE-75 & 33.23 &
    57.88 & 38.10 & 43.59 & \num{13.32} &
    55.43 & 3638 & 43.93 & \num{25255} \\
    
    CDE-100 & 35.77 &
    55.62 & 41.48 & 45.28 & \num{13.32} &
    53.49 & 39.72 & 45.59 & \num{25255} \\
    
    \hline
    
    FB-0 & 3.19 &
    100.00 & 03.19 & 3.19 & \num{18.76} &
    100.00 & 0.00 & 0.00 & \num{15312} \\
    
    FB-25 & 27.41 &
    73.46 & 30.56 & 37.06 & \num{18.76} &
    69.28 & 30.15 & 42.02 & \num{15312} \\
    
    FB-50 & 33.39 &
    68.36 & 37.61 & 42.90 & \num{18.76} &
    64.90 & 36.76 & 46.94 & \num{15312} \\
    
    FB-75 & 36.22 &
    64.93 & 41.37 & 45.14 & \num{18.76} &
    62.76 & 40.82 & 49.47 & \num{15312} \\
    
    FB-100 & 38.43 &
    63.08 & 44.34 & 46.97 & \num{18.76} &
    60.98 & 43.51 & 50.79 & \num{15312} \\
    
    \hline
\end{tabular}

    }
    \caption{Final Aggregator results, i.e. final results for the Power model. The results of the Ruler and Texter, whose predictions the Aggregator combines, are also shown for comparison. Although the Aggregator does not outperform its respective Ruler and Texter in terms of F1 score, it does for mAP.}
    \label{tab:5_experiments/5_aggregator/results}
\end{table}

As the mAP values show, the Aggregator performs several percentage points better than the Ruler and Texter on their own, with the improvement on the CDE split being more obvious. However, the relatively small increase on the FB split suggests that the true positives of Ruler and Texter almost coincide there. For the CDE split, on the other hand, manually peeking into the predictions reveals that the improved mAP mainly results from complementary true positives -- and not so much from improved ranks of joint predictions. Looking at the values of simple and attentive Texter, it is also noticeable that the lead of the simple Texter over the attentive Texter shrinks when adding the Ruler. Likewise, the lead of the text sets with many sentences and with high-quality sentences shrinks. Finally, the different aptitudes for Ruler and Overall, the Aggregator results are even similar between the two splits, while previously, models performed significantly better on the FB split.

Two experiments that will be mentioned only briefly here, because of their unspectacular results, concerning the calculation of the Aggregator's confidence as per \autoref{eq:4_approach/3_aggregator/conf_aggregator}: First, in the beginning, experiments were conducted on the computation of the combined confidence $conf_{Aggregator}$ in cases where facts are predicted by Ruler and Texter. As combining methods, calculating the maximum and the mean of $conf_{Ruler}$ and $conf_{Texter}$ were evaluated, but it soon became apparent that summing them up much better accommodates the fact that a fact predicted by Ruler and Texter deserves very high confidence. Second, experiments showed that taking into account the weight parameter $\alpha$ between Ruler and Texter yields only marginal performance improvements in the tenths of a percent range because the confidence values of Ruler and Texter seem to be very comparable after all and thus always yield $\alpha$ values close to 0.5. In detail, Ruler and Texer were both a bit too optimistic about their predictions in the experiments -- but they were equally overconfident.
