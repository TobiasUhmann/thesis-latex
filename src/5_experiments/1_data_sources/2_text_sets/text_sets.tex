In addition to the open-world fact splits of FB15k-237 and CoDEx-M, Hamann provides multiple text sets for each split's entities. Thereby, the text sets' contents vary in quality and quantity, ranging from text sets that offer single, very specific entity descriptions to text sets that provide multiple low-quality contexts not necessarily describing the entity directly. Some of the text sets contain plain text, some mark the entity's mentions within the text via special tokens, and some mask the entity mention. Together with the choice between FB15k-237 and CoDEx-M, this allows for graph-text combinations suiting different real-world scenarios.

Table~\ref{tab:5_experiments/1_data_sources/2_text_sets/text_sets_table} lists some example sentences from selected text sets and shortly describes the naming schema behind the text sets' names that will be used throughout this chapter, while Table~\ref{tab:a_appendix/text_sets_all} in Appendix~\ref{ch:a_appendix} lists all text sets. For example, the text set named "cde-irt-5-marked" indictates that it contains up to five marked sentences for each entity in the CoDEx-M graph. Thereby, the four dash-separated name parts denote (1) the graph for whose entities texts are provided, (2) the texts' origin, (3) the maximum number of sentences per entity and (4) whether the sentences are marked masked. There are three types of sentences that are distinguished by their source:

\begin{itemize}
    \item The \emph{CDE sentences} were provided by the authors of the CoDEx paper~\cite{}. They are the entities' first sentence from their respective Wikipedia page and thus very specific.
    \item The \emph{IRT sentences} have been introduced in Hamann's IRT paper~\cite{}. They are randomly sampled entity contexts from the English Wikipedia that mention the entity in a more or less meaningful way anywhere in the sentence.
    \item The \emph{OWE sentences} are very compact entity descriptions, often consisting of only a few words. They were created by Villmov et al. during the work on their OWE model for open-world KGC~\cite{Shah2019AnOE}.
\end{itemize}

\begin{table}
    \centering
    \begin{tabularx}{\textwidth}{ l >{\hsize=.42\hsize}X >{\hsize=.58\hsize}X }
    \toprule
    
    \multicolumn{1}{l}{\textbf{Text Set}} &
    \multicolumn{1}{l}{\textbf{Description}} &
    \multicolumn{1}{l}{\textbf{Example}} \\
    
    \midrule
    
    cde-cde-1-clean & One CDE sentence per CDE entity &
    A Few Good Men is a 1992 American legal drama film directed by Rob Reiner and starring \dots \\ 
    
    \midrule
    
    cde-irt-5-marked & Up to five marked IRT sentences per CDE entity &
    \dots for Best Film Editing for the feature film, [START] A Few Good Men [END] (1992). \\ 

    \midrule
    
    fb-irt-30-masked & Up to 30 masked IRT sentences per FB entity &
    The [MASK] has qualified one male and one female athlete in the artistic gymnastics competition. \\ 

    \midrule
    
    fb-owe-1-clean & One OWE sentence per FB entity &
    Country in the caribbean. \\

    \bottomrule
\end{tabularx}

    \caption{Example sentences from some of the text sets - in some text sets the entity mention is marked or masked via special tokens}
    \label{tab:5_experiments/1_data_sources/2_text_sets/text_sets_table}
\end{table}
