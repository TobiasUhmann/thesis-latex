In addition to the open-world fact splits of FB15k-237 and CoDExM, Hamann provides multiple text sets for each split's entities. Thereby, the text sets' contents vary in quality and quantity, ranging from text sets that offer single, very specific entity descriptions to text sets that provide multiple low-quality contexts not necessarily describing the entity directly. Some of the text sets contain plain text, some mark the entity's mentions within the text via special tokens, and some mask the entity mention. Together with the choice between CoDEx-M and FB15k-237, this allows for graph-text combinations suiting different real-world scenarios.

Table~\ref{tab:5_experiments/1_data_sources/2_text_sets/text_sets_table} lists some example sentences from selected text sets and shortly describes the naming schema behind the text sets' names that will be used throughout this chapter. For example, the text set named "cde-irt-5-marked" indictates that it contains up to five marked sentences for each of the CoDEx-M entities. Thereby, "irt" marks the origin of the sentences: The \emph{IRT sentences} introduced in the Hamann's IRT paper~\cite{} are randomly sampled entity contexts from the English Wikipedia that mention the entity in a more or less meaningful way anywhere in the sentence. In contrast, the \emph{OWE sentences} are very compact entity descriptions, often consisting of only a few words, created by Villmov et al. for their work on open-world KGC~\cite{Shah2019AnOE}. A middle between the vagueness of the IRT sentences and the compactness of the OWE sentences are the \emph{CDE sentences} provided by the authors of the CoDEx paper~\cite{}, which are the entities' first sentence from their respective Wikipedia page. Table~\ref{tab:5_experiments/1_data_sources/2_text_sets/text_sets_table} in Appendix~\ref{ch:a_appendix} shows the full list of all text sets used during evaluation.

\begin{table}
    \centering
    \begin{tabularx}{\textwidth}{ l >{\hsize=.42\hsize}X >{\hsize=.58\hsize}X }
    \toprule
    
    \multicolumn{1}{l}{\textbf{Text Set}} &
    \multicolumn{1}{l}{\textbf{Description}} &
    \multicolumn{1}{l}{\textbf{Example}} \\
    
    \midrule
    
    cde-cde-1-clean & One CDE sentence per CDE entity &
    A Few Good Men is a 1992 American legal drama film directed by Rob Reiner and starring \dots \\ 
    
    \midrule
    
    cde-irt-5-marked & Up to five marked IRT sentences per CDE entity &
    \dots for Best Film Editing for the feature film, [START] A Few Good Men [END] (1992). \\ 

    \midrule
    
    fb-irt-30-masked & Up to 30 masked IRT sentences per FB entity &
    The [MASK] has qualified one male and one female athlete in the artistic gymnastics competition. \\ 

    \midrule
    
    fb-owe-1-clean & One OWE sentence per FB entity &
    Country in the caribbean. \\

    \bottomrule
\end{tabularx}

    \caption{Example sentences from some of the text sets - in some text sets the entity mention is marked or masked via special tokens}
    \label{tab:5_experiments/1_data_sources/2_text_sets/text_sets_table}
\end{table}
