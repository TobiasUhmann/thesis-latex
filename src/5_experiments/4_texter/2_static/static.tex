The traditional way to implement the embedding block in both, the simple and the attentive Texter, is using static word embeddings, i.e. a word is embedded independent from where it occurs in the sentence. The advantages include ease of implementation using a simple look-up table, few hyperparameters and quick training. However, although it is possible to somehow include a word's immediate context by forming n-grams consisting of multiple adjacent tokens, the sentence embedding cannot properly capture the meaning of markings and maskings via special tokens such as those in sentences from the "cde-irt-5-marked" or "cde-irt-5-masked" text sets exemplified in Table~\ref{tab:5_experiments/1_base_datasets/2_text_sets/text_sets_table}. Therefore all experiments in this section are conducted on the clean text sets containing no special tokens. Table~\ref{tab:5_experiments/4_texter/2_static/results} shows the final results over all clean text sets for both the baseline and the Power model.

\begin{table}[h]
    \centering
    \begin{tabular}{| l | r | r | r | r | r | r | r | r | r |}
    \hline
    
    \multicolumn{1}{|c|}{\textbf{Split}} &
    \multicolumn{1}{|c|}{\textbf{mAP}} &
    \multicolumn{4}{|c|}{\textbf{Macro over ents}} &
    \multicolumn{4}{|c|}{\textbf{Micro over facts}} \\
    
    \multicolumn{1}{|c|}{} &
    \multicolumn{1}{|c|}{} &
    \multicolumn{1}{|c|}{\textbf{Prec}} &
    \multicolumn{1}{|c|}{\textbf{Rec}} &
    \multicolumn{1}{|c|}{\textbf{F1}} &
    \multicolumn{1}{|c|}{\textbf{Supp}} &
    \multicolumn{1}{|c|}{\textbf{Prec}} &
    \multicolumn{1}{|c|}{\textbf{Rec}} &
    \multicolumn{1}{|c|}{\textbf{F1}} &
    \multicolumn{1}{|c|}{\textbf{Supp}} \\
    
    \hline \hline
    
    CDE-0 & 0.84 &
    100.00 & 0.84 & 0.84 & \num{13.32} &
    100.00 & 0.00 & 0.00 & \num{25255} \\
    
    CDE-25 & 22.06 &
    66.90 & 24.27 & 32.02 & \num{13.32} &
    62.67 & 22.93 & 33.57 & \num{25255} \\
    
    CDE-50 & 29.26 &
    61.41 & 33.14 & 40.29 & \num{13.32} &
    58.47 & 31.52 & 40.96 & \num{25255} \\
    
    CDE-75 & 33.23 &
    57.88 & 38.10 & 43.59 & \num{13.32} &
    55.43 & 3638 & 43.93 & \num{25255} \\
    
    CDE-100 & 35.77 &
    55.62 & 41.48 & 45.28 & \num{13.32} &
    53.49 & 39.72 & 45.59 & \num{25255} \\
    
    \hline
    
    FB-0 & 3.19 &
    100.00 & 03.19 & 3.19 & \num{18.76} &
    100.00 & 0.00 & 0.00 & \num{15312} \\
    
    FB-25 & 27.41 &
    73.46 & 30.56 & 37.06 & \num{18.76} &
    69.28 & 30.15 & 42.02 & \num{15312} \\
    
    FB-50 & 33.39 &
    68.36 & 37.61 & 42.90 & \num{18.76} &
    64.90 & 36.76 & 46.94 & \num{15312} \\
    
    FB-75 & 36.22 &
    64.93 & 41.37 & 45.14 & \num{18.76} &
    62.76 & 40.82 & 49.47 & \num{15312} \\
    
    FB-100 & 38.43 &
    63.08 & 44.34 & 46.97 & \num{18.76} &
    60.98 & 43.51 & 50.79 & \num{15312} \\
    
    \hline
\end{tabular}

    \caption{Final results for simple and attentive Texter using static word embeddings. All values are in percent.}
    \label{tab:5_experiments/4_texter/2_static/results}
\end{table}

The numbers show that both, the simple and the attentive Texter, outperform the zero rule baselines by far. The attentive version, however, is not far ahead of the simple one. On text sets with few sentences per entity, the simple Texter performs slightly better, while its the ther way around on text sets with many sentences. The largest difference can be observed on the "fb-irt-30-clean" text set, where the attentive model leads by 1.5\%. Similar to other text sets with multiple sentences per entity, the higher F1 score results from an improved precision that overcompensates the worse recall that exists at the same time. The attentive Texter benefits more from increasing the number of sentences, but the effect wears off beyond 15 sentences per entity. By far the best results are achieved on the qualitative single-sentence text sets. On the CDE split, the CDE sentences, i.e. the first sentence of an entity's Wikipedia page, seems to be more descriptive than 30 randomly samples contexts. Even greater is the success of the short OWE descriptions on the FB split that cause a 3\% advance over the next best attentive performance on 30 randomly sampled contexts. This can be regarded two ways: On the one hand, the text quality plays a significant role. On the other hand, On the other hand, a sufficiently large set of low-quality sentences can substitute a good description.

In addition to the test results, Figure~\ref{figure:experiments/static_f1} shows the development of the F1 score during training for selected text sets on the (a) CDE and (b) FB splits. On both, CDE and FB, the attentive Texter's curves start below the simple Texter's curves and catch up over time. The diagrams show, that the validation curves converge towards the test results after 30 episodes. In total, all trainings were run for 100 episodes. During the 70 episodes not shown, performance increases monotonically by another 3\% and the validation loss starts to increase slowly. It can be assumed that the test results could be minimally improved if training was continued for another 100 episodes.

%\begin{figure}[t]
%    \centering
%
%    \subfloat[Precision (micro)]{
%        \includegraphics[width=0.48\textwidth]{images/experiments/static_cde_f1.png}
%        \label{figure:experiments/static_cde_f1}
%    }
%    \subfloat[Recall (micro)]{
%        \includegraphics[width=0.48\textwidth]{images/experiments/static_fb_f1.png}
%        \label{figure:experiments/static_fb_f1}
%    }
%
%    \caption{Development of the F1 score on the validation data during training}
%    \label{figure:experiments/static_f1}
%\end{figure}

What the F1 curves in Figure~\ref{figure:experiments/static_f1} don't answer, however, is how well the texter can handle rare classes. This is addressed by Figure~\ref{fig:5_experiments/4_texter/2_static/static_classes} which presents the Texter's performance for individual classes. It compares the validation F1 scores between the most common classes, the least common classes and the macro F1 score that results from averaging the scores of all classes. Furthermore, the results are compared between the simple and the attentive Texter.

\begin{figure}[t]
    \centering
    \subfloat[Simple Texter]{
    \input{5_experiments/4_texter/2_static/static_classes/cde_irt_5_simple}
    \label{fig:5_experiments/4_texter/2_static/static_classes/cde_irt_5_simple}
}
\hskip 5pt
\subfloat[Attentive Texter]{
    \input{5_experiments/4_texter/2_static/static_classes/cde_irt_5_attentive}
    \label{fig:5_experiments/4_texter/2_static/static_classes/cde_irt_5_attentive}
}

    \caption{Per-class validation F1 scores during training; Comparison between most common classes (red, orange, yellow), least common classes (green, blue, purple) and the averaged value over all classes (black)}
    \label{fig:5_experiments/4_texter/2_static/static_classes}
\end{figure}

The graphs reveal that predictions for common classes are much more reliable. The three most common classes on the CDE split, with frequencies of almonst 20\% each, reach performances between 50\% and 80\%, whereas the three least common classes, with frequencies of around 1\% each, reach strongly divergent values, from 0\% to 60\%. Despite the tendency that frequent classes perform better, however, there is no close correlation between frequency and performance. The least common classes have very similar frequencies but perform very differently. For the second least common class prediction does not work at all, while the third least common class performs even better than the third most common class. There are also significant performance differences between the three most frequent classes although the frequencies are similar. Another, more obvious, finding is that frequent classes are learned within fewer episodes than less common classes. Interestingly, the attentive Texter seems to learn faster than the simple Texter. Finally, the graph shows that both Texters converge to similar values for all classes, as might have been assumed from the similar macro F1 values.

Looking at the results for the other datasets shown in Tables~\ref{fig:a_appendix/static_classes_1}~--~\ref{fig:a_appendix/static_classes_3} in the appendix, the same patterns can be detected with only a few exceptions where class results differ significantly between simple and attentive Texter. However, although there are no major differences between the Texter variants, it is noticeable that the results for rare classes fluctuate between IRT text sets instead of improving with increasing number of sentences per entity. This assumption was confirmed in another experiment: When forming new fb-irt-1-clean text sets from the fb-irt-30-clean text set by randomly taking one of the 30 sentences for each entity the rare classes' F1 curves vary seemingly randomly between the text sets. The average and the frequent classes' scores, on the other hand, remain practically unaffected.

With the final results for static word embeddings at hand, the following subsections show what impact various model changes and hyperparameters have. The affected model components are examined in the order in which they are invoked during training, beginning with the tokenizer and ending with the optimizer. Finally, the attentive Texter's attention mechanism is investigated to explain why it does not improve upon the simple Texter as hoped.

\subsubsection{Changing the tokenizer}
\label{subsubsec:5_experiments/4_texter/2_static/1_tokenizer}
The first variable component in both the simple and the attentive Texter is the tokenizer. A naive tokenizer might split the input sentences on any whitespace such as spaces, tabs and line breaks. The problem with that simple approach is the large number of resulting tokens blowing up the vocabulary. For example, the sentence "Hello, world!" would be split into the two tokens "Hello," and "world!", which would be different from "Hello" and "world". As a consequence, none of a word's "impure" occurrences contribute to learning the words embedding, but rather offer the model a way to overfit. One possible solution would be deleting punctuation characters, numbers, dates and other "polluting" characters and unique tokens. However, this could also lead to the destruction of correct tokens, such as "U.K.". Instead, this work uses a tokenizer from the NLP library SpaCy \cite{SpaCy} which splits the sentence "Welcome to the U.K.!" into the tokens "Welcome", "to", "the", "U.K." and "!".

%\begin{table}[h]
%    \centering
%    \begin{tabular}{| l | l | r | r | r | r |}
    \hline

    \multicolumn{1}{|c|}{\multirow{2}{*}{\textbf{Text Set}}} &
    \multicolumn{1}{|c|}{\multirow{2}{*}{\textbf{Texter}}} &
    \multicolumn{4}{|c|}{\textbf{Pooling}} \\

    &
    &
    \multicolumn{1}{|c|}{\textbf{CLS}} &
    \multicolumn{1}{|c|}{\textbf{No CLS}} &
    \multicolumn{1}{|c|}{\textbf{No Pad}} &
    \multicolumn{1}{|c|}{\textbf{All}} \\

    \hline \hline

    \multirow{2}{*}{cde-cde-1-clean}
    & Simple &  &  &  &  \\
    & Attend &  &  &  &  \\ \hline

    \multirow{2}{*}{cde-irt-1-marked}
    & Simple &  &  &  &  \\
    & Attend &  &  &  &  \\ \hline

    \multirow{2}{*}{cde-irt-5-marked}
    & Simple &  &  &  &  \\
    & Attend &  &  &  &  \\ \hline

    \multirow{2}{*}{cde-irt-15-marked}
    & Simple &  &  &  &  \\
    & Attend &  &  &  &  \\ \hline

    \multirow{2}{*}{cde-irt-30-marked}
    & Simple &  &  &  &  \\
    & Attend &  &  &  &  \\ \hline \hline

    \multirow{2}{*}{fb-irt-1-marked}
    & Simple &  &  &  &  \\
    & Attend &  &  &  &  \\ \hline

    \multirow{2}{*}{fb-irt-5-marked}
    & Simple &  &  &  &  \\
    & Attend &  &  &  &  \\ \hline

    \multirow{2}{*}{fb-irt-15-marked}
    & Simple &  &  &  &  \\
    & Attend &  &  &  &  \\ \hline

    \multirow{2}{*}{fb-irt-30-marked}
    & Simple &  &  &  &  \\
    & Attend &  &  &  &  \\ \hline

    \multirow{2}{*}{fb-owe-1-clean}
    & Simple &  &  &  &  \\
    & Attend &  &  &  &  \\ \hline

\end{tabular}

%    \caption{Impact of different tokenizers on vocabulary size and evaluation results}
%    \label{table:experiments/static/tokenizer}
%\end{table}

Table~\ref{table:experiments/static/tokenizer} shows the tokenizer choice's impact on the vocabulary size and the evaluation results for two text sets on FB15k-237. For the large text set "fb-irt-30", the vocabulary's size halves when using the SpaCy tokenizer rather than splitting on whitespace and the F1 score increases slightly. On "fb-owe-1", the vocabulary's size increases by only 30\%, but the performance gap is over 5\%, probably because every "lost" token means a bigger loss on the small text set. Table~\ref{table:static_tokenizers_all} in Appendix~\ref{ch:a_appendix} contains the full evaluation over all text sets where similar results can be observed on the "cde-irt-30" and "cde-cde-1" text sets.


\subsubsection{Initializing word embeddings randomly}
\label{subsubsec:5_experiments/4_texter/2_static/2_emb_size}
Upon tokenization, the sentences' tokens are embedded -- in the context of this chapter using static word embeddings. For this, randomly initialized, or as shown in the next subsection, pre-trained embeddings can be used. When using randomly initialized embeddings, the questions of what embedding size and which random distribution to chose arises. As random distribution the standard normal distribution $N(0, 1)$ is set. The optimal embedding size is determined via grid search. It is related to, among other things, the amount of available training data and is determined via grid search. Small embeddings can hold little specific information and generalize well, whereas large embeddings can represent subtleties of individual tokens but are also more susceptible to overfitting. The grid search covers a wide range of embedding sizes around the expected optimal embedding size of about 200, a usual size in current models, including unrealistically small sizes to find out to what degree it is worth increasing the size. Overall, expectations for the experiment were limited because of the moderately large amount of training. When using the maximum embedding size of 1000, performance was expected to decrease due to overfitting. \autoref{tab:5_experiments/3_texter/2_static/2_emb_size/grid_search} shows the experiment's results.

\begin{table}[t]
    \centering
    \begin{tabular}{| l | l | r | r | r | r |}
    \hline

    \multicolumn{1}{|c|}{\multirow{2}{*}{\textbf{Text Set}}} &
    \multicolumn{1}{|c|}{\multirow{2}{*}{\textbf{Texter}}} &
    \multicolumn{4}{|c|}{\textbf{Pooling}} \\

    &
    &
    \multicolumn{1}{|c|}{\textbf{CLS}} &
    \multicolumn{1}{|c|}{\textbf{No CLS}} &
    \multicolumn{1}{|c|}{\textbf{No Pad}} &
    \multicolumn{1}{|c|}{\textbf{All}} \\

    \hline \hline

    \multirow{2}{*}{cde-cde-1-clean}
    & Simple &  &  &  &  \\
    & Attend &  &  &  &  \\ \hline

    \multirow{2}{*}{cde-irt-1-marked}
    & Simple &  &  &  &  \\
    & Attend &  &  &  &  \\ \hline

    \multirow{2}{*}{cde-irt-5-marked}
    & Simple &  &  &  &  \\
    & Attend &  &  &  &  \\ \hline

    \multirow{2}{*}{cde-irt-15-marked}
    & Simple &  &  &  &  \\
    & Attend &  &  &  &  \\ \hline

    \multirow{2}{*}{cde-irt-30-marked}
    & Simple &  &  &  &  \\
    & Attend &  &  &  &  \\ \hline \hline

    \multirow{2}{*}{fb-irt-1-marked}
    & Simple &  &  &  &  \\
    & Attend &  &  &  &  \\ \hline

    \multirow{2}{*}{fb-irt-5-marked}
    & Simple &  &  &  &  \\
    & Attend &  &  &  &  \\ \hline

    \multirow{2}{*}{fb-irt-15-marked}
    & Simple &  &  &  &  \\
    & Attend &  &  &  &  \\ \hline

    \multirow{2}{*}{fb-irt-30-marked}
    & Simple &  &  &  &  \\
    & Attend &  &  &  &  \\ \hline

    \multirow{2}{*}{fb-owe-1-clean}
    & Simple &  &  &  &  \\
    & Attend &  &  &  &  \\ \hline

\end{tabular}

    \caption{Static Texter with randomly initialized word embeddings of varying size. Numbers show F1 scores. Best result per row marked bold. The simple Texter profits from very large embeddings, while the attentive Texter's performance decreases from medium sizes on.}
    \label{tab:5_experiments/3_texter/2_static/2_emb_size/grid_search}
\end{table}

At first glance, one can see that training does actually work and that the simple Texter performs much better with randomly initialized embeddings. Against expectations, the attentive Texter reaches its top performance at an embedding size of only around 30 to 100, while the simple model benefits from very large embeddings beyond 300. Apart from one outlier, the performances of both models evolve similarly for small embedding sizes until the attentive Texter stagnates early at medium sizes. For the attentive model, the poor performance seems reasonable since semantically similar words, onto which the same class embedding should match, have completely different initial values, which should complicate the class embeddings' training. Another aspect that catches the eye when looking at \autoref{tab:5_experiments/3_texter/2_static/2_emb_size/grid_search} is that even one-dimensional embeddings deliver surprisingly good results. In the case of the CDE split, the appearance is deceptive, as the F1 values are actually below the best zero-rule baseline, which reaches about 20\%, but for the FB split, the results are indeed better than the best zero-rule baseline with only 9\%. Overall, it can be said that randomly initialized embeddings can work, but, despite the good results for the simple Texter, they are not considered in further experiments, as pre-trained embeddings yield better results for both models as shown in the next subsection.


\subsubsection{Using pre-trained word embeddings}
\label{subsubsec:5_experiments/4_texter/2_static/3_pre_trained}
Studies show that pre-trained word embeddings can produce better results after a shorter training period - especially when little training data is available~\cite{}. Especially for the attentive model, there was the assumption that pre-trained embeddings could help the model to focus on training the class embeddings. Until the unexpectedly good experiment with randomly initialized embeddings, pre-trained embeddings even seemed indispensable.

When pre-trained embeddings are provided, Texter's embedding block initializes as many of the vocabulary's words with them as possible. Tokens for which no pre-trained embeddings are available are still initialized randomly. Therefore, when using pre-trained embeddings, it is important to use a tokenizer that produces tokens similar to the ones produced during pre-training - another reason why SpaCy's tokenizer led to better results than the Whitespace tokenizer.

Just as the models used in pre-training use different tokenizers, the resulting pre-trained embeddings also differ. In the experiment at hand, a total of 13 variants of three different types of pre-trained embedding sets were tried. The embedding sets vary in embedding size, vocabulary size and the text corpus pre-training was conducted on. Table~\ref{tab:5_experiments/4_texter/2_static/3_pre_trained/vector_sets} lists all embedding sets tried during the experiment.

\begin{table}[h]
    \centering
    \begin{tabular}{ l r r }
    \toprule
    
    \multicolumn{1}{l}{\textbf{Vectors}} &
    \multicolumn{1}{c}{\textbf{Emb Size}} &
    \multicolumn{1}{c}{\textbf{Vocab Size}} \\
    
    \midrule
    
    charngram.100d & 100 & \num{874474} \\
    
    \addlinespace
    
    fasttext.simple.300d & 300 & \num{111051} \\
    
    \addlinespace
    
    fasttext.en.300d & 300 & \num{2519370} \\
    
    \addlinespace
    
    glove.6B.50d  &  50 & \multirow{4}{*}{\num{400000}} \\
    glove.6B.100d & 100 &                               \\
    glove.6B.200d & 200 &                               \\
    glove.6B.300d & 300 &                               \\
    
    \addlinespace
    
    glove.twitter.27B.25d  &  25 & \multirow{4}{*}{\num{1917494}} \\
    glove.twitter.27B.50d  &  50 &                                \\
    glove.twitter.27B.100d & 100 &                                \\
    glove.twitter.27B.200d & 200 &                                \\
    
    \addlinespace
    
    glove.42B.300d & 300 & \num{1917494} \\
    
    \addlinespace
    
    glove.840B.300d & 300 & \num{2196017} \\
    
    \bottomrule
\end{tabular}

    \caption{Pre-trained word embedding sets considered for evaluation}
    \label{tab:5_experiments/4_texter/2_static/3_pre_trained/vector_sets}
\end{table}

The three types of embedding differ in how word embeddings are obtained:

\begin{itemize}
    \item \textbf{\emph{GloVe}}~\cite{Pennington2014GloveGV}, coined from "global vectors", is an unsupervised learning algorithm dedicated to obtaining the popular, equally named embeddings. GloVe learns embeddings for whole words so that co-occurring, semantically similar words are close in embedding space. Thereby, remarkable relations between related word embeddings emerge, such as the equation $king - queen = man - woman$. Since English has many different words, GloVe has a very large vocabulary.

    \item \textbf{\emph{fastText}}~\cite{Bojanowski2017EnrichingWV,Mikolov2018AdvancesIP}  is based on the idea that words can be viewed as sums of n-grams they exist of. Leveraging the internal structure of words has the advantage that rare or even unknown words can be handled. For example, different declensions of verbs can be related without resorting to stemming, that is, without reducing words to their root. The approach of composing words from n-grams also offers the potential advantage of a small vocabulary.

    \item \textbf{\emph{Charagram}}, referred to as charngram in the library used for implementing Power's Texter, forms word embeddings, as the name suggests, from charachter n-grams as well. Charagram was introduced at the same time as fastText and serves as an alternative to fastText in the experiment.
\end{itemize}

In the experiment all 13 embedding sets were evaluated. For brevity, Table~\ref{tab:5_experiments/4_texter/2_static/3_pre_trained/vector_sets} is limited to the evaluation results for the Charagram, fastText and GloVe embeddings obtained from training on Wikipedia and Gigaword. Thus, the selected embedding sets cover all three embedding types and give an impression of the influence the embedding size has by comparing the four otherwise equal GloVe embedding sets. Table~\ref{} in Appendix~\ref{ch:a_appendix} contains the second part of the table listing the results for the remaining six GloVe embedding sets.

The expected outcome of the experiment was a significant performance increase for both, simple and attentive Texter, just as other works did. Between the embedding sets, medium deviations were assumed, depending on how similar the data during pre-training resembled those of the IRT text sets. In case of the text sets with CDE and IRT sentences taken from Wikipedia, the GloVe embeddings were therefore the favorite candidate. Regarding the embedding type, there were no clear expectations, as both a large GloVe vocabulary and compound charachter n-grams should be able to cover the vocabulary of the natural language texts. However, it was assumed that a larger text corpus used for pre-training would lead to a noteable performance improvement, for example for fasttext.en.300d compared to fasttext.simple.300d or glove.840B.300d compared to glove.42B.300d. Finally, it was expected that an increase in embedding size would have a similar strong effect as for randomly initialized embeddings.

\begin{table}[h]
    \begin{tabular}{| l | l | r | r | r | r |}
    \hline

    \multicolumn{1}{|c|}{\multirow{2}{*}{\textbf{Text Set}}} &
    \multicolumn{1}{|c|}{\multirow{2}{*}{\textbf{Texter}}} &
    \multicolumn{4}{|c|}{\textbf{Pooling}} \\

    &
    &
    \multicolumn{1}{|c|}{\textbf{CLS}} &
    \multicolumn{1}{|c|}{\textbf{No CLS}} &
    \multicolumn{1}{|c|}{\textbf{No Pad}} &
    \multicolumn{1}{|c|}{\textbf{All}} \\

    \hline \hline

    \multirow{2}{*}{cde-cde-1-clean}
    & Simple &  &  &  &  \\
    & Attend &  &  &  &  \\ \hline

    \multirow{2}{*}{cde-irt-1-marked}
    & Simple &  &  &  &  \\
    & Attend &  &  &  &  \\ \hline

    \multirow{2}{*}{cde-irt-5-marked}
    & Simple &  &  &  &  \\
    & Attend &  &  &  &  \\ \hline

    \multirow{2}{*}{cde-irt-15-marked}
    & Simple &  &  &  &  \\
    & Attend &  &  &  &  \\ \hline

    \multirow{2}{*}{cde-irt-30-marked}
    & Simple &  &  &  &  \\
    & Attend &  &  &  &  \\ \hline \hline

    \multirow{2}{*}{fb-irt-1-marked}
    & Simple &  &  &  &  \\
    & Attend &  &  &  &  \\ \hline

    \multirow{2}{*}{fb-irt-5-marked}
    & Simple &  &  &  &  \\
    & Attend &  &  &  &  \\ \hline

    \multirow{2}{*}{fb-irt-15-marked}
    & Simple &  &  &  &  \\
    & Attend &  &  &  &  \\ \hline

    \multirow{2}{*}{fb-irt-30-marked}
    & Simple &  &  &  &  \\
    & Attend &  &  &  &  \\ \hline

    \multirow{2}{*}{fb-owe-1-clean}
    & Simple &  &  &  &  \\
    & Attend &  &  &  &  \\ \hline

\end{tabular}

    \makebox[\textwidth][c]{
    }
    \caption{Pre-trained embs}
    \label{tab:5_experiments/4_texter/2_static/3_pre_trained/grid_search}
\end{table}

As Tables~\ref{tab:5_experiments/4_texter/2_static/3_pre_trained/vector_sets} and~\ref{table:appendix/static_vectors_2} show, many of the assumptions were not met to the extent expected. Overall, it does not matter very much which embedding set is chosen. All three embedding types offer similar top performances with their respective best embedding sets, a multiplication of pre-training data is not necessarily better, as a look into Table~\ref{table:appendix/static_vectors_2} reveals, and even increasing the embedding size causes only a marginal improvement. Overall, pre-trained embeddings yield better results than randomly initialized ones, so that the last statement can also be formulated the other way around, namely that pre-trained embeddings already work well with small embedding sizes. Besides the small difference between the embedding sets, it is noticeable that the attentive Texter benefits particularly strongly from pre-trained embeddings, although the simple model still performs better on most text sets. On closer inspection, there is a tendency for the attentive model to work better with charachter n-grams, while the simple model might work slightly better with GloVe embeddings. However, the latter performance gain is so small that all further experiments were performed with character n-grams for the sake of clarity. Among the n-gram based Charagram and fastText embeddings, the slightly better fastText embeddings were chosen, and among these again the smaller fasttext.simple.300d embeddings set whose performance is barely distinguishable from fasttext.en.300d.


\subsubsection{Freezing pre-trained embeddings}
\label{subsubsec:5_experiments/4_texter/2_static/4_update_vectors}
\begin{table}[t!]
    \makebox[\textwidth][c]{
        \begin{tabular}{| l | l | r | r | r | r |}
    \hline

    \multicolumn{1}{|c|}{\multirow{2}{*}{\textbf{Text Set}}} &
    \multicolumn{1}{|c|}{\multirow{2}{*}{\textbf{Texter}}} &
    \multicolumn{4}{|c|}{\textbf{Pooling}} \\

    &
    &
    \multicolumn{1}{|c|}{\textbf{CLS}} &
    \multicolumn{1}{|c|}{\textbf{No CLS}} &
    \multicolumn{1}{|c|}{\textbf{No Pad}} &
    \multicolumn{1}{|c|}{\textbf{All}} \\

    \hline \hline

    \multirow{2}{*}{cde-cde-1-clean}
    & Simple &  &  &  &  \\
    & Attend &  &  &  &  \\ \hline

    \multirow{2}{*}{cde-irt-1-marked}
    & Simple &  &  &  &  \\
    & Attend &  &  &  &  \\ \hline

    \multirow{2}{*}{cde-irt-5-marked}
    & Simple &  &  &  &  \\
    & Attend &  &  &  &  \\ \hline

    \multirow{2}{*}{cde-irt-15-marked}
    & Simple &  &  &  &  \\
    & Attend &  &  &  &  \\ \hline

    \multirow{2}{*}{cde-irt-30-marked}
    & Simple &  &  &  &  \\
    & Attend &  &  &  &  \\ \hline \hline

    \multirow{2}{*}{fb-irt-1-marked}
    & Simple &  &  &  &  \\
    & Attend &  &  &  &  \\ \hline

    \multirow{2}{*}{fb-irt-5-marked}
    & Simple &  &  &  &  \\
    & Attend &  &  &  &  \\ \hline

    \multirow{2}{*}{fb-irt-15-marked}
    & Simple &  &  &  &  \\
    & Attend &  &  &  &  \\ \hline

    \multirow{2}{*}{fb-irt-30-marked}
    & Simple &  &  &  &  \\
    & Attend &  &  &  &  \\ \hline

    \multirow{2}{*}{fb-owe-1-clean}
    & Simple &  &  &  &  \\
    & Attend &  &  &  &  \\ \hline

\end{tabular}

    }
    \caption{Static Texters when (not) freezing the pre-trained embeddings. Numbers show F1 scores. Best result per row marked bold. Freezing pre-trained word embeddings leads to worse results in every case.}
    \label{tab:5_experiments/3_texter/2_static/4_update_vectors/grid_search}
\end{table}

When using pre-trained word embeddings, the embeddings adapt to the new training data during fine-tuning and may lose their special properties~\cite{He2019AnalyzingTF}. To prevent this, the new training data used for fine-tuning is sometimes mixed in with the training data used during pre-training. Another option is to freeze the pre-trained word embeddings during training so that other parameters must align themselves more strongly while the word embeddings remain constant. The latter approach was considered in an attempt to enhance the attentive Texter's class embeddings. The outcome of the experiment was uncertain. On the one hand, the model is deprived of a large part of its parameters, leaving only the class embeddings and linear layers to be learned, on the other hand, it could focus the training on the class embeddings and prevent overfitting. \autoref{tab:5_experiments/3_texter/2_static/4_update_vectors/grid_search} shows the results. In addition to the fasttext.simple.300d embeddings, charngram.300d, and glove.6B.300d were also examined, as it was conceivable that embedding freezing would have different effects depending on the embedding type.

As the numbers inevitably show, freezing the pre-trained embeddings results in worse performance for every dataset and every type of embedding which is why no further experiments make use of it. Nevertheless, it is interesting to observe how different the negative effects are: CharNGram shows immense performance losses while GloVe embeddings handle the restriction well on text sets with few IRT sentences per entity. Nonetheless, even if the negative effects cannot be compensated, the assumption that freezing the word embeddings has a positive effect on the attention mechanism seems to be valid, since the attentive Texter's performance drops less on text sets with multiple IRT sentences compared to the simple model.


\subsubsection{Pooling}
\label{subsubsec:5_experiments/4_texter/2_static/5_pooling}
from embedded tokens multiple ways to chose from
for classification usually take class embedding, as done by bert authors (?), captures whole sentence meaning
alternative idea is to average word embeddings instead
while on it, also tried average cls + word embs and average all embs including out of sentence, i.e. padding
table \ref{tab:5_experiments/4_texter/3_context/2_pooling/grid_search} shows results

\begin{table}[h]
    \centering
    \begin{tabular}{| l | l | r | r | r | r |}
    \hline

    \multicolumn{1}{|c|}{\multirow{2}{*}{\textbf{Text Set}}} &
    \multicolumn{1}{|c|}{\multirow{2}{*}{\textbf{Texter}}} &
    \multicolumn{4}{|c|}{\textbf{Pooling}} \\

    &
    &
    \multicolumn{1}{|c|}{\textbf{CLS}} &
    \multicolumn{1}{|c|}{\textbf{No CLS}} &
    \multicolumn{1}{|c|}{\textbf{No Pad}} &
    \multicolumn{1}{|c|}{\textbf{All}} \\

    \hline \hline

    \multirow{2}{*}{cde-cde-1-clean}
    & Simple &  &  &  &  \\
    & Attend &  &  &  &  \\ \hline

    \multirow{2}{*}{cde-irt-1-marked}
    & Simple &  &  &  &  \\
    & Attend &  &  &  &  \\ \hline

    \multirow{2}{*}{cde-irt-5-marked}
    & Simple &  &  &  &  \\
    & Attend &  &  &  &  \\ \hline

    \multirow{2}{*}{cde-irt-15-marked}
    & Simple &  &  &  &  \\
    & Attend &  &  &  &  \\ \hline

    \multirow{2}{*}{cde-irt-30-marked}
    & Simple &  &  &  &  \\
    & Attend &  &  &  &  \\ \hline \hline

    \multirow{2}{*}{fb-irt-1-marked}
    & Simple &  &  &  &  \\
    & Attend &  &  &  &  \\ \hline

    \multirow{2}{*}{fb-irt-5-marked}
    & Simple &  &  &  &  \\
    & Attend &  &  &  &  \\ \hline

    \multirow{2}{*}{fb-irt-15-marked}
    & Simple &  &  &  &  \\
    & Attend &  &  &  &  \\ \hline

    \multirow{2}{*}{fb-irt-30-marked}
    & Simple &  &  &  &  \\
    & Attend &  &  &  &  \\ \hline

    \multirow{2}{*}{fb-owe-1-clean}
    & Simple &  &  &  &  \\
    & Attend &  &  &  &  \\ \hline

\end{tabular}

    \caption{Pooling}
    \label{tab:5_experiments/4_texter/3_context/2_pooling/grid_search}
\end{table}

best is to average all embs followed by word embs followed by cls default
but difference not big, absolute ?\%


\subsubsection{Activation Function}
\label{subsubsec:5_experiments/4_texter/2_static/6_activation}
In case of the attentive Texter, the sentence embeddings generated by the embedding block are subsequently passed on to the attention block where they are scalar multiplied with the class embeddings to obtain the attention values that specify how well a class embedding matches each of the entity's sentences. For normalization purposes a classes' attentions are further pushed through a non-linear activation function before the resulting values serve as weight factors in calculating the class-specific entity embeddings returned from the attention block.

Initially, the softmax function was considered as the activation function, so that the attention mechanism is forced to compare all of an entities sentencies to each other. On the other hand, it is problematic that the attention weights sum to 1 even if none of the sentences, or all of them, fit the class. In the first case, the class embedding converges towards unrelated sentence embeddings during backpropagation and in the second case several insightful sentences are not fully utilized, because each one's gradient is lower than possible. Therefore, the sigmoid function was considered as an alternative, allowing each sentence embedding to be weighted by its attention value in $[0, 1]$ individually. While on it, the gererally popular ReLu function was also tested. Out of interest in the principle utility of a non-linear activation function, it was also measured what happens when no activation function is applied at all. Both the relu function and the omission of an activation function allow outliers with a large scalar product to have great influence on the learning process, which was expected to have a negative impact. In the comparison between Softmax and Sigmoid, the outcome was uncertain. \autoref{tab:5_experiments/3_texter/2_static/6_activation/grid_search} shows the results of varying the activation function for the attentive Texter only, as the simple model does not have an attention block.

\begin{table}[t]
    \centering
    \begin{tabular}{| l | l | r | r | r | r |}
    \hline

    \multicolumn{1}{|c|}{\multirow{2}{*}{\textbf{Text Set}}} &
    \multicolumn{1}{|c|}{\multirow{2}{*}{\textbf{Texter}}} &
    \multicolumn{4}{|c|}{\textbf{Pooling}} \\

    &
    &
    \multicolumn{1}{|c|}{\textbf{CLS}} &
    \multicolumn{1}{|c|}{\textbf{No CLS}} &
    \multicolumn{1}{|c|}{\textbf{No Pad}} &
    \multicolumn{1}{|c|}{\textbf{All}} \\

    \hline \hline

    \multirow{2}{*}{cde-cde-1-clean}
    & Simple &  &  &  &  \\
    & Attend &  &  &  &  \\ \hline

    \multirow{2}{*}{cde-irt-1-marked}
    & Simple &  &  &  &  \\
    & Attend &  &  &  &  \\ \hline

    \multirow{2}{*}{cde-irt-5-marked}
    & Simple &  &  &  &  \\
    & Attend &  &  &  &  \\ \hline

    \multirow{2}{*}{cde-irt-15-marked}
    & Simple &  &  &  &  \\
    & Attend &  &  &  &  \\ \hline

    \multirow{2}{*}{cde-irt-30-marked}
    & Simple &  &  &  &  \\
    & Attend &  &  &  &  \\ \hline \hline

    \multirow{2}{*}{fb-irt-1-marked}
    & Simple &  &  &  &  \\
    & Attend &  &  &  &  \\ \hline

    \multirow{2}{*}{fb-irt-5-marked}
    & Simple &  &  &  &  \\
    & Attend &  &  &  &  \\ \hline

    \multirow{2}{*}{fb-irt-15-marked}
    & Simple &  &  &  &  \\
    & Attend &  &  &  &  \\ \hline

    \multirow{2}{*}{fb-irt-30-marked}
    & Simple &  &  &  &  \\
    & Attend &  &  &  &  \\ \hline

    \multirow{2}{*}{fb-owe-1-clean}
    & Simple &  &  &  &  \\
    & Attend &  &  &  &  \\ \hline

\end{tabular}

    \caption{Static Texters with various activation functions in the attention block. Numbers show F1 scores. Best result per row marked bold. The sigmoid function works best. Using no activation function works surprisingly well.}
    \label{tab:5_experiments/3_texter/2_static/6_activation/grid_search}
\end{table}

In line with expectations, the Sigmoid and Softmax functions produce the best results. For all text sets that provide multiple sentences per entity, however, Sigmoid is ahead by a few percentage points. On single-sentence text sets Sigmoid and Softmax perform similiarly. Interestingly, not using an activation function performs only slightly worse than Softmax on the CDE split and even similarly well on the FB split. On the text set, even the best results are achieved without an activation function. The otherwise promising ReLu function was inferior to the other functions in this experiment. For all further experiments, the sigmoid function was used as the standard.


\subsubsection{Appplying class weights}
\label{subsubsec:5_experiments/4_texter/2_static/7_weight_factor}
After the entity embeddings have been formed from the sentence embeddings in the attention block, they are passed through the classification block which outputs the final class logits of the forward() function. In case of the simple model, the attention block is omitted and the sentence embeddings are used directly for classification. In either case, the forward() function's class logits are taken as input by the loss function during training to calculate the model's loss in regard to the ground truth class labels. To make the logits comparable with the usual 0 and 1 ground truth labels, they are normalized to range $[0, 1]$ during the loss calculation - usually by applying the sigmoid function.

For the multilabel problem at hand, the binary cross-entropy (BCE) loss function is used. A convenient property of the BCE function is that it produces very large loss values for very wrong predictions which contributes to an initially fast learning process. To counteract the unbalanced classes in the Power dataset, the weighted binary cross entropy loss function (wBCE) shown in Equation~\ref{eq:5_experiments/4_texter/2_static/7_weight_factor/wbce} is used. It calculates the loss for the multi-label output logits $x$ and the respective ground truth labels $y$, both of which are c-dimensional vectors with $c$ being the number of output classes. Without the weights, the model would learn that it gets off best by always making negative predictions for very rare classes. This would be good for high accuracy, but is bad for the metrics used during evaluation, which do not measure true negatives. A class' weight $w_c$ is calculated as the reciprocal of the class' frequence it occurs in the training data with. A class that is true for every fifth entity, for example, would be assigned a class weight of five.

\begin{align}
    wBCE(x, y) = - \frac{1}{C} \sum_{c = 1}^C w_c \cdot log(\sigma(x)) + (1 - y) \cdot log(1 - \sigma(x))
    \label{eq:5_experiments/4_texter/2_static/7_weight_factor/wbce}
\end{align}

In an experiment, whose results fill Table~\ref{tab:5_experiments/4_texter/2_static/7_weight_factor/grid_search}, it should be verified that the use of class weights has a positive effect. Furthermore, it should be ensured that the choice to calculate the weights as the reciprocal of the class frequencies is optimal. Therefore, also smaller and larger weights were included in the comparison by halving and doubling the classes' reciprocals, respectively.

\begin{table}[t]
    \centering
    \begin{tabular}{| l | l | r | r | r | r |}
    \hline

    \multicolumn{1}{|c|}{\multirow{2}{*}{\textbf{Text Set}}} &
    \multicolumn{1}{|c|}{\multirow{2}{*}{\textbf{Texter}}} &
    \multicolumn{4}{|c|}{\textbf{Pooling}} \\

    &
    &
    \multicolumn{1}{|c|}{\textbf{CLS}} &
    \multicolumn{1}{|c|}{\textbf{No CLS}} &
    \multicolumn{1}{|c|}{\textbf{No Pad}} &
    \multicolumn{1}{|c|}{\textbf{All}} \\

    \hline \hline

    \multirow{2}{*}{cde-cde-1-clean}
    & Simple &  &  &  &  \\
    & Attend &  &  &  &  \\ \hline

    \multirow{2}{*}{cde-irt-1-marked}
    & Simple &  &  &  &  \\
    & Attend &  &  &  &  \\ \hline

    \multirow{2}{*}{cde-irt-5-marked}
    & Simple &  &  &  &  \\
    & Attend &  &  &  &  \\ \hline

    \multirow{2}{*}{cde-irt-15-marked}
    & Simple &  &  &  &  \\
    & Attend &  &  &  &  \\ \hline

    \multirow{2}{*}{cde-irt-30-marked}
    & Simple &  &  &  &  \\
    & Attend &  &  &  &  \\ \hline \hline

    \multirow{2}{*}{fb-irt-1-marked}
    & Simple &  &  &  &  \\
    & Attend &  &  &  &  \\ \hline

    \multirow{2}{*}{fb-irt-5-marked}
    & Simple &  &  &  &  \\
    & Attend &  &  &  &  \\ \hline

    \multirow{2}{*}{fb-irt-15-marked}
    & Simple &  &  &  &  \\
    & Attend &  &  &  &  \\ \hline

    \multirow{2}{*}{fb-irt-30-marked}
    & Simple &  &  &  &  \\
    & Attend &  &  &  &  \\ \hline

    \multirow{2}{*}{fb-owe-1-clean}
    & Simple &  &  &  &  \\
    & Attend &  &  &  &  \\ \hline

\end{tabular}

    \caption{Applying different class weights}
    \label{tab:5_experiments/4_texter/2_static/7_weight_factor/grid_search}
\end{table}

The empirical values are in line with the theoretical expectations. Although omitting class weights produces useful values - especially on the FB split where the 100 most frequent classes have higher frequencies than on the CDE split - applying class weights significantly improve performance. Thereby, it seems more important to apply weights at all than to fine-tune the weights. In most cases, it does not matter much whether the weights are halved or doubled. On average, the plain reciprocals seem to be the best choice, which is why they are set as the default.


\subsubsection{Choosing an optimizer}
\label{subsubsec:5_experiments/4_texter/2_static/8_optimizer}
% TODO moved to approach

\begin{table}[t]
    \centering
    \begin{tabular}{| l | l | r | r | r | r |}
    \hline

    \multicolumn{1}{|c|}{\multirow{2}{*}{\textbf{Text Set}}} &
    \multicolumn{1}{|c|}{\multirow{2}{*}{\textbf{Texter}}} &
    \multicolumn{4}{|c|}{\textbf{Pooling}} \\

    &
    &
    \multicolumn{1}{|c|}{\textbf{CLS}} &
    \multicolumn{1}{|c|}{\textbf{No CLS}} &
    \multicolumn{1}{|c|}{\textbf{No Pad}} &
    \multicolumn{1}{|c|}{\textbf{All}} \\

    \hline \hline

    \multirow{2}{*}{cde-cde-1-clean}
    & Simple &  &  &  &  \\
    & Attend &  &  &  &  \\ \hline

    \multirow{2}{*}{cde-irt-1-marked}
    & Simple &  &  &  &  \\
    & Attend &  &  &  &  \\ \hline

    \multirow{2}{*}{cde-irt-5-marked}
    & Simple &  &  &  &  \\
    & Attend &  &  &  &  \\ \hline

    \multirow{2}{*}{cde-irt-15-marked}
    & Simple &  &  &  &  \\
    & Attend &  &  &  &  \\ \hline

    \multirow{2}{*}{cde-irt-30-marked}
    & Simple &  &  &  &  \\
    & Attend &  &  &  &  \\ \hline \hline

    \multirow{2}{*}{fb-irt-1-marked}
    & Simple &  &  &  &  \\
    & Attend &  &  &  &  \\ \hline

    \multirow{2}{*}{fb-irt-5-marked}
    & Simple &  &  &  &  \\
    & Attend &  &  &  &  \\ \hline

    \multirow{2}{*}{fb-irt-15-marked}
    & Simple &  &  &  &  \\
    & Attend &  &  &  &  \\ \hline

    \multirow{2}{*}{fb-irt-30-marked}
    & Simple &  &  &  &  \\
    & Attend &  &  &  &  \\ \hline

    \multirow{2}{*}{fb-owe-1-clean}
    & Simple &  &  &  &  \\
    & Attend &  &  &  &  \\ \hline

\end{tabular}

    \caption{Static Texter when applying different learning rates during training. Numbers show F1 scores. Best result per row marked bold. The simple Texter can be trained with high learning rates while the attentive Texter is more sensible, especially on datasets with many sentences per entity.}
    \label{tab:5_experiments/3_texter/2_static/8_optimizer/grid_search}
\end{table}

In this work, SGD with momentum~\cite{Qian1999OnTM} and Adam~\cite{Kingma2015AdamAM} are tried. SGD with Momentum serves as a representative of the classical, non-adaptive gradient descent methods for which are generally better suited to find the minimum of a loss function~\cite{Wilson2017TheMV}. The adaptive optimizer Adam, on the other hand, is particularly popular~\cite{AdamPopular} and is generally considered fast and good. In the experiment, SGD's momentum constant was set to 0.9.

Directly related to the optimizer is the learning rate. Both optimizers were tested with different learning rates from a range that should allow training within a reasonable time. It was expected that lower learning rates always lead to a better result given sufficient training time, but that the improvements become smaller and smaller as the learning rate decreases, so that a sufficiently good learning rate can be declared for each optimizer. SGD with momentum was expected perform slightly better after a longer training time. Depending on the ratio of additional training time to gained performance, a decision should be made between the optimizers.

\begin{figure}[t]
    \centering
    \subfloat[Simple, SGD]{
    \begin{tikzpicture}
        \begin{axis}[
            axis lines = middle,
            cycle list name = tb,
            grid = both,
            legend pos = outer north east,
            scale = 0.8,
            xlabel = epoch,
            ylabel = F1,
        ]
            \addplot table [x = Step, y = Value, col sep = comma] {5_experiments/4_texter/2_static/8_optimizer/sgd_vs_adam/simple_sgd/lr_1_0.csv};
            \addplot table [x = Step, y = Value, col sep = comma] {5_experiments/4_texter/2_static/8_optimizer/sgd_vs_adam/simple_sgd/lr_0_3.csv};
            \addplot table [x = Step, y = Value, col sep = comma] {5_experiments/4_texter/2_static/8_optimizer/sgd_vs_adam/simple_sgd/lr_0_1.csv};
            \addplot table [x = Step, y = Value, col sep = comma] {5_experiments/4_texter/2_static/8_optimizer/sgd_vs_adam/simple_sgd/lr_0_03.csv};
            \addplot table [x = Step, y = Value, col sep = comma] {5_experiments/4_texter/2_static/8_optimizer/sgd_vs_adam/simple_sgd/lr_0_01.csv};
            \addplot table [x = Step, y = Value, col sep = comma] {5_experiments/4_texter/2_static/8_optimizer/sgd_vs_adam/simple_sgd/lr_0_003.csv};
            \addplot table [x = Step, y = Value, col sep = comma] {5_experiments/4_texter/2_static/8_optimizer/sgd_vs_adam/simple_sgd/lr_0_001.csv};
        \end{axis}
    \end{tikzpicture}
    \label{fig:5_experiments/4_texter/2_static/8_optimizer/sgd_vs_adam/simple_sgd}
}
\hskip 5pt
\subfloat[Attentive, SGD]{
    \begin{tikzpicture}
        \begin{axis}[
            axis lines = middle,
            cycle list name = tb,
            grid = both,
            legend pos = outer north east,
            scale = 0.8,
            xlabel = epoch,
            ylabel = F1,
        ]
            \addplot table [x = Step, y = Value, col sep = comma] {5_experiments/4_texter/2_static/8_optimizer/sgd_vs_adam/attentive_sgd/lr_1_0.csv};
            \addplot table [x = Step, y = Value, col sep = comma] {5_experiments/4_texter/2_static/8_optimizer/sgd_vs_adam/attentive_sgd/lr_0_3.csv};
            \addplot table [x = Step, y = Value, col sep = comma] {5_experiments/4_texter/2_static/8_optimizer/sgd_vs_adam/attentive_sgd/lr_0_1.csv};
            \addplot table [x = Step, y = Value, col sep = comma] {5_experiments/4_texter/2_static/8_optimizer/sgd_vs_adam/attentive_sgd/lr_0_03.csv};
            \addplot table [x = Step, y = Value, col sep = comma] {5_experiments/4_texter/2_static/8_optimizer/sgd_vs_adam/attentive_sgd/lr_0_01.csv};
            \addplot table [x = Step, y = Value, col sep = comma] {5_experiments/4_texter/2_static/8_optimizer/sgd_vs_adam/attentive_sgd/lr_0_003.csv};
            \addplot table [x = Step, y = Value, col sep = comma] {5_experiments/4_texter/2_static/8_optimizer/sgd_vs_adam/attentive_sgd/lr_0_001.csv};
        \end{axis}
    \end{tikzpicture}
    \label{fig:5_experiments/4_texter/2_static/8_optimizer/sgd_vs_adam/attentive_sgd}
}

\subfloat[Simple, Adam]{
    \begin{tikzpicture}
        \begin{axis}[
            axis lines = middle,
            cycle list name = tb,
            grid = both,
            legend pos = outer north east,
            scale = 0.8,
            xlabel = epoch,
            ylabel = F1,
        ]
            \addplot table [x = Step, y = Value, col sep = comma] {5_experiments/4_texter/2_static/8_optimizer/sgd_vs_adam/simple_adam/lr_1_0.csv};
            \addplot table [x = Step, y = Value, col sep = comma] {5_experiments/4_texter/2_static/8_optimizer/sgd_vs_adam/simple_adam/lr_0_3.csv};
            \addplot table [x = Step, y = Value, col sep = comma] {5_experiments/4_texter/2_static/8_optimizer/sgd_vs_adam/simple_adam/lr_0_1.csv};
            \addplot table [x = Step, y = Value, col sep = comma] {5_experiments/4_texter/2_static/8_optimizer/sgd_vs_adam/simple_adam/lr_0_03.csv};
            \addplot table [x = Step, y = Value, col sep = comma] {5_experiments/4_texter/2_static/8_optimizer/sgd_vs_adam/simple_adam/lr_0_01.csv};
            \addplot table [x = Step, y = Value, col sep = comma] {5_experiments/4_texter/2_static/8_optimizer/sgd_vs_adam/simple_adam/lr_0_003.csv};
            \addplot table [x = Step, y = Value, col sep = comma] {5_experiments/4_texter/2_static/8_optimizer/sgd_vs_adam/simple_adam/lr_0_001.csv};
        \end{axis}
    \end{tikzpicture}
    \label{fig:5_experiments/4_texter/2_static/8_optimizer/sgd_vs_adam/simple_adam}
}
\hskip 5pt
\subfloat[Attentive, Adam]{
    \begin{tikzpicture}
        \begin{axis}[
            axis lines = middle,
            cycle list name = tb,
            grid = both,
            legend pos = outer north east,
            scale = 0.8,
            xlabel = epoch,
            ylabel = F1,
        ]
            \addplot table [x = Step, y = Value, col sep = comma] {5_experiments/4_texter/2_static/8_optimizer/sgd_vs_adam/attentive_adam/lr_1_0.csv};
            \addplot table [x = Step, y = Value, col sep = comma] {5_experiments/4_texter/2_static/8_optimizer/sgd_vs_adam/attentive_adam/lr_0_3.csv};
            \addplot table [x = Step, y = Value, col sep = comma] {5_experiments/4_texter/2_static/8_optimizer/sgd_vs_adam/attentive_adam/lr_0_1.csv};
            \addplot table [x = Step, y = Value, col sep = comma] {5_experiments/4_texter/2_static/8_optimizer/sgd_vs_adam/attentive_adam/lr_0_03.csv};
            \addplot table [x = Step, y = Value, col sep = comma] {5_experiments/4_texter/2_static/8_optimizer/sgd_vs_adam/attentive_adam/lr_0_01.csv};
            \addplot table [x = Step, y = Value, col sep = comma] {5_experiments/4_texter/2_static/8_optimizer/sgd_vs_adam/attentive_adam/lr_0_003.csv};
            \addplot table [x = Step, y = Value, col sep = comma] {5_experiments/4_texter/2_static/8_optimizer/sgd_vs_adam/attentive_adam/lr_0_001.csv};
        \end{axis}
    \end{tikzpicture}
    \label{fig:5_experiments/4_texter/2_static/8_optimizer/sgd_vs_adam/attentive_adam}
}

    \caption{Static Texter optimized via SGD and Adam with high to low learning rates (red = 1.0, orange, yellow, black, green, blue, purple = 0.001) on the fb-owe-1-clean text set. The Adam optimizer converges much faster, but does not work with high learning rates when training the attentive Texter.}
    \label{fig:5_experiments/3_texter/2_static/8_optimizer/sgd_vs_adam/sgd_vs_adam}
\end{figure}

Unfortunately, even at the highest learning rate, SGD converged too slowly to estimate definitive results. Therefore, \autoref{tab:5_experiments/3_texter/2_static/8_optimizer/grid_search} only shows the results for Adam. The four plots in \autoref{fig:5_experiments/3_texter/2_static/8_optimizer/sgd_vs_adam/sgd_vs_adam} illustrate the training of the simple and the attentive Texter using SGD and Adam with different learning rates. The upper plots show SGD's slowly converging curves while Adam's bottom plots paint a contrary picture. Even with the highest learning rate, SGD is slower than Adam with the lowest learning rate and would most likely need at least 200 more episodes to converge. An optimizer-independent observation is that the complex model appears to be more sensitive to learning rates that are too high. For SGD the highest learning rate leads to a slower training progress and for Adam the training does not work at all at the two highest learning rates. This information is also reflected in \autoref{tab:5_experiments/3_texter/2_static/8_optimizer/grid_search}, which also reveals another fact that is not directly evident from the plot: A lower learning rate does not automatically result in better performance. However, the nearest explanation is that training is simply not finished at that point. \autoref{tab:5_experiments/3_texter/2_static/8_optimizer/grid_search} suggests that the training of the attentive Texter is somewhat slower, but at a learning rate of 0.003, the training of both models appears to converge after the specified 200 epochs, which is why this learning rate is used in the other experiments.


\subsubsection{Inspecting the attention mechanism}
\label{subsubsec:5_experiments/4_texter/2_static/9_attention}
In the course of all the described experiments, the performance of both the simple and the complex Texter were steadily improved. What was missing was the hoped-for increase in performance of the attentive Texter compared to the simple version. Instead, the simple Texter actually performs better on many text sets. In order to get to the bottom of the cause, the attention mechanism should be investigated to find out whether the basic assumption that the class embeddings react to sentences in which class-typical keywords appear can be verified. In particular, it was examined (1) whether the trained class embeddings match keywords that are typical for the respective class, (2) whether the attention block favors sentences that would yield clear classification results on their own and (3) how the attention value related to a sentence results from the sentence's individual words.

Regarding the first question, the assumption was that the class embeddings converge against keywords that are particularly meaningful for a class during training. To confirm this conjecture, the class embeddings $class_c$, of a trained texter were scalar multiplied with all word embeddings $word_w$ of the vocabulary, the resulting scalar products $\langle class_c, word_w \rangle$ were sorted, and the values with the largest absolute values $|\langle class_c, word_w \rangle|$ were examined for plausibility. Thereby, the indices $1 <= c <= |C|$ and $1 <= w <= |V|$ specify one of the classes from the class set $C$ and a word from the vocabulary $V$, respectively. It was expected that among these top candidates mainly words would be found that clearly speak for or against the corresponding class. Table shows the top words the attentive Texter reacts to after being trained on the CDE split with cde-irt-5-clean texts.

\begin{table}[h]
    \centering
    \begin{tabular}{| l | l |}
    \hline

    \multicolumn{1}{|c|}{\textbf{Class}} &
    \multicolumn{1}{|c|}{\textbf{Top matching words}} \\

    \hline \hline

    (country of citizenship, United States of America) & nhs, mid-engined, ruislip, filipina   \\
    (occupation, writer)                               & recepter, 1/4, makovsky, pairwise     \\
    (languages spoken, written, or signed, English)    & republish, in-demand, brahmos, steyr \\

    \hline
\end{tabular}

    \caption{Top words}
    \label{tab:5_experiments/4_texter/2_static/9_attention/top_words}
\end{table}

Contrary to the assumption, the top words do not show a clear relation to the respective classes. With a few exceptions such as "Maryland" and "BBC" which appear further down in the suggestions for (from, USA) or (speaks, English), the words appear rather random. Given the rather infrequent occurrence of top words, overfitting seemed the most obvious explanation. However, the top words change with repeated training runs, so it seemed more likely that the word embeddings simply happened to be close to the learned class embeddings. As long as this only affects a small proportion of top words and the class embeddings are mostly close to relevant words, this should have little impact on the generalizability of the model. Whether this relation exists to more frequent keywords was investigated in the third experiment.

Before that, the second question was clarified whether the attention mechanism weights the sentences most heavily that are also the most meaningful on their own. For this purpose, the attentive Texter was extended by a method that bypasses the attention mechanism during inference. The trained model is still passed multiple sentences per entity when called, but instead of these being calculated entity embeddings in the attention block, the sentence embeddings are pushed directly through the classification block. That function, denoted as $\psi$ in the following, finally returns the non-aggregated class logits for each individual sentence. However, since the linear layers of the complex model were actually trained on the entity embeddings, this experiment only works with a texter in whose attention block the softmax function is used. This is because the Sigmoid version, unlike the Softmax version, does not have the property that the individual set embeddings have the same order of magnitude as the entity embeddings. Therefore, the results of the attentive model trained with Softmax on the cde-irt-5-clean texts are shown below. However, since the performance of the Softmax version is almost as high as that of the Sigmoid version, the results should be representative. Table~\ref{tab:5_experiments/4_texter/2_static/9_attention/kari_soft} presents exemplary values for the entity Kari Hotakainen, a Finnish writer. In addition to the five rather short sentences about Kari as well as the model predictions and the ground truth concerning the four most frequent classes in the CDE split, for each class-sentence combination it is stated to which proportion the sentence embedding normally enters into the entity embedding and which result the function $\psi$ yields when the attention mechanism is bypassed. Considering only the selected classes, the positive predictions for logits above the zero threshold in the example result in a precision of 33\%, a recall of 100\%, and thus an overall F1 score of 33\% and thereby close to the cde-irt-5-clean text set's average value of 37.6\%. However, it must be mentioned that the observable effectiveness of the attention mechanism is rather above average. The example was chosen mainly because of the short sentences.

\begin{table}[h]
    \centering
    \begin{tabular}{|l|l|r|r|r|r|}
\hline
\multicolumn{1}{|c|}{\multirow{2}{*}{\textbf{Sentence}}} &
   &
  \multicolumn{4}{c|}{\textbf{Class}} \\ \cline{2-6} 
\multicolumn{1}{|c|}{} &
   &
  \multicolumn{1}{c|}{\textbf{\begin{tabular}[c]{@{}c@{}}from\\ USA\end{tabular}}} &
  \multicolumn{1}{c|}{\textbf{\begin{tabular}[c]{@{}c@{}}is\\ writer\end{tabular}}} &
  \multicolumn{1}{c|}{\textbf{\begin{tabular}[c]{@{}c@{}}speaks\\ English\end{tabular}}} &
  \multicolumn{1}{c|}{\textbf{\begin{tabular}[c]{@{}c@{}}is\\ actor\end{tabular}}} \\ \cline{2-6} 
 &
  $\phi_c(S)$ &
  -1.26 &
  1.98 &
  0.22 &
  0.55 \\ \cline{2-6} 
 &
  GT &
  \multicolumn{1}{c|}{0} &
  \multicolumn{1}{c|}{1} &
  \multicolumn{1}{c|}{0} &
  \multicolumn{1}{c|}{0} \\ \hline
\begin{tabular}[c]{@{}l@{}}the film was written by\\ kari \_ and juha \_\end{tabular} &
  \multirow{5}{*}{\begin{tabular}[c]{@{}l@{}} \\ \\ \\ \\ $A_{cs}$ \\ $\psi_c(s)$\end{tabular}} &
  \begin{tabular}[c]{@{}r@{}} 0.16 \\ -0.16 \end{tabular} &
  \begin{tabular}[c]{@{}r@{}} 0.11 \\  2.01 \end{tabular} &
  \begin{tabular}[c]{@{}r@{}} 0.12 \\  0.88 \end{tabular} &
  \begin{tabular}[c]{@{}r@{}} 0.13 \\  0.11 \end{tabular} \\ \cline{1-1} \cline{3-6} 
\begin{tabular}[c]{@{}l@{}}it is based on kari \_ semi-\\ autobiographical novel \_\end{tabular} &
   &
  \begin{tabular}[c]{@{}r@{}} 0.22 \\ -1.06 \end{tabular} &
  \begin{tabular}[c]{@{}r@{}} \textbf{0.27} \\  \textbf{2.69} \end{tabular} &
  \begin{tabular}[c]{@{}r@{}} 0.21 \\  0.57 \end{tabular} &
  \begin{tabular}[c]{@{}r@{}} 0.22 \\ -0.06 \end{tabular} \\ \cline{1-1} \cline{3-6} 
\begin{tabular}[c]{@{}l@{}}the film is inspired by kari \\ \_ novel of the same name.\end{tabular} &
   &
  \begin{tabular}[c]{@{}r@{}} 0.12 \\  0.16 \end{tabular} &
  \begin{tabular}[c]{@{}r@{}} 0.26 \\  2.40 \end{tabular} &
  \begin{tabular}[c]{@{}r@{}} 0.17 \\  \textbf{1.24} \end{tabular} &
  \begin{tabular}[c]{@{}r@{}} 0.11 \\  0.02 \end{tabular} \\ \cline{1-1} \cline{3-6} 
\begin{tabular}[c]{@{}l@{}}the unknown \_ \_ () is an\\ authorised biography on \\ finnish racing driver \_ \_ \\ by kari \_\end{tabular} &
   &
  \begin{tabular}[c]{@{}r@{}} \textbf{0.32} \\ \textbf{-2.87} \end{tabular} &
  \begin{tabular}[c]{@{}r@{}} 0.23 \\  1.04 \end{tabular} &
  \begin{tabular}[c]{@{}r@{}} \textbf{0.29} \\ -0.64 \end{tabular} &
  \begin{tabular}[c]{@{}r@{}} 0.26 \\  1.01 \end{tabular} \\ \cline{1-1} \cline{3-6} 
\begin{tabular}[c]{@{}l@{}}\_ is a 2002 novel by \\ finnish author kari \_\end{tabular} &
   &
  \begin{tabular}[c]{@{}r@{}} 0.18 \\ -0.57 \end{tabular} &
  \begin{tabular}[c]{@{}r@{}} 0.13 \\  1.24 \end{tabular} &
  \begin{tabular}[c]{@{}r@{}} 0.22 \\ -0.07 \end{tabular} &
  \begin{tabular}[c]{@{}r@{}} \textbf{0.28} \\  \textbf{1.03} \end{tabular} \\ \hline
\end{tabular}

    \caption{Kari}
    \label{tab:5_experiments/4_texter/2_static/9_attention/kari_soft}
\end{table}

For the exmample entity, the attentive Texter does indeed prefer those sentences that would give a clear result on their own in case of three of the four classes. In the fourth case, the model heavily incorporates a sentence that has a rather uncertain outcome. Despite that the respective prediction would have been correct on its own, the model should not have focused on it. Looking at other sentences that have been heavily weighted, further misjudgments become apparent, such as the cases where the texter relied on the fifth and second sentences to predict the classes (speaks, English) and (is, actor), respectively, which both lead to unreliable decisions on their own. Overall, however, it is noticeable that meaningful sentences are more strongly included in the entity embeddings.

However, the effective focus on sentences with a clear prediction alone is of no use if the unambiguous decision made on the basis of the individual sentences is wrong. For example, in Table~\ref{tab:5_experiments/4_texter/2_static/9_attention/kari_soft}, the first three sentences incorrectly indicate that Kari Hotakainen speaks English. Also, he is predicted to be an actor which is not the case which is probably due to the (is, actor) class's general high posterior probability as the dataset contains many actors. Otherwise, the sentences lead to largely correct classifications and was less certain about its two false positives than about its two correct predictions.

What is still missing, however, is a clear prioritization of the sentences. For example, as a human being, one would give the last sentence a much higher weighting when it comes to the negation of the class (from, USA) due to the fragment "by finnish author kari hotakainen". But, using static word embeddings, the model cannot recognize at this point that the word "finnish" is much more relevant than in the preceding sentence in which the same word does not refer to the entity itself. When the experiment is repeated, it is also noticeable that the class logits for the entire entity fluctuate only slightly, but change noticeably between individual sentences, which also changes the prioritization of the sentences.

This uncertainty about the order of the sentences also exists in the final model that uses the sigmoid function, although to a lesser extent. Table~\ref{tab:5_experiments/4_texter/2_static/9_attention/kari} shows, analogous to Table~\ref{tab:5_experiments/4_texter/2_static/9_attention/kari_soft}, the attention values after the sigmoid function has been applied instead of the softmax, which is why the probabilities no longer sum up to 100\% columnwise. Furthermore, Table~\ref{tab:5_experiments/4_texter/2_static/9_attention/kari_soft} misses the results of the $\psi$ function which cannot be calculated here as explained before. The shown figures imply that the sigmoid version achieves a higher overall precision due to the correct prediction of (is, actor) and that the inclusion of uncertain sentences is generally lower, but even here the order between the relevant sentences varies among training runs.

\begin{table}[h]
    \centering
    \begin{tabular}{|l|l|r|r|r|r|}
\hline
\multicolumn{1}{|c|}{\multirow{2}{*}{\textbf{Sentence}}} &
   &
  \multicolumn{4}{c|}{\textbf{Class}} \\ \cline{2-6} 
\multicolumn{1}{|c|}{} &
   &
  \multicolumn{1}{c|}{\textbf{\begin{tabular}[c]{@{}c@{}}from\\ USA\end{tabular}}} &
  \multicolumn{1}{c|}{\textbf{\begin{tabular}[c]{@{}c@{}}is\\ writer\end{tabular}}} &
  \multicolumn{1}{c|}{\textbf{\begin{tabular}[c]{@{}c@{}}speaks\\ English\end{tabular}}} &
  \multicolumn{1}{c|}{\textbf{\begin{tabular}[c]{@{}c@{}}is\\ actor\end{tabular}}} \\ \cline{2-6} 
 &
  $\phi_c(S)$ &
  -1.43 &
  2.19 &
  0.85 &
  -0.14 \\ \cline{2-6} 
 &
  GT &
  \multicolumn{1}{c|}{0} &
  \multicolumn{1}{c|}{1} &
  \multicolumn{1}{c|}{0} &
  \multicolumn{1}{c|}{0} \\ \hline
\begin{tabular}[c]{@{}l@{}}the film was written by\\ kari \_ and juha \_\end{tabular} &
  \multirow{5}{*}{\begin{tabular}[c]{@{}l@{}} \\ \\ \\ \\ $\sigma(\langle e_c, e_s \rangle)$ \end{tabular}} &
  0.47 &
  \textbf{0.74} &
  0.27 &
  0.36 \\ \cline{1-1} \cline{3-6} 
\begin{tabular}[c]{@{}l@{}}it is based on kari \_ semi-\\ autobiographical novel \_\end{tabular} &
   &
  0.40 &
  0.54 &
  0.23 &
  0.23 \\ \cline{1-1} \cline{3-6} 
\begin{tabular}[c]{@{}l@{}}the film is inspired by kari \\ \_ novel of the same name.\end{tabular} &
   &
  0.36 &
  0.45 &
  \textbf{0.13} &
  \textbf{0.21} \\ \cline{1-1} \cline{3-6} 
\begin{tabular}[c]{@{}l@{}}the unknown \_ \_ () is an\\ authorised biography on \\ finnish racing driver \_ \_ \\ by kari \_\end{tabular} &
   &
  0.31 &
  0.62 &
  0.20 &
  0.30 \\ \cline{1-1} \cline{3-6} 
\begin{tabular}[c]{@{}l@{}}\_ is a 2002 novel by \\ finnish author kari \_\end{tabular} &
   &
  \textbf{0.27} &
  0.57 &
  0.27 &
  0.38 \\ \hline
\end{tabular}

    \caption{Kari}
    \label{tab:5_experiments/4_texter/2_static/9_attention/kari}
\end{table}

Another interesting insight is provided by the third experiment on the attention mechanism, which investigates how attention values derive from word embeddings. Prerequisite for the experiment is the usage of mean pooling, i.e. the calculation of the sentence embeddings as the mean of the respective word embeddings. In this case the scalar product of a class embedding $class_c$ and a sentence embedding $sent_s$ can be broken down to the mean of the scalar products between the class embedding and the sentence's word embeddings $word_w$ where $1 <= w <= N$ is the words position within the sentence:

\[
    \langle class_c, sent_s \rangle
    = \langle class_c, \frac{1}{N} \sum_{w=1}^N word_w \rangle
    = \frac{1}{N} \sum_{w=1}^N \langle class_c, word_w \rangle
\]

Leveraging this relationship, Figure~\ref{fig:5_experiments/4_texter/2_static/9_attention/kari_softmax} illustrates which words influence the overall scalar products between the class and sentence embeddings of the example entity Kari Hotakainen the most. Words that have a similar or opposite embedding to the class embedding are highlighted in yellow and purple, respectively. The expectation was that class-specific keywords would play a large role. During the experiment, it was noticed by coincidence that the result for Softmax- and Sigmoid-based attention mechanisms differ significantly. Therefore, each sentence is shown twice per class - once for Softmax and once for Sigmoid.

\begin{figure}
    \centering
    \includegraphics[width=\textwidth]{5_experiments/4_texter/2_static/9_attention/kari_softmax}
    \caption{Sent}
    \label{fig:5_experiments/4_texter/2_static/9_attention/kari_softmax}
\end{figure}

Although the use of the softmax or the sigmoid function in the attention block leads to very similar F1 scores the training of the word embeddings seems to differ significantly in both cases. When using sigmoid, the learned class embeddings attend on class-relevant keywords as expected whereas this is not the case with softmax. For example, the sigmoid version learns that the words "novel," "author," and "biography" are closely related to class (is, author) and that Finns tend to be likely to speak English. Conversely, "Finnish" and the name "Juha" militate against an origin from the US, and, apparently, authors are rarely actors at the same time. On the other hand, the softmax version attends to the names "Kimi" and "Kari" whose causal relationsships to (from, USA) and (is, actor) is less obvious.

In summary, the assumptions regarding the attention mechanism are only partially supported empirically and no reliable prioritization between an entity's sentences is established, which is probably the reason for the missing performance improvement of the attentive Texter compared to the attention-less version. One possible reason has already been mentioned: With static word embeddings it is not possible for the model to detect whether a potentially class-relevant keyword refers to the entity or not. This becomes even more obvious with a long sentence like the following of which there are many in the IRT text sentences:

\begin{displayquote}
    The Chilean writer Ricardo Cuadros said that McOndo irreverence for Latin American literary tradition, its thematic–stylistic concentration upon the pop culture of the United States, and the literatures’ apolitical tone, are dismissive of the literary ideas, writing style, and narrative techniques of the generation of Latin American writers (García Márquez, Vargas Llosa, Carpentier, Fuentes, et al.) who lived under, opposed, and (occasionally) were repressed by dictators.
\end{displayquote}

The sentece is associated with a Latin American writer who was classified as an American due to the keywords "United", "States" and the twice occurring "American" whose relation to "Latin" cannot be captured by static word embeddings. When trying to identify the entity in question, a second problem reveals itself in long sentences: Even as a human, it is not recognizable that, with all the named entities, the text is about the Peruvian writer Mario Vargas Llosa and not the Chilean Ricardo Cuadros. To address these problems, the next step is to use transformers that can recognize relationships between words and include special markers in the text sets that have not been used so far.

