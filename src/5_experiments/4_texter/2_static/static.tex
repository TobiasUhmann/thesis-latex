The classical way of implementing the embedding block in both, the simple and the attentive Texter, is using static word embeddings, i.e. a word is embedded independent from where it occurs in the sentence. The advantages include ease of implementation using a simple look-up table, few hyperparameters and quick training. However, although it is possible to somehow include a word's immediate context by forming n-grams consisting of multiple adjacent tokens, the sentence embedding cannot properly capture the meaning of markings and maskings via special tokens such as those in sentences from the "cde-irt-5-marked" or "cde-irt-5-masked" text sets exemplified in Table~\ref{tab:5_experiments/1_base_datasets/2_text_sets/text_sets_table}. Therefore, all experiments in this section are conducted on the clean text sets containing no special tokens. Table~\ref{tab:5_experiments/4_texter/2_static/results} shows the final results over all clean text sets for both the simple and the attentive versions of the Texter.

\begin{table}[h]
    \centering
    \begin{tabular}{ l l c r r r }
    \toprule

    \multicolumn{1}{l}{\textbf{Text Set}} &
    \multicolumn{1}{l}{\textbf{Texter}} & \phantom &
    \multicolumn{3}{c}{\textbf{Macro over Classes}} \\

    \cmidrule{4-6}

    & 
    &&
    \multicolumn{1}{c}{\textbf{Prec}} &
    \multicolumn{1}{c}{\textbf{Rec}} &
    \multicolumn{1}{c}{\textbf{F1}} \\
    
    \midrule

    \multirow{2}{*}{cde-cde-1-clean}
    & Simple    && \textbf{49.02} & 47.57 & 47.67 \\
    & Attentive && 46.86 & \textbf{51.09} & \textbf{47.98} \\

    \addlinespace

    \multirow{2}{*}{cde-irt-1-clean}
    & Simple    && 25.20 & \textbf{34.18} & \textbf{28.13} \\
    & Attentive && \textbf{26.70} & 31.41 & 27.43 \\

    \addlinespace

    \multirow{2}{*}{cde-irt-5-clean}
    & Simple    && 36.49 & \textbf{44.38} & \textbf{38.98} \\
    & Attentive && \textbf{39.20} & 37.11 & 36.98 \\

    \addlinespace

    \multirow{2}{*}{cde-irt-15-clean}
    & Simple    && 41.83 & \textbf{48.69} & \textbf{44.07} \\
    & Attentive && \textbf{44.63} & 37.39 & 39.78 \\

    \addlinespace

    \multirow{2}{*}{cde-irt-30-clean}
    & Simple    && 40.73 & \textbf{50.09} & \textbf{44.11} \\
    & Attentive && \textbf{43.60} & 36.10 & 38.78 \\
    
    \midrule

    \multirow{2}{*}{fb-owe-1-clean}
    & Simple    && 42.36 & \textbf{86.72} & 54.03 \\
    & Attentive && \textbf{45.21} & 84.03 & \textbf{56.16} \\

    \addlinespace

    \multirow{2}{*}{fb-irt-1-clean}
    & Simple    && 26.40 & 46.68 & 32.51 \\
    & Attentive && \textbf{27.01} & \textbf{49.90} & \textbf{34.26} \\

    \addlinespace

    \multirow{2}{*}{fb-irt-5-clean}
    & Simple    && 34.37 & \textbf{55.95} & 40.88 \\
    & Attentive && \textbf{39.21} & 50.63 & \textbf{43.50} \\

    \addlinespace

    \multirow{2}{*}{fb-irt-15-clean}
    & Simple    && \textbf{48.95} & \textbf{54.85} & \textbf{50.06} \\
    & Attentive && 48.89 & 52.75 & 49.92 \\

    \addlinespace

    \multirow{2}{*}{fb-irt-30-clean}
    & Simple    && 43.90 & \textbf{63.31} & \textbf{50.55} \\
    & Attentive && \textbf{50.57} & 48.47 & 48.79 \\
    
    \bottomrule
\end{tabular}

    \caption{Final evluation results for the simple and the attentive Texter using static word embeddings - for each text set, the better model is marked in terms of precision, recall and F1}
    \label{tab:5_experiments/4_texter/2_static/results}
\end{table}

The numbers show that both, the simple and the attentive Texter, outperform the zero rule baselines by far. However, the attentive model does not outperform the simple version as hoped. On most datasets, the attentive model achieves better precision, but the simple Texter has higher recall. Overall, the simple modell performs better, especially on datasets with multiple sentences per entity which should have been in favor of the attentive model. Interestingly, both Texters perform best given the short OWE sentence due to high recall scores, which implies that a single, good entity description is more useful than many vague contexts for knowledge graph completion. Similar to the OWE sentences, the CDE sentences from the entities' Wikipedia introductions perform best on the CDE. However, the results are not far ahead of those for the text set with 30 IRT sentences, so that the observation could also be formulated in reverse: 30 randomly sampled entity contexts are just as good as a selected high-quality description.

In addition to the final evaluation results on the test data, Figure~\ref{fig:5_experiments/4_texter/2_static/plot_valid_curves} shows the development of the F1 score for different text sets during training on the FB graph. After 200 episodes of training, the simple and attentive versions of the Texter converge against similar values. The only notable difference between the two is that the simple model reaches its optimal training time earlier before it starts overfitting on most datasets. Thereby, the IRT text sets with few sentences are more prone to overfitting than the ones with 15 or 30 sentences per entity. The short OWE texts, on the other hand, seem to be immune to overfitting. Although training could be stopped after 50 episodes without major defficiencies in this case, all following experiments are performed with 200 episodes to ensure that all models reach their full potential even if changes lead to slower training.

\begin{figure}[t]
    \centering
    \subfloat[Simple Model]{
    \begin{tikzpicture}
    \begin{axis}[
        axis lines = middle,
        cycle list name = tb,
        grid = both,
        legend pos = outer north east,
        scale = 0.6,
        xlabel = epoch,
        ylabel = F1,
    ]
        \addplot table [x = Step, y = Value, col sep = comma] {5_experiments/3_texter/2_static/plot_valid_curves/a_simple/fb-owe-1-clean.csv};
        \addplot table [x = Step, y = Value, col sep = comma] {5_experiments/3_texter/2_static/plot_valid_curves/a_simple/fb-irt-30-clean.csv};
        \addplot table [x = Step, y = Value, col sep = comma] {5_experiments/3_texter/2_static/plot_valid_curves/a_simple/fb-irt-15-clean.csv};
        \addplot table [x = Step, y = Value, col sep = comma] {5_experiments/3_texter/2_static/plot_valid_curves/a_simple/fb-irt-5-clean.csv};
        \addplot table [x = Step, y = Value, col sep = comma] {5_experiments/3_texter/2_static/plot_valid_curves/a_simple/fb-irt-1-clean.csv};
    \end{axis}
\end{tikzpicture}

    \label{fig:5_experiments/3_texter/2_static/plot_valid_curves/a_simple}
}
\hskip 5pt
\subfloat[Attentive Model]{
    \begin{tikzpicture}
    \begin{axis}[
        axis lines = middle,
        cycle list name = tb,
        grid = both,
        legend pos = outer north east,
        scale = 0.6,
        xlabel = epoch,
        ylabel = F1,
    ]
        \addplot table [x = Step, y = Value, col sep = comma] {5_experiments/3_texter/2_static/plot_valid_curves/b_attentive/fb-owe-1-clean.csv};
        \addplot table [x = Step, y = Value, col sep = comma] {5_experiments/3_texter/2_static/plot_valid_curves/b_attentive/fb-irt-30-clean.csv};
        \addplot table [x = Step, y = Value, col sep = comma] {5_experiments/3_texter/2_static/plot_valid_curves/b_attentive/fb-irt-15-clean.csv};
        \addplot table [x = Step, y = Value, col sep = comma] {5_experiments/3_texter/2_static/plot_valid_curves/b_attentive/fb-irt-5-clean.csv};
        \addplot table [x = Step, y = Value, col sep = comma] {5_experiments/3_texter/2_static/plot_valid_curves/b_attentive/fb-irt-1-clean.csv};
    \end{axis}
\end{tikzpicture}

    \label{fig:5_experiments/3_texter/2_static/plot_valid_curves/b_attentive}
}

    \caption{Validation F1 curves during training the Texter on the FB split with 1 OWE sentence (red), 30 IRT sentences (orange), 15 IRT sentences (yellow), 5 IRT sentences (black) or 1 IRT sentence (green) per entity}
    \label{fig:5_experiments/4_texter/2_static/plot_valid_curves}
\end{figure}

What is not visible in Figure~\ref{fig:5_experiments/4_texter/2_static/plot_valid_curves} is the information how the average F1 score is composed of the class-wise F1 scores, which is addressed by Figure~\ref{fig:5_experiments/4_texter/2_static/plot_class_curves}. It compares the F1 scores between the most common classes, the least common classes, and the average F1 score over all classes. The graphs reveal that predictions for common classes are much more reliable. The three most common classes on the CDE split, with frequencies of almonst 20\% each, reach performances between 50\% and 80\%, whereas the three least common classes, with frequencies of around 1\% each, reach strongly divergent values, from 0\% to 60\%. Despite the tendency that frequent classes perform better, however, there is no close correlation between frequency and performance. The least common classes have very similar frequencies but perform very differently. For the second least common class prediction does not work at all, while the third least common class performs even better than the third most common class. There are also significant performance differences between the three most frequent classes although the frequencies are similar. Another, more obvious, finding is that frequent classes are learned within fewer episodes than less common classes. Interestingly, the attentive Texter seems to learn faster than the simple Texter. Finally, the graph shows that both Texters converge to similar values for all classes, as might have been assumed from the similar macro F1 values.

\begin{figure}[t]
    \centering
    \subfloat[Simple Model]{
    \begin{tikzpicture}
    \begin{axis}[
        axis lines = middle,
        cycle list name = tb,
        grid = both,
        legend pos = outer north east,
        scale = 0.8,
        xlabel = epoch,
        ylabel = F1,
    ]
        \addplot table [x = Step, y = Value, col sep = comma] {plots/appendix/static_classes_all/cde_irt_5_simple/class_1.csv};
        \addplot table [x = Step, y = Value, col sep = comma] {plots/appendix/static_classes_all/cde_irt_5_simple/class_2.csv};
        \addplot table [x = Step, y = Value, col sep = comma] {plots/appendix/static_classes_all/cde_irt_5_simple/class_3.csv};
        \addplot table [x = Step, y = Value, col sep = comma] {plots/appendix/static_classes_all/cde_irt_5_simple/class_avg.csv};
        \addplot table [x = Step, y = Value, col sep = comma] {plots/appendix/static_classes_all/cde_irt_5_simple/class_98.csv};
        \addplot table [x = Step, y = Value, col sep = comma] {plots/appendix/static_classes_all/cde_irt_5_simple/class_99.csv};
        \addplot table [x = Step, y = Value, col sep = comma] {plots/appendix/static_classes_all/cde_irt_5_simple/class_100.csv};
    \end{axis}
\end{tikzpicture}

    \label{fig:5_experiments/4_texter/2_static/plot_class_curves/a_simple}
}
\hskip 5pt
\subfloat[Attentive Model]{
    \begin{tikzpicture}
    \begin{axis}[
        axis lines = middle,
        cycle list name = tb,
        grid = both,
        legend pos = outer north east,
        scale = 0.6,
        xlabel = epoch,
        ylabel = F1,
    ]
        \addplot table [x = Step, y = Value, col sep = comma] {a_appendix/static_classes_1/cde_irt_5_attentive/class_1.csv};
        \addplot table [x = Step, y = Value, col sep = comma] {a_appendix/static_classes_1/cde_irt_5_attentive/class_2.csv};
        \addplot table [x = Step, y = Value, col sep = comma] {a_appendix/static_classes_1/cde_irt_5_attentive/class_3.csv};
        \addplot table [x = Step, y = Value, col sep = comma] {a_appendix/static_classes_1/cde_irt_5_attentive/class_avg.csv};
        \addplot table [x = Step, y = Value, col sep = comma] {a_appendix/static_classes_1/cde_irt_5_attentive/class_98.csv};
        \addplot table [x = Step, y = Value, col sep = comma] {a_appendix/static_classes_1/cde_irt_5_attentive/class_99.csv};
        \addplot table [x = Step, y = Value, col sep = comma] {a_appendix/static_classes_1/cde_irt_5_attentive/class_100.csv};
    \end{axis}
\end{tikzpicture}

    \label{fig:5_experiments/4_texter/2_static/plot_class_curves/b_attentive}
}
    \caption{Class-wise validation F1 curves during training the Texter on the CDE split with 5 IRT sentences per entity - comparison between the most common classes (red, orange, yellow), the least common classes (green, blue, purple) and the average value over all classes (black)}
    \label{fig:5_experiments/4_texter/2_static/plot_class_curves}
\end{figure}

Looking at the results for the other datasets shown in Tables~\ref{fig:a_appendix/static_classes_1}~--~\ref{fig:a_appendix/static_classes_3} in Appendix~\ref{ch:a_appendix}, similar patterns can be observed on other text sets with only a few exceptions. However, although there are no major differences between the simple and the attentive Texter, it is noticeable that the results for rare classes do not improve steadily when increasing the number of sentences in case of the IRT text sets. Multiple training runs on the same data showed that this was not due to different initializations of the model. Instead, separate experiments showed that it is the random selection of IRT sentences that affects the performance of individual rare classes. In the respective experiment, new fb-irt-1-clean text sets were built by repeatedly selecting random sentences from the larger fb-irt-30-clean dataset. The overall F1 score over all classes, however, stayed the same for all runs.

With the final results for static word embeddings at hand, the following subsections show what impact various model changes and hyperparameters have. The affected model components are examined in the order in which they are invoked during training, beginning with the tokenizer and ending with the optimizer. Finally, the attentive Texter's attention mechanism is investigated to explain why it does not improve upon the simple Texter as hoped.

\subsubsection{Changing the tokenizer}
\label{subsubsec:5_experiments/4_texter/2_static/1_tokenizer}
The first variable component in both the simple and the attentive Texter is the tokenizer. A naive tokenizer might split the input sentences on any whitespace such as spaces, tabs and line breaks. The problem with that simple approach is the large number of resulting tokens blowing up the vocabulary. For example, the sentence "Hello, world!" would be split into the two tokens "Hello," and "world!", which would be different from "Hello" and "world". As a consequence, none of a word's "impure" occurrences contribute to learning the words embedding, but rather offer the model a way to overfit. One possible solution would be deleting punctuation characters, numbers, dates and other "polluting" characters and unique tokens. However, this could also lead to the destruction of correct tokens, such as "U.K.". Instead, this work uses a tokenizer from the NLP library SpaCy~\cite{SpaCy} which splits the sentence "Welcome to the U.K.!" into the tokens "Welcome", "to", "the", "U.K." and "!".

\begin{table}[h]
    \centering
    \begin{tabular}{| l | l | r | r |}
    \hline

    \multicolumn{1}{|c|}{\multirow{2}{*}{\textbf{Text Set}}} &
    \multicolumn{1}{|c|}{\multirow{2}{*}{\textbf{Model}}} &
    \multicolumn{2}{|c|}{\textbf{Tokenizer}} \\

    &
    &
    \multicolumn{1}{|c|}{\textbf{Whitespace}} &
    \multicolumn{1}{|c|}{\textbf{SpaCy}} \\

    \hline \hline

    \multirow{2}{*}{cde-cde-1-clean}
    & Simple    & 45.07 & \textbf{47.53} \\
    & Attentive & 46.20 & \textbf{48.34} \\ \hline

    \multirow{2}{*}{cde-irt-1-clean}
    & Simple    & 26.43 & \textbf{27.93} \\
    & Attentive & 24.09 & \textbf{27.27} \\ \hline

    \multirow{2}{*}{cde-irt-5-clean}
    & Simple    & 36.60 & \textbf{39.11} \\
    & Attentive & 32.86 & \textbf{37.08} \\ \hline

    \multirow{2}{*}{cde-irt-15-clean}
    & Simple    & 42.00 & \textbf{43.99} \\
    & Attentive & 35.54 & \textbf{39.59} \\ \hline

    \multirow{2}{*}{cde-irt-30-clean}
    & Simple    & 41.69 & \textbf{43.75} \\
    & Attentive & 33.86 & \textbf{38.69} \\ \hline \hline

    \multirow{2}{*}{fb-irt-1-clean}
    & Simple    & 31.77 & \textbf{32.38} \\
    & Attentive & 32.21 & \textbf{33.31} \\ \hline

    \multirow{2}{*}{fb-irt-5-clean}
    & Simple    & 40.11 & \textbf{41.19} \\
    & Attentive & 39.49 & \textbf{41.85} \\ \hline

    \multirow{2}{*}{fb-irt-15-clean}
    & Simple    & 46.02 & \textbf{50.50} \\
    & Attentive & 46.38 & \textbf{48.42} \\ \hline

    \multirow{2}{*}{fb-irt-30-clean}
    & Simple    & 49.36 & \textbf{50.33} \\
    & Attentive & 47.07 & \textbf{48.04} \\ \hline

    \multirow{2}{*}{fb-owe-1-clean}
    & Simple    & 53.81 & \textbf{54.43} \\
    & Attentive & 55.24 & \textbf{55.99} \\ \hline

\end{tabular}

    \caption{Static Texter with either whitespace or SpaCy tokenizer. Numbers show F1 scores. Best result per row marked bold. Using SpaCy always yields better results, especially for the attentive Texter.}
    \label{tab:5_experiments/3_texter/2_static/1_tokenizer/static_tokenizer_alt}
\end{table}

Table~\ref{tab:5_experiments/3_texter/2_static/1_tokenizer/static_tokenizer_alt} shows the tokenizer choice's impact on the vocabulary size and the evaluation results for two text sets on FB15k-237. For the large text set "fb-irt-30", the vocabulary's size halves when using the SpaCy tokenizer rather than splitting on whitespace and the F1 score increases slightly. On "fb-owe-1", the vocabulary's size increases by only 30\%, but the performance gap is over 5\%, probably because every "lost" token means a bigger loss on the small text set. Table~\ref{table:static_tokenizers_all} in Appendix~\ref{ch:a_appendix} contains the full evaluation over all text sets where similar results can be observed on the "cde-irt-30" and "cde-cde-1" text sets.


\subsubsection{Initializing word embeddings randomly}
\label{subsubsec:5_experiments/4_texter/2_static/2_emb_size}
Upon tokenization, the sentences' tokens are embedded -- in the context of this chapter using static word embeddings. For this, randomly initialized, or as shown in the next subsection, pre-trained embeddings can be used. When using randomly initialized embeddings, the questions of what embedding size and which random distribution to chose arises. As random distribution the standard normal distribution $N(0, 1)$ is set. The optimal embedding size is determined via grid search. It is related to, among other things, the amount of available training data and is determined via grid search. Small embeddings can hold little specific information and generalize well, whereas large embeddings can represent subtleties of individual tokens but are also more susceptible to overfitting. The grid search covers a wide range of embedding sizes around the expected optimal embedding size of about 200, a usual size in current models, including unrealistically small sizes to find out to what degree it is worth increasing the size. Overall, expectations for the experiment were limited because of the moderately large amount of training. When using the maximum embedding size of 1000, performance was expected to decrease due to overfitting. Table~\ref{tab:5_experiments/3_texter/2_static/2_emb_size/grid_search} shows the experiment's results.

\begin{table}[t]
    \centering
    \begin{tabular}{| l | l | r | r | r | r |}
    \hline

    \multicolumn{1}{|c|}{\multirow{2}{*}{\textbf{Text Set}}} &
    \multicolumn{1}{|c|}{\multirow{2}{*}{\textbf{Texter}}} &
    \multicolumn{4}{|c|}{\textbf{Sentence Length}} \\

    &
    &
    \multicolumn{1}{|c|}{\textbf{16}} &
    \multicolumn{1}{|c|}{\textbf{32}} &
    \multicolumn{1}{|c|}{\textbf{48}} &
    \multicolumn{1}{|c|}{\textbf{64}} \\

    \hline \hline

    \multirow{2}{*}{cde-cde-1-clean}
    & Simple & 31.22 & 42.40 & 46.00 & \textbf{47.16} \\
    & Attend & 31.54 & 44.37 & 48.18 & \textbf{49.78} \\ \hline

    \multirow{2}{*}{cde-irt-1-marked}
    & Simple & 27.21 & 33.51 & \textbf{36.35} & 36.28 \\
    & Attend & 25.35 & 32.13 & 34.80 & \textbf{35.39} \\ \hline

    \multirow{2}{*}{cde-irt-5-marked}
    & Simple & 39.06 & 42.65 & \textbf{43.11} & 43.06 \\
    & Attend & 36.22 & 40.22 & 41.42 & \textbf{42.12} \\ \hline

    \multirow{2}{*}{cde-irt-15-marked}
    & Simple & 43.87 & 44.21 & 44.31 & \textbf{44.82} \\
    & Attend & 40.96 & 44.17 & 44.72 & \textbf{44.85} \\ \hline

    \multirow{2}{*}{cde-irt-30-marked}
    & Simple &  &  &  &  \\
    & Attend &  &  &  &  \\ \hline \hline

    \multirow{2}{*}{fb-owe-1-clean}
    & Simple & 55.27 & 55.46 & 54.78 & \textbf{56.13} \\
    & Attend & 59.48 & \textbf{59.52} & 57.50 & 57.28 \\ \hline

    \multirow{2}{*}{fb-irt-1-marked}
    & Simple &  &  &  &  \\
    & Attend &  &  &  &  \\ \hline

    \multirow{2}{*}{fb-irt-5-marked}
    & Simple &  &  &  &  \\
    & Attend &  &  &  &  \\ \hline

    \multirow{2}{*}{fb-irt-15-marked}
    & Simple &  &  &  &  \\
    & Attend &  &  &  &  \\ \hline

    \multirow{2}{*}{fb-irt-30-marked}
    & Simple &  &  &  &  \\
    & Attend &  &  &  &  \\ \hline

\end{tabular}

    \caption{Evaluation results for static Texter with randomly initialized word embeddings of varying size - all entries show the macro F1 over all classes, the best results per text set are in bold}
    \label{tab:5_experiments/3_texter/2_static/2_emb_size/grid_search}
\end{table}

At first glance, one can see that training does actually work and that the simple Texter performs much better with randomly initialized embeddings. Against expectations, the attentive Texter reaches its top performance at an embedding size of only around 30 to 100, while the simple model benefits from very large embeddings beyond 300. Apart from one outlier, the performances of both models evolve similarly for small embedding sizes until the attentive Texter stagnates early at medium sizes. For the attentive model, the poor performance seems reasonable since semantically similar words, onto which the same class embedding should match, have completely different initial values, which should complicate the class embeddings' training. Another aspect that catches the eye when looking at Table~\ref{tab:5_experiments/3_texter/2_static/2_emb_size/grid_search} is that even one-dimensional embeddings deliver surprisingly good results. In the case of the CDE split, the appearance is deceptive, as the F1 values are actually below the best zero-rule baseline, which reaches about 20\%, but for the FB split, the results are indeed better than the best zero-rule baseline with only 9\%. Overall, it can be said that randomly initialized embeddings can work, but, despite the good results for the simple Texter, they are not considered in further experiments, as pre-trained embeddings yield better results for both models as shown in the next subsection.


\subsubsection{Using pre-trained word embeddings}
\label{subsubsec:5_experiments/4_texter/2_static/3_pre_trained}
Studies show that pre-trained word embeddings can produce better results after a shorter training period when little training data is available~\cite{Pan2010ASO}. Especially for the attentive model, there was the assumption that pre-trained embeddings could help the model to focus on training the class embeddings. Until the unexpectedly good experiment with randomly initialized embeddings, pre-trained embeddings even seemed indispensable.

When pre-trained embeddings are provided, the Texter's embedding block initializes as many of the vocabulary's words with them as possible. Tokens for which no pre-trained embeddings are available are still initialized randomly. Therefore, when using pre-trained embeddings, it is important to use a tokenizer that produces tokens similar to the ones produced during pre-training -- another reason why SpaCy's tokenizer led to better results than the whitespace tokenizer.

In the past, several research works have provided pre-trained embeddings for their models. In the experiment at hand, a total of 13 variants of three different types of pre-trained embedding sets were tried. The embedding sets vary in embedding size, vocabulary size, and the text corpus pre-training was conducted on. \autoref{tab:5_experiments/3_texter/2_static/3_pre_trained/vector_sets} lists all embedding sets tried during the experiment.

\begin{table}[t]
    \centering
    \begin{tabular}{| l | r | r | l |}
    \hline
    
    \multicolumn{1}{|c|}{\textbf{Vectors}} &
    \multicolumn{1}{|c|}{\textbf{Emb Size}} &
    \multicolumn{1}{|c|}{\textbf{Vocab Size}} &
    \multicolumn{1}{|c|}{\textbf{Trained on}} \\
    
    \hline \hline
    
    charngram.100d & 100 & \num{874474} & 
    ~200mio paraphrase pairs from the paraphrase database PPDB \cite{Ganitkevitch2013PPDBTP} \\
    
    \hline
    
    fasttext.simple.300d & 300 & \num{111051} & 
    ~6mio articles from English Wikipedia (en.wikipedia.org) \\
    
    \hline
    
    fasttext.en.300d & 300 & \num{2519370} & 
    190k articles from Wikipedia in simplified English (simple.wikipedia.org) \\
    
    \hline
    
    glove.6B.50d  &  50 & \multirow{4}{*}{\num{400000}} & \multirow{4}{*}{6bn tokens from Wikipedia and Gigaword 5 \cite{Gigaword}} \\
    glove.6B.100d & 100 &                               & \\
    glove.6B.200d & 200 &                               & \\
    glove.6B.300d & 300 &                               & \\
    
    \hline
    
    glove.twitter.27B.25d  &  25 & \multirow{4}{*}{\num{1917494}} &
    \multirow{4}[*]{27bn tokens from 2bn Twitter tweets}{}\\
    glove.twitter.27B.50d  &  50 &                                & \\
    glove.twitter.27B.100d & 100 &                                & \\
    glove.twitter.27B.200d & 200 &                                & \\
    
    \hline
    
    glove.42B.300d & 300 & \num{1917494} & 
    42bn tokens from Common Crawl \cite{CommonCrawl} \\
    
    \hline
    
    glove.840B.300d & 300 & \num{2196017} & 
    840bn tokens from Common Crawl \\
    
    \hline
\end{tabular}

    \caption{Pre-trained word embedding sets considered for evaluation}
    \label{tab:5_experiments/3_texter/2_static/3_pre_trained/vector_sets}
\end{table}

The three types of embedding differ in how word embeddings are obtained:

\begin{itemize}
    \item \textbf{\emph{GloVe}}~\cite{Pennington2014GloveGV}, coined from ``global vectors'', is an unsupervised learning algorithm dedicated to obtaining the popular, equally named embeddings. GloVe learns embeddings for whole words in such a way that co-occurring, semantically similar words are close in embedding space. Thereby, remarkable relations between related word embeddings emerge, such as the equation $king - queen = man - woman$. Since English has many different words, GloVe has a very large vocabulary.

    \item \textbf{\emph{fastText}}~\cite{Bojanowski2017EnrichingWV,Mikolov2018AdvancesIP}  is based on the idea that words can be viewed as sums of multiple n-grams they exist of. Leveraging the internal structure of words has the advantage that rare or even unknown words can be handled. For example, different declensions of verbs can be related without resorting to stemming, that is, without reducing words to their root. The approach of composing words from n-grams also offers the potential advantage of a small vocabulary.

    \item \textbf{\emph{Charagram}}, referred to as ``charngram'' in the library used for implementing Power's Texter, forms word embeddings, as the name suggests, from charachter n-grams as well. Charagram was introduced at the same time as fastText and serves as an alternative to fastText in the experiment.
\end{itemize}

In the experiment, all 13 embedding sets were evaluated. For brevity, \autoref{tab:5_experiments/3_texter/2_static/3_pre_trained/vector_sets} is limited to the evaluation results for the Charagram, fastText, and GloVe embeddings obtained from training on Wikipedia and Gigaword. Thus, the selected embedding sets cover all three embedding types and give an impression of the influence the embedding size has by comparing the four otherwise equal GloVe embedding sets. \autoref{} in \autoref{ch:a_appendix} contains the second part of the table listing the results for the remaining six GloVe embedding sets.

The expected outcome of the experiment was a significant performance increase for both, simple and attentive Texter, just as observed in other works. Between the embedding sets, medium deviations were assumed, depending on how similar the data during pre-training resembled those of the IRT text sets. In the case of the text sets with CDE and IRT sentences taken from Wikipedia, the GloVe embeddings were therefore the favorite candidate. Regarding the embedding type, there were no clear expectations, as both a large GloVe vocabulary and compound character n-grams should be able to cover the vocabulary of the natural language texts. However, it was assumed that a larger text corpus used for pre-training would lead to a notable performance improvement, for example for fasttext.en.300d compared to fasttext.simple.300d or glove.840B.300d compared to glove.42B.300d. Finally, it was expected that an increase in embedding size would have a similarly strong effect as for randomly initialized embeddings.

\begin{table}[t!]
    \makebox[\textwidth][c]{
        \begin{tabular}{| l | l | r | r | r | r |}
    \hline

    \multicolumn{1}{|c|}{\multirow{2}{*}{\textbf{Text Set}}} &
    \multicolumn{1}{|c|}{\multirow{2}{*}{\textbf{Texter}}} &
    \multicolumn{4}{|c|}{\textbf{Sentence Length}} \\

    &
    &
    \multicolumn{1}{|c|}{\textbf{16}} &
    \multicolumn{1}{|c|}{\textbf{32}} &
    \multicolumn{1}{|c|}{\textbf{48}} &
    \multicolumn{1}{|c|}{\textbf{64}} \\

    \hline \hline

    \multirow{2}{*}{cde-cde-1-clean}
    & Simple & 31.22 & 42.40 & 46.00 & \textbf{47.16} \\
    & Attend & 31.54 & 44.37 & 48.18 & \textbf{49.78} \\ \hline

    \multirow{2}{*}{cde-irt-1-marked}
    & Simple & 27.21 & 33.51 & \textbf{36.35} & 36.28 \\
    & Attend & 25.35 & 32.13 & 34.80 & \textbf{35.39} \\ \hline

    \multirow{2}{*}{cde-irt-5-marked}
    & Simple & 39.06 & 42.65 & \textbf{43.11} & 43.06 \\
    & Attend & 36.22 & 40.22 & 41.42 & \textbf{42.12} \\ \hline

    \multirow{2}{*}{cde-irt-15-marked}
    & Simple & 43.87 & 44.21 & 44.31 & \textbf{44.82} \\
    & Attend & 40.96 & 44.17 & 44.72 & \textbf{44.85} \\ \hline

    \multirow{2}{*}{cde-irt-30-marked}
    & Simple &  &  &  &  \\
    & Attend &  &  &  &  \\ \hline \hline

    \multirow{2}{*}{fb-owe-1-clean}
    & Simple & 55.27 & 55.46 & 54.78 & \textbf{56.13} \\
    & Attend & 59.48 & \textbf{59.52} & 57.50 & 57.28 \\ \hline

    \multirow{2}{*}{fb-irt-1-marked}
    & Simple &  &  &  &  \\
    & Attend &  &  &  &  \\ \hline

    \multirow{2}{*}{fb-irt-5-marked}
    & Simple &  &  &  &  \\
    & Attend &  &  &  &  \\ \hline

    \multirow{2}{*}{fb-irt-15-marked}
    & Simple &  &  &  &  \\
    & Attend &  &  &  &  \\ \hline

    \multirow{2}{*}{fb-irt-30-marked}
    & Simple &  &  &  &  \\
    & Attend &  &  &  &  \\ \hline

\end{tabular}

    }
    \caption{Static Texter with various pre-trained embeddings, part I (part II in \autoref{ch:a_appendix}). Numbers show F1 scores. Best entry per row marked bold if part II of tables does not contain better result. ``Best rand'' column shows best results with randomly initialized embeddings for comparison. ``GloVe *'' refers to the ``glove.6B.*d'' embedding set. All pre-trained embedding sets are similarly well-suited. Only the attentive Texter performs better than with random embeddings.}
    \label{tab:5_experiments/3_texter/2_static/3_pre_trained/grid_search}
\end{table}

As Tables~\ref{tab:5_experiments/3_texter/2_static/3_pre_trained/vector_sets} and~\ref{table:appendix/static_vectors_2} show, many of the assumptions were not met to the expected extent. Overall, it does not matter very much which embedding set is chosen. All three embedding types offer similar top performances with their respective best embedding sets, a multiplication of pre-training data is not necessarily better, and even increasing the embedding size causes only a marginal improvement. Overall, pre-trained embeddings yield better results than randomly initialized ones, so that the last statement can also be formulated the other way around, namely that pre-trained embeddings already work well with small embedding sizes. Besides the small difference between the embedding sets, it is noticeable that the attentive Texter benefits particularly strongly from pre-trained embeddings, although the simple model still performs better on most text sets. On closer inspection, there is a tendency for the attentive model to work better with character n-grams, while the simple model might work slightly better with GloVe embeddings. However, the latter performance gain is so small that all further experiments were performed with character n-grams for the sake of clarity. Among the n-gram based Charagram and fastText embeddings, the slightly better fastText embeddings were chosen, and among these again the smaller fasttext.simple.300d embedding set whose performance is barely distinguishable from fasttext.en.300d.


\subsubsection{Freezing pre-trained embeddings}
\label{subsubsec:5_experiments/4_texter/2_static/4_update_vectors}
\begin{table}[t!]
    \makebox[\textwidth][c]{
        \begin{tabular}{| l | l | r | r | r | r |}
    \hline

    \multicolumn{1}{|c|}{\multirow{2}{*}{\textbf{Text Set}}} &
    \multicolumn{1}{|c|}{\multirow{2}{*}{\textbf{Texter}}} &
    \multicolumn{4}{|c|}{\textbf{Sentence Length}} \\

    &
    &
    \multicolumn{1}{|c|}{\textbf{16}} &
    \multicolumn{1}{|c|}{\textbf{32}} &
    \multicolumn{1}{|c|}{\textbf{48}} &
    \multicolumn{1}{|c|}{\textbf{64}} \\

    \hline \hline

    \multirow{2}{*}{cde-cde-1-clean}
    & Simple & 31.22 & 42.40 & 46.00 & \textbf{47.16} \\
    & Attend & 31.54 & 44.37 & 48.18 & \textbf{49.78} \\ \hline

    \multirow{2}{*}{cde-irt-1-marked}
    & Simple & 27.21 & 33.51 & \textbf{36.35} & 36.28 \\
    & Attend & 25.35 & 32.13 & 34.80 & \textbf{35.39} \\ \hline

    \multirow{2}{*}{cde-irt-5-marked}
    & Simple & 39.06 & 42.65 & \textbf{43.11} & 43.06 \\
    & Attend & 36.22 & 40.22 & 41.42 & \textbf{42.12} \\ \hline

    \multirow{2}{*}{cde-irt-15-marked}
    & Simple & 43.87 & 44.21 & 44.31 & \textbf{44.82} \\
    & Attend & 40.96 & 44.17 & 44.72 & \textbf{44.85} \\ \hline

    \multirow{2}{*}{cde-irt-30-marked}
    & Simple &  &  &  &  \\
    & Attend &  &  &  &  \\ \hline \hline

    \multirow{2}{*}{fb-owe-1-clean}
    & Simple & 55.27 & 55.46 & 54.78 & \textbf{56.13} \\
    & Attend & 59.48 & \textbf{59.52} & 57.50 & 57.28 \\ \hline

    \multirow{2}{*}{fb-irt-1-marked}
    & Simple &  &  &  &  \\
    & Attend &  &  &  &  \\ \hline

    \multirow{2}{*}{fb-irt-5-marked}
    & Simple &  &  &  &  \\
    & Attend &  &  &  &  \\ \hline

    \multirow{2}{*}{fb-irt-15-marked}
    & Simple &  &  &  &  \\
    & Attend &  &  &  &  \\ \hline

    \multirow{2}{*}{fb-irt-30-marked}
    & Simple &  &  &  &  \\
    & Attend &  &  &  &  \\ \hline

\end{tabular}

    }
    \caption{Static Texters when (not) freezing the pre-trained embeddings. Numbers show F1 scores. Best result per row marked bold. Freezing pre-trained word embeddings leads to worse results in every case.}
    \label{tab:5_experiments/3_texter/2_static/4_update_vectors/grid_search}
\end{table}

When using pre-trained word embeddings, the embeddings adapt to the new training data during fine-tuning and may lose their special properties~\cite{He2019AnalyzingTF}. To prevent this, the new training data used for fine-tuning is sometimes mixed in with the training data used during pre-training. Another option is to freeze the pre-trained word embeddings during training so that other parameters must align themselves more strongly while the word embeddings remain constant. The latter approach was considered in an attempt to enhance the attentive Texter's class embeddings. The outcome of the experiment was uncertain. On the one hand, the model is deprived of a large part of its parameters, leaving only the class embeddings and linear layers to be learned, on the other hand, it could focus the training on the class embeddings and prevent overfitting. \autoref{tab:5_experiments/3_texter/2_static/4_update_vectors/grid_search} shows the results. In addition to the fasttext.simple.300d embeddings, charngram.300d, and glove.6B.300d were also examined, as it was conceivable that embedding freezing would have different effects depending on the embedding type.

As the numbers inevitably show, freezing the pre-trained embeddings results in worse performance for every dataset and every type of embedding which is why no further experiments make use of it. Nevertheless, it is interesting to observe how different the negative effects are: CharNGram shows immense performance losses while GloVe embeddings handle the restriction well on text sets with few IRT sentences per entity. Nonetheless, even if the negative effects cannot be compensated, the assumption that freezing the word embeddings has a positive effect on the attention mechanism seems to be valid, since the attentive Texter's performance drops less on text sets with multiple IRT sentences compared to the simple model.


\subsubsection{Pooling}
\label{subsubsec:5_experiments/4_texter/2_static/5_pooling}
The switch to DistilBert also opens up new possibilities for the embedding block's pooling layer. Although, after the experiments on static word embeddings, the choice of mean pooling remains, with a transformer, there are several possibilities which embeddings to average. As mentioned before, DistilBERT adds the [CLS] token to the input sentence, whose purpose is to capture the meaning of the sentence. Thus, the easiest way to receive a sentence embedding in the pooling layer is to simply take the [CLS] token's embedding, as it is also recommended in the BERT paper~\cite{Devlin2019BERTPO}. According to reports from various online forums, however, models achieved better results by using the mean of the actual word embedding instead. Therefore, it was tested which variant works better for the Power model. \autoref{tab:5_experiments/3_texter/3_context/2_pooling/grid_search} shows the results. While on it, two further variants were tested, namely averaging both the [CLS] embedding and the word embeddings, and averaging all embeddings including those created for the padding tokens.

\begin{table}[t!]
    \makebox[\textwidth][c]{
        \begin{tabular}{| l | l | r | r | r | r |}
    \hline

    \multicolumn{1}{|c|}{\multirow{2}{*}{\textbf{Text Set}}} &
    \multicolumn{1}{|c|}{\multirow{2}{*}{\textbf{Texter}}} &
    \multicolumn{4}{|c|}{\textbf{Sentence Length}} \\

    &
    &
    \multicolumn{1}{|c|}{\textbf{16}} &
    \multicolumn{1}{|c|}{\textbf{32}} &
    \multicolumn{1}{|c|}{\textbf{48}} &
    \multicolumn{1}{|c|}{\textbf{64}} \\

    \hline \hline

    \multirow{2}{*}{cde-cde-1-clean}
    & Simple & 31.22 & 42.40 & 46.00 & \textbf{47.16} \\
    & Attend & 31.54 & 44.37 & 48.18 & \textbf{49.78} \\ \hline

    \multirow{2}{*}{cde-irt-1-marked}
    & Simple & 27.21 & 33.51 & \textbf{36.35} & 36.28 \\
    & Attend & 25.35 & 32.13 & 34.80 & \textbf{35.39} \\ \hline

    \multirow{2}{*}{cde-irt-5-marked}
    & Simple & 39.06 & 42.65 & \textbf{43.11} & 43.06 \\
    & Attend & 36.22 & 40.22 & 41.42 & \textbf{42.12} \\ \hline

    \multirow{2}{*}{cde-irt-15-marked}
    & Simple & 43.87 & 44.21 & 44.31 & \textbf{44.82} \\
    & Attend & 40.96 & 44.17 & 44.72 & \textbf{44.85} \\ \hline

    \multirow{2}{*}{cde-irt-30-marked}
    & Simple &  &  &  &  \\
    & Attend &  &  &  &  \\ \hline \hline

    \multirow{2}{*}{fb-owe-1-clean}
    & Simple & 55.27 & 55.46 & 54.78 & \textbf{56.13} \\
    & Attend & 59.48 & \textbf{59.52} & 57.50 & 57.28 \\ \hline

    \multirow{2}{*}{fb-irt-1-marked}
    & Simple &  &  &  &  \\
    & Attend &  &  &  &  \\ \hline

    \multirow{2}{*}{fb-irt-5-marked}
    & Simple &  &  &  &  \\
    & Attend &  &  &  &  \\ \hline

    \multirow{2}{*}{fb-irt-15-marked}
    & Simple &  &  &  &  \\
    & Attend &  &  &  &  \\ \hline

    \multirow{2}{*}{fb-irt-30-marked}
    & Simple &  &  &  &  \\
    & Attend &  &  &  &  \\ \hline

\end{tabular}

    }
    \caption{Contextual Texters using various pooling methods. Sentence embeddings can be created by taking the [CLS] token's embedding (CLS), averaging the words' embeddings (Words), averaging the words' embeddings, including the [CLS] token (C+W), or by averaging all token embeddings, including embedded paddings (C+W+P). Overall, the choice does not make a big difference.}
    \label{tab:5_experiments/3_texter/3_context/2_pooling/grid_search}
\end{table}

Apparently, however, the difference between the approaches is negligible on all text sets. The involvement of all word embeddings plus the [CLS] embedding seems to yield minimally better results than the other approaches, so it has been set as the default pooling strategy.


\subsubsection{Activation Function}
\label{subsubsec:5_experiments/4_texter/2_static/6_activation}
In case of the attentive Texter, the sentence embeddings generated by the embedding block are subsequently passed on to the attention block where they are scalar multiplied with the class embeddings to obtain the attention values that specify how well a class embedding matches each of the entity's sentences. For normalization purposes a classes' attentions are further pushed through a non-linear activation function before the resulting values serve as weight factors in calculating the class-specific entity embeddings returned from the attention block.

Initially, the softmax function was considered as the activation function, so that the attention mechanism is forced to compare all of an entities sentencies to each other. On the other hand, it is problematic that the attention weights sum to 1 even if none of the sentences, or all of them, fit the class. In the first case, the class embedding converges towards unrelated sentence embeddings during backpropagation and in the second case several insightful sentences are not fully utilized, because each one's gradient is lower than possible. Therefore, the sigmoid function was considered as an alternative, allowing each sentence embedding to be weighted by its attention value in $[0, 1]$ individually. While on it, the gererally popular ReLu function was also tested. Out of interest in the principle utility of a non-linear activation function, it was also measured what happens when no activation function is applied at all. Both the relu function and the omission of an activation function allow outliers with a large scalar product to have great influence on the learning process, which was expected to have a negative impact. In the comparison between Softmax and Sigmoid, the outcome was uncertain. Table~\ref{tab:5_experiments/3_texter/2_static/6_activation/grid_search} shows the results of varying the activation function for the attentive Texter only, as the simple model does not have an attention block.

\begin{table}[t]
    \centering
    \begin{tabular}{| l | l | r | r | r | r |}
    \hline

    \multicolumn{1}{|c|}{\multirow{2}{*}{\textbf{Text Set}}} &
    \multicolumn{1}{|c|}{\multirow{2}{*}{\textbf{Texter}}} &
    \multicolumn{4}{|c|}{\textbf{Sentence Length}} \\

    &
    &
    \multicolumn{1}{|c|}{\textbf{16}} &
    \multicolumn{1}{|c|}{\textbf{32}} &
    \multicolumn{1}{|c|}{\textbf{48}} &
    \multicolumn{1}{|c|}{\textbf{64}} \\

    \hline \hline

    \multirow{2}{*}{cde-cde-1-clean}
    & Simple & 31.22 & 42.40 & 46.00 & \textbf{47.16} \\
    & Attend & 31.54 & 44.37 & 48.18 & \textbf{49.78} \\ \hline

    \multirow{2}{*}{cde-irt-1-marked}
    & Simple & 27.21 & 33.51 & \textbf{36.35} & 36.28 \\
    & Attend & 25.35 & 32.13 & 34.80 & \textbf{35.39} \\ \hline

    \multirow{2}{*}{cde-irt-5-marked}
    & Simple & 39.06 & 42.65 & \textbf{43.11} & 43.06 \\
    & Attend & 36.22 & 40.22 & 41.42 & \textbf{42.12} \\ \hline

    \multirow{2}{*}{cde-irt-15-marked}
    & Simple & 43.87 & 44.21 & 44.31 & \textbf{44.82} \\
    & Attend & 40.96 & 44.17 & 44.72 & \textbf{44.85} \\ \hline

    \multirow{2}{*}{cde-irt-30-marked}
    & Simple &  &  &  &  \\
    & Attend &  &  &  &  \\ \hline \hline

    \multirow{2}{*}{fb-owe-1-clean}
    & Simple & 55.27 & 55.46 & 54.78 & \textbf{56.13} \\
    & Attend & 59.48 & \textbf{59.52} & 57.50 & 57.28 \\ \hline

    \multirow{2}{*}{fb-irt-1-marked}
    & Simple &  &  &  &  \\
    & Attend &  &  &  &  \\ \hline

    \multirow{2}{*}{fb-irt-5-marked}
    & Simple &  &  &  &  \\
    & Attend &  &  &  &  \\ \hline

    \multirow{2}{*}{fb-irt-15-marked}
    & Simple &  &  &  &  \\
    & Attend &  &  &  &  \\ \hline

    \multirow{2}{*}{fb-irt-30-marked}
    & Simple &  &  &  &  \\
    & Attend &  &  &  &  \\ \hline

\end{tabular}

    \caption{Static Texters with various activation functions in the attention block. Numbers show F1 scores. Best result per row marked bold. The sigmoid function works best. Using no activation function works surprisingly well.}
    \label{tab:5_experiments/3_texter/2_static/6_activation/grid_search}
\end{table}

In line with expectations, the Sigmoid and Softmax functions produce the best results. For all text sets that provide multiple sentences per entity, however, Sigmoid is ahead by a few percentage points. On single-sentence text sets Sigmoid and Softmax perform similiarly. Interestingly, not using an activation function performs only slightly worse than Softmax on the CDE split and even similarly well on the FB split. On the text set, even the best results are achieved without an activation function. The otherwise promising ReLu function was inferior to the other functions in this experiment. For all further experiments, the sigmoid function was used as the standard.


\subsubsection{Appplying class weights}
\label{subsubsec:5_experiments/4_texter/2_static/7_weight_factor}
\begin{table}[t]
    \centering
    \begin{tabular}{| l | l | r | r | r | r |}
    \hline

    \multicolumn{1}{|c|}{\multirow{2}{*}{\textbf{Text Set}}} &
    \multicolumn{1}{|c|}{\multirow{2}{*}{\textbf{Texter}}} &
    \multicolumn{4}{|c|}{\textbf{Sentence Length}} \\

    &
    &
    \multicolumn{1}{|c|}{\textbf{16}} &
    \multicolumn{1}{|c|}{\textbf{32}} &
    \multicolumn{1}{|c|}{\textbf{48}} &
    \multicolumn{1}{|c|}{\textbf{64}} \\

    \hline \hline

    \multirow{2}{*}{cde-cde-1-clean}
    & Simple & 31.22 & 42.40 & 46.00 & \textbf{47.16} \\
    & Attend & 31.54 & 44.37 & 48.18 & \textbf{49.78} \\ \hline

    \multirow{2}{*}{cde-irt-1-marked}
    & Simple & 27.21 & 33.51 & \textbf{36.35} & 36.28 \\
    & Attend & 25.35 & 32.13 & 34.80 & \textbf{35.39} \\ \hline

    \multirow{2}{*}{cde-irt-5-marked}
    & Simple & 39.06 & 42.65 & \textbf{43.11} & 43.06 \\
    & Attend & 36.22 & 40.22 & 41.42 & \textbf{42.12} \\ \hline

    \multirow{2}{*}{cde-irt-15-marked}
    & Simple & 43.87 & 44.21 & 44.31 & \textbf{44.82} \\
    & Attend & 40.96 & 44.17 & 44.72 & \textbf{44.85} \\ \hline

    \multirow{2}{*}{cde-irt-30-marked}
    & Simple &  &  &  &  \\
    & Attend &  &  &  &  \\ \hline \hline

    \multirow{2}{*}{fb-owe-1-clean}
    & Simple & 55.27 & 55.46 & 54.78 & \textbf{56.13} \\
    & Attend & 59.48 & \textbf{59.52} & 57.50 & 57.28 \\ \hline

    \multirow{2}{*}{fb-irt-1-marked}
    & Simple &  &  &  &  \\
    & Attend &  &  &  &  \\ \hline

    \multirow{2}{*}{fb-irt-5-marked}
    & Simple &  &  &  &  \\
    & Attend &  &  &  &  \\ \hline

    \multirow{2}{*}{fb-irt-15-marked}
    & Simple &  &  &  &  \\
    & Attend &  &  &  &  \\ \hline

    \multirow{2}{*}{fb-irt-30-marked}
    & Simple &  &  &  &  \\
    & Attend &  &  &  &  \\ \hline

\end{tabular}

    \caption{Static Texters when multiplying the class weights with different factors. Numbers show F1 scores. Best result per row marked bold. The ``None'' column shows the results when applying no class weights at all. Applying class weights in the first place is more important than the weight factor.}
    \label{tab:5_experiments/3_texter/2_static/7_weight_factor/grid_search}
\end{table}

After the entity embeddings have been formed from the sentence embeddings in the attention block, they are passed through the classification block which outputs the final class logits. In the case of the simple model, the attention block is omitted and the sentence embeddings are used directly for classification. In either case, the class logits are taken as input by the loss function during training to calculate the model's loss in regard to the ground truth class labels.

In an experiment, whose results fill \autoref{tab:5_experiments/3_texter/2_static/7_weight_factor/grid_search}, it was tested whether the use of class weights really has a positive effect. Furthermore, it was tested whether the choice to calculate the weights as the reciprocal of the class frequencies is optimal. Therefore, the binary cross-entropy loss function in \autoref{eq:2_basics/1_neural_networks/bce} has been extended by a \emph{weight factor} $\omega$ that multiplies the class weights by a constant value. Thus, by setting $\omega \neq 1$, it can be tested whether increasing or decreasing the class weights enhances the training process. \autoref{tab:5_experiments/3_texter/2_static/7_weight_factor/grid_search} shows the observed results.

\begin{align}
    L_{\omega wBCE}(\textbf{y}, \hat{\textbf{y}}) = - \frac{1}{|C|} \sum_{c = 1}^{|C|} \omega \cdot \textbf{w}_c \cdot log(\sigma(\textbf{y}_c)) + (1 - \hat{\textbf{y}}_c) \cdot log(1 - \sigma(\textbf{y}_c))
    \label{eq:5_experiments/3_texter/2_static/7_weight_factor/wwbce}
\end{align}

The empirical values are in line with the theoretical expectations. Although omitting class weights produces useful results - especially on the FB split where the 100 most frequent classes have higher frequencies than on the CDE split, applying class weights significantly improves performance. Thereby, it seems more important to apply weights at all than to fine-tune the weights. In most cases, it does not matter much whether the weights are halved or doubled. On average, the plain reciprocals of the class frequencies seem to be the best choice, which is why they are set as the class weights.


\subsubsection{Choosing an optimizer}
\label{subsubsec:5_experiments/4_texter/2_static/8_optimizer}
Besides the loss function, another important aspect of training is the way the gradients calculated from the loss are applied during backpropagation, i.e. the way to perform gradient descent. Plain batch gradient descent is generally too slow and while stochastic gradient descent and mini-batch gradient descent improve in this matter, they still cannot be considered fast. Therefore, a whole series of extensions have been developed over time to optimize Gradient Descent~\cite{Ruder2016AnOO}.

\begin{table}[t]
    \centering
    \begin{tabular}{| l | l | r | r | r | r |}
    \hline

    \multicolumn{1}{|c|}{\multirow{2}{*}{\textbf{Text Set}}} &
    \multicolumn{1}{|c|}{\multirow{2}{*}{\textbf{Texter}}} &
    \multicolumn{4}{|c|}{\textbf{Sentence Length}} \\

    &
    &
    \multicolumn{1}{|c|}{\textbf{16}} &
    \multicolumn{1}{|c|}{\textbf{32}} &
    \multicolumn{1}{|c|}{\textbf{48}} &
    \multicolumn{1}{|c|}{\textbf{64}} \\

    \hline \hline

    \multirow{2}{*}{cde-cde-1-clean}
    & Simple & 31.22 & 42.40 & 46.00 & \textbf{47.16} \\
    & Attend & 31.54 & 44.37 & 48.18 & \textbf{49.78} \\ \hline

    \multirow{2}{*}{cde-irt-1-marked}
    & Simple & 27.21 & 33.51 & \textbf{36.35} & 36.28 \\
    & Attend & 25.35 & 32.13 & 34.80 & \textbf{35.39} \\ \hline

    \multirow{2}{*}{cde-irt-5-marked}
    & Simple & 39.06 & 42.65 & \textbf{43.11} & 43.06 \\
    & Attend & 36.22 & 40.22 & 41.42 & \textbf{42.12} \\ \hline

    \multirow{2}{*}{cde-irt-15-marked}
    & Simple & 43.87 & 44.21 & 44.31 & \textbf{44.82} \\
    & Attend & 40.96 & 44.17 & 44.72 & \textbf{44.85} \\ \hline

    \multirow{2}{*}{cde-irt-30-marked}
    & Simple &  &  &  &  \\
    & Attend &  &  &  &  \\ \hline \hline

    \multirow{2}{*}{fb-owe-1-clean}
    & Simple & 55.27 & 55.46 & 54.78 & \textbf{56.13} \\
    & Attend & 59.48 & \textbf{59.52} & 57.50 & 57.28 \\ \hline

    \multirow{2}{*}{fb-irt-1-marked}
    & Simple &  &  &  &  \\
    & Attend &  &  &  &  \\ \hline

    \multirow{2}{*}{fb-irt-5-marked}
    & Simple &  &  &  &  \\
    & Attend &  &  &  &  \\ \hline

    \multirow{2}{*}{fb-irt-15-marked}
    & Simple &  &  &  &  \\
    & Attend &  &  &  &  \\ \hline

    \multirow{2}{*}{fb-irt-30-marked}
    & Simple &  &  &  &  \\
    & Attend &  &  &  &  \\ \hline

\end{tabular}

    \caption{Different learning rates}
    \label{tab:5_experiments/4_texter/2_static/8_optimizer/grid_search}
\end{table}

In this work, SGD with momentum~\cite{Qian1999OnTM} and Adam~\cite{Kingma2015AdamAM} are tried. SGD with Momentum serves as a representative of the classical, non-adaptive gradient descent methods for which are generally better suited to find the minimum of a loss function~\cite{Wilson2017TheMV}. The adaptive optimizer Adam, on the other hand, is particularly popular~\cite{AdamPopular} and is generally considered fast and good. In the experiment, SGD's momentum constant was set to 0.9.

Directly related to the optimizer is the learning rate. Both optimizers were tested with different learning rates from a range that should allow training within a reasonable time. It was expected that lower learning rates always lead to a better result given sufficient training time, but that the improvements become smaller and smaller as the learning rate decreases, so that a sufficiently good learning rate can be declared for each optimizer. SGD with momentum was expected perform slightly better after a longer training time. Depending on the ratio of additional training time to gained performance, a decision should be made between the optimizers.

\begin{figure}[t]
    \centering
    \subfloat[Simple, SGD]{
    \begin{tikzpicture}
        \begin{axis}[
            axis lines = middle,
            cycle list name = tb,
            grid = both,
            legend pos = outer north east,
            scale = 0.8,
            xlabel = epoch,
            ylabel = F1,
        ]
            \addplot table [x = Step, y = Value, col sep = comma] {5_experiments/4_texter/2_static/8_optimizer/sgd_vs_adam/simple_sgd/lr_1_0.csv};
            \addplot table [x = Step, y = Value, col sep = comma] {5_experiments/4_texter/2_static/8_optimizer/sgd_vs_adam/simple_sgd/lr_0_3.csv};
            \addplot table [x = Step, y = Value, col sep = comma] {5_experiments/4_texter/2_static/8_optimizer/sgd_vs_adam/simple_sgd/lr_0_1.csv};
            \addplot table [x = Step, y = Value, col sep = comma] {5_experiments/4_texter/2_static/8_optimizer/sgd_vs_adam/simple_sgd/lr_0_03.csv};
            \addplot table [x = Step, y = Value, col sep = comma] {5_experiments/4_texter/2_static/8_optimizer/sgd_vs_adam/simple_sgd/lr_0_01.csv};
            \addplot table [x = Step, y = Value, col sep = comma] {5_experiments/4_texter/2_static/8_optimizer/sgd_vs_adam/simple_sgd/lr_0_003.csv};
            \addplot table [x = Step, y = Value, col sep = comma] {5_experiments/4_texter/2_static/8_optimizer/sgd_vs_adam/simple_sgd/lr_0_001.csv};
        \end{axis}
    \end{tikzpicture}
    \label{fig:5_experiments/4_texter/2_static/8_optimizer/sgd_vs_adam/simple_sgd}
}
\hskip 5pt
\subfloat[Attentive, SGD]{
    \begin{tikzpicture}
        \begin{axis}[
            axis lines = middle,
            cycle list name = tb,
            grid = both,
            legend pos = outer north east,
            scale = 0.8,
            xlabel = epoch,
            ylabel = F1,
        ]
            \addplot table [x = Step, y = Value, col sep = comma] {5_experiments/4_texter/2_static/8_optimizer/sgd_vs_adam/attentive_sgd/lr_1_0.csv};
            \addplot table [x = Step, y = Value, col sep = comma] {5_experiments/4_texter/2_static/8_optimizer/sgd_vs_adam/attentive_sgd/lr_0_3.csv};
            \addplot table [x = Step, y = Value, col sep = comma] {5_experiments/4_texter/2_static/8_optimizer/sgd_vs_adam/attentive_sgd/lr_0_1.csv};
            \addplot table [x = Step, y = Value, col sep = comma] {5_experiments/4_texter/2_static/8_optimizer/sgd_vs_adam/attentive_sgd/lr_0_03.csv};
            \addplot table [x = Step, y = Value, col sep = comma] {5_experiments/4_texter/2_static/8_optimizer/sgd_vs_adam/attentive_sgd/lr_0_01.csv};
            \addplot table [x = Step, y = Value, col sep = comma] {5_experiments/4_texter/2_static/8_optimizer/sgd_vs_adam/attentive_sgd/lr_0_003.csv};
            \addplot table [x = Step, y = Value, col sep = comma] {5_experiments/4_texter/2_static/8_optimizer/sgd_vs_adam/attentive_sgd/lr_0_001.csv};
        \end{axis}
    \end{tikzpicture}
    \label{fig:5_experiments/4_texter/2_static/8_optimizer/sgd_vs_adam/attentive_sgd}
}

\subfloat[Simple, Adam]{
    \begin{tikzpicture}
        \begin{axis}[
            axis lines = middle,
            cycle list name = tb,
            grid = both,
            legend pos = outer north east,
            scale = 0.8,
            xlabel = epoch,
            ylabel = F1,
        ]
            \addplot table [x = Step, y = Value, col sep = comma] {5_experiments/4_texter/2_static/8_optimizer/sgd_vs_adam/simple_adam/lr_1_0.csv};
            \addplot table [x = Step, y = Value, col sep = comma] {5_experiments/4_texter/2_static/8_optimizer/sgd_vs_adam/simple_adam/lr_0_3.csv};
            \addplot table [x = Step, y = Value, col sep = comma] {5_experiments/4_texter/2_static/8_optimizer/sgd_vs_adam/simple_adam/lr_0_1.csv};
            \addplot table [x = Step, y = Value, col sep = comma] {5_experiments/4_texter/2_static/8_optimizer/sgd_vs_adam/simple_adam/lr_0_03.csv};
            \addplot table [x = Step, y = Value, col sep = comma] {5_experiments/4_texter/2_static/8_optimizer/sgd_vs_adam/simple_adam/lr_0_01.csv};
            \addplot table [x = Step, y = Value, col sep = comma] {5_experiments/4_texter/2_static/8_optimizer/sgd_vs_adam/simple_adam/lr_0_003.csv};
            \addplot table [x = Step, y = Value, col sep = comma] {5_experiments/4_texter/2_static/8_optimizer/sgd_vs_adam/simple_adam/lr_0_001.csv};
        \end{axis}
    \end{tikzpicture}
    \label{fig:5_experiments/4_texter/2_static/8_optimizer/sgd_vs_adam/simple_adam}
}
\hskip 5pt
\subfloat[Attentive, Adam]{
    \begin{tikzpicture}
        \begin{axis}[
            axis lines = middle,
            cycle list name = tb,
            grid = both,
            legend pos = outer north east,
            scale = 0.8,
            xlabel = epoch,
            ylabel = F1,
        ]
            \addplot table [x = Step, y = Value, col sep = comma] {5_experiments/4_texter/2_static/8_optimizer/sgd_vs_adam/attentive_adam/lr_1_0.csv};
            \addplot table [x = Step, y = Value, col sep = comma] {5_experiments/4_texter/2_static/8_optimizer/sgd_vs_adam/attentive_adam/lr_0_3.csv};
            \addplot table [x = Step, y = Value, col sep = comma] {5_experiments/4_texter/2_static/8_optimizer/sgd_vs_adam/attentive_adam/lr_0_1.csv};
            \addplot table [x = Step, y = Value, col sep = comma] {5_experiments/4_texter/2_static/8_optimizer/sgd_vs_adam/attentive_adam/lr_0_03.csv};
            \addplot table [x = Step, y = Value, col sep = comma] {5_experiments/4_texter/2_static/8_optimizer/sgd_vs_adam/attentive_adam/lr_0_01.csv};
            \addplot table [x = Step, y = Value, col sep = comma] {5_experiments/4_texter/2_static/8_optimizer/sgd_vs_adam/attentive_adam/lr_0_003.csv};
            \addplot table [x = Step, y = Value, col sep = comma] {5_experiments/4_texter/2_static/8_optimizer/sgd_vs_adam/attentive_adam/lr_0_001.csv};
        \end{axis}
    \end{tikzpicture}
    \label{fig:5_experiments/4_texter/2_static/8_optimizer/sgd_vs_adam/attentive_adam}
}

    \caption{Simple and attentive Texter optimized via SGD and Adam with high to low learning rates (red = 1.0, orange, yellow, black, green, blue, purple = 0.001) on the fb-owe-1-clean text set}
    \label{fig:5_experiments/4_texter/2_static/8_optimizer/sgd_vs_adam/sgd_vs_adam}
\end{figure}

Unfortunately, even at the highest learning rate, SGD converged too slowly to estimate definitive results. Therefore,~\ref{tab:5_experiments/4_texter/2_static/8_optimizer/grid_search} only shows the results for Adam. The four plots in Figure~\ref{fig:5_experiments/4_texter/2_static/8_optimizer/sgd_vs_adam/sgd_vs_adam} illustrate the training of the simple and the attentive Texter using SGD and Adam with different learning rates. The upper plots show SGD's slowly converging curves while Adam's bottom plots paint a contrary picture. Even with the highest learning rate, SGD is slower than Adam with the lowest learning rate and would most likely need at least 200 more episodes to converge. An optimizer-independent observation is that the complex model appears to be more sensitive to learning rates that are too high. For SGD the highest learning rate leads to a slower training progress and for Adam the training does not work at all at the two highest learning rates. This information is also reflected in Table~\ref{tab:5_experiments/4_texter/2_static/8_optimizer/grid_search}, which also reveals another fact that is not directly evident from the plot: A lower learning rate does not automatically result in better performance. However, the nearest explanation is that training is simply not finished at that point. Figure~\ref{tab:5_experiments/4_texter/2_static/8_optimizer/grid_search} suggests that the training of the attentive Texter is somewhat slower, but at a learning rate of 0.003, the training of both models appears to converge after the specified 200 epochs, which is why this learning rate is used in the other experiments.


\subsubsection{Inspecting the attention mechanism}
\label{subsubsec:5_experiments/4_texter/2_static/9_attention}
Does not work

