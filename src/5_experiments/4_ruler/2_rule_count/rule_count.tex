Concerning the Ruler's training, the number of mined rules should play a decisive role over the success in inference. With longer mining, new rules should be found steadily in the graph up to a certain point. The expectation was that at the beginning most rules should be found per time interval and that the rate at which new rules are added should flatten over time until the information from the graph structure is exhausted. At the same time, it was expected that the number of useful rules with $conf_{min} > 0.5$ would be exhausted earlier and that only rules with lower confidence would be found thereafter. Until then, recall was expected to grow quickly. How the quality of the rules would change during the finding process was not certain. It was assumed that rules with high support would be found at the beginning so that the support of new rules would decrease in the course of mining. Regarding rule confidence, there was no clear expectation. Overall, it was assumed that the changing quality during the mining process should have a negligible effect compared to the increasing recall and that F1 and mAP should therefore increase continuously. A decrease in F1 and mAP was excluded because rules with $conf_{min} > 0.5$ should generally have a positive effect. \autoref{tab:5_experiments/4_ruler/2_rule_count/results} shows the empirical results.

\begin{table}
    \makebox[\textwidth][c]{
        \begin{tabular}{| l | r | r | r | r | r | r | r | r | r |}
    \hline
    
    \multicolumn{1}{|c|}{\textbf{Split}} &
    \multicolumn{1}{|c|}{\textbf{mAP}} &
    \multicolumn{4}{|c|}{\textbf{Macro over ents}} &
    \multicolumn{4}{|c|}{\textbf{Micro over facts}} \\
    
    \multicolumn{1}{|c|}{} &
    \multicolumn{1}{|c|}{} &
    \multicolumn{1}{|c|}{\textbf{Prec}} &
    \multicolumn{1}{|c|}{\textbf{Rec}} &
    \multicolumn{1}{|c|}{\textbf{F1}} &
    \multicolumn{1}{|c|}{\textbf{Supp}} &
    \multicolumn{1}{|c|}{\textbf{Prec}} &
    \multicolumn{1}{|c|}{\textbf{Rec}} &
    \multicolumn{1}{|c|}{\textbf{F1}} &
    \multicolumn{1}{|c|}{\textbf{Supp}} \\
    
    \hline \hline
    
    CDE-0 & 0.84 &
    100.00 & 0.84 & 0.84 & \num{13.32} &
    100.00 & 0.00 & 0.00 & \num{25255} \\
    
    CDE-25 & 22.06 &
    66.90 & 24.27 & 32.02 & \num{13.32} &
    62.67 & 22.93 & 33.57 & \num{25255} \\
    
    CDE-50 & 29.26 &
    61.41 & 33.14 & 40.29 & \num{13.32} &
    58.47 & 31.52 & 40.96 & \num{25255} \\
    
    CDE-75 & 33.23 &
    57.88 & 38.10 & 43.59 & \num{13.32} &
    55.43 & 3638 & 43.93 & \num{25255} \\
    
    CDE-100 & 35.77 &
    55.62 & 41.48 & 45.28 & \num{13.32} &
    53.49 & 39.72 & 45.59 & \num{25255} \\
    
    \hline
    
    FB-0 & 3.19 &
    100.00 & 03.19 & 3.19 & \num{18.76} &
    100.00 & 0.00 & 0.00 & \num{15312} \\
    
    FB-25 & 27.41 &
    73.46 & 30.56 & 37.06 & \num{18.76} &
    69.28 & 30.15 & 42.02 & \num{15312} \\
    
    FB-50 & 33.39 &
    68.36 & 37.61 & 42.90 & \num{18.76} &
    64.90 & 36.76 & 46.94 & \num{15312} \\
    
    FB-75 & 36.22 &
    64.93 & 41.37 & 45.14 & \num{18.76} &
    62.76 & 40.82 & 49.47 & \num{15312} \\
    
    FB-100 & 38.43 &
    63.08 & 44.34 & 46.97 & \num{18.76} &
    60.98 & 43.51 & 50.79 & \num{15312} \\
    
    \hline
\end{tabular}

    }
    \caption{Ruler results when leveraging rules mined after various mining times. Ruler keeps rules that fulfill $supp_{min} = 2$ and $conf_{min} = 0.5$. The number of mined rules grows in proportion to mining time, but after a certain point, more rules do not increase performance much further.}
    \label{tab:5_experiments/4_ruler/2_rule_count/results}
\end{table}

Surprisingly, not only did the number of totally mined rules increased linearly with time, but also the number of useful rules with $conf_{min} > 0.5$. The unbroken trend suggests that both the CDE and FB graphs still offer much room upwards for more rules. Nonetheless, most of the later mined rules seem to predict facts that are already predicted by earlier mined rules. At least, that is what the recall values suggest which begin to converge after mining time $t = 1000s$ on the CDE split. The F1 and mAP scores, however, approach their peak values already after $t = 300s$, because precision decreases and counteracts the increasing recall. The precision decrease speaks for the hypothesis of decreasing rule quality over time. The influence of rule support and confidence is therefore further studied in the following \autoref{subsec:5_experiments/4_ruler/2_rule_count}. The top values after mining time $t = 1000s$ are the final test results for the Ruler. The version after $t = 100s$ used for the other experiments differs only by a few percentage points in F1 and mAP\@.
