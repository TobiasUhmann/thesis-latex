The assumption that 50\% of a test entity's facts are known during inference leads to stable evaluation results, especially if the data set provides only a few facts per entity. However, if many facts are available, the scenario no longer corresponds to a few-shot use case, which is particularly interesting in practice. For a test entity from the FB split, 50\% known test facts already correspond to an average of nine facts per entity.

To get an idea of how much the number of known test facts affects the Ruler's performance, this experiment evaluates all of the Power splits introduced in \autoref{sec:5_experiments/2_power_splits}, including the splits which include none or all of the entities' facts in the known test set. Test entities without known test facts correspond to open-world entities. Although the Ruler cannot predict facts for them, they were processed to assure that the evaluation code would calculate precision to 100\% and recall to 0\%. The splits that include all of an entity's test facts as known facts do not represent a practical use case for the Ruler, neither. Still, it is included to show how much increasing the share of known facts beyond 50\% would bring. In practice, the most interesting splits are the ones with a low share of known test facts. The expectation was that recall would increase with the number of known test facts, while precision should be independent of the number of known facts. Thus, F1 and mAP were expected to increase with the proportion of known facts. \autoref{tab:5_experiments/4_ruler/1_unknown/all_results} shows the observed results.

\begin{table}
    \centering
    \begin{tabular}{ l l c r r r }
    \toprule

    \multicolumn{1}{l}{\textbf{Text Set}} &
    \multicolumn{1}{l}{\textbf{Texter}} & \phantom &
    \multicolumn{3}{c}{\textbf{Macro over Classes}} \\

    \cmidrule{4-6}

    & 
    &&
    \multicolumn{1}{c}{\textbf{Prec}} &
    \multicolumn{1}{c}{\textbf{Rec}} &
    \multicolumn{1}{c}{\textbf{F1}} \\
    
    \midrule

    \multirow{2}{*}{cde-cde-1-clean}
    & Simple    && \textbf{49.02} & 47.57 & 47.67 \\
    & Attentive && 46.86 & \textbf{51.09} & \textbf{47.98} \\

    \addlinespace

    \multirow{2}{*}{cde-irt-1-clean}
    & Simple    && 25.20 & \textbf{34.18} & \textbf{28.13} \\
    & Attentive && \textbf{26.70} & 31.41 & 27.43 \\

    \addlinespace

    \multirow{2}{*}{cde-irt-5-clean}
    & Simple    && 36.49 & \textbf{44.38} & \textbf{38.98} \\
    & Attentive && \textbf{39.20} & 37.11 & 36.98 \\

    \addlinespace

    \multirow{2}{*}{cde-irt-15-clean}
    & Simple    && 41.83 & \textbf{48.69} & \textbf{44.07} \\
    & Attentive && \textbf{44.63} & 37.39 & 39.78 \\

    \addlinespace

    \multirow{2}{*}{cde-irt-30-clean}
    & Simple    && 40.73 & \textbf{50.09} & \textbf{44.11} \\
    & Attentive && \textbf{43.60} & 36.10 & 38.78 \\
    
    \midrule

    \multirow{2}{*}{fb-owe-1-clean}
    & Simple    && 42.36 & \textbf{86.72} & 54.03 \\
    & Attentive && \textbf{45.21} & 84.03 & \textbf{56.16} \\

    \addlinespace

    \multirow{2}{*}{fb-irt-1-clean}
    & Simple    && 26.40 & 46.68 & 32.51 \\
    & Attentive && \textbf{27.01} & \textbf{49.90} & \textbf{34.26} \\

    \addlinespace

    \multirow{2}{*}{fb-irt-5-clean}
    & Simple    && 34.37 & \textbf{55.95} & 40.88 \\
    & Attentive && \textbf{39.21} & 50.63 & \textbf{43.50} \\

    \addlinespace

    \multirow{2}{*}{fb-irt-15-clean}
    & Simple    && \textbf{48.95} & \textbf{54.85} & \textbf{50.06} \\
    & Attentive && 48.89 & 52.75 & 49.92 \\

    \addlinespace

    \multirow{2}{*}{fb-irt-30-clean}
    & Simple    && 43.90 & \textbf{63.31} & \textbf{50.55} \\
    & Attentive && \textbf{50.57} & 48.47 & 48.79 \\
    
    \bottomrule
\end{tabular}

    \caption{Ruler results for Power splits with various numbers of facts availalbe for rule application. Ruler uses rules mined after $t = 100s$ and keeps rules that fulfill $supp_{min} = 2$ and $conf_{min} = 0.5$. The Ruler already achieves usable results when tested on the few-shot splits CDE-30 and FB-15.}
    \label{tab:5_experiments/4_ruler/1_unknown/all_results}
\end{table}

As expected, recall does increase with the number of test facts available for rule application. Precision, on the other hand, does not stagnate but decreases instead. The reason is the overlooked fact that true predictions about the same fact produced by rules based on multiple known test facts do not add up. In contrast, false predictions from multiple known facts usually do not overlap. However, the increase in recall prevails so that F1 increases with the number of known facts as does mAP. The results on the FB split are higher for few known facts because rounding off fact numbers affects the lower fact numbers on CDE more heavily compared to the FB split. Recall does not equal zero on the ``0'' splits, because they contain entities without any test facts. In those cases, the empty prediction set produced by the Ruler is correct and leads to 100\% recall.
