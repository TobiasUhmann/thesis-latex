While training the Texter means repeatedly applying backpropagation to the Texter's neural network, the Ruler's equivalent is mining rules on the known train facts. The found rules are then used during inference to infer new facts from the ones known about a query entity. In contrast to the Texter's training, all training facts are used for rule mining and all test facts are considered during evaluation. Thus, the macro F1 and mAP metrics calculated over all test entities that were used for the contextual Texter's final evaluation are the default metrics applied to the Ruler.

For rule mining, 138k and 238k training facts are available in the CDE and FB splits, respectively. Depending on how long AnyBURL runs, more rules can be mined than the graph contains facts. The Ruler used in the final evaluation leveraged 1.74mio rules on the CDE split and 2.51mio rules on the FB split. Mining those rules took $t = 1000s$ on an Intel i7 quad-core processor. The vast majority of the rules have a rule body consisting of a single fact, such as the rule $(X, lives~in, Norway) => (X, speaks, English)$. Most of the rules have a confidence below the minimum confidence threshold $conf_{min} = 0.5$, meaning that the facts they imply are probably wrong. Those rules are therefore discarded, leaving 785k supposedly useful rules on CDE and 916k useful rules on the FB split. AnyBURL's default minimum support threshold $supp_{min} = 2$ was kept for the final evaluation.

The useful rules are then used for prediction by searching for rule groundings within the known facts, which include all train facts in addition to the query entity's known test facts. The heads of the rules for which a grounding could be found then yield the predicted facts. \autoref{tab:5_experiments/4_ruler/results} shows the final evaluation results obtained when evaluating the predictions against all test facts. The ``50'' suffix denotes that 50\% of the test facts may be used to apply rules during inference. As one would expect, the FB results are once again better than the CDE ones, due to the larger training set and the higher number of facts per entity.

\begin{table}[h]
    \centering
    \begin{tabular}{ l l c r r r }
    \toprule

    \multicolumn{1}{l}{\textbf{Text Set}} &
    \multicolumn{1}{l}{\textbf{Texter}} & \phantom &
    \multicolumn{3}{c}{\textbf{Macro over Classes}} \\

    \cmidrule{4-6}

    & 
    &&
    \multicolumn{1}{c}{\textbf{Prec}} &
    \multicolumn{1}{c}{\textbf{Rec}} &
    \multicolumn{1}{c}{\textbf{F1}} \\
    
    \midrule

    \multirow{2}{*}{cde-cde-1-clean}
    & Simple    && \textbf{49.02} & 47.57 & 47.67 \\
    & Attentive && 46.86 & \textbf{51.09} & \textbf{47.98} \\

    \addlinespace

    \multirow{2}{*}{cde-irt-1-clean}
    & Simple    && 25.20 & \textbf{34.18} & \textbf{28.13} \\
    & Attentive && \textbf{26.70} & 31.41 & 27.43 \\

    \addlinespace

    \multirow{2}{*}{cde-irt-5-clean}
    & Simple    && 36.49 & \textbf{44.38} & \textbf{38.98} \\
    & Attentive && \textbf{39.20} & 37.11 & 36.98 \\

    \addlinespace

    \multirow{2}{*}{cde-irt-15-clean}
    & Simple    && 41.83 & \textbf{48.69} & \textbf{44.07} \\
    & Attentive && \textbf{44.63} & 37.39 & 39.78 \\

    \addlinespace

    \multirow{2}{*}{cde-irt-30-clean}
    & Simple    && 40.73 & \textbf{50.09} & \textbf{44.11} \\
    & Attentive && \textbf{43.60} & 36.10 & 38.78 \\
    
    \midrule

    \multirow{2}{*}{fb-owe-1-clean}
    & Simple    && 42.36 & \textbf{86.72} & 54.03 \\
    & Attentive && \textbf{45.21} & 84.03 & \textbf{56.16} \\

    \addlinespace

    \multirow{2}{*}{fb-irt-1-clean}
    & Simple    && 26.40 & 46.68 & 32.51 \\
    & Attentive && \textbf{27.01} & \textbf{49.90} & \textbf{34.26} \\

    \addlinespace

    \multirow{2}{*}{fb-irt-5-clean}
    & Simple    && 34.37 & \textbf{55.95} & 40.88 \\
    & Attentive && \textbf{39.21} & 50.63 & \textbf{43.50} \\

    \addlinespace

    \multirow{2}{*}{fb-irt-15-clean}
    & Simple    && \textbf{48.95} & \textbf{54.85} & \textbf{50.06} \\
    & Attentive && 48.89 & 52.75 & 49.92 \\

    \addlinespace

    \multirow{2}{*}{fb-irt-30-clean}
    & Simple    && 43.90 & \textbf{63.31} & \textbf{50.55} \\
    & Attentive && \textbf{50.57} & 48.47 & 48.79 \\
    
    \bottomrule
\end{tabular}

    \caption{Final Ruler results on the CDE and FB splits with rules mined after $t = 1000s$ with $supp_{min} = 2$ and $conf_{min} = 0.5$}
    \label{tab:5_experiments/4_ruler/results}
\end{table}

In the following, \autoref{subsec:5_experiments/4_ruler/1_known} demonstrates how well the Ruler works in few shot scenarios, \autoref{subsec:5_experiments/4_ruler/2_rule_count} shows how much the number of mined rules affects the Ruler's performance, and \autoref{subsec:5_experiments/4_ruler/3_rule_quality} explains why removing low-quality rules leads to worse performance. All further evaluations were performed on Rulers that were trained on rule sets obtained from mining for $t = 100s$. \autoref{tab:5_experiments/4_ruler/2_rule_count/results} in \autoref{subsec:5_experiments/4_ruler/2_rule_count} can be used to estimate performances for longer mining times. The other training parameters are kept the same as for the final evaluation if not specified otherwise, i.e. $supp_{min} = 2$ and $conf_{min} = 0.5$.

\subsection{Known Test Facts}
\label{subsec:5_experiments/4_ruler/1_known}
The assumption that 50\% of a test entity's facts are known during inference leads to stable evaluation results, especially if the data set provides only few facts per entity. However, if many facts are available, the assumption no longer corresponds to a few-shot scenario which is particularly interesting in practice. For a test entity from the FB split, 50\% known test facts already corresponds to an average of nine facts per entity.

To get an idea of how much the number of known test facts affects the Ruler's performance, this experiment evaluates all of the Power splits introduced in Section~\ref{sec:5_experiments/2_power_splits}, including the splits which include none or all of the entities' facts in the known test set. Test entities without known test facts correspond to open-world entities. Although, the Ruler cannot predict facts for them, they were processed to assure that the evaluation code would calculate precision and recall to 100\% and 0\%, respectively. The splits that include all of an entity's test facts as known facts does not represent a practical use case for the Ruler, neither. Still, it is included to show how much increasing the share of known facts beyond 50\% would bring. In practice, the most interesting splits are the ones with a low percentage of known test facts. The expectation was that recall would increase logarithmically with the number of known test facts, as the first fact about an entity reveals much more information than the last one, while precision should be independent from the choice of random known facts. Thus, F1 and mAP were expected to increase logarithmically with the proportion of known facts. Table~\ref{tab:5_experiments/5_ruler/1_unknown/all_results} shows the actual results.

\begin{table}
    \centering
    \begin{tabular}{ l l c r r r }
    \toprule

    \multicolumn{1}{l}{\textbf{Text Set}} &
    \multicolumn{1}{l}{\textbf{Texter}} & \phantom &
    \multicolumn{3}{c}{\textbf{Macro over Classes}} \\

    \cmidrule{4-6}

    & 
    &&
    \multicolumn{1}{c}{\textbf{Prec}} &
    \multicolumn{1}{c}{\textbf{Rec}} &
    \multicolumn{1}{c}{\textbf{F1}} \\
    
    \midrule

    \multirow{2}{*}{cde-cde-1-clean}
    & Simple    && \textbf{49.02} & 47.57 & 47.67 \\
    & Attentive && 46.86 & \textbf{51.09} & \textbf{47.98} \\

    \addlinespace

    \multirow{2}{*}{cde-irt-1-clean}
    & Simple    && 25.20 & \textbf{34.18} & \textbf{28.13} \\
    & Attentive && \textbf{26.70} & 31.41 & 27.43 \\

    \addlinespace

    \multirow{2}{*}{cde-irt-5-clean}
    & Simple    && 36.49 & \textbf{44.38} & \textbf{38.98} \\
    & Attentive && \textbf{39.20} & 37.11 & 36.98 \\

    \addlinespace

    \multirow{2}{*}{cde-irt-15-clean}
    & Simple    && 41.83 & \textbf{48.69} & \textbf{44.07} \\
    & Attentive && \textbf{44.63} & 37.39 & 39.78 \\

    \addlinespace

    \multirow{2}{*}{cde-irt-30-clean}
    & Simple    && 40.73 & \textbf{50.09} & \textbf{44.11} \\
    & Attentive && \textbf{43.60} & 36.10 & 38.78 \\
    
    \midrule

    \multirow{2}{*}{fb-owe-1-clean}
    & Simple    && 42.36 & \textbf{86.72} & 54.03 \\
    & Attentive && \textbf{45.21} & 84.03 & \textbf{56.16} \\

    \addlinespace

    \multirow{2}{*}{fb-irt-1-clean}
    & Simple    && 26.40 & 46.68 & 32.51 \\
    & Attentive && \textbf{27.01} & \textbf{49.90} & \textbf{34.26} \\

    \addlinespace

    \multirow{2}{*}{fb-irt-5-clean}
    & Simple    && 34.37 & \textbf{55.95} & 40.88 \\
    & Attentive && \textbf{39.21} & 50.63 & \textbf{43.50} \\

    \addlinespace

    \multirow{2}{*}{fb-irt-15-clean}
    & Simple    && \textbf{48.95} & \textbf{54.85} & \textbf{50.06} \\
    & Attentive && 48.89 & 52.75 & 49.92 \\

    \addlinespace

    \multirow{2}{*}{fb-irt-30-clean}
    & Simple    && 43.90 & \textbf{63.31} & \textbf{50.55} \\
    & Attentive && \textbf{50.57} & 48.47 & 48.79 \\
    
    \bottomrule
\end{tabular}

    \caption{Ruler test results for different Power splits on rules mined after $t = 100s$ with $supp_{min} = 2$ and $conf_{min} = 0.5$}
    \label{tab:5_experiments/5_ruler/1_unknown/all_results}
\end{table}

As expected, recall does increase with the number of test facts available for rule application. Precision, on the other hand does not stagnate, but decreases instead. The reason is the overlooked fact that true predictions about the same fact produced by rules based on multiple known test facts do not add up. In contrast, false predictions from multiple known facts usually do not overlap. However, the increase in recall prevails so that F1 increases with the number of known facts as does mean average precision. The results on the FB split are higher for low known facts shares because rounding off fact numbers affects the lower fact numbers on CDE more heavily than on the FB split. Recall does not equal zero on the "0" splits, because they contain entities without any test facts. In those cases the empty prediction set produced by the Ruler is correct and leads to 100\% recall.


\subsection{Number of Rules}
\label{subsec:5_experiments/4_ruler/2_rule_count}
Concerning the Ruler's training, the number of mined rules should play a decisive role over the success in inference. With longer mining, new rules should be found steadily in the graph up to a certain point. The expectation was that at the beginning most rules should be found per time interval and that the rate at which new rules are added should flatten over time until the information from the graph structure is exhausted. At the same time, it was expected that the number of useful rules with $conf_{min} > 0.5$ would be exhausted earlier and that only rules with lower confidence would be found thereafter. Until then, recall was expected to grow quickly. How the quality of the rules would change during the finding process was not certain. It was assumed that rules with high support would be found at the beginning so that the support of new rules would decrease in the course of mining. Regarding rule confidence, there was no clear expectation. Overall, it was assumed that the changing quality during the mining process should have a negligible effect compared to the increasing recall and that F1 and mAP should therefore increase continuously. A decrease in F1 and mAP was excluded because rules with $conf_{min} > 0.5$ should generally have a positive effect. \autoref{tab:5_experiments/4_ruler/2_rule_count/results} shows the empirical results.

\begin{table}
    \makebox[\textwidth][c]{
        \begin{tabular}{ l l c r r r }
    \toprule

    \multicolumn{1}{l}{\textbf{Text Set}} &
    \multicolumn{1}{l}{\textbf{Texter}} & \phantom &
    \multicolumn{3}{c}{\textbf{Macro over Classes}} \\

    \cmidrule{4-6}

    & 
    &&
    \multicolumn{1}{c}{\textbf{Prec}} &
    \multicolumn{1}{c}{\textbf{Rec}} &
    \multicolumn{1}{c}{\textbf{F1}} \\
    
    \midrule

    \multirow{2}{*}{cde-cde-1-clean}
    & Simple    && \textbf{49.02} & 47.57 & 47.67 \\
    & Attentive && 46.86 & \textbf{51.09} & \textbf{47.98} \\

    \addlinespace

    \multirow{2}{*}{cde-irt-1-clean}
    & Simple    && 25.20 & \textbf{34.18} & \textbf{28.13} \\
    & Attentive && \textbf{26.70} & 31.41 & 27.43 \\

    \addlinespace

    \multirow{2}{*}{cde-irt-5-clean}
    & Simple    && 36.49 & \textbf{44.38} & \textbf{38.98} \\
    & Attentive && \textbf{39.20} & 37.11 & 36.98 \\

    \addlinespace

    \multirow{2}{*}{cde-irt-15-clean}
    & Simple    && 41.83 & \textbf{48.69} & \textbf{44.07} \\
    & Attentive && \textbf{44.63} & 37.39 & 39.78 \\

    \addlinespace

    \multirow{2}{*}{cde-irt-30-clean}
    & Simple    && 40.73 & \textbf{50.09} & \textbf{44.11} \\
    & Attentive && \textbf{43.60} & 36.10 & 38.78 \\
    
    \midrule

    \multirow{2}{*}{fb-owe-1-clean}
    & Simple    && 42.36 & \textbf{86.72} & 54.03 \\
    & Attentive && \textbf{45.21} & 84.03 & \textbf{56.16} \\

    \addlinespace

    \multirow{2}{*}{fb-irt-1-clean}
    & Simple    && 26.40 & 46.68 & 32.51 \\
    & Attentive && \textbf{27.01} & \textbf{49.90} & \textbf{34.26} \\

    \addlinespace

    \multirow{2}{*}{fb-irt-5-clean}
    & Simple    && 34.37 & \textbf{55.95} & 40.88 \\
    & Attentive && \textbf{39.21} & 50.63 & \textbf{43.50} \\

    \addlinespace

    \multirow{2}{*}{fb-irt-15-clean}
    & Simple    && \textbf{48.95} & \textbf{54.85} & \textbf{50.06} \\
    & Attentive && 48.89 & 52.75 & 49.92 \\

    \addlinespace

    \multirow{2}{*}{fb-irt-30-clean}
    & Simple    && 43.90 & \textbf{63.31} & \textbf{50.55} \\
    & Attentive && \textbf{50.57} & 48.47 & 48.79 \\
    
    \bottomrule
\end{tabular}

    }
    \caption{Ruler results when leveraging rules mined after various mining times. Ruler keeps rules that fulfill $supp_{min} = 2$ and $conf_{min} = 0.5$. The number of mined rules grows in proportion to mining time, but after a certain point, more rules do not increase performance much further.}
    \label{tab:5_experiments/4_ruler/2_rule_count/results}
\end{table}

Surprisingly, not only did the number of totally mined rules increased linearly with time, but also the number of useful rules with $conf_{min} > 0.5$. The unbroken trend suggests that both the CDE and FB graphs still offer much room upwards for more rules. Nonetheless, most of the later mined rules seem to predict facts that are already predicted by earlier mined rules. At least, that is what the recall values suggest which begin to converge after mining time $t = 1000s$ on the CDE split. The F1 and mAP scores, however, approach their peak values already after $t = 300s$, because precision decreases and counteracts the increasing recall. The precision decrease speaks for the hypothesis of decreasing rule quality over time. The influence of rule support and confidence is therefore further studied in the following \autoref{subsec:5_experiments/4_ruler/2_rule_count}. The top values after mining time $t = 1000s$ are the final test results for the Ruler. The version after $t = 100s$ used for the other experiments differs only by a few percentage points in F1 and mAP\@.


\subsection{Rule Quality}
\label{subsec:5_experiments/4_ruler/3_rule_quality}
Besides the number of mined rules, the rules' quality plays a crucial role for the Ruler's performance. By default, rules with support equals of at least $supp_{min} = 2$ and confidence of at least $conf_{min} = 0.5$ are considered. It was assumed that the Ruler's predictions might improve the rules were further restricted to those with high support and confidence. Therefore, two independent experiments were conducted in which minimal support and confidence were gradually increased. Both experiments were also evaluated over Power splits with varing number of known test facts during inference to whether high-quality rules were increasingly beneficial in few-shot scenarios.

For the experiment on varying the minimal support threshold it was expected that precision would not vary between low- and high-support rules, but that recall would decrease strongly due to a restriction to few high-support rules. Therefore, F1 and mAP were assumed to drop in accordance with recall. \autoref{tab:5_experiments/4_ruler/3_rule_quality/supp_results} shows the measurements for minimum support thresholds ranging from the default value of 2 up to 1000.

\begin{table}
    \centering
    \begin{tabular}{| l | r | r | r | r | r | r | r |}
    \hline

    \multicolumn{1}{|c|}{\textbf{Split}} &
    \multicolumn{7}{|c|}{\textbf{Min Support}} \\

    \multicolumn{1}{|c|}{} &
    \multicolumn{1}{|c|}{\textbf{1}} &
    \multicolumn{1}{|c|}{\textbf{3}} &
    \multicolumn{1}{|c|}{\textbf{10}} &
    \multicolumn{1}{|c|}{\textbf{30}} &
    \multicolumn{1}{|c|}{\textbf{100}} &
    \multicolumn{1}{|c|}{\textbf{300}} &
    \multicolumn{1}{|c|}{\textbf{1000}} \\

    \hline \hline

    CDE-0   &  &  &  &  &  &  &  \\
    CDE-5   &  &  &  &  &  &  &  \\
    CDE-15  &  &  &  &  &  &  &  \\
    CDE-30  &  &  &  &  &  &  &  \\
    CDE-50  &  &  &  &  &  &  &  \\
    CDE-100 &  &  &  &  &  &  &  \\

    \hline

    FB-0    &  &  &  &  &  &  &  \\
    FB-5    &  &  &  &  &  &  &  \\
    FB-15   &  &  &  &  &  &  &  \\
    FB-30   &  &  &  &  &  &  &  \\
    FB-50   &  &  &  &  &  &  &  \\
    FB-100  &  &  &  &  &  &  &  \\

    \hline
\end{tabular}

    \caption{Ruler results for various minimum support thresholds. Ruler ues rules mined after $t = 100s$ and keeps rules with $conf_{min} = 0.5$. Restricting to rules with high support does not improve performance.}
    \label{tab:5_experiments/4_ruler/3_rule_quality/supp_results}
\end{table}

The first numbers that stick out, but are not surprising, are the low numbers of high-confidence rules, especially on the CDE split. More interesting, however, is that precision does actually increase for high-support rules and recall does not suffer as much as expected, because few high-support rules seem to be able to stemm many predictions on their own. But still, the decrease in recall outweighs the increase in precision which is why the other experiments as well as the final evaluation keep all useful rules regardless of their support.

In case of rule confidence, expectations were higher. It was assumed that the decreasing performance in the experiment on different rule minig times, presented in \autoref{subsec:5_experiments/4_ruler/2_rule_count}, was due to a decreasing confidence of newly found rules and that this effect could be counteracted by increasing the minimum confidence threshold for rules. While this would reduce the number of usable rules, it was hoped that there were enough high-confidence rules to keep up the correct predictions. At a certain confidence threshold, a turning point was expected at which the decreasing recall would become too severe. Table A shows the results for confidence thresholds between 0.5 and 1.0.

\begin{table}
    \centering
    \begin{tabular}{| l | l | r | r | r | r | r | r |}
    \hline

    \multirow{2}{*}{\textbf{Split}} &
    &
    \multicolumn{6}{|c|}{\textbf{Minimal Confidence}} \\

    &
    &
    \multicolumn{1}{|c|}{\textbf{0.5}} &
    \multicolumn{1}{|c|}{\textbf{0.6}} &
    \multicolumn{1}{|c|}{\textbf{0.7}} &
    \multicolumn{1}{|c|}{\textbf{0.8}} &
    \multicolumn{1}{|c|}{\textbf{0.9}} &
    \multicolumn{1}{|c|}{\textbf{1.0}} \\

    \hline \hline

    \multirow{5}{*}{CDE-50}
    & \textbf{Rules} & \num{73034} & \num{53474} & \num{35139} & \num{25801} & \num{17621} & \num{13828} \\ \cline{2-8}
    & \textbf{Prec}  & 59.71       & 69.31       & 77.10       & 83.75       & 90.55       & 94.64       \\
    & \textbf{Rec}   & 34.29       & 29.47       & 24.48       & 18.95       & 13.46       & 5.83        \\
    & \textbf{F1}    & 40.89       & 38.52       & 34.32       & 28.08       & 20.55       & 8.41        \\
    & \textbf{mAP}   & 30.05       & 26.67       & 22.74       & 18.03       & 12.99       & 8.41        \\ \hline \hline

    \multirow{5}{*}{FB-50}
    & \textbf{Rules} & \num{86273} & \num{69707} & \num{51791} & \num{38544} & \num{24781} & \num{19619} \\ \cline{2-8}
    & \textbf{Prec}  & 66.76       & 73.68       & 79.55       & 84.63       & 91.23       & 93.20       \\
    & \textbf{Rec}   & 38.22       & 35.35       & 31.76       & 25.31       & 15.28       & 9.36        \\
    & \textbf{F1}    & 42.88       & 41.95       & 39.57       & 33.97       & 22.47       & 13.66       \\
    & \textbf{mAP}   & 33.91       & 31.80       & 28.92       & 23.50       & 14.75       & 9.12        \\ \hline

\end{tabular}

    \caption{Ruler results for various minimum confidence thresholds. Ruler ues rules mined after $t = 100s$ and keeps rules with $supp_{min} = 2$. Restricting to rules with high confidence does not improve performance.}
    \label{tab:5_experiments/4_ruler/3_rule_quality/conf_results}
\end{table}

What stands out is the large number of rules with a confidence of 100\% which make up roughly a fifth of all rules on both fact splits. Overall, confidence values are distributed quite uniformly over rules with $conf >= 0.5$. As expected, precision increases significantly with confidence, but contrary to expectations, recall drops faster from the first confidence raise on so that the inital conidence threshold of 0.5 is kept as default.

