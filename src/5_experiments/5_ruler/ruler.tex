While training the Texter means repeatedly applying backpropagation to the Texter's neural network, the Ruler's equivalent is mining rules on the known train facts. The found rules are then used during inference to infer new facts from the ones known about a query entity. In contrast to the Texter's training, all train facts are used for rule mining and all test facts are considered during evaluation - only one experiment deviates from this. Thus, the macro F1 and mAP metrics calculated over all test entities that were used for the contextual Texter's final evaluation are the default metrics applied to the Ruler.

For rule mining, \num{137738} and \num{238191} training facts are available in the CDE and FB splits, respectively. Depending on how long AnyBURL runs, more rules can be mined than the graph contains train facts. The final Rulers for the CDE and FB splits leverage \num{1743694} and \num{2510667} rules, respectively, that were mined after \num{1000} seconds on an Intel i7 quad-core processor. The vast majority of these rules have a rule body consisting of a single fact, such as (lives in, Norway) => (speaks, English), for example. Most of the rules have a confidence below 0.5, i.e. the fact they is probably wrong. Those facts are discarded, leaving \num{785514} useful facts on CDE and \num{} ones on the FB split.

The useful rules are then used for prediction by searching for rule groundings within the known facts, which include all train facts in addition to the query entity's known test facts. The heads of the rules for which a grounding could be found then yield the predicted facts. Table~\ref{tab:5_experiments/5_ruler/results} shows the results obtained when evaluating the predictions against all test facts. The "50" suffix refers to the fact that 50\% of the test facts may be used to apply rules during inference. Section~\ref{subsec:5_experiments/5_ruler/1_known} presents the results for different percentages. As one would expect, the FB15K-237-related results are once again better than the CoDEx-M ones, due to the larger training set and the higher number of facts per entity.

\begin{table}[h]
    \centering
    \begin{tabular}{| l | r | r | r | r | r | r | r | r | r |}
    \hline
    
    \multicolumn{1}{|c|}{\textbf{Split}} &
    \multicolumn{1}{|c|}{\textbf{mAP}} &
    \multicolumn{4}{|c|}{\textbf{Macro over ents}} &
    \multicolumn{4}{|c|}{\textbf{Micro over facts}} \\
    
    \multicolumn{1}{|c|}{} &
    \multicolumn{1}{|c|}{} &
    \multicolumn{1}{|c|}{\textbf{Prec}} &
    \multicolumn{1}{|c|}{\textbf{Rec}} &
    \multicolumn{1}{|c|}{\textbf{F1}} &
    \multicolumn{1}{|c|}{\textbf{Supp}} &
    \multicolumn{1}{|c|}{\textbf{Prec}} &
    \multicolumn{1}{|c|}{\textbf{Rec}} &
    \multicolumn{1}{|c|}{\textbf{F1}} &
    \multicolumn{1}{|c|}{\textbf{Supp}} \\
    
    \hline \hline
    
    CDE-0 & 0.84 &
    100.00 & 0.84 & 0.84 & \num{13.32} &
    100.00 & 0.00 & 0.00 & \num{25255} \\
    
    CDE-25 & 22.06 &
    66.90 & 24.27 & 32.02 & \num{13.32} &
    62.67 & 22.93 & 33.57 & \num{25255} \\
    
    CDE-50 & 29.26 &
    61.41 & 33.14 & 40.29 & \num{13.32} &
    58.47 & 31.52 & 40.96 & \num{25255} \\
    
    CDE-75 & 33.23 &
    57.88 & 38.10 & 43.59 & \num{13.32} &
    55.43 & 3638 & 43.93 & \num{25255} \\
    
    CDE-100 & 35.77 &
    55.62 & 41.48 & 45.28 & \num{13.32} &
    53.49 & 39.72 & 45.59 & \num{25255} \\
    
    \hline
    
    FB-0 & 3.19 &
    100.00 & 03.19 & 3.19 & \num{18.76} &
    100.00 & 0.00 & 0.00 & \num{15312} \\
    
    FB-25 & 27.41 &
    73.46 & 30.56 & 37.06 & \num{18.76} &
    69.28 & 30.15 & 42.02 & \num{15312} \\
    
    FB-50 & 33.39 &
    68.36 & 37.61 & 42.90 & \num{18.76} &
    64.90 & 36.76 & 46.94 & \num{15312} \\
    
    FB-75 & 36.22 &
    64.93 & 41.37 & 45.14 & \num{18.76} &
    62.76 & 40.82 & 49.47 & \num{15312} \\
    
    FB-100 & 38.43 &
    63.08 & 44.34 & 46.97 & \num{18.76} &
    60.98 & 43.51 & 50.79 & \num{15312} \\
    
    \hline
\end{tabular}

    \caption{Ruler evaluation against known+unknown. Codex 50 means 50\% known test facts}
    \label{tab:5_experiments/5_ruler/results}
\end{table}

In the following, Section~\ref{subsec:5_experiments/5_ruler/1_known} shows how well the Ruler works in a few shot scenarios and presents an alternative evaluation scenario, Section~\ref{subsec:5_experiments/5_ruler/2_rule_count} shows how important the number of rules is the performance of the Ruler, and Section~\ref{subsec:5_experiments/5_ruler/3_rule_quality} explains why restriction to high-quality rules leads to worse performance. All further evaluations were performed on Rulers that were trained on the rule set mined after 10 seconds, which does not influence the qualitative statements. Table~\ref{tab:5_experiments/5_ruler/2_rule_count/results} in Section~\ref{subsec:5_experiments/5_ruler/2_rule_count} can be used to estimate performances for larger rule sets.

\subsection{Known Test Facts}
\label{subsec:5_experiments/5_ruler/1_known}
Die Annahme, dass während der Inferenz 50 der Fakten einer Testentität bekannt seien führt zu einem stabilen Evaluationsergebnis, insbesondere wenn das Datenset nur wenige Fakten pro Entität bereithält. Stehen jedoch viele Fakten zur Verfügung entspricht die Annahme nicht mehr dem in der Praxis interessanten Few-Shot-Szenario. Bei einer Testentität aus dem FB-Split entsprechen 50 im Schnitt bereits neun bekannten Fakten pro Entität.

The assumption that 50\% of a test entity's facts are known during inference leads to a stable evaluation result, especially if the data set provides only a few facts per entity. However, if many facts are available, the assumption no longer corresponds to a few-shot scenario which is particularly interesting in practice. For a test entity from the FB split, 50\% known test facts already corresponds to an average of nine facts per entity.

To get an idea how much the number of known test facts affects the Ruler's function, this experiment evaluates all of the Power splits introduced in Section~\ref{sec:5_experiments/2_power_datasets}, including the splits which include none or all of the entities' facts in the known test set. Test entities without known test facts correspond to open-world entities. Although, the Ruler cannot predict facts for them, they were processed to assure that the evaluation code would calculate precision and recall to 100\% and 0\%, respectively. The splits that include all of an entity's test facts as known facts does not represent a practical use case for the Ruler, neither. Still, it has been included to show how much increasing the share of known facts beyond 50\% would bring. In practice, the most interesting use cases, include the splits with a low percentage of known test facts. The expectation was that recall would increase logarithmically with the number of known test facts, as the first fact about an entity reveals much more information than the last one, while precision should be independent from the choice of random known facts. Thus, F1 and mAP were expected to increase logarithmically with the proportion of known facts.

\begin{table}
    \centering
    \begin{tabular}{| l | r | r | r | r | r | r | r | r | r |}
    \hline
    
    \multicolumn{1}{|c|}{\textbf{Split}} &
    \multicolumn{1}{|c|}{\textbf{mAP}} &
    \multicolumn{4}{|c|}{\textbf{Macro over ents}} &
    \multicolumn{4}{|c|}{\textbf{Micro over facts}} \\
    
    \multicolumn{1}{|c|}{} &
    \multicolumn{1}{|c|}{} &
    \multicolumn{1}{|c|}{\textbf{Prec}} &
    \multicolumn{1}{|c|}{\textbf{Rec}} &
    \multicolumn{1}{|c|}{\textbf{F1}} &
    \multicolumn{1}{|c|}{\textbf{Supp}} &
    \multicolumn{1}{|c|}{\textbf{Prec}} &
    \multicolumn{1}{|c|}{\textbf{Rec}} &
    \multicolumn{1}{|c|}{\textbf{F1}} &
    \multicolumn{1}{|c|}{\textbf{Supp}} \\
    
    \hline \hline
    
    CDE-0 & 0.84 &
    100.00 & 0.84 & 0.84 & \num{13.32} &
    100.00 & 0.00 & 0.00 & \num{25255} \\
    
    CDE-25 & 22.06 &
    66.90 & 24.27 & 32.02 & \num{13.32} &
    62.67 & 22.93 & 33.57 & \num{25255} \\
    
    CDE-50 & 29.26 &
    61.41 & 33.14 & 40.29 & \num{13.32} &
    58.47 & 31.52 & 40.96 & \num{25255} \\
    
    CDE-75 & 33.23 &
    57.88 & 38.10 & 43.59 & \num{13.32} &
    55.43 & 3638 & 43.93 & \num{25255} \\
    
    CDE-100 & 35.77 &
    55.62 & 41.48 & 45.28 & \num{13.32} &
    53.49 & 39.72 & 45.59 & \num{25255} \\
    
    \hline
    
    FB-0 & 3.19 &
    100.00 & 03.19 & 3.19 & \num{18.76} &
    100.00 & 0.00 & 0.00 & \num{15312} \\
    
    FB-25 & 27.41 &
    73.46 & 30.56 & 37.06 & \num{18.76} &
    69.28 & 30.15 & 42.02 & \num{15312} \\
    
    FB-50 & 33.39 &
    68.36 & 37.61 & 42.90 & \num{18.76} &
    64.90 & 36.76 & 46.94 & \num{15312} \\
    
    FB-75 & 36.22 &
    64.93 & 41.37 & 45.14 & \num{18.76} &
    62.76 & 40.82 & 49.47 & \num{15312} \\
    
    FB-100 & 38.43 &
    63.08 & 44.34 & 46.97 & \num{18.76} &
    60.98 & 43.51 & 50.79 & \num{15312} \\
    
    \hline
\end{tabular}

    \caption{Ruler evaluation against unknown. Codex 50 means 50\% known test facts}
    \label{tab:5_experiments/5_ruler/1_unknown/all_results}
\end{table}

Wie erwartet steigt

Für die Splits ohne bekannte Testfakten ist der Recall größer null, weil es Entitäten ohne Ground Truth Fakten gibt. Der Ruler, der ohne bekannte Testfakten keine Vorhersage machen kann, liegt also für diese wenigen Entitäten richtig wenn er keine Vorhersage macht.

could evaluate against unknown valid facts, close to use case, or could eval against all valid facts as usual []
ok to validate against known + unknown facts because known not seen during training
eval against unknown means filter out predicted known facts
table ? shows comparison between both eval methods
when evaluating against known + unknown, best result for 100
when evaluating against unknwon, best result for 50 as balance between known and remaining for prediction. 0 = open world = no preds, 100 = no left for prediction



\subsection{Number of Rules}
\label{subsec:5_experiments/5_ruler/2_rule_count}
better results the more train rules as shown by table ? (?)

\begin{table}
    \centering
    \begin{tabular}{| l | l | r | r | r | r |}
    \hline

    \multicolumn{1}{|c|}{\multirow{2}{*}{\textbf{Text Set}}} &
    \multicolumn{1}{|c|}{\multirow{2}{*}{\textbf{Texter}}} &
    \multicolumn{4}{|c|}{\textbf{Pooling}} \\

    &
    &
    \multicolumn{1}{|c|}{\textbf{CLS}} &
    \multicolumn{1}{|c|}{\textbf{No CLS}} &
    \multicolumn{1}{|c|}{\textbf{No Pad}} &
    \multicolumn{1}{|c|}{\textbf{All}} \\

    \hline \hline

    \multirow{2}{*}{cde-cde-1-clean}
    & Simple &  &  &  &  \\
    & Attend &  &  &  &  \\ \hline

    \multirow{2}{*}{cde-irt-1-marked}
    & Simple &  &  &  &  \\
    & Attend &  &  &  &  \\ \hline

    \multirow{2}{*}{cde-irt-5-marked}
    & Simple &  &  &  &  \\
    & Attend &  &  &  &  \\ \hline

    \multirow{2}{*}{cde-irt-15-marked}
    & Simple &  &  &  &  \\
    & Attend &  &  &  &  \\ \hline

    \multirow{2}{*}{cde-irt-30-marked}
    & Simple &  &  &  &  \\
    & Attend &  &  &  &  \\ \hline \hline

    \multirow{2}{*}{fb-irt-1-marked}
    & Simple &  &  &  &  \\
    & Attend &  &  &  &  \\ \hline

    \multirow{2}{*}{fb-irt-5-marked}
    & Simple &  &  &  &  \\
    & Attend &  &  &  &  \\ \hline

    \multirow{2}{*}{fb-irt-15-marked}
    & Simple &  &  &  &  \\
    & Attend &  &  &  &  \\ \hline

    \multirow{2}{*}{fb-irt-30-marked}
    & Simple &  &  &  &  \\
    & Attend &  &  &  &  \\ \hline

    \multirow{2}{*}{fb-owe-1-clean}
    & Simple &  &  &  &  \\
    & Attend &  &  &  &  \\ \hline

\end{tabular}

    \caption{more train rules = better metrics, evaluated against unknown}
    \label{tab:5_experiments/5_ruler/2_rule_count/grid_search}
\end{table}


\subsection{Rule Quality}
\label{subsec:5_experiments/5_ruler/3_rule_quality}
Standardmäßig werden Regeln mit Support größer eins und Konfidenz größer 0.001 (?) beibehalten, also praktisch alle Regeln. Durch eine Beschränkung auf qualitativ hochwertigere Regeln kann die Laufzeit des Rulers verringert werden. Welche Auswirkungen dies auf die Qualität der Fakten-Vorhersagen hat wird in diesem Experiment überprüft in dem die Hemmschwellen für Support und Konfidenz höher angesetzt werden. Die Variation von Support und Konfidenz Schwellen erfolgt unabhängig voneinander. Die Erwartung ist, dass das Entfernen von Regeln mit niedrigem Support einen negative Effekt auf die Vorhersagequalität hat, da seltene Regeln veloren gehen könnten die bestimmte Randfälle sehr gut beschreiben. Ein Anheben der Konfidenzschwelle sollte zu einer Verbesserung von Präzision, Recall und somit F1 führen, gleichzeit aber die letztendlich wichtige mAP schwächen, da qualitativ niederwertige Regeln nicht zu Vorhersagen führen können die bei den Top-Fakten-Vorhersagen stören.

different confidence thresholds

<results>

cut out rules with low support

<results>

